\documentclass{exam}
\usepackage{amsmath, amsfonts}
\usepackage{verbatim}
\usepackage{graphicx}
\usepackage[super]{nth}

\DeclareMathOperator*{\argmin}{argmin}

\usepackage[hyperfootnotes=false]{hyperref}

\usepackage[usenames,dvipsnames]{color}
\newcommand{\note}[1]{
	\noindent~\\
	\vspace{0.25cm}
	\fcolorbox{Red}{Orange}{\parbox{0.99\textwidth}{#1\\}}
	%{\parbox{0.99\textwidth}{#1\\}}
	\vspace{0.25cm}
}


\renewcommand{\vec}[1]{\mathbf{#1}}
\newcommand{\lecture}{ML4AAD}
\newcommand{\lecturelong}{Machine Learning for Automated Algorithm Design}
\newcommand{\semester}{WS 2018/19}
\newcommand{\assignment}[1]{\nth{#1} Assignment}
\newcommand{\lectors}{M. Lindauer \& A. Biedenkapp}
\newcommand{\hide}[1]{}


\newcommand{\gccs}{\paragraph{General constraints for code submissions}
    
    \begin{itemize}
        \item The program can be called as stated on the exercise sheet.
        \item The program exactly returns the required output (neither less nor more) -- please use a \texttt{--verbose} option to increase the verbosity level for debugging\footnote{You might want to use \texttt{argparse} for simplicity's sake.}.
        \item Your scripts should be commented to be readable for the tutors. All functions and classes are documented with a docstring. 
        \item Provide a README ($\to$ how to install requirements and run your program(s)) and (if necessary) an installation script if your program requires any other packages.
        \item Programs are to be submitted in python $3.5$ or newer.
        \item Adding new packages to the requirements.txt is fine. If you do this however, you'll have to give a brief description why you use that package and a link to it's documentation or github page.
        \item All prepared unittests have to pass.
        \item We don't accept ipython notebook submissions.
        \item Points will be deducted if you don't fullfill these constraints.
        \item You are allowed (sometimes required) to reuse code from previous exercises.
    \end{itemize}
    \rule{\textwidth}{.5pt}
    \smallskip\\
    \noindent}
%\renewcommand{\hide}[1]{#1}

\qformat{\thequestion. \textbf{\thequestiontitle}\hfill[\thepoints]}
\bonusqformat{\thequestion. \textbf{\thequestiontitle}\hfill[\thepoints]}

\pagestyle{headandfoot}

%%%%%% MODIFY FOR EACH SHEET!!!! %%%%%%
\newcommand{\duedate}{24.05.19 (10:00)}
\newcommand{\due}{{\bf This assignment is due on \duedate.} }
\firstpageheader
{Due: \duedate \\ Points: 5}
{{\bf\lecture}\\ \assignment{4}}
{\lectors\\ \semester}

\runningheader
{Due: \duedate}
{\assignment{4}}
{\semester}
%%%%%% MODIFY FOR EACH SHEET!!!! %%%%%%

\firstpagefooter
{}
{\thepage}
{}

\runningfooter
{}
{\thepage}
{}

\headrule
\pointsinrightmargin
\bracketedpoints
\marginpointname{pt.}

\begin{document}
	\gccs
	Now that you have learned about hyperparameter optimization techniques such as Bayesian optimization (BO) you will implement this loop yourself.
	\begin{questions}
		\titledquestion{Bayesian Optimization for HPO}[4]
		We provide you with a rough structure of the BO loop using a Gaussian Process. You will implement the remaining parts to \textbf{minimize} a synthetic 1D function.
		\begin{parts}
		\part[1] Implement the acquisition functions \textit{Expected Improvement} and \textit{Lower Confidence Bound}\footnote{Similar to \textit{Lower Confidence Bound}, but for minimizing an objective value} as presented in the lecture (use \texttt{NumPy} and \texttt{SciPy} wherever possible for efficiency). Keep in mind that you will use \texttt{scipy.minimize} to optimize the acquisition function.
		\part[1] Generate plots that demonstrate the functionality of the BO loop and of the implemented acquisition functions, e.g. best-so-far seen function value over time and the acquisition function values at different time steps. Add all plots to a PDF and briefly discuss what you can learn from these plots.
		\part[1] Implement Grid Search and Random Search.
		\part[1] Compare your implementations of BO against Random Search and Grid Search for at most 50 function evaluations\footnote{Hint: Your implementations of BO should perform better than Random Search.}
	\end{parts}
		
		\titledquestion{Feedback}[Bonus: 1]
		For each question in this assignment, state:
		\begin{itemize}
			\item How long you worked on it.
			\item What you learned.
			\item Anything you would improve in this question if you were teaching the course.
		\end{itemize}
	\end{questions}
	
	\noindent
	\due Submit your solution for the tasks by uploading a PDF to your groups BitBucket repository. The PDF has to include the name of the submitter(s).
\end{document}