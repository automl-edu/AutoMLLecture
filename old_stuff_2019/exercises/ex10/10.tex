\documentclass{exam}
\usepackage{amsmath, amsfonts}
\usepackage{bm}
\usepackage{verbatim}
\usepackage{graphicx}
\usepackage[super]{nth}
\usepackage{booktabs}

\DeclareMathOperator*{\argmin}{argmin}

\usepackage[hyperfootnotes=false]{hyperref}

\usepackage[usenames,dvipsnames]{color}
\newcommand{\note}[1]{
	\noindent~\\
	\vspace{0.25cm}
	\fcolorbox{Red}{Orange}{\parbox{0.99\textwidth}{#1\\}}
	%{\parbox{0.99\textwidth}{#1\\}}
	\vspace{0.25cm}
}


\renewcommand{\vec}[1]{\mathbf{#1}}
\newcommand{\lecture}{ML4AAD}
\newcommand{\lecturelong}{Machine Learning for Automated Algorithm Design}
\newcommand{\semester}{WS 2018/19}
\newcommand{\assignment}[1]{\nth{#1} Assignment}
\newcommand{\lectors}{M. Lindauer \& A. Biedenkapp}
\newcommand{\hide}[1]{}


\newcommand{\gccs}{\paragraph{General constraints for code submissions}
    
    \begin{itemize}
        \item The program can be called as stated on the exercise sheet.
        \item The program exactly returns the required output (neither less nor more) -- please use a \texttt{--verbose} option to increase the verbosity level for debugging\footnote{You might want to use \texttt{argparse} for simplicity's sake.}.
        \item Your scripts should be commented to be readable for the tutors. All functions and classes are documented with a docstring. 
        \item Provide a README ($\to$ how to install requirements and run your program(s)) and (if necessary) an installation script if your program requires any other packages.
        \item Programs are to be submitted in python $3.5$ or newer.
        \item Adding new packages to the requirements.txt is fine. If you do this however, you'll have to give a brief description why you use that package and a link to it's documentation or github page.
        \item All prepared unittests have to pass.
        \item We don't accept ipython notebook submissions.
        \item Points will be deducted if you don't fullfill these constraints.
        \item You are allowed (sometimes required) to reuse code from previous exercises.
    \end{itemize}
    \rule{\textwidth}{.5pt}
    \smallskip\\
    \noindent}
%\renewcommand{\hide}[1]{#1}

\qformat{\thequestion. \textbf{\thequestiontitle}\hfill[\thepoints]}
\bonusqformat{\thequestion. \textbf{\thequestiontitle}\hfill[\thepoints]}

\pagestyle{headandfoot}

%%%%%% MODIFY FOR EACH SHEET!!!! %%%%%%
\newcommand{\duedate}{12.07.19 (10:00)}
\newcommand{\due}{{\bf This assignment is due on \duedate.} }
\firstpageheader
{Due: \duedate \\ Points: 9}
{{\bf\lecture}\\ \assignment{9}}
{\lectors\\ \semester}

\runningheader
{Due: \duedate}
{\assignment{9}}
{\semester}
%%%%%% MODIFY FOR EACH SHEET!!!! %%%%%%

\firstpagefooter
{}
{\thepage}
{}

\runningfooter
{}
{\thepage}
{}

\headrule
\pointsinrightmargin
\bracketedpoints
\marginpointname{pt.}




\newcommand{\parents}{p}
\newcommand{\negation}[1]{\overline{#1}}
%\newcommand{\tuple}[1]{\langle #1 \rangle}
\newcommand{\tuple}[1]{\left<#1\right>}
\newcommand{\dom}[1]{dom(#1)}              % domain

\newcommand{\false}{false}
\newcommand{\true}{true}
\newcommand{\TRUE}{{\mbox{\scriptsize \em TRUE}}}
\newcommand{\FALSE}{{\mbox{\scriptsize \em FALSE}}}

\newcommand{\bSigma}{\bm{\Sigma}}
\newcommand{\bmu}{\bm{\mu}}
\newcommand{\bx}{\bm{x}}
\newcommand{\by}{\bm{y}}
\newcommand{\bX}{\bm{X}}
\newcommand{\bI}{\bm{I}}
\newcommand{\bw}{\bm{w}}
\newcommand{\ba}{\bm{a}}
\newcommand{\bb}{\bm{b}}
\newcommand{\bk}{\bm{k}}  
\newcommand{\inv}{^{-1}}

\newcommand{\norm}{{\mathcal{N}}}

\newcommand\transpose{^{\textrm{\tiny{\sf{T}}}}}

\begin{document}
	\gccs
	Now that you have learned about the application of reinforcement learning for learning to learn and algorithm control, you will implement
	a basic RL agent yourself.
	\begin{questions}
		\titledquestion{Q-Learning}[9]
		We present you a simple environment (based on the OpenAI gym standard), in which your agent has to learn to approximate a Sigmoid function.
		
		\begin{parts}
		\part[2] Implement a tabular function approximator that is able to handle float data, by mapping float state feature to the closest integer.
		
		\part[5] Implement the Q-learning algorithm. Train your agent on the provided environment in its default configuration for $10\,000$ training episodes.
			Your agent will have to learn a policy of length 11 with action space $\{0, 1, 2\}$. The optimal policy will result in a reward of roughly $10.16$
			and looks like $0, 0, 0, 0, 1, 1, 1, 2, 2, 2, 2$.
			
			If your agent at first does not produce the optimal policy play around with the learning rate $\alpha$ and the exploration factor $\epsilon$.
			In your PDF report all learning rates and exploration factors you tried and the resulting policies.
			
			In your own words explain how the learning rate and the exploration factor influence the Q-learning on this deterministic environment.

		\part[2] In the same folder you called \texttt{main.py} from, the results of training will be stored. Plot the true reward against the expected reward for all
			learning rates and exploration factors you tried and add them to your PDF. 

		\end{parts}

		\titledquestion{Feedback}[Bonus: 0.5]
		For each question in this assignment, state:
		\begin{itemize}
			\item How long you worked on it.
			\item What you learned.
			\item Anything you would improve in this question if you were teaching the course.
		\end{itemize}
	\end{questions}
	
	\noindent
	\due Submit your solution for the tasks by uploading a PDF to your groups BitBucket repository. The PDF has to include the name of the submitter(s).
\end{document}
