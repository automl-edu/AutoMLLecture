\pdfminorversion=4 % for acroread
\documentclass[aspectratio=169,t,xcolor={usenames,dvipsnames}]{beamer}
%\documentclass[t,handout,xcolor={usenames,dvipsnames}]{beamer}
\usepackage{../beamerstyle}
\usepackage{dsfont}
\usepackage{bm}
\usepackage[english]{babel}
\usepackage[utf8]{inputenc}
\usepackage{graphicx}
\usepackage{algorithm}
\usepackage[ruled,vlined,algo2e,linesnumbered]{algorithm2e}
%\usepackage[boxed,vlined]{algorithm2e}
\usepackage{hyperref}
\usepackage{booktabs}
\usepackage{mathtools}

\usepackage{amsmath,amssymb}
\usepackage{listings}
\lstset{frame=lines,framesep=3pt,numbers=left,numberblanklines=false,basicstyle=\ttfamily\small}

\usepackage{subfig}
\usepackage{multicol}
%\usepackage{appendixnumberbeamer}
%
\usepackage{tcolorbox}

\usepackage{pgfplots}
\usepackage{tikz}
\usetikzlibrary{trees} 
\usetikzlibrary{shapes.geometric}
\usetikzlibrary{positioning,shapes,shadows,arrows,calc,mindmap}
\usetikzlibrary{positioning,fadings,through}
\usetikzlibrary{decorations.pathreplacing}
\usetikzlibrary{intersections}
\usetikzlibrary{positioning,fit,calc,shadows,backgrounds}
\pgfdeclarelayer{background}
\pgfdeclarelayer{foreground}
\pgfsetlayers{background,main,foreground}
\tikzstyle{activity}=[rectangle, draw=black, rounded corners, text centered, text width=8em]
\tikzstyle{data}=[rectangle, draw=black, text centered, text width=8em]
\tikzstyle{myarrow}=[->, thick, draw=black]

% Define the layers to draw the diagram
\pgfdeclarelayer{background}
\pgfdeclarelayer{foreground}
\pgfsetlayers{background,main,foreground}

%\usepackage{listings}
%\lstset{numbers=left,
%  showstringspaces=false,
%  frame={tb},
%  captionpos=b,
%  lineskip=0pt,
%  basicstyle=\ttfamily,
%%  extendedchars=true,
%  stepnumber=1,
%  numberstyle=\small,
%  xleftmargin=1em,
%  breaklines
%}

 
\definecolor{blue}{RGB}{0, 74, 153}

\usetheme{Boadilla}
%\useinnertheme{rectangles}
\usecolortheme{whale}
\setbeamercolor{alerted text}{fg=blue}
\useoutertheme{infolines}
\setbeamertemplate{navigation symbols}{\vspace{-5pt}} % to lower the logo
\setbeamercolor{date in head/foot}{bg=white} % blue
\setbeamercolor{date in head/foot}{fg=white}
\setbeamercolor{author  in head/foot}{bg=white} %blue
\setbeamercolor{title in head/foot}{bg=white} % blue
\setbeamercolor{title}{fg=white, bg=blue}
\setbeamercolor{block title}{fg=white,bg=blue}
\setbeamercolor{block body}{bg=blue!10}
\setbeamercolor{frametitle}{fg=white, bg=blue}
\setbeamercovered{invisible}

\makeatletter
\setbeamertemplate{footline}
{
  \leavevmode%
  \hbox{%
  \begin{beamercolorbox}[wd=.333333\paperwidth,ht=2.25ex,dp=1ex,center]{author in head/foot}%
%    \usebeamerfont{author in head/foot}\insertshortauthor
  \end{beamercolorbox}%
  \begin{beamercolorbox}[wd=.333333\paperwidth,ht=2.25ex,dp=1ex,center]{title in head/foot}%
    \usebeamerfont{title in head/foot}\insertshorttitle
  \end{beamercolorbox}%
  \begin{beamercolorbox}[wd=.333333\paperwidth,ht=2.25ex,dp=1ex,right]{date in head/foot}%
    \usebeamerfont{date in head/foot}\insertshortdate{}\hspace*{2em}
%    \insertframenumber\hspace*{2ex} 
  \end{beamercolorbox}}%
  \vskip0pt%
}
\makeatother

%\pgfdeclareimage[height=1.2cm]{automl}{images/logos/automl.png}
%\pgfdeclareimage[height=1.2cm]{freiburg}{images/logos/freiburg}

%\logo{\pgfuseimage{freiburg}}

\newcommand{\comment}[1]{
	\noindent
	%\vspace{0.25cm}
	{\color{red}{\textbf{TODO:} #1}}
	%\vspace{0.25cm}
}
\renewcommand{\comment}[1]{}
\newcommand{\hide}[1]{}
\newcommand{\cemph}[2]{\emph{\textcolor{#1}{#2}}}

\newcommand{\lit}[1]{{\footnotesize\color{black!70}[#1]}}

\newcommand{\litw}[1]{{\footnotesize\color{black!20}[#1]}}


\newcommand{\myframe}[2]{\begin{frame}[c]{#1}#2\end{frame}}
\newcommand{\myframetop}[2]{\begin{frame}{#1}#2\end{frame}}
\newcommand{\myit}[1]{\begin{itemize}#1\end{itemize}}
\newcommand{\myblock}[2]{\begin{block}{#1}#2\end{block}}


\newcommand{\votepurple}[1]{\textcolor{Purple}{$\bigstar$}}
\newcommand{\voteyellow}[1]{\textcolor{Goldenrod}{$\bigstar$}}
\newcommand{\voteblue}[1]{\textcolor{RoyalBlue}{$\bigstar$}}
\newcommand{\votepink}[1]{\textcolor{Pink}{$\bigstar$}}

\newcommand{\diff}{\mathop{}\!\mathrm{d}}
\newcommand{\refstyle}[1]{{\small{\textcolor{gray}{#1}}}}
\newcommand{\hands}[0]{\includegraphics[height=1.5em]{images/hands}}
\newcommand{\transpose}[0]{{\textrm{\tiny{\sf{T}}}}}
\newcommand{\norm}{{\mathcal{N}}}
\newcommand{\cutoff}[0]{\kappa}
\newcommand{\instD}[0]{\dataset}
\newcommand{\insts}[0]{\mathcal{I}}
\newcommand{\inst}[0]{i}
\newcommand{\pcs}[0]{\mathbf{\Lambda}}
\newcommand{\bx}[0]{\conf}
\newcommand{\conf}[0]{\mathbf{\lambda}}
\newcommand{\defconf}[0]{\mathbf{\lambda}_{\text{def}}}
\newcommand{\finconf}[0]{\mathbf{\lambda}^*}
\newcommand{\incumbent}[0]{\finconf}
\newcommand{\confs}[0]{\pcs}
%\newcommand{\vlambda}[0]{\bm{\lambda}}
%\newcommand{\vLambda}[0]{\bm{\Lambda}}
\newcommand{\dataset}[0]{\mathcal{D}}
\newcommand{\datasets}[0]{\mathbf{D}}
\newcommand{\loss}[0]{\mathcal{L}}

% \renewcommand{\vec}[1]{\mathbf{#1}}
\newcommand{\hist}[0]{\mathcal{H}}
\newcommand{\param}[0]{p}
\newcommand{\algo}[0]{\mathcal{A}}
\newcommand{\algos}[0]{\mathbf{A}}
%\newcommand{\nn}[0]{N}
\newcommand{\feats}[0]{\mathcal{F}}
\newcommand{\feat}[0]{\vec{f}}
\newcommand{\cluster}[0]{\vec{h}}
\newcommand{\clusters}[0]{\vec{H}}
\newcommand{\perf}[0]{\mathbb{R}}
%\newcommand{\surro}[0]{\mathcal{S}}
\newcommand{\surro}[0]{\hat{f}}
\newcommand{\func}[0]{f}
\newcommand{\epm}[0]{\surro}
\newcommand{\portfolio}[0]{\mathcal{P}}
\newcommand{\schedule}[0]{\mathcal{S}}
\newcommand{\mdata}[0]{\dataset_{\text{meta}}}

% Deep Learning
\newcommand{\weights}[0]{\theta}
\newcommand{\metaweights}[0]{\phi}


% reinforcement learning
\newcommand{\policies}[0]{\Pi}
\newcommand{\policy}[0]{\pi}
\newcommand{\actionRL}[0]{a}
\newcommand{\stateRL}[0]{s}
\newcommand{\statesRL}[0]{\mathcal{S}}
\newcommand{\rewardRL}[0]{r}
\newcommand{\rewardfuncRL}[0]{\mathcal{R}}

\RestyleAlgo{algoruled}
\DontPrintSemicolon
\LinesNumbered
\SetAlgoVlined
\SetFuncSty{textsc}

\SetKwInOut{Input}{Input}
\SetKwInOut{Output}{Output}
\SetKw{Return}{return}

%\newcommand{\changed}[1]{{\color{red}#1}}

%\newcommand{\citeN}[1]{\citeauthor{#1}~(\citeyear{#1})}

\renewcommand{\vec}[1]{\mathbf{#1}}
\DeclareMathOperator*{\argmin}{arg\,min}
\DeclareMathOperator*{\argmax}{arg\,max}

\newcommand{\aqme}{\textit{AQME}}
\newcommand{\aslib}{\textit{ASlib}}
\newcommand{\llama}{\textit{LLAMA}}
\newcommand{\satzilla}{\textit{SATzilla}}
\newcommand{\satzillaY}[1]{\textit{SATzilla'{#1}}}
\newcommand{\snnap}{\textit{SNNAP}}
\newcommand{\claspfolioTwo}{\textit{claspfolio~2}}
\newcommand{\flexfolio}{\textit{FlexFolio}}
\newcommand{\claspfolioOne}{\textit{claspfolio~1}}
\newcommand{\isac}{\textit{ISAC}}
\newcommand{\eisac}{\textit{EISAC}}
\newcommand{\sss}{\textit{3S}}
\newcommand{\sunny}{\textit{Sunny}}
\newcommand{\ssspar}{\textit{3Spar}}
\newcommand{\cshc}{\textit{CSHC}}  
\newcommand{\cshcpar}{\textit{CSHCpar}}  
\newcommand{\measp}{\textit{ME-ASP}} 
\newcommand{\aspeed}{\textit{aspeed}}
\newcommand{\autofolio}{\textit{AutoFolio}}
\newcommand{\cedalion}{\textit{Cedalion}}
\newcommand{\fanova}{\textit{fANOVA}}
\newcommand{\sbs}{\textit{SB}}
\newcommand{\oracle}{\textit{VBS}}

% like approaches
\newcommand{\claspfoliolike}[1]{\texttt{claspfolio-#1-like}}
\newcommand{\satzillalike}[1]{\texttt{SATzilla'#1-like}}
\newcommand{\isaclike}{\texttt{ISAC-like}}
\newcommand{\ssslike}{\texttt{3S-like}}
\newcommand{\measplike}{\texttt{ME-ASP-like}}

\newcommand{\aspCoseal}{\textit{ASP-POTASSCO}}
\newcommand{\cspCoseal}{\textit{CSP-2010}}
\newcommand{\maxsatCoseal}{\textit{MAXSAT12-PMS}}
\newcommand{\premarCoseal}{\textit{PRE\-MARSHALLING}}
\newcommand{\qbfCoseal}{\textit{QBF-2011}}
\newcommand{\satallTwelveCoseal}{\textit{SAT12-ALL}}
\newcommand{\sathandTwelveCoseal}{\textit{SAT12-HAND}}
\newcommand{\satinduTwelveCoseal}{\textit{SAT12-INDU}}
\newcommand{\satrandTwelveCoseal}{\textit{SAT12-RAND}}
\newcommand{\sathandElevenCoseal}{\textit{SAT11-HAND}}
\newcommand{\satinduElevenCoseal}{\textit{SAT11-INDU}}
\newcommand{\satrandElevenCoseal}{\textit{SAT11-RAND}}
\newcommand{\proteusCoseal}{\textit{PROTEUS-2014}}

\newcommand{\irace}{\textit{I/F-race}}
\newcommand{\gga}{\textit{GGA}}
\newcommand{\smac}{\textit{SMAC}}
\newcommand{\paramils}{\textit{ParamILS}}
\newcommand{\spearmint}{\textit{Spearmint}}
\newcommand{\tpe}{\textit{TPE}}

\newcommand{\gringo}{\textit{gringo}}
\newcommand{\clasp}{\textit{clasp}}
\newcommand{\lingeling}{\textit{lingeling}}

\newcommand{\hydra}{\textit{Hydra}}

\newcommand{\plingeling}{\textit{Plingeling}}
\newcommand{\ccasat}{\textit{CCASat}}

\usepackage{pifont}
\newcommand{\itarrow}{\mbox{\Pisymbol{pzd}{229}}}
\newcommand{\ithook}{\mbox{\Pisymbol{pzd}{52}}}
\newcommand{\itcross}{\mbox{\Pisymbol{pzd}{56}}}
\newcommand{\ithand}{\mbox{\raisebox{-1pt}{\Pisymbol{pzd}{43}}}}

%\DeclareMathOperator*{\argmax}{arg\,max}

\newcommand{\ie}{{\it{}i.e.\/}}
\newcommand{\eg}{{\it{}e.g.\/}}
\newcommand{\cf}{{\it{}cf.\/}}
\newcommand{\wrt}{\mbox{w.r.t.}}
\newcommand{\vs}{{\it{}vs\/}}
\newcommand{\vsp}{{\it{}vs\/}}
\newcommand{\etc}{{\copyedit{etc.}}}
\newcommand{\etal}{{\it{}et al.\/}}

\newcommand{\pscProc}{{\bf procedure}}
\newcommand{\pscBegin}{{\bf begin}}
\newcommand{\pscEnd}{{\bf end}}
\newcommand{\pscEndIf}{{\bf endif}}
\newcommand{\pscFor}{{\bf for}}
\newcommand{\pscEach}{{\bf each}}
\newcommand{\pscThen}{{\bf then}}
\newcommand{\pscElse}{{\bf else}}
\newcommand{\pscWhile}{{\bf while}}
\newcommand{\pscIf}{{\bf if}}
\newcommand{\pscRepeat}{{\bf repeat}}
\newcommand{\pscUntil}{{\bf until}}
\newcommand{\pscWithProb}{{\bf with probability}}
\newcommand{\pscOtherwise}{{\bf otherwise}}
\newcommand{\pscDo}{{\bf do}}
\newcommand{\pscTo}{{\bf to}}
\newcommand{\pscOr}{{\bf or}}
\newcommand{\pscAnd}{{\bf and}}
\newcommand{\pscNot}{{\bf not}}
\newcommand{\pscFalse}{{\bf false}}
\newcommand{\pscEachElOf}{{\bf each element of}}
\newcommand{\pscReturn}{{\bf return}}

%\newcommand{\param}[1]{{\sl{}#1}}
\newcommand{\var}[1]{{\it{}#1}}
\newcommand{\cond}[1]{{\sf{}#1}}
%\newcommand{\state}[1]{{\sf{}#1}}
%\newcommand{\func}[1]{{\sl{}#1}}
\newcommand{\set}[1]{{\Bbb #1}}
%\newcommand{\inst}[1]{{\tt{}#1}}
\newcommand{\myurl}[1]{{\small\sf #1}}

\newcommand{\Nats}{{\Bbb N}}
\newcommand{\Reals}{{\Bbb R}}
\newcommand{\extset}[2]{\{#1 \; | \; #2\}}

\newcommand{\vbar}{$\,\;|$\hspace*{-1em}\raisebox{-0.3mm}{$\,\;\;|$}}
\newcommand{\vendbar}{\raisebox{+0.4mm}{$\,\;|$}}
\newcommand{\vend}{$\,\:\lfloor$}


\newcommand{\goleft}[2][.7]{\parbox[t]{#1\linewidth}{\strut\raggedright #2\strut}}
\newcommand{\rightimage}[2][.3]{\mbox{}\hfill\raisebox{1em-\height}[0pt][0pt]{\includegraphics[width=#1\linewidth]{#2}}\vspace*{-\baselineskip}}





\newcommand{\inducer}{\mathcal{I}}
\newcommand{\R}{\mathds{R}}

%The following might look confusing but allows us to switch the notation of the optimization problem independently from the notation of the hyper parameter optimization
\newcommand{\xx}{\conf} %x of the optimizer
\newcommand{\xxi}[1][i]{\conf_{#1}} %i-th component of xx (not confuse with i-th individual)
\newcommand{\XX}{\pcs} %search space / domain of f
\newcommand{\f}{\cost} %objective function

\newenvironment{blocki}[1] % itemize block
{
 \begin{block}{#1}\begin{itemize}
}
{
\end{itemize}\end{block}
}

\title[AutoML: Hyperparameter Optimization]{AutoML: Hyperparameter Optimization}
%\subtitle{Overview for this Week} %To be defined in source!
%TODO: change authors!
\author[Marius Lindauer]{\underline{Bernd Bischl} \and Frank Hutter \and Lars Kotthoff\newline \and Marius Lindauer \and Joaquin Vanschoren}
\institute{}
\date{}

\subtitle{Example and Practical Hints}



\begin{document}

\maketitle


%----------------------------------------------------------------------
%----------------------------------------------------------------------

\begin{frame}{Expert Knowledge}

\begin{itemize}
	\item Knowledge about hyperparameters can help to guide the optimization
	\item For example, some hyperparameters might not be sampled uniformly
\end{itemize}

    \vspace{0.5cm}
Example: regularization hyperparameter ($C$ or \emph{cost}) of SVM: $[0.001, 1000.0]$

\begin{itemize}
	\item The distance between $999.9$ and $1000.0$ should not be the same as between $0.1$ and $0.2$.
  \item We might want to sample here from from a log-scale, e.g., $[10^{\conf_l}, 10^{\conf_u}]$ with $\conf_l = -3$ and $\conf_u = 3$.
\end{itemize}

\begin{figure}[htb]
\centering
  \begin{tikzpicture}[auto]%[scale=1.5]
    \draw [->](-0.3,0)-- (12.3,0) coordinate;
    \draw [->](-0.3,-2)-- (12.3,-2) coordinate;
    \foreach \x/\xtext/\xxtext in {-3/-3/0.001, -2/-2/ , -1/-1/, 0/0/, 1/1/, 2/2/100, 3/3/1000} {
      \draw [very thick] ({\x*2+6},-2pt) -- ++(0, 4pt) node[xshift = -6pt, yshift=-3pt,anchor=south west,baseline]{\strut$\xtext$};
      \draw [very thick] ({10^(\x)*0.012},-2cm+2pt) -- ++(0,-4pt) node[anchor=north]{$\xxtext$};
      \draw [->] ({\x*2+6},-2pt) .. controls ({\x*2+6},-0.5) and ({10^(\x)*0.012},-1.5) .. ({10^(\x)*0.012},-2cm+2pt);
    }
    \node[] at (-0.7,-0.1) (t1) {$\conf$};
    \node[] at (-0.7,-1.9) (t2) {$10^{\conf}$};
  \end{tikzpicture}
\end{figure}


\end{frame}
\begin{frame}{Tuning Example}
Tuning $(\text{cost},\gamma) \in [10^{-3},10^{3}]^2$ for the \emph{SVM} with \emph{random search}, \emph{grid search} and \emph{CMAES}\footnote{A popular EA that samples offspring from a multi-variate normal distribution} using a 5-fold CV on the \texttt{spam} data set for AUC:
\begin{columns}
\begin{column}{0.45\textwidth}
  \vspace{1em}
  % \resizebox{\linewidth}{!}{
  %   \begin{tabular}{l|l|l|l}
  %   Parameter&Type & Min & Max \\
  %   \hline
  %   \texttt{cost}  & double & $10^{-3}$ & $10^{3}$ \\
  %   \texttt{gamma} & double & $10^{-3}$ & $10^{3}$ \\
  %   \end{tabular}
  % }

  We notice here:

  \begin{itemize}
    \item \emph{Grid search} has many evaluations with bad performance ($\gamma>1$).
    \item \emph{Random search} can lead to underexplored areas even in promising regions.
    \item \emph{CMAES} only explores a small region.
  \end{itemize}
\end{column}%
\begin{column}{0.5\textwidth}
  \vspace{-1em}
  \begin{figure}
  \includegraphics[width=0.9\textwidth]{images/benchmark_scatter.png}
  \end{figure}
\end{column}
\end{columns}
\end{frame}

\begin{frame}{Tuning Example (cont.)}
The \emph{optimization curve} shows the best found configuration until a given time point.
\begin{columns}
\begin{column}{0.4\textwidth}
  \footnotesize
  \only<1>{

    Note:

    \begin{itemize}
      \item For \emph{random search} and \emph{grid search} the chronological order on the x-axis is arbitrary.
      \item The curve shows the performance on the tuning validation (\emph{inner resampling}) on a single fold
    \end{itemize}
  }
  \only<2-3>{
    \begin{itemize}
      \item<2-> The outer 10-fold CV gives us 10 optimization curves.
      \item<3-> The median at each time point gives us an estimate of the average optimization progress.
    \end{itemize}
  }
  \only<4>{
    \begin{itemize}
      \item Remember: The final model will be trained on the \emph{outer training set} with the configuration $\finconf$ that lead to the best performance on the \emph{inner test set}.
      \item To compare the effectiveness of the tuning strategies we have to look at the performance that $\finconf$ gives us on the \emph{outer test set}.
    \end{itemize}
  }
\end{column}
\begin{column}{0.6\textwidth}
  \vspace{-1em}
  \begin{figure}
  \only<1>{\includegraphics[width=\textwidth]{images/benchmark_curve_iter_1.png}}
  \only<2>{\includegraphics[width=\textwidth]{images/benchmark_curve_iter_all.png}}
  \only<3>{\includegraphics[width=\textwidth]{images/benchmark_curve_median.png}}
  \only<4>{\includegraphics[width=\textwidth]{images/benchmark_curve_iter_all_median.png}}
  \end{figure}
\end{column}
\end{columns}


  %   \footnotesize
  %     \item Tuning does not necessarily improve the performance of the learner, because e.g.\ tuning is badly configured or default values are determined by \emph{clever} heuristics.
  %     \item Tuning error can be overly optimistic (see \emph{nested resampling}).
  %   \end{itemize}
  % }
  % \only<2>{
  %   \begin{itemize}
  %     \item Effect of the chosen resampling split (objective noise) can dominate tuning effect.
  %   \end{itemize}
  % }
\end{frame}

\begin{frame}{Tuning Example: Validation}

\begin{columns}
\begin{column}{0.4\textwidth}
  \footnotesize

  The box plots show the distribution of the AUC-values that were measured on the \emph{outer test set} with a 10-fold CV.

  Note:

  \begin{itemize}
    \item The box plots do not indicate significant differences.
    \item The performance differs from the results obtained on the inner resampling.
  \end{itemize}

\end{column}
\begin{column}{0.6\textwidth}
  \vspace{-1em}
  \begin{figure}
  \includegraphics[width=\textwidth]{images/benchmark_boxplot_tuners.png}
  \end{figure}
\end{column}
\end{columns}
\end{frame}

\begin{frame}{Tuning Example: Validation}

\begin{columns}
\begin{column}{0.4\textwidth}
  \footnotesize

  \begin{itemize}
    \item<1-> Comparison of \textit{self-tuning} \emph{SVMs} with an SVM that was configured with $\conf = (\text{cost},\gamma) = (1,1)$ shows that tuning is necessary.
    \item<2-> However, some \emph{SVM} implementations have \emph{clever} heuristics:
        \begin{itemize}
            \item $\gamma = \frac{1}{p*var(X)}$
        \end{itemize}
  \end{itemize}
\end{column}
\begin{column}{0.6\textwidth}
  \begin{figure}
  \only<1>{\includegraphics[width=\textwidth]{images/benchmark_boxplot_default.png}}
  \only<2>{\includegraphics[width=\textwidth]{images/benchmark_boxplot_all.png}}
  \end{figure}
\end{column}
\end{columns}
\end{frame}

%\begin{frame}{Practical Hints: Nested Resampling}
%\begin{itemize}
%   \item For small data sets the relative size of the \emph{inner training set} should not differ much from \emph{outer training set}. Example: \\$n = 200$,
%   \begin{itemize}
%     \item outer resampling: 5-fold CV, inner resampling: 3-fold CV: $n_{\text{inner train}} = \frac{4}{5} \cdot {2}{3} \cdot n = 0.53 * n = 107$
%     \item inner and outer resampling: 10-fold CV: $\frac{9}{10} \cdot \frac{9}{10} \cdot n = 0.81 * n = 162$
%   \end{itemize}
%   \item Resampling strategies depend on dataset sizes:
%  \begin{itemize}
%    \item Small datasets: More resampling iterations necessary to obtain reliable performance estimate (e.g.\ repeated CV).
%    \item Large datasets: Less resampling iterations possible due to runtime, but holdout can be sufficient to estimate performance.
%    \item For unbalanced and multi-class datasets $n$ has to be higher to obtain reliable performance estimates, i.e.\ they should be treated like small datasets if not sufficiently big.
%  \end{itemize}
%\end{itemize}
%\end{frame}

\begin{frame}[allowframebreaks]{More Practical Hints}
\begin{itemize}
  % \item Tuning can lead to overfitting.
  \item The optimal hyperparameter setting can depent on the size of the training data. This can be a reason why the optimal configuration determined on the \emph{inner resampling} data does not have to be optimal for the \emph{outer resampling}.

  \item If $\finconf$ is close to the border of $\pcs$, consider log-scales or widening the search space.

  \item \begin{minipage}[t]{.55\linewidth}\raggedright
          \emph{Grid search} worked well on our example because $\pcs \in \mathbb{R}^2$ and the search space was defined with expert knowledge. In general grid search is ineffective as unimportant hyperparameters are explored while other parameters are kept constant.
        \end{minipage}%
        \begin{minipage}[t]{.4\linewidth}
        \raisebox{-\height}{\includegraphics[width = 0.8\linewidth]{images/grid_search_im.png}}
        \end{minipage}

\frambreak

  \item Tuning algorithm and implementation differ in the type of hyperparameters they support, e.g., real-valued, integer, categorical, mixed, hierarchical.

  \item The selection of $\finconf$ should consider the stochastic characteristic of the objective.

  \item Use parallelization right:
  \begin{itemize}
    \item An embarrassingly parallel tuner (e.g.\ random search) can be more efficient then an \emph{smart} optimizer that is purely sequential if many cores are available.
    \item Make use of the internal parallelization of the learner before using parallel tuning (unless it is embarrassingly parallel).
  \end{itemize}

\end{itemize}
\end{frame}

\end{document}
