\pdfminorversion=4 % for acroread
\documentclass[aspectratio=169,t,xcolor={usenames,dvipsnames}]{beamer}
%\documentclass[t,handout,xcolor={usenames,dvipsnames}]{beamer}
\usepackage{../beamerstyle}
\usepackage{dsfont}
\usepackage{bm}
\usepackage[english]{babel}
\usepackage[utf8]{inputenc}
\usepackage{graphicx}
\usepackage{algorithm}
\usepackage[ruled,vlined,algo2e,linesnumbered]{algorithm2e}
%\usepackage[boxed,vlined]{algorithm2e}
\usepackage{hyperref}
\usepackage{booktabs}
\usepackage{mathtools}

\usepackage{amsmath,amssymb}
\usepackage{listings}
\lstset{frame=lines,framesep=3pt,numbers=left,numberblanklines=false,basicstyle=\ttfamily\small}

\usepackage{subfig}
\usepackage{multicol}
%\usepackage{appendixnumberbeamer}
%
\usepackage{tcolorbox}

\usepackage{pgfplots}
\usepackage{tikz}
\usetikzlibrary{trees} 
\usetikzlibrary{shapes.geometric}
\usetikzlibrary{positioning,shapes,shadows,arrows,calc,mindmap}
\usetikzlibrary{positioning,fadings,through}
\usetikzlibrary{decorations.pathreplacing}
\usetikzlibrary{intersections}
\usetikzlibrary{positioning,fit,calc,shadows,backgrounds}
\pgfdeclarelayer{background}
\pgfdeclarelayer{foreground}
\pgfsetlayers{background,main,foreground}
\tikzstyle{activity}=[rectangle, draw=black, rounded corners, text centered, text width=8em]
\tikzstyle{data}=[rectangle, draw=black, text centered, text width=8em]
\tikzstyle{myarrow}=[->, thick, draw=black]

% Define the layers to draw the diagram
\pgfdeclarelayer{background}
\pgfdeclarelayer{foreground}
\pgfsetlayers{background,main,foreground}

%\usepackage{listings}
%\lstset{numbers=left,
%  showstringspaces=false,
%  frame={tb},
%  captionpos=b,
%  lineskip=0pt,
%  basicstyle=\ttfamily,
%%  extendedchars=true,
%  stepnumber=1,
%  numberstyle=\small,
%  xleftmargin=1em,
%  breaklines
%}

 
\definecolor{blue}{RGB}{0, 74, 153}

\usetheme{Boadilla}
%\useinnertheme{rectangles}
\usecolortheme{whale}
\setbeamercolor{alerted text}{fg=blue}
\useoutertheme{infolines}
\setbeamertemplate{navigation symbols}{\vspace{-5pt}} % to lower the logo
\setbeamercolor{date in head/foot}{bg=blue} % blue
\setbeamercolor{date in head/foot}{fg=white}
\setbeamercolor{author in head/foot}{bg=blue} %blue
\setbeamercolor{title in head/foot}{bg=blue} % blue
\setbeamercolor{title}{fg=white, bg=blue}
\setbeamercolor{block title}{fg=white,bg=blue}
\setbeamercolor{block body}{bg=blue!10}
\setbeamercolor{frametitle}{fg=white, bg=blue}
\setbeamercovered{invisible}

\makeatletter
\setbeamertemplate{footline}
{
  \leavevmode%
  \hbox{%
  \begin{beamercolorbox}[wd=.333333\paperwidth,ht=2.25ex,dp=1ex,center]{author in head/foot}%
    \usebeamerfont{author in head/foot}\insertshortauthor
  \end{beamercolorbox}%
  \begin{beamercolorbox}[wd=.333333\paperwidth,ht=2.25ex,dp=1ex,center]{title in head/foot}%
    \usebeamerfont{title in head/foot}\insertshorttitle
  \end{beamercolorbox}%
  \begin{beamercolorbox}[wd=.333333\paperwidth,ht=2.25ex,dp=1ex,right]{date in head/foot}%
    \usebeamerfont{date in head/foot}Week \@week, Topic \@topicnumber, Slide \insertframenumber{}\hspace*{2em}
%    \insertframenumber\hspace*{2ex} 
  \end{beamercolorbox}}%
  \vskip0pt%
}

\newcommand{\@week}{0}
\newcommand{\@topicnumber}{0}
\newcommand{\week}[1]{\renewcommand{\@week}{#1}}
\newcommand{\topicnumber}[1]{\renewcommand{\@topicnumber}{#1}}

\makeatother

%\pgfdeclareimage[height=1.2cm]{automl}{images/logos/automl.png}
%\pgfdeclareimage[height=1.2cm]{freiburg}{images/logos/freiburg}

%\logo{\pgfuseimage{freiburg}}

\input{../latex_main/macros}




\newcommand{\inducer}{\mathcal{I}}
\newcommand{\R}{\mathds{R}}

%The following might look confusing but allows us to switch the notation of the optimization problem independently from the notation of the hyper parameter optimization
\newcommand{\xx}{\conf} %x of the optimizer
\newcommand{\xxi}[1][i]{\conf_{#1}} %i-th component of xx (not confuse with i-th individual)
\newcommand{\XX}{\pcs} %search space / domain of f
\newcommand{\f}{\cost} %objective function

\newenvironment{blocki}[1] % itemize block
{
 \begin{block}{#1}\begin{itemize}
}
{
\end{itemize}\end{block}
}

\title[AutoML: Hyperparameter Optimization]{AutoML: Hyperparameter Optimization}
%\subtitle{Overview for this Week} %To be defined in source!
%TODO: change authors!
\author[Jakob Richter]{Bernd Bischl \and Frank Hutter \and Lars Kotthoff\newline \and Marius Lindauer}
\institute{}
\date{}

\subtitle{Grid and Random Search}
\week{4}
\topicnumber{2}


\begin{document}

\maketitle


%----------------------------------------------------------------------
%----------------------------------------------------------------------


\begin{frame}[containsverbatim,allowframebreaks]{Grid search}

\begin{columns}
\begin{column}{0.49\textwidth}
\begin{itemize}
\item Simple technique which is still quite popular, tries all
HP combinations on a multi-dimensional discretized grid
\item For each hyperparameter a finite set of candidates is predefined
\item Then, we simply search all possible combinations in arbitrary order
\end{itemize}
\end{column}
\begin{column}{0.49\textwidth}
\vspace*{-0.8cm}
\begin{center}
\begin{figure}
\includegraphics[width=0.9\textwidth]{images/grid.png}
\caption*{Grid search over 10x10 points}
\end{figure}
\end{center}
\end{column}
\end{columns}

\framebreak

\begin{blocki}{Advantages}
\item Very easy to implement
\item All parameter types possible
\item Parallelizing computation is trivial
\end{blocki}

\begin{blocki}{Disadvantages}
\item Scales badly: Combinatorial explosion
\item Inefficient: Searches large irrelevant areas
\item Low resolution in each dimension
\item Arbitrary: Which values / discretization?
\end{blocki}
\end{frame}


\begin{frame}[containsverbatim,allowframebreaks]{Random search}



\begin{columns}
\begin{column}{0.49\textwidth}
\begin{itemize}
\item Small variation of grid search
\item Uniformly sample from the region-of-interest
\end{itemize}
\end{column}
\begin{column}{0.49\textwidth}
\vspace*{-0.8cm}
\begin{center}
\begin{figure}
\includegraphics[width=0.9\textwidth]{images/random.png}
\caption*{Random search over 100 points}
\end{figure}
\end{center}
\end{column}
\end{columns}

\framebreak

\begin{blocki}{Advantages}
\item Like grid search: Very easy to implement, all parameter types possible, trivial parallelization
\item Anytime algorithm: Can stop the search whenever our budget for computation is exhausted, or continue until we reach our performance goal.
\item No discretization: each individual parameter is tried with a different value every time
\end{blocki}

\begin{blocki}{Disadvantages}
\item Inefficient: many evaluations in areas with low likelihood for improvement
\item Scales badly: high dimensional hyperparameter spaces need \emph{lots} of samples to cover.
\end{blocki}
\end{frame}

\begin{frame}{Grid search vs. Random search}

\begin{columns}
\begin{column}{0.49\textwidth}
\begin{itemize}
    \item With a tuning budget of $T$ only $T^{\frac{1}{d}}$ unique hyperparmeter values are explored for each $\conf_1, \dots, \conf_d$ in a grid search.
    \item Random search will (most likely) see $T$ different values for each hyperparameter.
    \item Grid search can be disadvantageous if some hyperparameters have little or no influence on the model.
\end{itemize}
\end{column}
\begin{column}{0.49\textwidth}
\vspace*{-0.8cm}
\begin{center}
\begin{figure}
\includegraphics[width=0.9\textwidth]{images/rsvsgs.pdf}
    \caption*{Comparison of grid search and random search. \lit{\href{https://www.automl.org/wp-content/uploads/2019/05/AutoML\_Book.pdf}{Hutter et al. 2019}}}
\end{figure}
\end{center}
\end{column}
\end{columns}

\end{frame}

\end{document}
