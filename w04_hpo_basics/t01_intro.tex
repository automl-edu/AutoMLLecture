\pdfminorversion=4 % for acroread
\documentclass[aspectratio=169,t,xcolor={usenames,dvipsnames}]{beamer}
%\documentclass[t,handout,xcolor={usenames,dvipsnames}]{beamer}
\usepackage{../beamerstyle}
\usepackage{dsfont}
\usepackage{bm}
\usepackage[english]{babel}
\usepackage[utf8]{inputenc}
\usepackage{graphicx}
\usepackage{algorithm}
\usepackage[ruled,vlined,algo2e,linesnumbered]{algorithm2e}
%\usepackage[boxed,vlined]{algorithm2e}
\usepackage{hyperref}
\usepackage{booktabs}
\usepackage{mathtools}

\usepackage{amsmath,amssymb}
\usepackage{listings}
\lstset{frame=lines,framesep=3pt,numbers=left,numberblanklines=false,basicstyle=\ttfamily\small}

\usepackage{subfig}
\usepackage{multicol}
%\usepackage{appendixnumberbeamer}
%
\usepackage{tcolorbox}

\usepackage{pgfplots}
\usepackage{tikz}
\usetikzlibrary{trees} 
\usetikzlibrary{shapes.geometric}
\usetikzlibrary{positioning,shapes,shadows,arrows,calc,mindmap}
\usetikzlibrary{positioning,fadings,through}
\usetikzlibrary{decorations.pathreplacing}
\usetikzlibrary{intersections}
\usetikzlibrary{positioning,fit,calc,shadows,backgrounds}
\pgfdeclarelayer{background}
\pgfdeclarelayer{foreground}
\pgfsetlayers{background,main,foreground}
\tikzstyle{activity}=[rectangle, draw=black, rounded corners, text centered, text width=8em]
\tikzstyle{data}=[rectangle, draw=black, text centered, text width=8em]
\tikzstyle{myarrow}=[->, thick, draw=black]

% Define the layers to draw the diagram
\pgfdeclarelayer{background}
\pgfdeclarelayer{foreground}
\pgfsetlayers{background,main,foreground}

%\usepackage{listings}
%\lstset{numbers=left,
%  showstringspaces=false,
%  frame={tb},
%  captionpos=b,
%  lineskip=0pt,
%  basicstyle=\ttfamily,
%%  extendedchars=true,
%  stepnumber=1,
%  numberstyle=\small,
%  xleftmargin=1em,
%  breaklines
%}

 
\definecolor{blue}{RGB}{0, 74, 153}

\usetheme{Boadilla}
%\useinnertheme{rectangles}
\usecolortheme{whale}
\setbeamercolor{alerted text}{fg=blue}
\useoutertheme{infolines}
\setbeamertemplate{navigation symbols}{\vspace{-5pt}} % to lower the logo
\setbeamercolor{date in head/foot}{bg=blue} % blue
\setbeamercolor{date in head/foot}{fg=white}
\setbeamercolor{author in head/foot}{bg=blue} %blue
\setbeamercolor{title in head/foot}{bg=blue} % blue
\setbeamercolor{title}{fg=white, bg=blue}
\setbeamercolor{block title}{fg=white,bg=blue}
\setbeamercolor{block body}{bg=blue!10}
\setbeamercolor{frametitle}{fg=white, bg=blue}
\setbeamercovered{invisible}

\makeatletter
\setbeamertemplate{footline}
{
  \leavevmode%
  \hbox{%
  \begin{beamercolorbox}[wd=.333333\paperwidth,ht=2.25ex,dp=1ex,center]{author in head/foot}%
    \usebeamerfont{author in head/foot}\insertshortauthor
  \end{beamercolorbox}%
  \begin{beamercolorbox}[wd=.333333\paperwidth,ht=2.25ex,dp=1ex,center]{title in head/foot}%
    \usebeamerfont{title in head/foot}\insertshorttitle
  \end{beamercolorbox}%
  \begin{beamercolorbox}[wd=.333333\paperwidth,ht=2.25ex,dp=1ex,right]{date in head/foot}%
    \usebeamerfont{date in head/foot}Week \@week, Topic \@topicnumber, Slide \insertframenumber{}\hspace*{2em}
%    \insertframenumber\hspace*{2ex} 
  \end{beamercolorbox}}%
  \vskip0pt%
}

\newcommand{\@week}{0}
\newcommand{\@topicnumber}{0}
\newcommand{\week}[1]{\renewcommand{\@week}{#1}}
\newcommand{\topicnumber}[1]{\renewcommand{\@topicnumber}{#1}}

\makeatother

%\pgfdeclareimage[height=1.2cm]{automl}{images/logos/automl.png}
%\pgfdeclareimage[height=1.2cm]{freiburg}{images/logos/freiburg}

%\logo{\pgfuseimage{freiburg}}

\input{../latex_main/macros}




\newcommand{\inducer}{\mathcal{I}}
\newcommand{\R}{\mathds{R}}

%The following might look confusing but allows us to switch the notation of the optimization problem independently from the notation of the hyper parameter optimization
\newcommand{\xx}{\conf} %x of the optimizer
\newcommand{\xxi}[1][i]{\conf_{#1}} %i-th component of xx (not confuse with i-th individual)
\newcommand{\XX}{\pcs} %search space / domain of f
\newcommand{\f}{\cost} %objective function

\newenvironment{blocki}[1] % itemize block
{
 \begin{block}{#1}\begin{itemize}
}
{
\end{itemize}\end{block}
}

\title[AutoML: Hyperparameter Optimization]{AutoML: Hyperparameter Optimization}
%\subtitle{Overview for this Week} %To be defined in source!
%TODO: change authors!
\author[Jakob Richter]{Bernd Bischl \and Frank Hutter \and Lars Kotthoff\newline \and Marius Lindauer}
\institute{}
\date{}

\subtitle{Overview and Introduction}
\week{4}
\topicnumber{1}

\begin{document}

\maketitle


%----------------------------------------------------------------------
%----------------------------------------------------------------------

\begin{frame}[containsverbatim,allowframebreaks]{Motivating Example}

\begin{itemize}
\item Given a dataset, we want to train a classification tree.
\item We feel that a maximum tree depth of $4$ has worked out well for us previously, so we decide to set this hyperparameter to $4$.
\item The learner ("inducer") $\inducer$ takes the input data, internally performs \textbf{empirical risk minimization}, and returns a fitted tree model $\fh(\x) = f(\x, \hat{\theta})$ of at most depth $\conf = 4$ that minimizes the empirical risk.
\end{itemize}

% FIGURE SOURCE: https://docs.google.com/presentation/d/14xwcs5zncTjFL4hIHAprjZMmyGIPqk5vs8DS32vEAvQ/edit?usp=sharing
\begin{center}
\begin{figure}
\includegraphics[width=0.4\textwidth]{images/riskmin_bilevel1.png}
\end{figure}
\end{center}

\framebreak

\begin{itemize}
\item We are \textbf{actually} interested in the \textbf{generalization performance} $GE\left(\fh\right)$ of the estimated model on new, previously unseen data.
\item We estimate the generalization performance by evaluating the model $\fh$ on a test set $\datasettest$: $$
\widehat{GE}_{\datasettest}\left(\fh\right) = \frac{1}{|\datasettest|} \sum\limits_{(\x, y) \in \datasettest} L\left(y, \fh(\x) \right)
$$
\end{itemize}
\vspace*{-0.6cm}
% FIGURE SOURCE: https://docs.google.com/presentation/d/14xwcs5zncTjFL4hIHAprjZMmyGIPqk5vs8DS32vEAvQ/edit?usp=sharing
\begin{center}
\begin{figure}
\includegraphics[width=0.5\textwidth]{images/riskmin_bilevel2.png}
\end{figure}
\end{center}

\framebreak

\begin{itemize}
\vfill
\item But many ML algorithms are sensitive w.r.t. a good setting of their hyperparameters,
and generalization performance might be bad, if we have chosen a suboptimal configuration:
\begin{itemize}
\vfill
\item The data may be too complex to be modeled by a tree of depth $4$
\vfill
\item The data may be much simpler than we thought, and a tree of depth $4$ overfits
\end{itemize}
\vfill
\item[$\implies$] algorithmically try out different values for the tree depth. For each maximal depth $\conf$, we have to train the model \textbf{to completion} and evaluate its performance on the test set.
\vfill
\item We choose the tree depth $\conf$ that is \textbf{optimal} w.r.t. the generalization error of the model.
\end{itemize}

% $\to$ Finding the hyperparameter $\lambda$ that is optimal w.r.t. the generalization performance constitutes an optimization problem.

\end{frame}

% \begin{frame}[containsverbatim,allowframebreaks]{The role of hyperparameters}

% \begin{itemize}
% \item Hyperparameters often control the complexity of a model, i.e., how flexible the model is.
% \item But they can in principle influence any structural property of a model or computational part of the training process.
% \item If a model is not flexible enough, it cannot approximate the relationship between the features and the output well and will underfit.
% \item If a model is too flexible so that it simply \textit{memorizes} the training data, we will face the dreaded problem of overfitting.
% \item[$\implies$] Hence, controlling the model's capacity, i.e., finding suitable hyperparameters,
% can prevent overfitting the model on the training set.
% \end{itemize}

% \end{frame}

\begin{frame}[containsverbatim,allowframebreaks]{Model Parameters vs. Hyperparameters}

It is critical to understand the difference between model parameters and hyperparameters.

\vspace{0.5cm}

\textbf{Model parameters} are optimized during training, typically via loss minimization. They are an \textbf{output} of the training. Examples:
\begin{itemize}
\item The splits and terminal node constants of a tree learner
\item Coefficients $\theta$ of a linear model $\fx = \theta^\top\x$
\end{itemize}

\framebreak

In contrast, \textbf{hyperparameters} (HPs) are not decided during training. They must be specified before the training, they are an \textbf{input} of the training.
Hyperparameters often control the complexity of a model, i.e., how flexible the model is.
But they can in principle influence any structural property of a model or computational part of the training process.

\vspace{0.5cm}

Examples:

\begin{itemize}
\item Tree: The maximum depth of a tree
\item $k$ Nearest Neighbours: Number of neighbours $k$ and distance measure
\item Linear regression: Number and maximal order of interactions
\end{itemize}

\end{frame}


\begin{frame}[containsverbatim,allowframebreaks]{Types of hyperparameters}

We summarize all hyperparameters we want to tune over in a vector $\conf \in \pcs$ of (possibly) mixed type. HPs can have different types:

\begin{itemize}
\item Real-valued parameters, e.g.:
\begin{itemize}
\item Minimal error improvement in a tree to accept a split
\item Bandwidths of the kernel density estimates for Naive Bayes
\end{itemize}
\item Integer parameters, e.g.:
\begin{itemize}
\item Neighbourhood size $k$ for $k$-NN
\item Minimum number of samples for a split in a random forest
\end{itemize}
\item Categorical parameters, e.g.:
\begin{itemize}
\item Which split criterion for classification trees?
\item Which distance measure for $k$-NN?
\end{itemize}
\end{itemize}

Hyperparameters are often \textbf{hierarchically dependent} on each other, e.g., \emph{if} we use
a kernel-density estimate for Naive Bayes, what is its width?
% with polynomials of the features up to a certain maximal degree $d$, then
% we must specify whether to also include polynomial interaction terms like e.g. $x_j^{d-d'}x_m^{d'}$ or not
% and up to which degree $d' \leq d$.
\end{frame}

\begin{frame}{Tuning}

\vskip 3em
Recall: \textbf{Hyperparameters} $\conf$ are parameters that are \emph{inputs} to the
training problem, in which a learner $\inducer$ minimizes the empirical risk on a training data set in order
to find optimal \textbf{model parameters} $\theta$ which define the fitted model $\fh$.
\vskip 2em

\textbf{(Hyperparameter) Tuning} is the process of finding good model hyperparameters $\conf$.

% \begin{frame}[containsverbatim,allowframebreaks]{{Hyperparameter Tuning}
% \begin{itemize}
% \item Optimize hyperparameters for learner w.r.t. prediction error
% Tuner proposes configuration, eval by resampling, tuner receives performance, iterate
% \end{itemize}
% \begin{columns}[c, onlytextwidth]
% \column{0.45\textwidth}
% FIGURE SOURCE: No source
% \includegraphics[trim={0cm 0cm 0cm 0cm}, clip, width=1.2\textwidth]{images/chain.jpg}
% \column{0.45\textwidth}
% FIGURE SOURCE: https://drive.google.com/open?id=1wY3aUZxIMZPje3vR0t2yWiDMx_osXRCi
% \includegraphics[trim={1cm 0cm 1cm 0cm}, clip, width=1.2\textwidth]{images/tuning_process.jpg}
% \end{columns}

% \end{frame}

\end{frame}


\begin{frame}[containsverbatim,allowframebreaks]{Tuning: A bi-level optimization problem}

\vspace{0.2cm}

We face a \textbf{bi-level} optimization problem: The well-known risk minimization problem to find $\hat f$ is \textbf{nested} within the outer hyperparameter optimization (also called second-level problem):

\begin{center}
\begin{figure}
% FIGURE SOURCE: https://docs.google.com/presentation/d/14xwcs5zncTjFL4hIHAprjZMmyGIPqk5vs8DS32vEAvQ/edit?usp=sharing
\includegraphics[width=0.6\textwidth]{images/riskmin_bilevel3.png}
\end{figure}
\end{center}

\framebreak

\begin{itemize}
\item For a learning algorithm $\inducer$ (also inducer) with $d$ hyperparameters, the hyperparameter \textbf{configuration space} is:
$$\pcs=\pcs_{1} \times \pcs_{2} \times \ldots \times \pcs_{d}$$
where $\pcs_{i}$ is the domain of the $i$-th hyperparameter.
\item The domains can be continuous, discrete or categorical.
\item For practical reasons, the domain of a continuous or integer-valued hyperparameter is typically bounded.
\item A vector in this configuration space is denoted as $\conf \in \pcs$.
\item A learning algorithm $\inducer$ takes a (training) dataset $\dataset$ and a hyperparameter configuration $\conf \in \pcs$ and returns a trained model (through risk minimization).

\vspace*{-0.2cm}
\begin{eqnarray*}
\inducer: \left(\mathcal{X} \times \mathcal{Y}\right)^n \times \pcs &\to& \mathcal{H} \\
(\dataset, \conf) &\mapsto& \inducer(\dataset, \conf) = \hat f_{\dataset, \conf}
\end{eqnarray*}
% \item Additionally, some hyperparameters may only need to be specified if another hyperparameter (or combination of hyperparameters) takes on a certain value.
\end{itemize}

% \vspace{0.5cm}

% Note that

% In contrast to the first-level (empirical) risk minimization problem, hyperparameter optimization is also referred to as \textbf{second-level} optimization. The first-level problem can be seen as a subroutine called by the second-level problem: Each evaluation of $\pcs$ requires to solve the first-level optimization problem.


% \framebreak

% \begin{itemize}
% \item search for the \textbf{inducer} hyperparameter $\pcs$
% \item that minimizes the \textbf{generalization error}
% $$
% \min_{\pcs} \E_{\D_n \sim \Pxy, (\xv, y) \sim \Pxy} \left(V\left(y, \hat f_{\D, \pcs}(\xv)\right)\right).
% $$
% \end{itemize}

% We compare: In empirical risk minimization, we

% \begin{itemize}
% \item search for the \textbf{model} parameter $\thetab$
% \item that minimizes the \textbf{empirical risk}
% $$
% \min_{\thetab} \sum_{(\xi, \yi) \in \datasettrain} L\left(\yi, \fxi\right).
% $$
% \end{itemize}

% In hyperparameter optimization, we

% \begin{itemize}
% \item search for the \textbf{inducer} hyperparameter $\pcs$
% \item that minimizes the \textbf{test error}
% $$
% \min_{\pcs \in \pcs} \sum_{(\xi, \yi) \in \datasettest} V\left(\inducer(\datasettrain, \pcs)(\xi), \yi\right).
% $$
% \end{itemize}

% \framebreak

% \framebreak

% The hyperparameter optimization problem is difficult in many ways:

\framebreak

We formally state the nested hyperparameter tuning problem as:

$$
    \min_{\conf \in \pcs} \cost(\conf) = \widehat{GE}_{\datasettest}\left(\inducer(\datasettrain, \conf)\right)
$$

\begin{itemize}
\item The learner $\inducer(\datasettrain, \conf)$ takes a training dataset as well as hyperparameter settings $\pcs$ (e.g.\ the maximal depth of a classification tree) as an input.
\item $\inducer(\datasettrain, \conf)$ performs empirical risk minimization on the training data and returns the optimal model $\hat f$ for the given hyperparameters.
\item Note that for the estimation of the generalization error, more sophisticated resampling strategies like cross-validation can be used.
\end{itemize}

\framebreak

The components of a tuning problem are:

\begin{itemize}
\item The dataset
\item The learner (possibly: several competing learners?) that is tuned %(e.g. a decision tree classifier)
\item The learner's hyperparameters and their respective regions-of-interest over which we optimize % (e.g. $\texttt{tree depth} \in \{1, 2, ..., 20\}$)
\item The performance measure, as determined by the application.\\ Not necessarily identical to the loss function that defines the risk minimization problem for the learner!\\
% We could even be interested in multiple measures simultaneously, e.g., accuracy and computation time of our model, TPR and PPV, etc.
\item A (resampling) procedure for estimating the predictive performance according to the performance measure.
 % The expected performance on unseen data can be estimated by holdout (i.e., a single train-test-split) or more advanced techniques like cross-validation.
% More on this later.
\end{itemize}

% \framebreak

% \begin{center}
% \begin{figure}
% FIGURE SOURCE: https://docs.google.com/presentation/d/1JUtguuVBgidcqD0IdFFIiKH9zqYzM6YRjCqC53V90dA/edit?usp=sharing
% \includegraphics[width=1.2\textwidth]{images/autotune_in_model_fit.pdf}
% \end{figure}
% \end{center}

\end{frame}

\begin{frame}{Why is tuning so hard?}
\begin{itemize}
\item Tuning is derivative-free (black box problem): It is usually impossible to compute derivatives of the objective (i.e., the resampled performance measure) that we optimize with regard to the HPs. All we can do is evaluate the performance for a given hyperparameter configuration.
\item Every evaluation requires one or multiple train and predict steps of the learner. I.e., every evaluation is very \textbf{expensive}.
\item Even worse: the answer we get from that evaluation is \textbf{not exact, but stochastic} in most settings, as we use resampling (and often stochastic learners).
\item Categorical and dependent hyperparameters aggravate our difficulties: the space of hyperparameters we optimize over has a non-metric, complicated structure.
\end{itemize}

\end{frame}

% \framebreak

% Possible scenarios for finding default hyperparameters:

% \begin{itemize}
% \item If the learner's performance is fairly insensitive to changes of a hyperparameter, we don't really have to worry as long as we remain within the range of reasonable values.
% \item Constant default: we can benchmark the learner across a broad range of data sets and scenarios and try to find hyperparameter values that work well in many different situations. Quite optimistic?
% \item Dynamic (heuristic) default: We can benchmark the learner across a broad range of data sets and scenarios and try to find an easily computable function that sets the hyperparameter in a data dependent way,
% e.g. using \texttt{mtry}$ = p/3$ for RF.\\
% How to construct or learn that heuristic function, though...?
% \item In some cases, can try to set hyperparameters optimally by extracting more info from the fitted model. E.g. \texttt{ntrees} for a random forest (does OOB error increase or decrease if you remove trees from the ensemble?).
% \end{itemize}
% \end{frame}


\end{document}
