% !TeX spellcheck = en_US

\pdfminorversion=4 % for acroread
\documentclass[aspectratio=169,t,xcolor={usenames,dvipsnames}]{beamer}
%\documentclass[t,handout,xcolor={usenames,dvipsnames}]{beamer}
\usepackage{../beamerstyle}
\usepackage{dsfont}
\usepackage{bm}
\usepackage[english]{babel}
\usepackage[utf8]{inputenc}
\usepackage{graphicx}
\usepackage{algorithm}
\usepackage[ruled,vlined,algo2e,linesnumbered]{algorithm2e}
%\usepackage[boxed,vlined]{algorithm2e}
\usepackage{hyperref}
\usepackage{booktabs}
\usepackage{mathtools}

\usepackage{amsmath,amssymb}
\usepackage{listings}
\lstset{frame=lines,framesep=3pt,numbers=left,numberblanklines=false,basicstyle=\ttfamily\small}

\usepackage{subfig}
\usepackage{multicol}
%\usepackage{appendixnumberbeamer}
%
\usepackage{tcolorbox}

\usepackage{pgfplots}
\usepackage{tikz}
\usetikzlibrary{trees} 
\usetikzlibrary{shapes.geometric}
\usetikzlibrary{positioning,shapes,shadows,arrows,calc,mindmap}
\usetikzlibrary{positioning,fadings,through}
\usetikzlibrary{decorations.pathreplacing}
\usetikzlibrary{intersections}
\usetikzlibrary{positioning,fit,calc,shadows,backgrounds}
\pgfdeclarelayer{background}
\pgfdeclarelayer{foreground}
\pgfsetlayers{background,main,foreground}
\tikzstyle{activity}=[rectangle, draw=black, rounded corners, text centered, text width=8em]
\tikzstyle{data}=[rectangle, draw=black, text centered, text width=8em]
\tikzstyle{myarrow}=[->, thick, draw=black]

% Define the layers to draw the diagram
\pgfdeclarelayer{background}
\pgfdeclarelayer{foreground}
\pgfsetlayers{background,main,foreground}

%\usepackage{listings}
%\lstset{numbers=left,
%  showstringspaces=false,
%  frame={tb},
%  captionpos=b,
%  lineskip=0pt,
%  basicstyle=\ttfamily,
%%  extendedchars=true,
%  stepnumber=1,
%  numberstyle=\small,
%  xleftmargin=1em,
%  breaklines
%}

 
\definecolor{blue}{RGB}{0, 74, 153}

\usetheme{Boadilla}
%\useinnertheme{rectangles}
\usecolortheme{whale}
\setbeamercolor{alerted text}{fg=blue}
\useoutertheme{infolines}
\setbeamertemplate{navigation symbols}{\vspace{-5pt}} % to lower the logo
\setbeamercolor{date in head/foot}{bg=blue} % blue
\setbeamercolor{date in head/foot}{fg=white}
\setbeamercolor{author in head/foot}{bg=blue} %blue
\setbeamercolor{title in head/foot}{bg=blue} % blue
\setbeamercolor{title}{fg=white, bg=blue}
\setbeamercolor{block title}{fg=white,bg=blue}
\setbeamercolor{block body}{bg=blue!10}
\setbeamercolor{frametitle}{fg=white, bg=blue}
\setbeamercovered{invisible}

\makeatletter
\setbeamertemplate{footline}
{
  \leavevmode%
  \hbox{%
  \begin{beamercolorbox}[wd=.333333\paperwidth,ht=2.25ex,dp=1ex,center]{author in head/foot}%
    \usebeamerfont{author in head/foot}\insertshortauthor
  \end{beamercolorbox}%
  \begin{beamercolorbox}[wd=.333333\paperwidth,ht=2.25ex,dp=1ex,center]{title in head/foot}%
    \usebeamerfont{title in head/foot}\insertshorttitle
  \end{beamercolorbox}%
  \begin{beamercolorbox}[wd=.333333\paperwidth,ht=2.25ex,dp=1ex,right]{date in head/foot}%
    \usebeamerfont{date in head/foot}Week \@week, Topic \@topicnumber, Slide \insertframenumber{}\hspace*{2em}
%    \insertframenumber\hspace*{2ex} 
  \end{beamercolorbox}}%
  \vskip0pt%
}

\newcommand{\@week}{0}
\newcommand{\@topicnumber}{0}
\newcommand{\week}[1]{\renewcommand{\@week}{#1}}
\newcommand{\topicnumber}[1]{\renewcommand{\@topicnumber}{#1}}

\makeatother

%\pgfdeclareimage[height=1.2cm]{automl}{images/logos/automl.png}
%\pgfdeclareimage[height=1.2cm]{freiburg}{images/logos/freiburg}

%\logo{\pgfuseimage{freiburg}}

\input{../latex_main/macros}





\title[AutoML: Learned LRs]{AutoML: Dynamic Configuration \& Learning}
\subtitle{Learning to Adjust Learning Rates}
\author[Marius Lindauer]{Bernd Bischl \and Frank Hutter \and Lars Kotthoff\newline \and \underline{Marius Lindauer} \and Joaquin Vanschoren}
\institute{}
\date{}
\week{11}
\topicnumber{3}



% \AtBeginSection[] % Do nothing for \section*
% {
%   \begin{frame}{Outline}
%     \bigskip
%     \vfill
%     \tableofcontents[currentsection]
%   \end{frame}
% }

\begin{document}
	
	\maketitle
	
	
%----------------------------------------------------------------------
%----------------------------------------------------------------------
\begin{frame}[c]{Learning Problem \litw{\href{https://www.aaai.org/ocs/index.php/AAAI/AAAI16/paper/view/11763/11767}{Daniel et al. 2016}}}


\begin{itemize}
	\item Optimization of a function:
\end{itemize}
\begin{equation}
	\weights \in \argmin F(\mathbf{X}; \weights) \nonumber
\end{equation}

where $\mathbf{X}$ is an input matrix and f is parameterized by $\weights$.


\pause
\medskip

\begin{equation}
F(\mathbf{X}; \weights) = \frac{1}{N} \sum_{i=1}^N f(\xI{i}; \weights) \nonumber
\end{equation}




\end{frame}
%----------------------------------------------------------------------
%----------------------------------------------------------------------
\begin{frame}[c]{Learning Step Size Policies \litw{\href{https://www.aaai.org/ocs/index.php/AAAI/AAAI16/paper/view/11763/11767}{Daniel et al. 2016}}}

\begin{itemize}
\item \alert{Idea:} Learn the hyperparameters of the weight update (short notation)
\end{itemize} 

\begin{eqnarray}
\weights^{(t+1)} = \weights^{(t)} - \alpha^{(t)} \nabla F(\weights^{(t)}) \nonumber\\
\nabla F(\weights^{(t)}) = \frac{1}{N} \sum_{i=1}^N \nabla f_i(\weights^{(t)})\nonumber
\end{eqnarray}


\begin{itemize}
\pause
\item For SGD, this would be for example the learning rate $\alpha$
\pause
\item \alert{Note (i)}: $\alpha$ have to be adapted in the course of the training
\begin{itemize}
\item similar to learning rate schedules (e.g., cosine annealing)
\end{itemize}
\pause
\item \alert{Note(ii)}: later we denote the learnt hyperparameters as $\lambda$
\medskip
\pause
\item \alert{Idea:} Use reinforcement learning to learn a policy $\policy: \stateRL \mapsto \actionRL$ to control the learning rate (or other adaptive hyperparameters)
\end{itemize}



\end{frame}
%----------------------------------------------------------------------
%----------------------------------------------------------------------
\begin{frame}[c]{Recap: Reinforcement Learning for Dynamic Algorithm Configuration}

\begin{center}
\input{tikz/control.tex}
\end{center}

\bigskip
To apply that, we need to define:
\begin{enumerate}
	\item State description
	\item Action space
	\item Reward function
\end{enumerate}

\end{frame}

%----------------------------------------------------------------------
%----------------------------------------------------------------------
\begin{frame}[c]{RL for Step Size Policies: State \litw{\href{https://www.aaai.org/ocs/index.php/AAAI/AAAI16/paper/view/11763/11767}{Daniel et al. 2016}}}

\textbf{Predictive change in function value:}

$$s_1 = \log \left( \text{Var}(\Delta \tilde{f}_i ) \right)$$
$$\Delta \tilde{f}_i = \tilde{f}(\xI{i} ; \theta + \delta \theta) - f(\xI{i} ; \theta)$$

where $\tilde{f}(\xI{i} ; \theta + \delta \theta)$ is done by a first order Taylor expansion

\pause
\smallskip
\textbf{Disagreement of function values:}
$$ s_2 = \log \left(\text{Var}(f(\xI{i} ; \theta)) \right)$$

\pause

\textbf{Discounted Average} (smoothing noise from mini-batches):
$$\hat{s}_i \leftarrow \gamma \hat{s_i} + (1 - \gamma) s_i$$

\pause

\textbf{Uncertainty Estimate} (noise level):
$$s_{K+i} \leftarrow \gamma s_{K+i} + (1-\gamma) (s_i - \hat{s}_i)^2$$


\end{frame}
%----------------------------------------------------------------------
%----------------------------------------------------------------------
\begin{frame}[c]{RL for Step Size Policies: Learning \litw{\href{https://www.aaai.org/ocs/index.php/AAAI/AAAI16/paper/view/11763/11767}{Daniel et al. 2016}}}

Reward (average loss improvement over time):

$$\rewardRL = \frac{1}{T-1} \sum_{t=2}^T \left(\log(\loss^{(t-1)}) - \log(\loss^{(t)})\right)$$

\pause

Optimal Policy:

$$\policy^*(\lambda \mid \stateRL) \in \argmax_{\policy} \int \int p(\stateRL) \policy(\conf \mid \stateRL)r(\conf,\stateRL) \diff\stateRL \diff\conf $$

\pause


\begin{itemize}
\item can be learnt for example via Relative Entropy Policy Search (REPS) \lit{\href{https://www.aaai.org/ocs/index.php/AAAI/AAAI10/paper/viewFile/1851/2264}{Peter et al. 2010}}
\end{itemize}

\end{frame}
%----------------------------------------------------------------------
%----------------------------------------------------------------------
\begin{frame}[c]{RL for Step Size Policies: Training \litw{\href{https://www.aaai.org/ocs/index.php/AAAI/AAAI16/paper/view/11763/11767}{Daniel et al. 2016}}}

\begin{itemize}
\item Goal: obtain robust policies,\\ i.e., good performance for many different DNN architectures
\begin{itemize}
\item[$\leadsto$] Sample architectures e.g., with different numbers of filters and layers
\item[$\leadsto$] (Sub-)Sample dataset
\item[$\leadsto$] Sample number of optimization steps
\end{itemize}
\end{itemize}

\pause 
\medskip
\centering
\includegraphics[width=0.55\textwidth]{images/l2stepsizecontroler_mnist_training.png}

"Ours" refers to the approach by \lit{\href{https://www.aaai.org/ocs/index.php/AAAI/AAAI16/paper/view/11763/11767}{Daniel et al. 2016}} and $\eta$ is the learning rate

\end{frame}
%----------------------------------------------------------------------

\end{document}