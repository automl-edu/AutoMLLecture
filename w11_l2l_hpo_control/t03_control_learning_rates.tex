% !TeX spellcheck = en_US

\pdfminorversion=4 % for acroread
\documentclass[aspectratio=169,t,xcolor={usenames,dvipsnames}]{beamer}
%\documentclass[t,handout,xcolor={usenames,dvipsnames}]{beamer}
\usepackage{../beamerstyle}
\usepackage{dsfont}
\usepackage{bm}
\usepackage[english]{babel}
\usepackage[utf8]{inputenc}
\usepackage{graphicx}
\usepackage{algorithm}
\usepackage[ruled,vlined,algo2e,linesnumbered]{algorithm2e}
%\usepackage[boxed,vlined]{algorithm2e}
\usepackage{hyperref}
\usepackage{booktabs}
\usepackage{mathtools}

\usepackage{amsmath,amssymb}
\usepackage{listings}
\lstset{frame=lines,framesep=3pt,numbers=left,numberblanklines=false,basicstyle=\ttfamily\small}

\usepackage{subfig}
\usepackage{multicol}
%\usepackage{appendixnumberbeamer}
%
\usepackage{tcolorbox}

\usepackage{pgfplots}
\usepackage{tikz}
\usetikzlibrary{trees} 
\usetikzlibrary{shapes.geometric}
\usetikzlibrary{positioning,shapes,shadows,arrows,calc,mindmap}
\usetikzlibrary{positioning,fadings,through}
\usetikzlibrary{decorations.pathreplacing}
\usetikzlibrary{intersections}
\usetikzlibrary{positioning,fit,calc,shadows,backgrounds}
\pgfdeclarelayer{background}
\pgfdeclarelayer{foreground}
\pgfsetlayers{background,main,foreground}
\tikzstyle{activity}=[rectangle, draw=black, rounded corners, text centered, text width=8em]
\tikzstyle{data}=[rectangle, draw=black, text centered, text width=8em]
\tikzstyle{myarrow}=[->, thick, draw=black]

% Define the layers to draw the diagram
\pgfdeclarelayer{background}
\pgfdeclarelayer{foreground}
\pgfsetlayers{background,main,foreground}

%\usepackage{listings}
%\lstset{numbers=left,
%  showstringspaces=false,
%  frame={tb},
%  captionpos=b,
%  lineskip=0pt,
%  basicstyle=\ttfamily,
%%  extendedchars=true,
%  stepnumber=1,
%  numberstyle=\small,
%  xleftmargin=1em,
%  breaklines
%}

 
\definecolor{blue}{RGB}{0, 74, 153}

\usetheme{Boadilla}
%\useinnertheme{rectangles}
\usecolortheme{whale}
\setbeamercolor{alerted text}{fg=blue}
\useoutertheme{infolines}
\setbeamertemplate{navigation symbols}{\vspace{-5pt}} % to lower the logo
\setbeamercolor{date in head/foot}{bg=white} % blue
\setbeamercolor{date in head/foot}{fg=white}
\setbeamercolor{author  in head/foot}{bg=white} %blue
\setbeamercolor{title in head/foot}{bg=white} % blue
\setbeamercolor{title}{fg=white, bg=blue}
\setbeamercolor{block title}{fg=white,bg=blue}
\setbeamercolor{block body}{bg=blue!10}
\setbeamercolor{frametitle}{fg=white, bg=blue}
\setbeamercovered{invisible}

\makeatletter
\setbeamertemplate{footline}
{
  \leavevmode%
  \hbox{%
  \begin{beamercolorbox}[wd=.333333\paperwidth,ht=2.25ex,dp=1ex,center]{author in head/foot}%
%    \usebeamerfont{author in head/foot}\insertshortauthor
  \end{beamercolorbox}%
  \begin{beamercolorbox}[wd=.333333\paperwidth,ht=2.25ex,dp=1ex,center]{title in head/foot}%
    \usebeamerfont{title in head/foot}\insertshorttitle
  \end{beamercolorbox}%
  \begin{beamercolorbox}[wd=.333333\paperwidth,ht=2.25ex,dp=1ex,right]{date in head/foot}%
    \usebeamerfont{date in head/foot}\insertshortdate{}\hspace*{2em}
%    \insertframenumber\hspace*{2ex} 
  \end{beamercolorbox}}%
  \vskip0pt%
}
\makeatother

%\pgfdeclareimage[height=1.2cm]{automl}{images/logos/automl.png}
%\pgfdeclareimage[height=1.2cm]{freiburg}{images/logos/freiburg}

%\logo{\pgfuseimage{freiburg}}

\newcommand{\comment}[1]{
	\noindent
	%\vspace{0.25cm}
	{\color{red}{\textbf{TODO:} #1}}
	%\vspace{0.25cm}
}
\renewcommand{\comment}[1]{}
\newcommand{\hide}[1]{}
\newcommand{\cemph}[2]{\emph{\textcolor{#1}{#2}}}

\newcommand{\lit}[1]{{\footnotesize\color{black!70}[#1]}}

\newcommand{\litw}[1]{{\footnotesize\color{black!20}[#1]}}


\newcommand{\myframe}[2]{\begin{frame}[c]{#1}#2\end{frame}}
\newcommand{\myframetop}[2]{\begin{frame}{#1}#2\end{frame}}
\newcommand{\myit}[1]{\begin{itemize}#1\end{itemize}}
\newcommand{\myblock}[2]{\begin{block}{#1}#2\end{block}}


\newcommand{\votepurple}[1]{\textcolor{Purple}{$\bigstar$}}
\newcommand{\voteyellow}[1]{\textcolor{Goldenrod}{$\bigstar$}}
\newcommand{\voteblue}[1]{\textcolor{RoyalBlue}{$\bigstar$}}
\newcommand{\votepink}[1]{\textcolor{Pink}{$\bigstar$}}

\newcommand{\diff}{\mathop{}\!\mathrm{d}}
\newcommand{\refstyle}[1]{{\small{\textcolor{gray}{#1}}}}
\newcommand{\hands}[0]{\includegraphics[height=1.5em]{images/hands}}
\newcommand{\transpose}[0]{{\textrm{\tiny{\sf{T}}}}}
\newcommand{\norm}{{\mathcal{N}}}
\newcommand{\cutoff}[0]{\kappa}
\newcommand{\instD}[0]{\dataset}
\newcommand{\insts}[0]{\mathcal{I}}
\newcommand{\inst}[0]{i}
\newcommand{\pcs}[0]{\mathbf{\Lambda}}
\newcommand{\bx}[0]{\conf}
\newcommand{\conf}[0]{\mathbf{\lambda}}
\newcommand{\defconf}[0]{\mathbf{\lambda}_{\text{def}}}
\newcommand{\finconf}[0]{\mathbf{\lambda}^*}
\newcommand{\incumbent}[0]{\finconf}
\newcommand{\confs}[0]{\pcs}
%\newcommand{\vlambda}[0]{\bm{\lambda}}
%\newcommand{\vLambda}[0]{\bm{\Lambda}}
\newcommand{\dataset}[0]{\mathcal{D}}
\newcommand{\datasets}[0]{\mathbf{D}}
\newcommand{\loss}[0]{\mathcal{L}}

% \renewcommand{\vec}[1]{\mathbf{#1}}
\newcommand{\hist}[0]{\mathcal{H}}
\newcommand{\param}[0]{p}
\newcommand{\algo}[0]{\mathcal{A}}
\newcommand{\algos}[0]{\mathbf{A}}
%\newcommand{\nn}[0]{N}
\newcommand{\feats}[0]{\mathcal{F}}
\newcommand{\feat}[0]{\vec{f}}
\newcommand{\cluster}[0]{\vec{h}}
\newcommand{\clusters}[0]{\vec{H}}
\newcommand{\perf}[0]{\mathbb{R}}
%\newcommand{\surro}[0]{\mathcal{S}}
\newcommand{\surro}[0]{\hat{f}}
\newcommand{\func}[0]{f}
\newcommand{\epm}[0]{\surro}
\newcommand{\portfolio}[0]{\mathcal{P}}
\newcommand{\schedule}[0]{\mathcal{S}}
\newcommand{\mdata}[0]{\dataset_{\text{meta}}}

% Deep Learning
\newcommand{\weights}[0]{\theta}
\newcommand{\metaweights}[0]{\phi}


% reinforcement learning
\newcommand{\policies}[0]{\Pi}
\newcommand{\policy}[0]{\pi}
\newcommand{\actionRL}[0]{a}
\newcommand{\stateRL}[0]{s}
\newcommand{\statesRL}[0]{\mathcal{S}}
\newcommand{\rewardRL}[0]{r}
\newcommand{\rewardfuncRL}[0]{\mathcal{R}}

\RestyleAlgo{algoruled}
\DontPrintSemicolon
\LinesNumbered
\SetAlgoVlined
\SetFuncSty{textsc}

\SetKwInOut{Input}{Input}
\SetKwInOut{Output}{Output}
\SetKw{Return}{return}

%\newcommand{\changed}[1]{{\color{red}#1}}

%\newcommand{\citeN}[1]{\citeauthor{#1}~(\citeyear{#1})}

\renewcommand{\vec}[1]{\mathbf{#1}}
\DeclareMathOperator*{\argmin}{arg\,min}
\DeclareMathOperator*{\argmax}{arg\,max}

\newcommand{\aqme}{\textit{AQME}}
\newcommand{\aslib}{\textit{ASlib}}
\newcommand{\llama}{\textit{LLAMA}}
\newcommand{\satzilla}{\textit{SATzilla}}
\newcommand{\satzillaY}[1]{\textit{SATzilla'{#1}}}
\newcommand{\snnap}{\textit{SNNAP}}
\newcommand{\claspfolioTwo}{\textit{claspfolio~2}}
\newcommand{\flexfolio}{\textit{FlexFolio}}
\newcommand{\claspfolioOne}{\textit{claspfolio~1}}
\newcommand{\isac}{\textit{ISAC}}
\newcommand{\eisac}{\textit{EISAC}}
\newcommand{\sss}{\textit{3S}}
\newcommand{\sunny}{\textit{Sunny}}
\newcommand{\ssspar}{\textit{3Spar}}
\newcommand{\cshc}{\textit{CSHC}}  
\newcommand{\cshcpar}{\textit{CSHCpar}}  
\newcommand{\measp}{\textit{ME-ASP}} 
\newcommand{\aspeed}{\textit{aspeed}}
\newcommand{\autofolio}{\textit{AutoFolio}}
\newcommand{\cedalion}{\textit{Cedalion}}
\newcommand{\fanova}{\textit{fANOVA}}
\newcommand{\sbs}{\textit{SB}}
\newcommand{\oracle}{\textit{VBS}}

% like approaches
\newcommand{\claspfoliolike}[1]{\texttt{claspfolio-#1-like}}
\newcommand{\satzillalike}[1]{\texttt{SATzilla'#1-like}}
\newcommand{\isaclike}{\texttt{ISAC-like}}
\newcommand{\ssslike}{\texttt{3S-like}}
\newcommand{\measplike}{\texttt{ME-ASP-like}}

\newcommand{\aspCoseal}{\textit{ASP-POTASSCO}}
\newcommand{\cspCoseal}{\textit{CSP-2010}}
\newcommand{\maxsatCoseal}{\textit{MAXSAT12-PMS}}
\newcommand{\premarCoseal}{\textit{PRE\-MARSHALLING}}
\newcommand{\qbfCoseal}{\textit{QBF-2011}}
\newcommand{\satallTwelveCoseal}{\textit{SAT12-ALL}}
\newcommand{\sathandTwelveCoseal}{\textit{SAT12-HAND}}
\newcommand{\satinduTwelveCoseal}{\textit{SAT12-INDU}}
\newcommand{\satrandTwelveCoseal}{\textit{SAT12-RAND}}
\newcommand{\sathandElevenCoseal}{\textit{SAT11-HAND}}
\newcommand{\satinduElevenCoseal}{\textit{SAT11-INDU}}
\newcommand{\satrandElevenCoseal}{\textit{SAT11-RAND}}
\newcommand{\proteusCoseal}{\textit{PROTEUS-2014}}

\newcommand{\irace}{\textit{I/F-race}}
\newcommand{\gga}{\textit{GGA}}
\newcommand{\smac}{\textit{SMAC}}
\newcommand{\paramils}{\textit{ParamILS}}
\newcommand{\spearmint}{\textit{Spearmint}}
\newcommand{\tpe}{\textit{TPE}}

\newcommand{\gringo}{\textit{gringo}}
\newcommand{\clasp}{\textit{clasp}}
\newcommand{\lingeling}{\textit{lingeling}}

\newcommand{\hydra}{\textit{Hydra}}

\newcommand{\plingeling}{\textit{Plingeling}}
\newcommand{\ccasat}{\textit{CCASat}}

\usepackage{pifont}
\newcommand{\itarrow}{\mbox{\Pisymbol{pzd}{229}}}
\newcommand{\ithook}{\mbox{\Pisymbol{pzd}{52}}}
\newcommand{\itcross}{\mbox{\Pisymbol{pzd}{56}}}
\newcommand{\ithand}{\mbox{\raisebox{-1pt}{\Pisymbol{pzd}{43}}}}

%\DeclareMathOperator*{\argmax}{arg\,max}

\newcommand{\ie}{{\it{}i.e.\/}}
\newcommand{\eg}{{\it{}e.g.\/}}
\newcommand{\cf}{{\it{}cf.\/}}
\newcommand{\wrt}{\mbox{w.r.t.}}
\newcommand{\vs}{{\it{}vs\/}}
\newcommand{\vsp}{{\it{}vs\/}}
\newcommand{\etc}{{\copyedit{etc.}}}
\newcommand{\etal}{{\it{}et al.\/}}

\newcommand{\pscProc}{{\bf procedure}}
\newcommand{\pscBegin}{{\bf begin}}
\newcommand{\pscEnd}{{\bf end}}
\newcommand{\pscEndIf}{{\bf endif}}
\newcommand{\pscFor}{{\bf for}}
\newcommand{\pscEach}{{\bf each}}
\newcommand{\pscThen}{{\bf then}}
\newcommand{\pscElse}{{\bf else}}
\newcommand{\pscWhile}{{\bf while}}
\newcommand{\pscIf}{{\bf if}}
\newcommand{\pscRepeat}{{\bf repeat}}
\newcommand{\pscUntil}{{\bf until}}
\newcommand{\pscWithProb}{{\bf with probability}}
\newcommand{\pscOtherwise}{{\bf otherwise}}
\newcommand{\pscDo}{{\bf do}}
\newcommand{\pscTo}{{\bf to}}
\newcommand{\pscOr}{{\bf or}}
\newcommand{\pscAnd}{{\bf and}}
\newcommand{\pscNot}{{\bf not}}
\newcommand{\pscFalse}{{\bf false}}
\newcommand{\pscEachElOf}{{\bf each element of}}
\newcommand{\pscReturn}{{\bf return}}

%\newcommand{\param}[1]{{\sl{}#1}}
\newcommand{\var}[1]{{\it{}#1}}
\newcommand{\cond}[1]{{\sf{}#1}}
%\newcommand{\state}[1]{{\sf{}#1}}
%\newcommand{\func}[1]{{\sl{}#1}}
\newcommand{\set}[1]{{\Bbb #1}}
%\newcommand{\inst}[1]{{\tt{}#1}}
\newcommand{\myurl}[1]{{\small\sf #1}}

\newcommand{\Nats}{{\Bbb N}}
\newcommand{\Reals}{{\Bbb R}}
\newcommand{\extset}[2]{\{#1 \; | \; #2\}}

\newcommand{\vbar}{$\,\;|$\hspace*{-1em}\raisebox{-0.3mm}{$\,\;\;|$}}
\newcommand{\vendbar}{\raisebox{+0.4mm}{$\,\;|$}}
\newcommand{\vend}{$\,\:\lfloor$}


\newcommand{\goleft}[2][.7]{\parbox[t]{#1\linewidth}{\strut\raggedright #2\strut}}
\newcommand{\rightimage}[2][.3]{\mbox{}\hfill\raisebox{1em-\height}[0pt][0pt]{\includegraphics[width=#1\linewidth]{#2}}\vspace*{-\baselineskip}}






\title[AutoML: Learned LRs]{AutoML: Dynamic Configuration \& Learning}
\subtitle{Learning to Adjust Learning Rates}
\author[Marius Lindauer]{Bernd Bischl \and Frank Hutter \and Lars Kotthoff\newline \and \underline{Marius Lindauer} \and Joaquin Vanschoren}
\institute{}
\date{}



% \AtBeginSection[] % Do nothing for \section*
% {
%   \begin{frame}{Outline}
%     \bigskip
%     \vfill
%     \tableofcontents[currentsection]
%   \end{frame}
% }

\begin{document}
	
	\maketitle
	
	
%----------------------------------------------------------------------
%----------------------------------------------------------------------
\begin{frame}[c]{Learning Problem \litw{\href{https://www.aaai.org/ocs/index.php/AAAI/AAAI16/paper/view/11763}{Daniel et al'16}}}


\begin{itemize}
	\item Optimization of a function:
\end{itemize}
\begin{equation}
	\weights \in \argmin F(\mathbf{X}; \weights) \nonumber
\end{equation}

where $\mathbf{X}$ is an input matrix and f is parameterized by $\weights$.


\pause
\medskip

\begin{equation}
F(\mathbf{X}; \weights) = \frac{1}{N} \sum_{i=1}^N f(\xI{i}; \weights) \nonumber
\end{equation}




\end{frame}
%----------------------------------------------------------------------
%----------------------------------------------------------------------
\begin{frame}[c]{Learning Step Size Policies \litw{\href{https://www.aaai.org/ocs/index.php/AAAI/AAAI16/paper/view/11763}{Daniel et al'16}}}

\begin{itemize}
\item \alert{Idea:} Learn the hyperparameters of the weight update (short notation)
\end{itemize} 

\begin{eqnarray}
\weights^{(t+1)} = \weights^{(t)} - \alpha^{(t)} \nabla F(\weights^{(t)}) \nonumber\\
\nabla F(\weights^{(t)}) = \frac{1}{N} \sum_{i=1}^N \nabla f_i(\weights^{(t)})\nonumber
\end{eqnarray}


\begin{itemize}
\pause
\item For SGD, this would be for example the learning rate $\alpha$
\pause
\item \alert{Note (i)}: $\alpha$ have to be adapted in the course of the training
\begin{itemize}
\item similar to learning rate schedules (e.g., cosine annealing)
\end{itemize}
\pause
\item \alert{Note(ii)}: later we denote the learnt hyperparameters as $\lambda$
\medskip
\pause
\item \alert{Idea:} Use reinforcement learning to learn a policy $\policy: \stateRL \mapsto \actionRL$ to control the learning rate (or other adaptive hyperparameters)
\end{itemize}



\end{frame}
%----------------------------------------------------------------------
%----------------------------------------------------------------------
\begin{frame}[c]{Recap: Reinforcement Learning for Dynamic Algorithm Configuration}

\begin{center}
\begin{tikzpicture}[node distance=2.1cm]
        			%PreProcessing
        			
        			\node (Agent) [activity] {Policy $\policy$};
        			
        			\node (Algo) [activity, right of=Agent, xshift=3cm] {Algorithm};
        			
        			\begin{pgfonlayer}{background}
        			\path (Agent -| Agent.west)+(-0.12,1.125) node (resUL) {};
        			\path (Algo.east |- Algo.south)+(0.125,-1.125) node (resBR) {};
        			
        			% Context
        			\path [rounded corners, draw=black!50, fill=white] ($(resUL)+(0.5, -0.5)$) rectangle ($(resBR)+(0.5, -0.5)$);
        			\path [rounded corners, draw=black!50, fill=white] ($(resUL)+(0.375, -0.375)$) rectangle ($(resBR)+(0.375, -0.375)$);
        			\path [rounded corners, draw=black!50, fill=white] ($(resUL)+(0.25, -0.25)$) rectangle ($(resBR)+(0.25, -0.25)$);
        			\path [rounded corners, draw=black!50, fill=white] ($(resUL)+(0.125, -0.125)$) rectangle ($(resBR)+(0.125, -0.125)$);
        			
        			% Top level
        			\path [rounded corners, draw=black!50, fill=white] (resUL) rectangle (resBR);
        			\path (resBR)+(-1.3,0.175) node [text=black!75] {instance $\inst \in \insts$};
        			\path (resUL.east |- resBR.north)+(+.9,0.075) node [text=black!75] {control of $h$};
        			
        			\end{pgfonlayer}
        			
        			%        \draw[myarrow] (feat.south) -- ($(feat.south |- Agent)+(0,1.125)$);
        			
        			\draw[myarrow] (Agent.north) -- ($(Agent.north)+(0.0,+0.35)$) -- ($(Algo.north)+(0.0,+0.35)$) node [above,pos=0.5] {apply action $a_t$} node [below,pos=0.5] {$\lambda_h = v$} -- (Algo.north);
        			\draw[myarrow, dashed] ($(Algo.south)+(-0.25, 0)$) -- ($(Algo.south)+(-0.25, -0.35)$) -- ($(Agent.south)+(0.25, -0.35)$) node [above,pos=0.5] {state $s_{t+1}$} -- ($(Agent.south)+(0.25, 0)$);
        			\draw[myarrow] ($(Algo.south)+(0.25, 0)$) -- ($(Algo.south)+(0.25, -0.55)$) -- ($(Agent.south)+(-0.25, -0.55)$) node [below,pos=0.5] {reward $r_{t+1}$} -- ($(Agent.south)+(-0.25, 0)$);
        			
        			\draw[<->, thick, draw=black!32.5] (resBR.east |- resUL.center) -- ($(resBR.east |- resUL.center)+(0.5, -0.5)$);
        			\path (resBR.east |- resUL.center)+(.5, -0.175) node [text=black!32.5] {$\insts$};
        			% This path is only needed so the figure stays centered (the text is not visible)
        			\path (resUL.west |- resUL.center)+(-.5, -0.175) node [text=black!0] {$\insts$}; 
        			
        			\end{tikzpicture}
\end{center}

\bigskip
To apply that, we need to define:
\begin{enumerate}
	\item State description
	\item Action space
	\item Reward function
\end{enumerate}

\end{frame}

%----------------------------------------------------------------------
%----------------------------------------------------------------------
\begin{frame}[c]{RL for Step Size Policies: State \litw{\href{https://www.aaai.org/ocs/index.php/AAAI/AAAI16/paper/view/11763}{Daniel et al'16}}}

\textbf{Predictive change in function value:}

$$s_1 = \log \left( \text{Var}(\Delta \tilde{f}_i ) \right)$$
$$\Delta \tilde{f}_i = \tilde{f}(x_i; \theta + \delta \theta) - f(x_i; \theta)$$

where $\tilde{f}(x_i; \theta + \delta \theta)$ is done by a first order Taylor expansion

\pause
\smallskip
\textbf{Disagreement of function values:}
$$ s_2 = \log \left(\text{Var}(f(x_i; \theta)) \right)$$

\pause

\textbf{Discounted Average} (smoothing noise from mini-batches):
$$\hat{s}_i \leftarrow \gamma \hat{s_i} + (1 - \gamma) s_i$$

\pause

\textbf{Uncertainty Estimate} (noise level):
$$s_{K+i} \leftarrow \gamma s_{K+i} + (1-\gamma) (s_i - \hat{s}_i)^2$$


\end{frame}
%----------------------------------------------------------------------
%----------------------------------------------------------------------
\begin{frame}[c]{RL for Step Size Policies: Learning \litw{\href{https://www.aaai.org/ocs/index.php/AAAI/AAAI16/paper/view/11763}{Daniel et al'16}}}

Reward (average loss improvement over time):

$$\rewardRL = \frac{1}{T-1} \sum_{t=2}^T \left(\log(\loss^{(t-1)}) - \log(\loss^{(t)})\right)$$

\pause

Optimal Policy:

$$\policy^*(\lambda \mid \stateRL) \in \argmax_{\policy} \int \int p(\stateRL) \policy(\conf \mid \stateRL)r(\conf,\stateRL) \diff\stateRL \diff\conf $$

\pause


\begin{itemize}
\item can be learnt for example via Relative Entropy Policy Search (REPS) \lit{\href{https://www.aaai.org/ocs/index.php/AAAI/AAAI10/paper/viewFile/1851/2264}{Peter et al. 2010}}
\end{itemize}

\end{frame}
%----------------------------------------------------------------------
%----------------------------------------------------------------------
\begin{frame}[c]{RL for Step Size Policies: Training \litw{\href{https://www.aaai.org/ocs/index.php/AAAI/AAAI16/paper/view/11763}{Daniel et al'16}}}

\begin{itemize}
\item Goal: obtain robust policies,\\ i.e., good performance for many different DNN architectures
\begin{itemize}
\item[$\leadsto$] Sample architectures e.g., with different numbers of filters and layers
\item[$\leadsto$] (Sub-)Sample dataset
\item[$\leadsto$] Sample number of optimization steps
\end{itemize}
\end{itemize}

\pause 
\medskip
\centering
\includegraphics[width=0.55\textwidth]{images/l2stepsizecontroler_mnist_training.png}

"Ours" refers to the approach by \lit{\href{https://www.aaai.org/ocs/index.php/AAAI/AAAI16/paper/view/11763}{Daniel et al'16}} and $\eta$ is the learning rate

\end{frame}
%----------------------------------------------------------------------

\end{document}