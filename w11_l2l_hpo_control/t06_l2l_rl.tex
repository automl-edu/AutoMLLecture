% !TeX spellcheck = en_US

\pdfminorversion=4 % for acroread
\documentclass[aspectratio=169,t,xcolor={usenames,dvipsnames}]{beamer}
%\documentclass[t,handout,xcolor={usenames,dvipsnames}]{beamer}
\usepackage{../beamerstyle}
\usepackage{dsfont}
\usepackage{bm}
\usepackage[english]{babel}
\usepackage[utf8]{inputenc}
\usepackage{graphicx}
\usepackage{algorithm}
\usepackage[ruled,vlined,algo2e,linesnumbered]{algorithm2e}
%\usepackage[boxed,vlined]{algorithm2e}
\usepackage{hyperref}
\usepackage{booktabs}
\usepackage{mathtools}

\usepackage{amsmath,amssymb}
\usepackage{listings}
\lstset{frame=lines,framesep=3pt,numbers=left,numberblanklines=false,basicstyle=\ttfamily\small}

\usepackage{subfig}
\usepackage{multicol}
%\usepackage{appendixnumberbeamer}
%
\usepackage{tcolorbox}

\usepackage{pgfplots}
\usepackage{tikz}
\usetikzlibrary{trees} 
\usetikzlibrary{shapes.geometric}
\usetikzlibrary{positioning,shapes,shadows,arrows,calc,mindmap}
\usetikzlibrary{positioning,fadings,through}
\usetikzlibrary{decorations.pathreplacing}
\usetikzlibrary{intersections}
\usetikzlibrary{positioning,fit,calc,shadows,backgrounds}
\pgfdeclarelayer{background}
\pgfdeclarelayer{foreground}
\pgfsetlayers{background,main,foreground}
\tikzstyle{activity}=[rectangle, draw=black, rounded corners, text centered, text width=8em]
\tikzstyle{data}=[rectangle, draw=black, text centered, text width=8em]
\tikzstyle{myarrow}=[->, thick, draw=black]

% Define the layers to draw the diagram
\pgfdeclarelayer{background}
\pgfdeclarelayer{foreground}
\pgfsetlayers{background,main,foreground}

%\usepackage{listings}
%\lstset{numbers=left,
%  showstringspaces=false,
%  frame={tb},
%  captionpos=b,
%  lineskip=0pt,
%  basicstyle=\ttfamily,
%%  extendedchars=true,
%  stepnumber=1,
%  numberstyle=\small,
%  xleftmargin=1em,
%  breaklines
%}

 
\definecolor{blue}{RGB}{0, 74, 153}

\usetheme{Boadilla}
%\useinnertheme{rectangles}
\usecolortheme{whale}
\setbeamercolor{alerted text}{fg=blue}
\useoutertheme{infolines}
\setbeamertemplate{navigation symbols}{\vspace{-5pt}} % to lower the logo
\setbeamercolor{date in head/foot}{bg=blue} % blue
\setbeamercolor{date in head/foot}{fg=white}
\setbeamercolor{author in head/foot}{bg=blue} %blue
\setbeamercolor{title in head/foot}{bg=blue} % blue
\setbeamercolor{title}{fg=white, bg=blue}
\setbeamercolor{block title}{fg=white,bg=blue}
\setbeamercolor{block body}{bg=blue!10}
\setbeamercolor{frametitle}{fg=white, bg=blue}
\setbeamercovered{invisible}

\makeatletter
\setbeamertemplate{footline}
{
  \leavevmode%
  \hbox{%
  \begin{beamercolorbox}[wd=.333333\paperwidth,ht=2.25ex,dp=1ex,center]{author in head/foot}%
    \usebeamerfont{author in head/foot}\insertshortauthor
  \end{beamercolorbox}%
  \begin{beamercolorbox}[wd=.333333\paperwidth,ht=2.25ex,dp=1ex,center]{title in head/foot}%
    \usebeamerfont{title in head/foot}\insertshorttitle
  \end{beamercolorbox}%
  \begin{beamercolorbox}[wd=.333333\paperwidth,ht=2.25ex,dp=1ex,right]{date in head/foot}%
    \usebeamerfont{date in head/foot}Week \@week, Topic \@topicnumber, Slide \insertframenumber{}\hspace*{2em}
%    \insertframenumber\hspace*{2ex} 
  \end{beamercolorbox}}%
  \vskip0pt%
}

\newcommand{\@week}{0}
\newcommand{\@topicnumber}{0}
\newcommand{\week}[1]{\renewcommand{\@week}{#1}}
\newcommand{\topicnumber}[1]{\renewcommand{\@topicnumber}{#1}}

\makeatother

%\pgfdeclareimage[height=1.2cm]{automl}{images/logos/automl.png}
%\pgfdeclareimage[height=1.2cm]{freiburg}{images/logos/freiburg}

%\logo{\pgfuseimage{freiburg}}

\input{../latex_main/macros}




\title[AutoML: L2L RL]{AutoML: Dynamic Configuration \& Learning}
\subtitle{Learning to Learn: Reinforcement Learning}
\author[Marius Lindauer]{Bernd Bischl \and Frank Hutter \and Lars Kotthoff\newline \and \underline{Marius Lindauer} \and Joaquin Vanschoren}
\institute{}
\date{}
\week{11}
\topicnumber{6}



% \AtBeginSection[] % Do nothing for \section*
% {
%   \begin{frame}{Outline}
%     \bigskip
%     \vfill
%     \tableofcontents[currentsection]
%   \end{frame}
% }

\begin{document}
	
	\maketitle
	

%----------------------------------------------------------------------
%----------------------------------------------------------------------
\begin{frame}[c]{Learning to Optimize via Reinforcement Learning \litw{\href{https://arxiv.org/pdf/1606.01885.pdf}{Li and Malik. 2017}}}

\centering
\includegraphics[width=0.4\textwidth]{images/l2o_comic}

\tiny
Source: \url{https://bair.berkeley.edu/blog/2017/09/12/learning-to-optimize-with-rl/}

\end{frame}
%----------------------------------------------------------------------
%----------------------------------------------------------------------
\begin{frame}[c]{Learning to Optimize via Reinforcement Learning \litw{\href{https://arxiv.org/pdf/1606.01885.pdf}{Li and Malik. 2017}}}

\begin{block}{Reinforcement Learning for Learning to Optimize}
\begin{description}
\item[State] current location, objective values and gradients evaluated at the current and past locations
\pause
\item[Action] Step update $\Delta \x$
\pause
\item[Transition] $\x^{(t)} \leftarrow \x^{(t-1)} + \Delta \x$
\pause
\item[Cost/Reward] Objective value at the current location
\begin{itemize}
\item Since the RL agent will optimize the cumulative cost, this is equivalent to $\loss_{\text{sum}}$ \lit{\href{https://arxiv.org/pdf/1611.03824.pdf}{Chen et al. 2017}} ($\gamma=0$)
\item encourages the policy to reach the minimum of the objective function as quickly as possible
\end{itemize}
\pause
\item[Policy] DNN predicting $\mu_d$ of Gaussian (with constant variance $\sigma^2$)\\ for dimension $d$; sample $\Delta \x_d \sim \mathcal{N}(\mu_d, \sigma^2)$
\pause
\item[Training Set] randomly generated objective functions
\end{description}
\end{block}

\end{frame}
%----------------------------------------------------------------------
%----------------------------------------------------------------------
\begin{frame}[c]{Learning to Optimize via Reinforcement Learning Results \litw{\href{https://arxiv.org/pdf/1606.01885.pdf}{Li and Malik. 2017}}}

\centering
\includegraphics[width=0.4\textwidth]{images/l2o_dnn}

\begin{itemize}
\item 2-layer DNN with ReLUs
\item Training datasets for training RL agent:\\ four multivariate Gaussians and sampling 25 points from each
\begin{itemize}
\item[$\leadsto$] hard toy problem
\end{itemize}
\end{itemize}

\end{frame}
%----------------------------------------------------------------------
%----------------------------------------------------------------------
\begin{frame}[c, fragile]{Learning Acquisition Functions \litw{\href{https://arxiv.org/pdf/1904.02642.pdf}{Volpp et al. 2019}}}

\begin{itemize}
\item Instead of learning everything, it might be sufficient to \alert{learn hand-design heuristics}
\pause
\item In Bayesian Optimization (BO), the most critical hand-design heuristic is\\ the acquisition function
\begin{itemize}
\item trade-off between exploitation and exploration
\item Depending on the problem at hand, you might need a different acquisition function
\pause
\item Choices:
\begin{itemize}
\item probability of improvement (PI)
\item expected improvement (EI)
\item upper confidence bounds (UCB)
\item entropy search (ES) -- quite expensive!
\item knowledge gradient (KG)
\item ...
\end{itemize} 
\end{itemize}
\pause
\item \alert{Idea:} Learn a \emph{neural acquisition function} from data
\end{itemize}

$\leadsto$ Replace acquisition function 

\end{frame}
%----------------------------------------------------------------------
%-----------------------------------------------------------------------
\begin{frame}[c,fragile]{Bayesian Optimization: Algorithm}

\begin{algorithm}[H]
\Input{Search Space $\mathcal{X}$,
black box function $f$, 
\alert{acquisition function $\alpha$,}
maximal number of function evaluations $\bobudget$
}
\BlankLine
$\mathcal{D}^{(0)}$ $\leftarrow$ initial\_design($\mathcal{X}$); \\
\For{\bocount = $1, 2, \ldots \bobudget - |D_0|$}{
$\surro: \x \mapsto \cost(\x)$ $\leftarrow$ fit predictive model on $\mathcal{D}^{(\bocount-1)}$;\\
select $\x^{(\bocount)}$ by optimizing $\x^{(\bocount)} \in \argmax_{\x \in \mathcal{X}} \alert{\alpha(\x; \mathcal{D}^{(\bocount-1)}, \surro)}$;\\
Query $y^{(\bocount)} := f(\x^{(\bocount)})$;\\
Add observation to data $D^{(\bocount)} := D^{(\bocount-1)} \cup \{\langle \x^{(\bocount)}, y^{(\bocount)} \rangle \}$;
}
\Return{Best $\x$ according to $D$ or $\surro$}
\caption{Bayesian Optimization (BO)}
\end{algorithm}


\end{frame}
%-----------------------------------------------------------------------
%-----------------------------------------------------------------------
\begin{frame}[c]{Neural Acquisition Function  \litw{\href{https://arxiv.org/pdf/1904.02642.pdf}{Volpp et al. 2019}}}

Although the \alert{acquisition function $\alpha$} depends on the history $\mathcal{D}^{(\bocount-1)}$ and the predictive model $\surro$, $\alpha$ mainly makes use of the \alert{predictive mean $\mu$ and variance $\sigma^2$}.

\pause
\bigskip

Neural acquisition function (AF):

\begin{eqnarray}
\alpha_\weights(\x) = \alpha_\weights(\mu^{(\bocount)}(\x), \sigma^{(\bocount)}(\x), \x, \bocount, \bobudget) \nonumber
\end{eqnarray}

where $\weights$ are the parameters of a neural network, and $\mu$, $\sigma$, $\x$, $\bocount$, $\bobudget$ are its inputs.

%\pause 
%\begin{itemize}
%\item Since the input is not $x$, it allows to learn scalable acquisition function
%\item No calibration of hyperparameter necessary, once the neural AF is learnt
%\end{itemize}

\end{frame}
%-----------------------------------------------------------------------
%-----------------------------------------------------------------------
\begin{frame}[c]{RL to train Neural AF \litw{\href{https://arxiv.org/pdf/1904.02642.pdf}{Volpp et al. 2019}}}

\begin{description}
\item[Policy $\policy_\weights$:] Neural acquisition function $\alpha_\weights$
\pause
\item[Episode:] run of $\policy$ on $f\in \mathcal{F}'$
\begin{itemize}
\item $\mathcal{F}$ is a set of functions we can sample functions from
\end{itemize}
\pause
\item[State $\stateRL^{(\bocount)} $:] $\mu^{(\bocount)} $ and $\sigma^{(\bocount)} $ on a set of points $\xi^{(\bocount)} $, and $t$ and $T$
\pause
\item[Action $\actionRL^{(\bocount)} $:] Sampled point $\x^{(\bocount)}  \in \xi^{(\bocount)} $
\pause
\item[Reward $\rewardRL^{(\bocount)} $:] negative simple regret: $\rewardRL^{(\bocount)} = f(\x^*) - f(\hat{\x})$
\begin{itemize}
\item assumes that we can estimate the optimal $\x^*$ for \emph{training} functions
\end{itemize}
\pause
\item[Transition probability]: Noisy evaluation of $f$ and the predictive model update
\end{description}

\end{frame}
%-----------------------------------------------------------------------
%-----------------------------------------------------------------------
\begin{frame}[c]{Policy \litw{\href{https://arxiv.org/pdf/1904.02642.pdf}{Volpp et al. 2019}}}

\begin{itemize}
\item The state is described by a discrete set of points $\xi^{(\bocount)}  = \{\xi_n\}^N_{n=1}$
\pause
\item We feed these points through the predictive model and the neural AF to obtain $\alpha_\weights(\xi_n) = \alpha_\weights(\mu^{(\bocount)} (\xi_n), \sigma^{(\bocount)}(\xi_n), \xi_n, t, T),  $
\pause
\item $\alpha_\weights(\xi_i)$ are interpreted as the logits of categorical distribution, s.t.
$$\pi_\alpha(\cdot \mid s^{(\bocount)} ) = \text{Cat}\left[\alpha_\weights(\xi_1), \ldots, \alpha_\weights(\xi_N) \right] $$
\pause
\item Due to curse of dimensionality, we need a two step approach for~$\xi^{(\bocount)} $
\begin{enumerate}
\item sample $\xi_{\text{global}}$ using a coarse Sobol grid
\item sample $\xi_{\text{local}}$ using local optimization starting from the best samples in $\xi_{\text{global}}$
\end{enumerate}
\item[$\leadsto$] $\xi^{(\bocount)} = \xi_{\text{global}} \cup \xi_{\text{local}}$ 
\end{itemize}

\end{frame}
%-----------------------------------------------------------------------
%-----------------------------------------------------------------------
\begin{frame}[c,fragile]{Learning Acquisition Functions: Overview \litw{\href{https://arxiv.org/pdf/1904.02642.pdf}{Volpp et al. 2019}}}

\centering
\includegraphics[width=0.8\textwidth]{images/l2acq.png}


\end{frame}
%-----------------------------------------------------------------------
%-----------------------------------------------------------------------
\begin{frame}[c]{Results on Artificial Functions \litw{\href{https://arxiv.org/pdf/1904.02642.pdf}{Volpp et al. 2019}}}

\centering
\includegraphics[width=0.9\textwidth]{images/l2acq_results.png}

\medskip

\begin{itemize}
\item Approach by \lit{\href{https://arxiv.org/pdf/1904.02642.pdf}{Volpp et al. 2019}} called MetaBO
\item MetaBO performs better than other acquisition functions\\ (EI, GP-UCB, PI) and other baselines (Random, TAF)
\end{itemize}

\pause
\begin{flushleft}
\alert{Assumption}: You have a family of functions at hand that resembles your target function.	
\end{flushleft}


\end{frame}
%-----------------------------------------------------------------------


\end{document}
