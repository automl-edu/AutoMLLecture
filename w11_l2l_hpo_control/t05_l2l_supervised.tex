% !TeX spellcheck = en_US

\pdfminorversion=4 % for acroread
\documentclass[aspectratio=169,t,xcolor={usenames,dvipsnames}]{beamer}
%\documentclass[t,handout,xcolor={usenames,dvipsnames}]{beamer}
\usepackage{../beamerstyle}
\usepackage{dsfont}
\usepackage{bm}
\usepackage[english]{babel}
\usepackage[utf8]{inputenc}
\usepackage{graphicx}
\usepackage{algorithm}
\usepackage[ruled,vlined,algo2e,linesnumbered]{algorithm2e}
%\usepackage[boxed,vlined]{algorithm2e}
\usepackage{hyperref}
\usepackage{booktabs}
\usepackage{mathtools}

\usepackage{amsmath,amssymb}
\usepackage{listings}
\lstset{frame=lines,framesep=3pt,numbers=left,numberblanklines=false,basicstyle=\ttfamily\small}

\usepackage{subfig}
\usepackage{multicol}
%\usepackage{appendixnumberbeamer}
%
\usepackage{tcolorbox}

\usepackage{pgfplots}
\usepackage{tikz}
\usetikzlibrary{trees} 
\usetikzlibrary{shapes.geometric}
\usetikzlibrary{positioning,shapes,shadows,arrows,calc,mindmap}
\usetikzlibrary{positioning,fadings,through}
\usetikzlibrary{decorations.pathreplacing}
\usetikzlibrary{intersections}
\usetikzlibrary{positioning,fit,calc,shadows,backgrounds}
\pgfdeclarelayer{background}
\pgfdeclarelayer{foreground}
\pgfsetlayers{background,main,foreground}
\tikzstyle{activity}=[rectangle, draw=black, rounded corners, text centered, text width=8em]
\tikzstyle{data}=[rectangle, draw=black, text centered, text width=8em]
\tikzstyle{myarrow}=[->, thick, draw=black]

% Define the layers to draw the diagram
\pgfdeclarelayer{background}
\pgfdeclarelayer{foreground}
\pgfsetlayers{background,main,foreground}

%\usepackage{listings}
%\lstset{numbers=left,
%  showstringspaces=false,
%  frame={tb},
%  captionpos=b,
%  lineskip=0pt,
%  basicstyle=\ttfamily,
%%  extendedchars=true,
%  stepnumber=1,
%  numberstyle=\small,
%  xleftmargin=1em,
%  breaklines
%}

 
\definecolor{blue}{RGB}{0, 74, 153}

\usetheme{Boadilla}
%\useinnertheme{rectangles}
\usecolortheme{whale}
\setbeamercolor{alerted text}{fg=blue}
\useoutertheme{infolines}
\setbeamertemplate{navigation symbols}{\vspace{-5pt}} % to lower the logo
\setbeamercolor{date in head/foot}{bg=blue} % blue
\setbeamercolor{date in head/foot}{fg=white}
\setbeamercolor{author in head/foot}{bg=blue} %blue
\setbeamercolor{title in head/foot}{bg=blue} % blue
\setbeamercolor{title}{fg=white, bg=blue}
\setbeamercolor{block title}{fg=white,bg=blue}
\setbeamercolor{block body}{bg=blue!10}
\setbeamercolor{frametitle}{fg=white, bg=blue}
\setbeamercovered{invisible}

\makeatletter
\setbeamertemplate{footline}
{
  \leavevmode%
  \hbox{%
  \begin{beamercolorbox}[wd=.333333\paperwidth,ht=2.25ex,dp=1ex,center]{author in head/foot}%
    \usebeamerfont{author in head/foot}\insertshortauthor
  \end{beamercolorbox}%
  \begin{beamercolorbox}[wd=.333333\paperwidth,ht=2.25ex,dp=1ex,center]{title in head/foot}%
    \usebeamerfont{title in head/foot}\insertshorttitle
  \end{beamercolorbox}%
  \begin{beamercolorbox}[wd=.333333\paperwidth,ht=2.25ex,dp=1ex,right]{date in head/foot}%
    \usebeamerfont{date in head/foot}Week \@week, Topic \@topicnumber, Slide \insertframenumber{}\hspace*{2em}
%    \insertframenumber\hspace*{2ex} 
  \end{beamercolorbox}}%
  \vskip0pt%
}

\newcommand{\@week}{0}
\newcommand{\@topicnumber}{0}
\newcommand{\week}[1]{\renewcommand{\@week}{#1}}
\newcommand{\topicnumber}[1]{\renewcommand{\@topicnumber}{#1}}

\makeatother

%\pgfdeclareimage[height=1.2cm]{automl}{images/logos/automl.png}
%\pgfdeclareimage[height=1.2cm]{freiburg}{images/logos/freiburg}

%\logo{\pgfuseimage{freiburg}}

\input{../latex_main/macros}




\title[AutoML: L2L Supervised]{AutoML: Dynamic Configuration \& Learning}
\subtitle{Learning to Learn: Supervised}
\author[Marius Lindauer]{Bernd Bischl \and Frank Hutter \and Lars Kotthoff\newline \and \underline{Marius Lindauer} \and Joaquin Vanschoren}
\institute{}
\date{}
\week{11}
\topicnumber{5}



% \AtBeginSection[] % Do nothing for \section*
% {
%   \begin{frame}{Outline}
%     \bigskip
%     \vfill
%     \tableofcontents[currentsection]
%   \end{frame}
% }

\begin{document}
	
	\maketitle
	

%----------------------------------------------------------------------
%----------------------------------------------------------------------
\begin{frame}[c]{Learning to Learn}

\begin{block}{Idea}
\begin{itemize}
	\item Learn algorithms directly, e.g., how to search in the weight space
	\item First idea: learn weight updates of a neural network
\end{itemize}
\end{block}

\pause

\begin{block}{Learning to learn by gradient descent by gradient descent\newline \litw{\href{https://arxiv.org/pdf/1606.04474.pdf}{Andrychowicz et al. 2016}}}
Weight updates (note: $\weights$ denote DNN weights):
\begin{equation}
\weights^{(t+1)} = \weights^{(t)} - \alpha^{(t)} \nabla f(\weights^{(t)}) \nonumber
\end{equation}
\pause
Even more general:
\begin{equation}
\weights^{(t+1)} = \weights^{(t)} + g^{(t)}(\nabla f(\weights^{(t)}), \metaweights) \nonumber
\end{equation}
where $g$ is the optimizer and $\metaweights$ are the parameters of the optimizer $g$.\\
\pause
$\leadsto$ \alert{Goal: Optimize $f$ wrt $\weights$ by learning $g$ (resp. $\metaweights$)}
\end{block}

\end{frame}
%----------------------------------------------------------------------
%----------------------------------------------------------------------
\begin{frame}[c]{Learning to Learn: Objective \litw{\href{https://arxiv.org/pdf/1606.04474.pdf}{Andrychowicz et al. 2016}}}

\vspace{-0.5cm}
\begin{equation}
\loss(\metaweights) = \mathbb{E}\left[ f(\weights^*(f,\metaweights)) \right]\nonumber
\end{equation}

where $\loss$ is a loss function and $\weights^*(f,\metaweights)$ are the optimized weights $\weights^*$ by using the optimizer parameterized with $\metaweights$ on function $f$.

\pause

%\vspace{-0.5cm}
\begin{equation}
\loss(\metaweights) = \mathbb{E}\left[\sum_{t=1}^T w^{(t)} f(\weights^{(t)})\right]\nonumber
\end{equation}

\pause
where $w_t$ are arbitrary weights associated with each time step
and 

\pause
\vspace{-0.5cm}
\begin{eqnarray}
\weights^{(t+1)} = \weights^{(t)} + g^{(t)}\nonumber\\
\begin{pmatrix}g^{(t)}\\h^{(t+1)}\end{pmatrix} = m(\nabla_\weights f(\weights^{(t)}), h^{(t)}, \metaweights)\nonumber
\end{eqnarray}

\pause
$\leadsto$ Goal: Learn $m$ via $\metaweights$ by using gradient descent by optimizing $\loss$ \\
\pause
$\leadsto$ ``Learning to learn gradient descent by gradient descent''
\end{frame}
%----------------------------------------------------------------------
%----------------------------------------------------------------------
\begin{frame}[c]{Learning to Learn: LSTM approach \litw{\href{https://arxiv.org/pdf/1606.04474.pdf}{Andrychowicz et al. 2016}}}

\begin{description}
\item[Optimizee] Target network to be trained
\item[Optimizer] LSTM with hidden state $h_t$ that predicts weight updates $g_t$
\end{description}

\medskip

\centering
\includegraphics[width=0.8\textwidth]{images/learning_to_learn_lstm}

\end{frame}
%----------------------------------------------------------------------
%----------------------------------------------------------------------
\begin{frame}[c]{Learning to Learn: Coordinatewise LSTM optimizer \litw{\href{https://arxiv.org/pdf/1606.04474.pdf}{Andrychowicz et al. 2016}}}

\centering
\includegraphics[width=.5\textwidth]{images/l2l_shared_lstm.png}

\begin{itemize}
\item One LSTM for each coordinate (i.e., weight)
\item All LSTMs have shared parameters $\metaweights$
\item Each coordinate has its own separate hidden state
\pause
\item[$\leadsto$] We can train the LSTM on $k$ weights and apply it larger DNNs\\ with $k'$ weights, where $k \leq k'$
\end{itemize}

\end{frame}
%----------------------------------------------------------------------
%----------------------------------------------------------------------
\begin{frame}[c]{Learning to Learn with LSTM: Results \litw{\href{https://arxiv.org/pdf/1606.04474.pdf}{Andrychowicz et al. 2016}}}

\centering
\includegraphics[width=0.5\textwidth]{images/l2l_mnist_base}

\end{frame}
%----------------------------------------------------------------------
%----------------------------------------------------------------------
\begin{frame}[c]{Learning to Learn with LSTM: Results \litw{\href{https://arxiv.org/pdf/1606.04474.pdf}{Andrychowicz et al. 2016}}}

Changing the original architecture of the DNN:
\smallskip

\centering
\includegraphics[width=0.7\textwidth]{images/l2l_mnist_okchange}

$\leadsto$ learnt optimizer is robust against some architectural changes

\end{frame}
%----------------------------------------------------------------------
%----------------------------------------------------------------------
\begin{frame}[c]{Learning to Learn with LSTM: Results \litw{\href{https://arxiv.org/pdf/1606.04474.pdf}{Andrychowicz et al. 2016}}}

Changing the activation function to ReLU:
\smallskip

\centering
\includegraphics[width=0.4\textwidth]{images/l2l_mnist_relu}

$\leadsto$ fails on other activation functions

\end{frame}
%----------------------------------------------------------------------
%----------------------------------------------------------------------
\begin{frame}[c]{Learning Black-box Optimization~\litw{\href{https://arxiv.org/pdf/1611.03824.pdf}{Chen et al. 2017}}}

\begin{block}{Black Box Optimization Setting}

\begin{equation}
\x^* \in \argmin_{\x \in \mathcal{X}} f(\x) \nonumber
\end{equation}
	
\begin{enumerate}
\item Given the current state of knowledge $h^{(t)}$ propose a query point $\x^{(t)}$
\item Observe the response $y^{(t)}$
\item Update any internal statistics to produce $h^{(t+1)}$
\end{enumerate}
\end{block}


\end{frame}
%----------------------------------------------------------------------
%----------------------------------------------------------------------
\begin{frame}[c]{Learning Black-box Optimization~\litw{\href{https://arxiv.org/pdf/1611.03824.pdf}{Chen et al. 2017}}}


\begin{block}{Learning Black Box Optimization}
	Essentially, a similar idea as before:
	\begin{eqnarray}
	h^{(t)}, \x^{(t)} &=& \text{RNN}_\metaweights(h^{(t-1)}, \x^{(t-1)}, y^{(t)}) \nonumber \\
	y^{(t)} &\sim& p(y|\x^{(t)})\nonumber
	\end{eqnarray}
	
	\begin{itemize}
		\item Using recurrent neural network (RNN) to predict next $x_t$.
		\item $h^{(t)}$ is the internal hidden state 
	\end{itemize}
	
\end{block}



\end{frame}
%----------------------------------------------------------------------
%----------------------------------------------------------------------
\begin{frame}[c]{Learning Black-box Optimization: Loss Functions \litw{\href{https://arxiv.org/pdf/1611.03824.pdf}{Chen et al. 2017}}}

\begin{itemize}
\item Sum loss: Provides more information than final loss
\end{itemize}
\begin{equation}
\loss_{\text{sum}}(\metaweights) = \mathbb{E}_{f,y^{(1:T-1)}}\left[\sum_{t=1}^T f(\x^{(t)})\right]\nonumber
\end{equation}

\pause

\begin{itemize}
\item EI loss: Try to learn behavior of Bayesian optimizer based on expected improvement (EI)
\begin{itemize}
\item requires model (e.g., GP)
\end{itemize}
\end{itemize}
\begin{equation}
\loss_{\text{EI}}(\metaweights) = - \mathbb{E}_{f,y^{(1:T-1)}}\left[\sum_{t=1}^T \text{EI}(\x^{(t)} | y^{(1:t-1)})\right]\nonumber
\end{equation}

\pause

\begin{itemize}
\item Observed Improvement Loss:
\end{itemize}

\begin{equation}
\loss_{\text{OI}}(\metaweights) = \mathbb{E}_{f,y^{(1:T-1)}}\left[\sum_{t=1}^T \min \left\{f(\x^{(t)}) - \min_{i<t}(f(\x^{(i)})),0 \right\}  \right]\nonumber
\end{equation}

\end{frame}
%----------------------------------------------------------------------
%----------------------------------------------------------------------
\begin{frame}[c]{Learning Black-box Optimization: Results \litw{\href{https://arxiv.org/pdf/1611.03824.pdf}{Chen et al. 2017}}}

\centering
\includegraphics[width=0.4\textwidth]{images/l2bo_hartmann3}

\begin{itemize}
\item Hartmann3 is an artificial function with 3 dimensions
\pause
\item[$\leadsto$] $\loss_{\text{OI}}$ and $\loss_{\text{EI}}$ perform best
\item[$\leadsto$] $\loss_{\text{OI}}$ easier to compute than $\loss_{\text{EI}}$\\ because we need a predictive model to compute EI 
\end{itemize}

\end{frame}
%----------------------------------------------------------------------


\end{document}
