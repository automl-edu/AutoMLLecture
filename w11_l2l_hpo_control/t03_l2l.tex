
\pdfminorversion=4 % for acroread
\documentclass[aspectratio=169,t,xcolor={usenames,dvipsnames}]{beamer}
%\documentclass[t,handout,xcolor={usenames,dvipsnames}]{beamer}
\usepackage{../beamerstyle}
\usepackage{dsfont}
\usepackage{bm}
\usepackage[english]{babel}
\usepackage[utf8]{inputenc}
\usepackage{graphicx}
\usepackage{algorithm}
\usepackage[ruled,vlined,algo2e,linesnumbered]{algorithm2e}
%\usepackage[boxed,vlined]{algorithm2e}
\usepackage{hyperref}
\usepackage{booktabs}
\usepackage{mathtools}

\usepackage{amsmath,amssymb}
\usepackage{listings}
\lstset{frame=lines,framesep=3pt,numbers=left,numberblanklines=false,basicstyle=\ttfamily\small}

\usepackage{subfig}
\usepackage{multicol}
%\usepackage{appendixnumberbeamer}
%
\usepackage{tcolorbox}

\usepackage{pgfplots}
\usepackage{tikz}
\usetikzlibrary{trees} 
\usetikzlibrary{shapes.geometric}
\usetikzlibrary{positioning,shapes,shadows,arrows,calc,mindmap}
\usetikzlibrary{positioning,fadings,through}
\usetikzlibrary{decorations.pathreplacing}
\usetikzlibrary{intersections}
\usetikzlibrary{positioning,fit,calc,shadows,backgrounds}
\pgfdeclarelayer{background}
\pgfdeclarelayer{foreground}
\pgfsetlayers{background,main,foreground}
\tikzstyle{activity}=[rectangle, draw=black, rounded corners, text centered, text width=8em]
\tikzstyle{data}=[rectangle, draw=black, text centered, text width=8em]
\tikzstyle{myarrow}=[->, thick, draw=black]

% Define the layers to draw the diagram
\pgfdeclarelayer{background}
\pgfdeclarelayer{foreground}
\pgfsetlayers{background,main,foreground}

%\usepackage{listings}
%\lstset{numbers=left,
%  showstringspaces=false,
%  frame={tb},
%  captionpos=b,
%  lineskip=0pt,
%  basicstyle=\ttfamily,
%%  extendedchars=true,
%  stepnumber=1,
%  numberstyle=\small,
%  xleftmargin=1em,
%  breaklines
%}

 
\definecolor{blue}{RGB}{0, 74, 153}

\usetheme{Boadilla}
%\useinnertheme{rectangles}
\usecolortheme{whale}
\setbeamercolor{alerted text}{fg=blue}
\useoutertheme{infolines}
\setbeamertemplate{navigation symbols}{\vspace{-5pt}} % to lower the logo
\setbeamercolor{date in head/foot}{bg=white} % blue
\setbeamercolor{date in head/foot}{fg=white}
\setbeamercolor{author  in head/foot}{bg=white} %blue
\setbeamercolor{title in head/foot}{bg=white} % blue
\setbeamercolor{title}{fg=white, bg=blue}
\setbeamercolor{block title}{fg=white,bg=blue}
\setbeamercolor{block body}{bg=blue!10}
\setbeamercolor{frametitle}{fg=white, bg=blue}
\setbeamercovered{invisible}

\makeatletter
\setbeamertemplate{footline}
{
  \leavevmode%
  \hbox{%
  \begin{beamercolorbox}[wd=.333333\paperwidth,ht=2.25ex,dp=1ex,center]{author in head/foot}%
%    \usebeamerfont{author in head/foot}\insertshortauthor
  \end{beamercolorbox}%
  \begin{beamercolorbox}[wd=.333333\paperwidth,ht=2.25ex,dp=1ex,center]{title in head/foot}%
    \usebeamerfont{title in head/foot}\insertshorttitle
  \end{beamercolorbox}%
  \begin{beamercolorbox}[wd=.333333\paperwidth,ht=2.25ex,dp=1ex,right]{date in head/foot}%
    \usebeamerfont{date in head/foot}\insertshortdate{}\hspace*{2em}
%    \insertframenumber\hspace*{2ex} 
  \end{beamercolorbox}}%
  \vskip0pt%
}
\makeatother

%\pgfdeclareimage[height=1.2cm]{automl}{images/logos/automl.png}
%\pgfdeclareimage[height=1.2cm]{freiburg}{images/logos/freiburg}

%\logo{\pgfuseimage{freiburg}}

\newcommand{\comment}[1]{
	\noindent
	%\vspace{0.25cm}
	{\color{red}{\textbf{TODO:} #1}}
	%\vspace{0.25cm}
}
\renewcommand{\comment}[1]{}
\newcommand{\hide}[1]{}
\newcommand{\cemph}[2]{\emph{\textcolor{#1}{#2}}}

\newcommand{\lit}[1]{{\footnotesize\color{black!70}[#1]}}

\newcommand{\litw}[1]{{\footnotesize\color{black!20}[#1]}}


\newcommand{\myframe}[2]{\begin{frame}[c]{#1}#2\end{frame}}
\newcommand{\myframetop}[2]{\begin{frame}{#1}#2\end{frame}}
\newcommand{\myit}[1]{\begin{itemize}#1\end{itemize}}
\newcommand{\myblock}[2]{\begin{block}{#1}#2\end{block}}


\newcommand{\votepurple}[1]{\textcolor{Purple}{$\bigstar$}}
\newcommand{\voteyellow}[1]{\textcolor{Goldenrod}{$\bigstar$}}
\newcommand{\voteblue}[1]{\textcolor{RoyalBlue}{$\bigstar$}}
\newcommand{\votepink}[1]{\textcolor{Pink}{$\bigstar$}}

\newcommand{\diff}{\mathop{}\!\mathrm{d}}
\newcommand{\refstyle}[1]{{\small{\textcolor{gray}{#1}}}}
\newcommand{\hands}[0]{\includegraphics[height=1.5em]{images/hands}}
\newcommand{\transpose}[0]{{\textrm{\tiny{\sf{T}}}}}
\newcommand{\norm}{{\mathcal{N}}}
\newcommand{\cutoff}[0]{\kappa}
\newcommand{\instD}[0]{\dataset}
\newcommand{\insts}[0]{\mathcal{I}}
\newcommand{\inst}[0]{i}
\newcommand{\pcs}[0]{\mathbf{\Lambda}}
\newcommand{\bx}[0]{\conf}
\newcommand{\conf}[0]{\mathbf{\lambda}}
\newcommand{\defconf}[0]{\mathbf{\lambda}_{\text{def}}}
\newcommand{\finconf}[0]{\mathbf{\lambda}^*}
\newcommand{\incumbent}[0]{\finconf}
\newcommand{\confs}[0]{\pcs}
%\newcommand{\vlambda}[0]{\bm{\lambda}}
%\newcommand{\vLambda}[0]{\bm{\Lambda}}
\newcommand{\dataset}[0]{\mathcal{D}}
\newcommand{\datasets}[0]{\mathbf{D}}
\newcommand{\loss}[0]{\mathcal{L}}

% \renewcommand{\vec}[1]{\mathbf{#1}}
\newcommand{\hist}[0]{\mathcal{H}}
\newcommand{\param}[0]{p}
\newcommand{\algo}[0]{\mathcal{A}}
\newcommand{\algos}[0]{\mathbf{A}}
%\newcommand{\nn}[0]{N}
\newcommand{\feats}[0]{\mathcal{F}}
\newcommand{\feat}[0]{\vec{f}}
\newcommand{\cluster}[0]{\vec{h}}
\newcommand{\clusters}[0]{\vec{H}}
\newcommand{\perf}[0]{\mathbb{R}}
%\newcommand{\surro}[0]{\mathcal{S}}
\newcommand{\surro}[0]{\hat{f}}
\newcommand{\func}[0]{f}
\newcommand{\epm}[0]{\surro}
\newcommand{\portfolio}[0]{\mathcal{P}}
\newcommand{\schedule}[0]{\mathcal{S}}
\newcommand{\mdata}[0]{\dataset_{\text{meta}}}

% Deep Learning
\newcommand{\weights}[0]{\theta}
\newcommand{\metaweights}[0]{\phi}


% reinforcement learning
\newcommand{\policies}[0]{\Pi}
\newcommand{\policy}[0]{\pi}
\newcommand{\actionRL}[0]{a}
\newcommand{\stateRL}[0]{s}
\newcommand{\statesRL}[0]{\mathcal{S}}
\newcommand{\rewardRL}[0]{r}
\newcommand{\rewardfuncRL}[0]{\mathcal{R}}

\RestyleAlgo{algoruled}
\DontPrintSemicolon
\LinesNumbered
\SetAlgoVlined
\SetFuncSty{textsc}

\SetKwInOut{Input}{Input}
\SetKwInOut{Output}{Output}
\SetKw{Return}{return}

%\newcommand{\changed}[1]{{\color{red}#1}}

%\newcommand{\citeN}[1]{\citeauthor{#1}~(\citeyear{#1})}

\renewcommand{\vec}[1]{\mathbf{#1}}
\DeclareMathOperator*{\argmin}{arg\,min}
\DeclareMathOperator*{\argmax}{arg\,max}

\newcommand{\aqme}{\textit{AQME}}
\newcommand{\aslib}{\textit{ASlib}}
\newcommand{\llama}{\textit{LLAMA}}
\newcommand{\satzilla}{\textit{SATzilla}}
\newcommand{\satzillaY}[1]{\textit{SATzilla'{#1}}}
\newcommand{\snnap}{\textit{SNNAP}}
\newcommand{\claspfolioTwo}{\textit{claspfolio~2}}
\newcommand{\flexfolio}{\textit{FlexFolio}}
\newcommand{\claspfolioOne}{\textit{claspfolio~1}}
\newcommand{\isac}{\textit{ISAC}}
\newcommand{\eisac}{\textit{EISAC}}
\newcommand{\sss}{\textit{3S}}
\newcommand{\sunny}{\textit{Sunny}}
\newcommand{\ssspar}{\textit{3Spar}}
\newcommand{\cshc}{\textit{CSHC}}  
\newcommand{\cshcpar}{\textit{CSHCpar}}  
\newcommand{\measp}{\textit{ME-ASP}} 
\newcommand{\aspeed}{\textit{aspeed}}
\newcommand{\autofolio}{\textit{AutoFolio}}
\newcommand{\cedalion}{\textit{Cedalion}}
\newcommand{\fanova}{\textit{fANOVA}}
\newcommand{\sbs}{\textit{SB}}
\newcommand{\oracle}{\textit{VBS}}

% like approaches
\newcommand{\claspfoliolike}[1]{\texttt{claspfolio-#1-like}}
\newcommand{\satzillalike}[1]{\texttt{SATzilla'#1-like}}
\newcommand{\isaclike}{\texttt{ISAC-like}}
\newcommand{\ssslike}{\texttt{3S-like}}
\newcommand{\measplike}{\texttt{ME-ASP-like}}

\newcommand{\aspCoseal}{\textit{ASP-POTASSCO}}
\newcommand{\cspCoseal}{\textit{CSP-2010}}
\newcommand{\maxsatCoseal}{\textit{MAXSAT12-PMS}}
\newcommand{\premarCoseal}{\textit{PRE\-MARSHALLING}}
\newcommand{\qbfCoseal}{\textit{QBF-2011}}
\newcommand{\satallTwelveCoseal}{\textit{SAT12-ALL}}
\newcommand{\sathandTwelveCoseal}{\textit{SAT12-HAND}}
\newcommand{\satinduTwelveCoseal}{\textit{SAT12-INDU}}
\newcommand{\satrandTwelveCoseal}{\textit{SAT12-RAND}}
\newcommand{\sathandElevenCoseal}{\textit{SAT11-HAND}}
\newcommand{\satinduElevenCoseal}{\textit{SAT11-INDU}}
\newcommand{\satrandElevenCoseal}{\textit{SAT11-RAND}}
\newcommand{\proteusCoseal}{\textit{PROTEUS-2014}}

\newcommand{\irace}{\textit{I/F-race}}
\newcommand{\gga}{\textit{GGA}}
\newcommand{\smac}{\textit{SMAC}}
\newcommand{\paramils}{\textit{ParamILS}}
\newcommand{\spearmint}{\textit{Spearmint}}
\newcommand{\tpe}{\textit{TPE}}

\newcommand{\gringo}{\textit{gringo}}
\newcommand{\clasp}{\textit{clasp}}
\newcommand{\lingeling}{\textit{lingeling}}

\newcommand{\hydra}{\textit{Hydra}}

\newcommand{\plingeling}{\textit{Plingeling}}
\newcommand{\ccasat}{\textit{CCASat}}

\usepackage{pifont}
\newcommand{\itarrow}{\mbox{\Pisymbol{pzd}{229}}}
\newcommand{\ithook}{\mbox{\Pisymbol{pzd}{52}}}
\newcommand{\itcross}{\mbox{\Pisymbol{pzd}{56}}}
\newcommand{\ithand}{\mbox{\raisebox{-1pt}{\Pisymbol{pzd}{43}}}}

%\DeclareMathOperator*{\argmax}{arg\,max}

\newcommand{\ie}{{\it{}i.e.\/}}
\newcommand{\eg}{{\it{}e.g.\/}}
\newcommand{\cf}{{\it{}cf.\/}}
\newcommand{\wrt}{\mbox{w.r.t.}}
\newcommand{\vs}{{\it{}vs\/}}
\newcommand{\vsp}{{\it{}vs\/}}
\newcommand{\etc}{{\copyedit{etc.}}}
\newcommand{\etal}{{\it{}et al.\/}}

\newcommand{\pscProc}{{\bf procedure}}
\newcommand{\pscBegin}{{\bf begin}}
\newcommand{\pscEnd}{{\bf end}}
\newcommand{\pscEndIf}{{\bf endif}}
\newcommand{\pscFor}{{\bf for}}
\newcommand{\pscEach}{{\bf each}}
\newcommand{\pscThen}{{\bf then}}
\newcommand{\pscElse}{{\bf else}}
\newcommand{\pscWhile}{{\bf while}}
\newcommand{\pscIf}{{\bf if}}
\newcommand{\pscRepeat}{{\bf repeat}}
\newcommand{\pscUntil}{{\bf until}}
\newcommand{\pscWithProb}{{\bf with probability}}
\newcommand{\pscOtherwise}{{\bf otherwise}}
\newcommand{\pscDo}{{\bf do}}
\newcommand{\pscTo}{{\bf to}}
\newcommand{\pscOr}{{\bf or}}
\newcommand{\pscAnd}{{\bf and}}
\newcommand{\pscNot}{{\bf not}}
\newcommand{\pscFalse}{{\bf false}}
\newcommand{\pscEachElOf}{{\bf each element of}}
\newcommand{\pscReturn}{{\bf return}}

%\newcommand{\param}[1]{{\sl{}#1}}
\newcommand{\var}[1]{{\it{}#1}}
\newcommand{\cond}[1]{{\sf{}#1}}
%\newcommand{\state}[1]{{\sf{}#1}}
%\newcommand{\func}[1]{{\sl{}#1}}
\newcommand{\set}[1]{{\Bbb #1}}
%\newcommand{\inst}[1]{{\tt{}#1}}
\newcommand{\myurl}[1]{{\small\sf #1}}

\newcommand{\Nats}{{\Bbb N}}
\newcommand{\Reals}{{\Bbb R}}
\newcommand{\extset}[2]{\{#1 \; | \; #2\}}

\newcommand{\vbar}{$\,\;|$\hspace*{-1em}\raisebox{-0.3mm}{$\,\;\;|$}}
\newcommand{\vendbar}{\raisebox{+0.4mm}{$\,\;|$}}
\newcommand{\vend}{$\,\:\lfloor$}


\newcommand{\goleft}[2][.7]{\parbox[t]{#1\linewidth}{\strut\raggedright #2\strut}}
\newcommand{\rightimage}[2][.3]{\mbox{}\hfill\raisebox{1em-\height}[0pt][0pt]{\includegraphics[width=#1\linewidth]{#2}}\vspace*{-\baselineskip}}







\title[AutoML: Learning to Control]{AutoML: Automated Machine Learning}
\subtitle{Learning to Control}
\author{Marius Lindauer}
\institute{\vspace*{2em}\includegraphics[width=10em]{../latex_main/images/automl_hannover.png}}
\date{}



% \AtBeginSection[] % Do nothing for \section*
% {
%   \begin{frame}{Outline}
%     \bigskip
%     \vfill
%     \tableofcontents[currentsection]
%   \end{frame}
% }

\begin{document}
	
	\maketitle
	

%----------------------------------------------------------------------
%----------------------------------------------------------------------
\begin{frame}[c]{Learning to Learn}

\centering
\huge
Learning to learn by gradient descent by gradient descent\\
by Andrychowicz et al. '16


\end{frame}
%----------------------------------------------------------------------
%----------------------------------------------------------------------
\begin{frame}[c]{Learning to Learn}

\begin{block}{Idea}
\begin{itemize}
	\item Learn algorithms directly, e.g., how to search in the weight space
	\item First idea: learn weight updates of a neural network
\end{itemize}
\end{block}

\pause

\begin{block}{Learning to learn by gradient descent by gradient descent\newline \lit{Andrychowicz et al'16}}
Weight updates (note: \alert{$\theta$ denote DNN weights!}):
\begin{equation}
\theta_{t+1} = \theta_t - \alpha_t \nabla f(\theta_t) \nonumber
\end{equation}
\pause
Even more general:
\begin{equation}
\theta_{t+1} = \theta_t + g_t(\nabla f(\theta_t), \phi) \nonumber
\end{equation}
where $g$ is the optimizer and $\phi$ are the parameters of the optimizer $g$.\\
\pause
$\leadsto$ \alert{Goal: Optimize $f$ wrt $\theta$ by learning $g$ (resp. $\phi$)}
\end{block}

\end{frame}
%----------------------------------------------------------------------
%----------------------------------------------------------------------
\begin{frame}[c]{Learning to Learn: Objective \lit{Andrychowicz et al'16}}

\vspace{-0.5cm}
\begin{equation}
\mathcal{L}(\phi) = \mathbb{E}\left[ f(\theta^*(f,\phi)) \right]\nonumber
\end{equation}

where $\mathcal{L}$ is a loss function and $\theta^*(f,\phi)$ are the optimized weights $\theta^*$ by using the optimizer parameterized with $\phi$ on function $f$.

\pause

%\vspace{-0.5cm}
\begin{equation}
\mathcal{L}(\phi) = \mathbb{E}\left[\sum_{t=1}^T w_t f(\theta_t)\right]\nonumber
\end{equation}

\pause
where $w_t$ are arbitrary weights associated with each time step
and 

\pause
\vspace{-0.5cm}
\begin{eqnarray}
\theta_{t+1} = \theta_t + g_t\nonumber\\
\begin{pmatrix}g_t\\h_{t+1}\end{pmatrix} = m(\nabla_\theta f(\theta_t), h_t, \phi)\nonumber
\end{eqnarray}

\pause
$\leadsto$ Goal: Learn $m$ via $\phi$ by using gradient descent by optimizing $\mathcal{L}$ \\
\pause
$\leadsto$ ``Learning to learn gradient descent by gradient descent''
\end{frame}
%----------------------------------------------------------------------
%----------------------------------------------------------------------
\begin{frame}[c]{Learning to Learn: LSTM approach \lit{Andrychowicz et al'16}}

\begin{description}
\item[Optimizee] Target network to be trained
\item[Optimizer] LSTM with hidden state $h_t$ that predicts weight updates $g_t$
\end{description}

\medskip

\includegraphics[width=1\textwidth]{images/learning_to_learn_lstm}

\end{frame}
%----------------------------------------------------------------------
%----------------------------------------------------------------------
\begin{frame}[c]{Learning to Learn: Coordinatewise LSTM optimizer \lit{Andrychowicz et al'16}}

\centering
\includegraphics[width=.7\textwidth]{images/l2l_shared_lstm.png}

\begin{itemize}
\item One LSTM for each coordinate (i.e., weight)
\item All LSTMs have shared parameters $\phi$
\item Each coordinate has its own separate hidden state
\pause
\item[$\leadsto$] We can train the LSTM on $k$ weights and apply it larger DNNs\\ with $k'$ weights, where $k \leq k'$
\end{itemize}

\end{frame}
%----------------------------------------------------------------------
%----------------------------------------------------------------------
\begin{frame}[c]{Learning to Learn with LSTM: Results \lit{Andrychowicz et al'16}}

\centering
\includegraphics[width=0.7\textwidth]{images/l2l_mnist_base}

\end{frame}
%----------------------------------------------------------------------
%----------------------------------------------------------------------
\begin{frame}[c]{Learning to Learn with LSTM: Results \lit{Andrychowicz et al'16}}

Changing the original architecture of the DNN:
\smallskip

\centering
\includegraphics[width=0.9\textwidth]{images/l2l_mnist_okchange}

$\leadsto$ learnt optimizer is robust against some architectural changes

\end{frame}
%----------------------------------------------------------------------
%----------------------------------------------------------------------
\begin{frame}[c]{Learning to Learn with LSTM: Results\newline \lit{Andrychowicz et al'16}}

Changing the activation function to ReLU:
\smallskip

\centering
\includegraphics[width=0.6\textwidth]{images/l2l_mnist_relu}

$\leadsto$ fails on other activation functions

\end{frame}
%----------------------------------------------------------------------
%----------------------------------------------------------------------
\begin{frame}[c]{Learning Step Size Controllers \lit{Daniel et al'16}}

\huge
\centering
Learning Step Size Controllers for Robust Neural Network Training\\
by Daniel et al. '16



\end{frame}
%----------------------------------------------------------------------
%----------------------------------------------------------------------
\begin{frame}[c]{Learning Step Size Controllers \lit{Daniel et al'16}}

\begin{itemize}
\item \alert{Idea:} Learn the hyperparameters of the weight update 
\end{itemize}

\begin{equation}
\theta_{t+1} = \theta_t - \alpha_t \nabla f(\theta_t) \nonumber
\end{equation}

\begin{itemize}
\pause
\item For SGD, this would be for example the learning rate $\alpha$
\pause
\item \alert{Note}: $\alpha$ have to be adapted in the course of the training
\begin{itemize}
\item similar to learning rate schedules (e.g., cosine annealing)
\end{itemize}
\pause
\item \alert{Note(ii)}: later we denote the learnt hyperparameters as $\lambda$
\medskip
\pause
\item \alert{Idea:} Use reinforcement learning to learn a policy $\pi: s \mapsto a$ to control the learning rate (or other adaptive hyperparameters)
\end{itemize}



\end{frame}
%----------------------------------------------------------------------
%----------------------------------------------------------------------
\begin{frame}[c]{Recap: Reinforcement Learning \lit{Barto \& Sutton; RL Book}}

\begin{center}
\includegraphics[width=0.6\textwidth]{images/suttonbarto_rl.png}
\end{center}

\pause

\begin{eqnarray}
\pi^* \in \argmax_{\pi} \mathbb{E}_{\pi} \left[ \sum^{T}_{t=1} \gamma^{t-1} r_t \mid s_0 \right] \nonumber
\end{eqnarray}

The goal is to find the optimal policy $\pi^{*}: s \mapsto a$ starting in a state $s_0$\\ (or in expectation of $s_0$) s.t. following $\pi^{*}$ from $s$ maximizes the expected (discounted) reward~$R$.


\end{frame}
%----------------------------------------------------------------------
%----------------------------------------------------------------------
\begin{frame}[c]{RL for Step Size Controllers: State \lit{Daniel et al'16}}

\textbf{Predictive change in function value:}

$$s_1 = \log \left( \text{Var}(\Delta \tilde{f}_i ) \right)$$
$$\Delta \tilde{f}_i = \tilde{f}(x_i; \theta + \delta \theta) - f(x_i; \theta)$$

where $\tilde{f}(x_i; \theta + \delta \theta)$ is done by a first order Taylor expansion

\pause
\textbf{Disagreement of function values:}
$$ s_2 = \log \left(\text{Var}(f(x_i; \theta)) \right)$$

\pause

\textbf{Discounted Average} (smoothing noise from mini-batches):
$$\hat{s}_i \leftarrow \gamma \hat{s_i} + (1 - \gamma) s_i$$

\pause

\textbf{Uncertainty Estimate} (noise level):
$$s_{K+i} \leftarrow \gamma s_{K+i} + (1-\gamma) (s_i - \hat{s}_i)^2$$


\end{frame}
%----------------------------------------------------------------------
%----------------------------------------------------------------------
\begin{frame}[c]{RL for Step Size Controllers: Learning \lit{Daniel et al'16}}

Reward (average loss improvement over time):

$$r = \frac{1}{T-1} \sum_{t=2}^T \left(\log(\loss_{t-1}) - \log(\loss_t)\right)$$

\pause

Optimal Policy:

$$\pi^*(\lambda \mid s) \in \argmax_{\pi} \int \int p(s) \pi(\lambda\mid s)r(\lambda,s) \texttt{d}s \texttt{d}\lambda $$

\pause

The policy $\pi$ is learnt as a controller $\lambda = g(s, \phi)$; leading to

$$\pi^*(\phi) \in \argmax_{\pi} \int \int p(s) \pi(\phi)r(g(s;\phi),s) \texttt{d}s \texttt{d}\lambda $$

\pause

\begin{itemize}
\item can be learnt for example via Relative Entropy Policy Search (REPS) \lit{Peter et al. 2010}
\end{itemize}

\end{frame}
%----------------------------------------------------------------------
%----------------------------------------------------------------------
\begin{frame}[c]{RL for Step Size Controllers: Training \lit{Daniel et al'16}}

\begin{itemize}
\item Goal: obtain robust policies,\\ i.e., good performance for many different DNN architectures
\begin{itemize}
\item[$\leadsto$] Sample architectures e.g., with different numbers of filters and layers
\item[$\leadsto$] (Sub-)Sample dataset
\item[$\leadsto$] Sample number of optimization steps
\end{itemize}
\end{itemize}

\pause 
\medskip
\centering
\includegraphics[width=0.6\textwidth]{images/l2stepsizecontroler_mnist_training.png}

"Ours" refers to the approach by \lit{Daniel et al'16} and $\eta$ is the learning rate

\end{frame}
%----------------------------------------------------------------------
%----------------------------------------------------------------------
\begin{frame}[c]{Learning Black-box Optimization~\lit{Chen et al'17}}

\huge
\centering
Learning to Learn without Gradient Descent by Gradient Descent\\
by Chen et al'17


\end{frame}
%----------------------------------------------------------------------
%----------------------------------------------------------------------
\begin{frame}[c]{Learning Black-box Optimization~\lit{Chen et al'17}}

\begin{block}{Black Box Optimization Setting}
\begin{enumerate}
\item Given the current state of knowledge $h_t$ propose a query point $x_t$
\item Observe the response $y_t$
\item Update any internal statistics to produce $h_{t+1}$
\end{enumerate}
\end{block}

\pause

\begin{block}{Learning Black Box Optimization}
Essentially, same idea as before:
\begin{eqnarray}
h_t, x_t = \text{RNN}_\phi(h_{t-1}, x_{t-1}, y_{t-1}) \nonumber \\
y_t \sim p(y|x_t)\nonumber
\end{eqnarray}

\begin{itemize}
\item Using recurrent neural network (RNN) to predict next $x_t$.
\item $h_t$ is the internal hidden state 
\pause
\item \alert{Remark:} in a black-box setting, we don't have gradient information!
\end{itemize}

\end{block}



\end{frame}
%----------------------------------------------------------------------
%----------------------------------------------------------------------
\begin{frame}[c]{Learning Black-box Optimization: Loss Functions\newline \lit{Chen et al'17}}

\begin{itemize}
\item Sum loss: Provides more information than final loss
\end{itemize}
\begin{equation}
\mathcal{L}_{\text{sum}}(\phi) = \mathbb{E}_{f,y_{1:T-1}}\left[\sum_{t=1}^T f(x_t)\right]\nonumber
\end{equation}

\pause

\begin{itemize}
\item EI loss: Try to learn behavior of Bayesian optimizer based on expected improvement (EI)
\begin{itemize}
\item requires model (e.g., GP)
\end{itemize}
\end{itemize}
\begin{equation}
\mathcal{L}_{\text{EI}}(\phi) = - \mathbb{E}_{f,y_{1:T-1}}\left[\sum_{t=1}^T \text{EI}(x_t | y_{1:t-1})\right]\nonumber
\end{equation}

\pause

\begin{itemize}
\item Observed Improvement Loss:
\end{itemize}

\begin{equation}
\mathcal{L}_{\text{OI}}(\phi) = \mathbb{E}_{f,y_{1:T-1}}\left[\sum_{t=1}^T \min \left\{f(x_t) - \min_{i<t}(f(x_i)),0 \right\}  \right]\nonumber
\end{equation}

\end{frame}
%----------------------------------------------------------------------
%----------------------------------------------------------------------
\begin{frame}[c]{Learning Black-box Optimization: Results\newline \lit{Chen et al'17}}

\centering
\includegraphics[width=0.6\textwidth]{images/l2bo_hartmann3}

\begin{itemize}
\item Hartmann3 is an artifical function with 3 dimensions
\pause
\item[$\leadsto$] $\mathcal{L}_{\text{OI}}$ and $\mathcal{L}_{\text{EI}}$ perform best
\item[$\leadsto$] $\mathcal{L}_{\text{OI}}$ easier to compute than $\mathcal{L}_{\text{EI}}$\\ because we need a predictive model to compute EI 
\end{itemize}

\end{frame}
%----------------------------------------------------------------------
%----------------------------------------------------------------------
\begin{frame}[c]{Learning to Optimize via Reinforcement Learning\newline \lit{Li and Malik'17}}

\centering\huge
Learning to Optimize\\
by Li and Malik '17

\end{frame}
%----------------------------------------------------------------------
%----------------------------------------------------------------------
\begin{frame}[c]{Learning to Optimize via Reinforcement Learning\newline \lit{Li and Malik'17}}

\centering
\includegraphics[width=0.6\textwidth]{images/l2o_comic}

\tiny
Source: \url{https://bair.berkeley.edu/blog/2017/09/12/learning-to-optimize-with-rl/}

\end{frame}
%----------------------------------------------------------------------
%----------------------------------------------------------------------
\begin{frame}[c]{Learning to Optimize via Reinforcement Learning\newline \lit{Li and Malik'17}}

\begin{block}{Reinforcement Learning for Learning to Optimize}
\begin{description}
\item[State] current location, objective values and gradients evaluated at the current and past locations
\pause
\item[Action] Step update $\Delta x$
\pause
\item[Transition] $x_t \leftarrow x_{t-1} + \Delta x$
\pause
\item[Cost/Reward] Objective value at the current location
\begin{itemize}
\item Since the RL agent will optimize the cumulative cost, this is equivalent to $\mathcal{L}_{\text{sum}}$
\item encourages the policy to reach the minimum of the objective function as quickly as possible
\end{itemize}
\pause
\item[Policy] DNN predicting $\mu_d$ of Gaussian (with constant variance $\sigma^2$)\\ for dimension $d$; sample $\Delta x_d \sim \mathcal{N}(\mu_d, \sigma^2)$
\pause
\item[Training Set] randomly generated objective functions
\end{description}
\end{block}

\end{frame}
%----------------------------------------------------------------------
%----------------------------------------------------------------------
\begin{frame}[c]{Learning to Optimize via Reinforcement Learning\newline Results \lit{Li and Malik'17}}

\centering
\includegraphics[width=0.5\textwidth]{images/l2o_dnn}

\begin{itemize}
\item 2-layer DNN with ReLUs
\item Training datasets for training RL agent:\\ four multivariate Gaussians and sampling 25 points from each
\begin{itemize}
\item[$\leadsto$] hard toy problem
\end{itemize}
\end{itemize}

\end{frame}
%----------------------------------------------------------------------
%----------------------------------------------------------------------
\begin{frame}[c, fragile]{Learning Acquisition Functions \lit{Volpp et al.'19}}

\centering\huge
Meta-Learning Acquisition Functions for Bayesian Optimization\\
by Volpp et al '19

\end{frame}
%----------------------------------------------------------------------
%----------------------------------------------------------------------
\begin{frame}[c, fragile]{Learning Acquisition Functions \lit{Volpp et al.'19}}

\begin{itemize}
\item Instead of learning everything, it might be sufficient to \alert{learn hand-design heuristics}
\pause
\item In Bayesian Optimization (BO), the most critical hand-design heuristic is the acquisition function
\begin{itemize}
\item trade-off between exploitation and exploration
\item Depending on the problem at hand, you might need a different acquisition function
\pause
\item Choices:
\begin{itemize}
\item probability of improvement (PI)
\item expected improvement (EI)
\item upper confidence bounds (UCB)
\item entropy search (ES) -- quite expensive!
\item knowledge gradient (KG)
\item ...
\end{itemize} 
\end{itemize}
\pause
\item \alert{Idea:} Learn a \emph{neural acquisition function} from data
\end{itemize}

$\leadsto$ Replace acquisition function 

\end{frame}
%----------------------------------------------------------------------
%-----------------------------------------------------------------------
\begin{frame}[c,fragile]{Bayesian Optimization: Algorithm}

\begin{algorithm}[H]
\Input{Search Space $\mathcal{X}$,
black box function $f$, 
\alert{acquisition function $\alpha$,}
maximal number of function evaluations $m$
}
\BlankLine
$\mathcal{D}_0$ $\leftarrow$ initial\_design($\mathcal{X}$); \\
\For{n = $1, 2, \ldots m - |D_0|$}{
$\surro: \conf \mapsto y$ $\leftarrow$ fit predictive model on $\mathcal{D}_{n-1}$;\\
select $x_{n}$ by optimizing $x_{n} \in \argmax_{x \in \mathcal{X}} \alert{\alpha(x; \mathcal{D}_{n-1}, \surro)}$;\\
Query $y_{n} := f(x_{n})$;\\
Add observation to data $D_{n} := D_{n-1} \cup \{\langle x_{n}, y_{n} \rangle \}$;
}
\Return{Best $x$ according to $D_m$ or $\surro$}
\caption{Bayesian Optimization (BO)}
\end{algorithm}


\end{frame}
%-----------------------------------------------------------------------
%-----------------------------------------------------------------------
\begin{frame}[c]{Neural Acquisition Function \lit{Volpp et al.'19}}

Although the \alert{acquisition function $\alpha$} depends on the history $\mathcal{D}_{n-1}$ and the predictive model $\surro$, $\alpha$ mainly makes use of the \alert{predictive mean $\mu$ and variance $\sigma^2$}.

\pause
\bigskip

Neural acquisition function (AF):

\begin{eqnarray}
\alpha_\theta(x) = \alpha_\theta(\mu_t(x), \sigma_t(x)) \nonumber
\end{eqnarray}

where $\theta$ are the parameters of a neural network,\\ and $\mu$ and $sigma$ are its inputs.

\pause 
\begin{itemize}
\item Since the input is not $x$, it allows to learn scalable acquisition function
\item No calibration of hyperparameter necessary, once the neural AF is learnt
\end{itemize}

\end{frame}
%-----------------------------------------------------------------------
%-----------------------------------------------------------------------
\begin{frame}[c]{RL to train Neural AF \lit{Volpp et al.'19}}

\begin{description}
\item[Policy $\pi_\theta$:] Neural acquisition function $\alpha_\theta$
\pause
\item[Episode:] run of $\pi$ on $f\in \mathcal{F}'$
\begin{itemize}
\item $\mathcal{F}$ is a set of functions we can sample functions from
\end{itemize}
\pause
\item[State $s_t$:] $\mu_t$ and $\sigma_t$ on a set of points $\xi_t$
\pause
\item[Action $a_t$:] Sampled point $x_t$
\pause
\item[Reward $r_t$:] negative simple regret: $r_t = f(x^*) - f(\hat{x})$
\begin{itemize}
\item assumes that we can estimate the optimal $x^*$ for \emph{training} functions
\end{itemize}
\pause
\item[Transition probability]: Noisy evaluation of $f$ and the predictive model update
\end{description}

\end{frame}
%-----------------------------------------------------------------------
%-----------------------------------------------------------------------
\begin{frame}[c]{State \lit{Volpp et al.'19}}

\begin{itemize}
\item The state is described by a discrete set of points $\xi_t = \{\xi_n\}^N_{n=1}$
\pause
\item We feed these points through the predictive model and the neural AF to obtain $\alpha_\theta(\xi_i) = \alpha_\theta(\mu_t(\xi_i), \sigma_t(\xi_i)) $
\pause
\item $\alpha_\theta(\xi_i)$ are interpreted as the logits of multinomial distribution, s.t.
$$\pi_\alpha(\cdot \mid s_t) = \text{Mult}\left[\alpha_\theta(\xi_1), \ldots, \alpha_\theta(\xi_N) \right] $$
\pause
\item Due to curse of dimensionality, we need a two step approach for~$\xi_t$
\begin{enumerate}
\item sample $\xi_{\text{global}}$ using a coarse Sobol grid
\item sample $\xi_{\text{local}}$ using local optimization starting from the best samples in $\xi_{\text{global}}$
\end{enumerate}
\item[$\leadsto$] $\xi_t = \xi_{\text{global}} \cup \xi_{\text{lokal}}$ 
\end{itemize}

\end{frame}
%-----------------------------------------------------------------------
%-----------------------------------------------------------------------
\begin{frame}[c,fragile]{Learning Acquisition Functions\newline Results \lit{Volpp et al.'19}}

\centering
\includegraphics[width=1.0\textwidth]{images/l2acq.png}


\end{frame}
%-----------------------------------------------------------------------
%-----------------------------------------------------------------------
\begin{frame}[c]{Results on Artificial Functions \lit{Volpp et al.'19}}

\includegraphics[width=1.0\textwidth]{images/l2acq_results.png}

\medskip

\begin{itemize}
\item Approach by \lit{Volpp et al. '19} called MetaBO
\item MetaBO performs better than other acquisition functions\\ (EI, GP-UBC, PI) and other baselines (Random, TAF)
\end{itemize}

\pause

\alert{Assumption}: You have a family of functions at hand\\ that resembles your target functions.

\end{frame}
%-----------------------------------------------------------------------


\end{document}
