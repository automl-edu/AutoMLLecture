
\pdfminorversion=4 % for acroread
\documentclass[aspectratio=169,t,xcolor={usenames,dvipsnames}]{beamer}
%\documentclass[t,handout,xcolor={usenames,dvipsnames}]{beamer}
\usepackage{../beamerstyle}
\usepackage{dsfont}
\usepackage{bm}
\usepackage[english]{babel}
\usepackage[utf8]{inputenc}
\usepackage{graphicx}
\usepackage{algorithm}
\usepackage[ruled,vlined,algo2e,linesnumbered]{algorithm2e}
%\usepackage[boxed,vlined]{algorithm2e}
\usepackage{hyperref}
\usepackage{booktabs}
\usepackage{mathtools}

\usepackage{amsmath,amssymb}
\usepackage{listings}
\lstset{frame=lines,framesep=3pt,numbers=left,numberblanklines=false,basicstyle=\ttfamily\small}

\usepackage{subfig}
\usepackage{multicol}
%\usepackage{appendixnumberbeamer}
%
\usepackage{tcolorbox}

\usepackage{pgfplots}
\usepackage{tikz}
\usetikzlibrary{trees} 
\usetikzlibrary{shapes.geometric}
\usetikzlibrary{positioning,shapes,shadows,arrows,calc,mindmap}
\usetikzlibrary{positioning,fadings,through}
\usetikzlibrary{decorations.pathreplacing}
\usetikzlibrary{intersections}
\usetikzlibrary{positioning,fit,calc,shadows,backgrounds}
\pgfdeclarelayer{background}
\pgfdeclarelayer{foreground}
\pgfsetlayers{background,main,foreground}
\tikzstyle{activity}=[rectangle, draw=black, rounded corners, text centered, text width=8em]
\tikzstyle{data}=[rectangle, draw=black, text centered, text width=8em]
\tikzstyle{myarrow}=[->, thick, draw=black]

% Define the layers to draw the diagram
\pgfdeclarelayer{background}
\pgfdeclarelayer{foreground}
\pgfsetlayers{background,main,foreground}

%\usepackage{listings}
%\lstset{numbers=left,
%  showstringspaces=false,
%  frame={tb},
%  captionpos=b,
%  lineskip=0pt,
%  basicstyle=\ttfamily,
%%  extendedchars=true,
%  stepnumber=1,
%  numberstyle=\small,
%  xleftmargin=1em,
%  breaklines
%}

 
\definecolor{blue}{RGB}{0, 74, 153}

\usetheme{Boadilla}
%\useinnertheme{rectangles}
\usecolortheme{whale}
\setbeamercolor{alerted text}{fg=blue}
\useoutertheme{infolines}
\setbeamertemplate{navigation symbols}{\vspace{-5pt}} % to lower the logo
\setbeamercolor{date in head/foot}{bg=blue} % blue
\setbeamercolor{date in head/foot}{fg=white}
\setbeamercolor{author in head/foot}{bg=blue} %blue
\setbeamercolor{title in head/foot}{bg=blue} % blue
\setbeamercolor{title}{fg=white, bg=blue}
\setbeamercolor{block title}{fg=white,bg=blue}
\setbeamercolor{block body}{bg=blue!10}
\setbeamercolor{frametitle}{fg=white, bg=blue}
\setbeamercovered{invisible}

\makeatletter
\setbeamertemplate{footline}
{
  \leavevmode%
  \hbox{%
  \begin{beamercolorbox}[wd=.333333\paperwidth,ht=2.25ex,dp=1ex,center]{author in head/foot}%
    \usebeamerfont{author in head/foot}\insertshortauthor
  \end{beamercolorbox}%
  \begin{beamercolorbox}[wd=.333333\paperwidth,ht=2.25ex,dp=1ex,center]{title in head/foot}%
    \usebeamerfont{title in head/foot}\insertshorttitle
  \end{beamercolorbox}%
  \begin{beamercolorbox}[wd=.333333\paperwidth,ht=2.25ex,dp=1ex,right]{date in head/foot}%
    \usebeamerfont{date in head/foot}Week \@week, Topic \@topicnumber, Slide \insertframenumber{}\hspace*{2em}
%    \insertframenumber\hspace*{2ex} 
  \end{beamercolorbox}}%
  \vskip0pt%
}

\newcommand{\@week}{0}
\newcommand{\@topicnumber}{0}
\newcommand{\week}[1]{\renewcommand{\@week}{#1}}
\newcommand{\topicnumber}[1]{\renewcommand{\@topicnumber}{#1}}

\makeatother

%\pgfdeclareimage[height=1.2cm]{automl}{images/logos/automl.png}
%\pgfdeclareimage[height=1.2cm]{freiburg}{images/logos/freiburg}

%\logo{\pgfuseimage{freiburg}}

\input{../latex_main/macros}






\title[AutoML: DAC]{AutoML: Dynamic Configuration \& Learning}
\subtitle{Dynamic Configuration}
\author[Marius Lindauer]{Bernd Bischl \and Frank Hutter \and Lars Kotthoff\newline \and \underline{Marius Lindauer} \and Joaquin Vanschoren}
\institute{}
\date{}



% \AtBeginSection[] % Do nothing for \section*
% {
%   \begin{frame}{Outline}
%     \bigskip
%     \vfill
%     \tableofcontents[currentsection]
%   \end{frame}
% }

\begin{document}
	
	\maketitle
	

%----------------------------------------------------------------------
%----------------------------------------------------------------------
\begin{frame}[c]{Iterative Optimization Heuristics}
	
	\begin{itemize}
		\item Many iterative heuristics in algorithms are dynamic and adaptive
		\begin{enumerate}
			\item the algorithm's behavior changes over time
			\item the algorithm's behavior changes based on internal statistics
		\end{enumerate}
		\medskip
		\pause
		\item These heuristics might control other hyperparameters of the algorithms
		\pause
		\smallskip
		\item Example: learning rate schedules for training DNNs
		\begin{enumerate}
			\item exponential decaying learning rate: based on number of iterations, learning rate decreases
			\pause
			\item Reduce learning rate on plateaus: if the learning stagnates for some time,\\ the learning rate is decreased by a factor
		\end{enumerate}
		\pause
		\smallskip
		\item other examples: restart probability of search, mutation rate of evolutionary algorithms, \ldots  
		
	\end{itemize}
	
\end{frame}
%----------------------------------------------------------------------
%----------------------------------------------------------------------
\begin{frame}[c]{Parametrization of Learning Rate Schedules}
	
	\begin{itemize}
		\item How can we parameterize learning rate schedules?
		\begin{enumerate}
			\item exponential decaying learning rate:
			\begin{itemize}
				\item initial learning rate
				\item minimal learning rate
				\item multiplicative factor
			\end{itemize}
			\pause
			\item Reduce learning rate on plateaus:
			\begin{itemize}
				\item patience (in number of epochs)
				\item patience threshold
				\item decreasing factor
				\item cool-down break (in number of epochs)
			\end{itemize}
		\end{enumerate}
		\pause
		\medskip
		\item[$\leadsto$] Many hyperparameters only to control a single hyperparameter
		\pause   
		\item Still not guaranteed that optimal setting of e.g. learning rate schedules will lead to optimal learning behavior
		\begin{itemize}
			\item Learning rate schedules are only heuristics
		\end{itemize}
	\end{itemize}
	
\end{frame}
%----------------------------------------------------------------------
%----------------------------------------------------------------------
\begin{frame}[c]{Dynamic Algorithm Configuration}

\begin{itemize}
	\item So far, we assumed that an algorithm runs with static settings
	\item However, settings, such as learning rate, have to be adapted over time
\end{itemize}

\begin{block}{Definition}
	Let 
	\begin{itemize}
		\item $\conf \in \pcs$ be a hyperparameter configuration of an algorithm $\algo$,
		\pause
		\item $p(\dataset)$ be a probability distribution over meta datasets $\dataset \in \datasets$,
		\pause
		\item $\stateRL^{(t)}$ be a state description of $\algo$ solving $\dataset$ at time point $t$,
		\pause
		\item $\cost: \policies \times \datasets \to \perf$ be a cost metric assessing the cost of a control policy $\pi \in \Pi$ on $\dataset \in \datasets$
	\end{itemize}
	
	\pause
	the \emph{dynamic algorithm configuration problem} is to obtain a policy $\policy^* : \stateRL_t \times \dataset \mapsto \conf$ by optimizing its cost across a distribution of datasets:
	\begin{equation}
	\policy^* \in \argmin_{\policy \in \policies} \int_{\datasets} p(\dataset) \cost(\policy, \dataset) \diff \dataset \nonumber
	\end{equation}
\end{block}

\end{frame}
%-----------------------------------------------------------------------	

%----------------------------------------------------------------------
\begin{frame}[c]{Dynamic Algorithm Configuration as Contextual MDP \litw{\href{https://ml.informatik.uni-freiburg.de/papers/20-ECAI-DAC.pdf}{Biedenkapp et al. 2020}}}


\begin{description}
	\item[State $\stateRL^{(t)}$] are described by statistics gathered in the algorithm run
	\pause
	\item[Action $\actionRL^{(t)}$] change hyperparameters according to some control policy $\pi$
	\pause
	\item[Transition] run the algorithm from state $\stateRL^{(t)}$ to $\stateRL^{(t+1)}$ for a "short" moment by using the hyperparameters defined by $a^{(t)}$
	\pause
	\item[Reward $\rewardRL^{(t)}$] Return your current solution quality (or an approximation)
	\pause
	\item[Context $\dataset$] A given dataset (or task)
\end{description}

\bigskip
	
\centering
\input{tikz/control.tex}
	
	
\end{frame}
%----------------------------------------------------------------------
%----------------------------------------------------------------------
\begin{frame}[c]{Solving Dynamic Algorithm Configuration}

Solve unknown MDP by using reinforcement learning (RL):

\begin{equation}
\mathcal{V}_\dataset^\policy(\stateRL^{(t)}) =  \mathbb{E} \left[\sum_{k=0}^\infty\gamma^k \rewardRL^{(t+k+1)}_\dataset| \stateRL^{(t)}=\stateRL \right]\nonumber
\end{equation}
\pause

\begin{equation}
 = \mathbb{E} \left[ \rewardRL^{(t+1)}_\dataset+\gamma\mathcal{V}_\dataset^\policy(\stateRL^{(t+1)})| \stateRL^{(t+1)} \sim \mathcal{T}_\dataset(\stateRL^{(t)}, \policy(\stateRL^{(t)})) \right] \nonumber
\end{equation}
\pause
\begin{equation}
\policy^* \in
\argmax_{\policy \in \policies}
\int_{\datasets} p(\dataset) \int_{\statesRL^{(0)}} \Pr(\stateRL^{(0)}) \cdot \mathcal{V}^\policy_\dataset(\stateRL^{(0)}) \diff \stateRL^{(0)} \diff \dataset \nonumber
\end{equation}

\bigskip
\pause
$\leadsto$ equivalent to Dynamic Algorithm Configuration definition


\end{frame}
%----------------------------------------------------------------------
%----------------------------------------------------------------------
\begin{frame}[c]{Dynamic Algorithm Configuration across Datasets \litw{\href{https://ml.informatik.uni-freiburg.de/papers/20-ECAI-DAC.pdf}{Biedenkapp et al. 2020}}}
	
\begin{itemize}
	\item Challenge: Evaluating a policy on all datasets is often not feasible
	\pause
	\item Curriculum learning \lit{\href{https://dl.acm.org/doi/10.1145/1553374.1553380}{Bengio et al. 2009}} showed that we should have a curriculum of tasks we tackle
	\pause
	\item Self-paced learning \lit{\href{https://papers.nips.cc/paper/2010/file/e57c6b956a6521b28495f2886ca0977a-Paper.pdf}{Kumar et al. 2010}} tries to automatically find such as a curriculum
	\begin{itemize}
		\item Focus on "easy" tasks where the agent can improve most:
	\end{itemize}
\end{itemize}
	
\pause
\begin{equation} 
\label{spl_loss}
\max_{\policy,\vec{v}}\mathcal{C}(\policy, \vec{v}, K) = \sum^{|\datasets|}_{i=1} \vec{v}_i\mathcal{R}_i(\policy) - \frac{1}{K} \sum^{|\datasets|}_{i=1} \vec{v}_i \nonumber
\end{equation}

with $\weights$ being the agent's policy parameters and\newline $\vec{v}$ being a masking vector for choosing the tasks at hand.

%\pause
%\medskip
%
%Iterative greedy optimization of $\vec{v}$:
%\begin{equation}
%\vec{v}_i = \left\{
%\begin{array}{ll}
%1, &  \mathrm{if}\quad\mathcal{C}(\policy, \vec{v}_{i}:=0, K) \leq \mathcal{C}(\policy, \vec{v}_{i}:=1, K)\\
%0, & \mathrm{otherwise}
%\end{array}
%\right.\nonumber
%\end{equation}
\end{frame}
%----------------------------------------------------------------------

\end{document}
