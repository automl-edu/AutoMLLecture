
\pdfminorversion=4 % for acroread
\documentclass[aspectratio=169,t,xcolor={usenames,dvipsnames}]{beamer}
%\documentclass[t,handout,xcolor={usenames,dvipsnames}]{beamer}
\usepackage{../beamerstyle}
\usepackage{dsfont}
\usepackage{bm}
\usepackage[english]{babel}
\usepackage[utf8]{inputenc}
\usepackage{graphicx}
\usepackage{algorithm}
\usepackage[ruled,vlined,algo2e,linesnumbered]{algorithm2e}
%\usepackage[boxed,vlined]{algorithm2e}
\usepackage{hyperref}
\usepackage{booktabs}
\usepackage{mathtools}

\usepackage{amsmath,amssymb}
\usepackage{listings}
\lstset{frame=lines,framesep=3pt,numbers=left,numberblanklines=false,basicstyle=\ttfamily\small}

\usepackage{subfig}
\usepackage{multicol}
%\usepackage{appendixnumberbeamer}
%
\usepackage{tcolorbox}

\usepackage{pgfplots}
\usepackage{tikz}
\usetikzlibrary{trees} 
\usetikzlibrary{shapes.geometric}
\usetikzlibrary{positioning,shapes,shadows,arrows,calc,mindmap}
\usetikzlibrary{positioning,fadings,through}
\usetikzlibrary{decorations.pathreplacing}
\usetikzlibrary{intersections}
\usetikzlibrary{positioning,fit,calc,shadows,backgrounds}
\pgfdeclarelayer{background}
\pgfdeclarelayer{foreground}
\pgfsetlayers{background,main,foreground}
\tikzstyle{activity}=[rectangle, draw=black, rounded corners, text centered, text width=8em]
\tikzstyle{data}=[rectangle, draw=black, text centered, text width=8em]
\tikzstyle{myarrow}=[->, thick, draw=black]

% Define the layers to draw the diagram
\pgfdeclarelayer{background}
\pgfdeclarelayer{foreground}
\pgfsetlayers{background,main,foreground}

%\usepackage{listings}
%\lstset{numbers=left,
%  showstringspaces=false,
%  frame={tb},
%  captionpos=b,
%  lineskip=0pt,
%  basicstyle=\ttfamily,
%%  extendedchars=true,
%  stepnumber=1,
%  numberstyle=\small,
%  xleftmargin=1em,
%  breaklines
%}

 
\definecolor{blue}{RGB}{0, 74, 153}

\usetheme{Boadilla}
%\useinnertheme{rectangles}
\usecolortheme{whale}
\setbeamercolor{alerted text}{fg=blue}
\useoutertheme{infolines}
\setbeamertemplate{navigation symbols}{\vspace{-5pt}} % to lower the logo
\setbeamercolor{date in head/foot}{bg=blue} % blue
\setbeamercolor{date in head/foot}{fg=white}
\setbeamercolor{author in head/foot}{bg=blue} %blue
\setbeamercolor{title in head/foot}{bg=blue} % blue
\setbeamercolor{title}{fg=white, bg=blue}
\setbeamercolor{block title}{fg=white,bg=blue}
\setbeamercolor{block body}{bg=blue!10}
\setbeamercolor{frametitle}{fg=white, bg=blue}
\setbeamercovered{invisible}

\makeatletter
\setbeamertemplate{footline}
{
  \leavevmode%
  \hbox{%
  \begin{beamercolorbox}[wd=.333333\paperwidth,ht=2.25ex,dp=1ex,center]{author in head/foot}%
    \usebeamerfont{author in head/foot}\insertshortauthor
  \end{beamercolorbox}%
  \begin{beamercolorbox}[wd=.333333\paperwidth,ht=2.25ex,dp=1ex,center]{title in head/foot}%
    \usebeamerfont{title in head/foot}\insertshorttitle
  \end{beamercolorbox}%
  \begin{beamercolorbox}[wd=.333333\paperwidth,ht=2.25ex,dp=1ex,right]{date in head/foot}%
    \usebeamerfont{date in head/foot}Week \@week, Topic \@topicnumber, Slide \insertframenumber{}\hspace*{2em}
%    \insertframenumber\hspace*{2ex} 
  \end{beamercolorbox}}%
  \vskip0pt%
}

\newcommand{\@week}{0}
\newcommand{\@topicnumber}{0}
\newcommand{\week}[1]{\renewcommand{\@week}{#1}}
\newcommand{\topicnumber}[1]{\renewcommand{\@topicnumber}{#1}}

\makeatother

%\pgfdeclareimage[height=1.2cm]{automl}{images/logos/automl.png}
%\pgfdeclareimage[height=1.2cm]{freiburg}{images/logos/freiburg}

%\logo{\pgfuseimage{freiburg}}

\input{../latex_main/macros}






\title[AutoML: Overview]{AutoML: Dynamic Configuration \& Learning}
\subtitle{Overview}
\author[Marius Lindauer]{Bernd Bischl \and Frank Hutter \and Lars Kotthoff\newline \and \underline{Marius Lindauer} \and Joaquin Vanschoren}
\institute{}
\date{}
\week{11}
\topicnumber{1}



% \AtBeginSection[] % Do nothing for \section*
% {
%   \begin{frame}{Outline}
%     \bigskip
%     \vfill
%     \tableofcontents[currentsection]
%   \end{frame}
% }

\begin{document}
	
	\maketitle
	

%----------------------------------------------------------------------
%----------------------------------------------------------------------
\begin{frame}[c]{Black vs. Grey vs. White Box}

\begin{itemize}
	\item Often we treat AutoML as a \alert{black-box problem}
	\begin{itemize}
		\item Black box: We choose input to the black box and observe outcome
		\pause
		\item E.g., classical hyperparameter optimizer:\\ Input: Hyperparameter configuration $\to$ Output: Accuracy
	\end{itemize}
   \pause
   \smallskip
   \item We discussed how to extend AutoML to a more \alert{grey-box approach}:	
   \begin{itemize}
     	\item Grey Box: We still choose the input, but we can observe more than the outcome,\\ e.g, intermediate results
     	\pause
     	\item We might can control the ``box'' a bit,  e.g., early termination
     	\pause
        \item E.g., learning curve predictions, multi-fidelity optimization, \ldots
        \pause
        \item[$\leadsto$] often more efficient than black-box approaches (if done right)
   \end{itemize}
   \pause
   \smallskip
   \item Ultimately, we would like to treat AutoML as a \alert{white-box problem}
   \begin{itemize}
     	\item White-box: We can observe and control all details of an algorithm run
   \end{itemize}
	\pause
	\medskip
   \item[$\leadsto$] Goal: \alert{Replace algorithm components by learned policies}
\end{itemize}


\end{frame}
%-----------------------------------------------------------------------
%----------------------------------------------------------------------
\begin{frame}[c]{Iterative Optimization Heuristics}

\begin{block}{IOHs}
	Iterative Optimization Heuristics (IOHs) propose a set of solution candidates in each iteration
	based on previous evaluations.
\end{block}

\pause
\smallskip
Important Observations:
\begin{itemize}
	\item Many ML algorithms are  \alert{iterative} in nature, in particular for big data, e.g.:
	\begin{itemize}
		\item SGD (for linear models or for deep neural networks)
		\item Tree-based algorithms
	\end{itemize}
	\pause
	\smallskip
	\item Often we have only a \alert{single solution candidate}\newline (e.g., weights of neural network)
	\begin{itemize}
		\item If we use a evoluationary strategy as in neural evoluation,\newline we have a population of solution candidates
	\end{itemize}
	\pause
	\smallskip
	\item Hopefully, the \alert{quality} of solution candiates \alert{improves} in each iteration
	\begin{itemize}
		\item Update of the weights of a neural network
	\end{itemize}
	\pause
	\smallskip
	\item Main component is the \alert{heuristic for proposal mechanism} of new solution candidates
\end{itemize}

\end{frame}
%-----------------------------------------------------------------------
%----------------------------------------------------------------------
\begin{frame}[c]{Learning for IOHs}

\begin{block}{Dynamic Adaptation of Hyperparameters}
	The goal is to dynamically adapt hyperparameters based on some feedback from the algorithm.
\end{block}

\bigskip
\pause	

\begin{block}{Dynamic Algorithm Configuration: DAC}
	The goal of DAC is to learn a policy from data\\ that adapts the \alert{hyperparameter settings} of an IOH.
\end{block}

\bigskip
\pause

\begin{block}{Learning to Learn: L2L}
	The goal of L2L is to learn a \alert{proposal mechanism} from data.
\end{block}
	
\end{frame}
%-----------------------------------------------------------------------
%----------------------------------------------------------------------
%\begin{frame}[c]{Main Idea \& Challenges}
%
%\begin{enumerate}
%	\item \alert{Training data}: Run algorithm over and over again
%	\pause
%	\smallskip
%	\item \alert{Learn} algorithm component from offline training data
%	\pause
%	\smallskip
%	\item Apply to many new problems (a.k.a. instances, tasks, datasets) and assess the \alert{generalization} performance 
%\end{enumerate}
%	
%	
%\end{frame}
%-----------------------------------------------------------------------

\end{document}
