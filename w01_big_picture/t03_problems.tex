% !TeX spellcheck = en_US

\pdfminorversion=4 % for acroread
\documentclass[aspectratio=169,t,xcolor={usenames,dvipsnames}]{beamer}
%\documentclass[t,handout,xcolor={usenames,dvipsnames}]{beamer}
\usepackage{../beamerstyle}
\usepackage{dsfont}
\usepackage{bm}
\usepackage[english]{babel}
\usepackage[utf8]{inputenc}
\usepackage{graphicx}
\usepackage{algorithm}
\usepackage[ruled,vlined,algo2e,linesnumbered]{algorithm2e}
%\usepackage[boxed,vlined]{algorithm2e}
\usepackage{hyperref}
\usepackage{booktabs}
\usepackage{mathtools}

\usepackage{amsmath,amssymb}
\usepackage{listings}
\lstset{frame=lines,framesep=3pt,numbers=left,numberblanklines=false,basicstyle=\ttfamily\small}

\usepackage{subfig}
\usepackage{multicol}
%\usepackage{appendixnumberbeamer}
%
\usepackage{tcolorbox}

\usepackage{pgfplots}
\usepackage{tikz}
\usetikzlibrary{trees} 
\usetikzlibrary{shapes.geometric}
\usetikzlibrary{positioning,shapes,shadows,arrows,calc,mindmap}
\usetikzlibrary{positioning,fadings,through}
\usetikzlibrary{decorations.pathreplacing}
\usetikzlibrary{intersections}
\usetikzlibrary{positioning,fit,calc,shadows,backgrounds}
\pgfdeclarelayer{background}
\pgfdeclarelayer{foreground}
\pgfsetlayers{background,main,foreground}
\tikzstyle{activity}=[rectangle, draw=black, rounded corners, text centered, text width=8em]
\tikzstyle{data}=[rectangle, draw=black, text centered, text width=8em]
\tikzstyle{myarrow}=[->, thick, draw=black]

% Define the layers to draw the diagram
\pgfdeclarelayer{background}
\pgfdeclarelayer{foreground}
\pgfsetlayers{background,main,foreground}

%\usepackage{listings}
%\lstset{numbers=left,
%  showstringspaces=false,
%  frame={tb},
%  captionpos=b,
%  lineskip=0pt,
%  basicstyle=\ttfamily,
%%  extendedchars=true,
%  stepnumber=1,
%  numberstyle=\small,
%  xleftmargin=1em,
%  breaklines
%}

 
\definecolor{blue}{RGB}{0, 74, 153}

\usetheme{Boadilla}
%\useinnertheme{rectangles}
\usecolortheme{whale}
\setbeamercolor{alerted text}{fg=blue}
\useoutertheme{infolines}
\setbeamertemplate{navigation symbols}{\vspace{-5pt}} % to lower the logo
\setbeamercolor{date in head/foot}{bg=blue} % blue
\setbeamercolor{date in head/foot}{fg=white}
\setbeamercolor{author in head/foot}{bg=blue} %blue
\setbeamercolor{title in head/foot}{bg=blue} % blue
\setbeamercolor{title}{fg=white, bg=blue}
\setbeamercolor{block title}{fg=white,bg=blue}
\setbeamercolor{block body}{bg=blue!10}
\setbeamercolor{frametitle}{fg=white, bg=blue}
\setbeamercovered{invisible}

\makeatletter
\setbeamertemplate{footline}
{
  \leavevmode%
  \hbox{%
  \begin{beamercolorbox}[wd=.333333\paperwidth,ht=2.25ex,dp=1ex,center]{author in head/foot}%
    \usebeamerfont{author in head/foot}\insertshortauthor
  \end{beamercolorbox}%
  \begin{beamercolorbox}[wd=.333333\paperwidth,ht=2.25ex,dp=1ex,center]{title in head/foot}%
    \usebeamerfont{title in head/foot}\insertshorttitle
  \end{beamercolorbox}%
  \begin{beamercolorbox}[wd=.333333\paperwidth,ht=2.25ex,dp=1ex,right]{date in head/foot}%
    \usebeamerfont{date in head/foot}Week \@week, Topic \@topicnumber, Slide \insertframenumber{}\hspace*{2em}
%    \insertframenumber\hspace*{2ex} 
  \end{beamercolorbox}}%
  \vskip0pt%
}

\newcommand{\@week}{0}
\newcommand{\@topicnumber}{0}
\newcommand{\week}[1]{\renewcommand{\@week}{#1}}
\newcommand{\topicnumber}[1]{\renewcommand{\@topicnumber}{#1}}

\makeatother

%\pgfdeclareimage[height=1.2cm]{automl}{images/logos/automl.png}
%\pgfdeclareimage[height=1.2cm]{freiburg}{images/logos/freiburg}

%\logo{\pgfuseimage{freiburg}}

\input{../latex_main/macros}






\title[AutoML: Problems]{AutoML: Introduction}
\subtitle{Overview of AutoML Problems}
\author[Marius Lindauer]{Bernd Bischl \and Frank Hutter \and Lars Kotthoff\newline \and \underline{Marius Lindauer} \and Joaquin Vanschoren}
\institute{}
\date{}
\week{1}
\topicnumber{3}



% \AtBeginSection[] % Do nothing for \section*
% {
%   \begin{frame}{Outline}
%     \bigskip
%     \vfill
%     \tableofcontents[currentsection]
%   \end{frame}
% }

\begin{document}
	
	\maketitle
	

%----------------------------------------------------------------------
\begin{frame}[c]{Objectives of AutoML}

Different metrics to measure the success of ML and AutoML:

\begin{itemize}
	\item Accuracy on a validation dataset
	\pause
	\item AUC-ROC
	\pause
	\item precision \& recall
	\item \ldots
	\pause
	\item Memory consumption
	\item Inference time
	\item \ldots
\end{itemize}

\pause
\medskip

\begin{itemize}
	\item[$\leadsto$] AutoML optimizes an arbitrary cost function, denoted as $\cost$.
	\pause
	\item[$\leadsto$] $\cost(\cdot)$ can also return several cost metrics
\end{itemize}


\end{frame}
%-----------------------------------------------------------------------	
%----------------------------------------------------------------------
\begin{frame}[c]{Hyperparameters of an SVM}

\centering
\includegraphics[width=0.7\textwidth]{images/sklearn_svm_doc.png}

\end{frame}
%-----------------------------------------------------------------------	
%----------------------------------------------------------------------
\begin{frame}[c]{Hyperparameter Optimization}

\begin{block}{Definition}
	Let 
	\begin{itemize}
		\item $\conf$ be the hyperparameters of an ML algorithm $\algo$ with domain $\pcs$,
		\pause
		\item $\dataset_{opt}$ be a dataset which is split into $\datasettrain$ and $\datasetval$ 
		\pause
		\item $\cost(\algo_{\conf}, \dataset_{train}, \dataset_{valid})$ denote the cost of $\algo_{\conf}$ trained on $\datasettrain$ and evaluated on $\datasetval$.
	\end{itemize}
	\pause
	The \emph{hyper-parameter optimization (HPO)} problem is to find a hyper-parameter configuration that minimizes this cost:
	\begin{equation}
	\optconf \in \argmin_{\conf \in \pcs} \cost(\algo_{\conf}, \dataset_{train}, \dataset_{valid}) \nonumber  
	\end{equation}
\end{block}

\pause
Remarks: 

\begin{itemize}
	\item $\argmin$ returns a set of optimal points of a given function. It suffices to find one element of this set and thus we use $\in$ instead of $=$.
	\pause
	\item Sometimes, we want to optimize for different metrics, instead of one
	\begin{itemize}
		\item[$\leadsto$] multi-objective optimization and Pareto fronts
	\end{itemize}
\end{itemize}

\end{frame}
%-----------------------------------------------------------------------
%----------------------------------------------------------------------
\begin{frame}[c]{Choosing an Algorithm}

\begin{itemize}
	\item Over the years, many ML-algorithms were proposed
	\pause
	\item Most of these still have a reason for existence
	\pause
	\item Examples for classification:
	\begin{itemize}
		\item logistic regression
		\item k-nearest neighbor
		\item na\"ive Bayes
		\item support vector machine
		\item decision tree
		\item random forest
		\item gradient boosting
		\item multi-layer perceptron
		\item residual networks
		\item ...
	\end{itemize}
	\pause
	\item studied $179$ classifiers on $121$ datasets   	     \lit{\href{http://jmlr.org/papers/volume15/		  	 	delgado14a/delgado14a.pdf}{Fern\'andez-Delgado 		        et al. 2015}}
	\pause
	\item In practice, we actually want to jointly choose\newline the best ML-algorithm and its hyperparameters
\end{itemize}


\end{frame}
%-----------------------------------------------------------------------
%----------------------------------------------------------------------
\begin{frame}[c]{CASH: Combined Algorithm Selection and\newline Hyperparameter Optimization}

\begin{block}{Definition}
	Let
	\begin{itemize}
		\item \alert{$\algos = \{\algo_1, \algo_2, \ldots, \algo_k\}$ be a set of algorithms (a.k.a. portfolio)}
		\item $\pcs$ be a set of hyperparameters of each machine learning algorithm $\algo_i$
		\item $\dataset_{opt}$ be a dataset which is split into $\dataset_{train}$ and $\dataset_{valid}$ 
		\item $\cost(\algo_{\conf}, \dataset_{train}, \dataset_{valid})$ denote the cost of $\algo_{\conf}$ trained on $\dataset_{train}$ and evaluated on $\dataset_{valid}$.
	\end{itemize}
	we want to find \alert{the best combination of algorithm $\algo \in \mathbf{A}$ and its hyperparameter configuration $\conf \in \pcs$} minimizing:
	\begin{equation}
	(\algo^*, \optconf) \in \argmin_{\algo \in \algos, \conf \in \pcs} \cost(\algo_{\conf}, \dataset_{train}, \dataset_{valid}) \nonumber
	\end{equation}
\end{block}

\end{frame}
%-----------------------------------------------------------------------	
%----------------------------------------------------------------------
\begin{frame}[c]{Architectures of Neural Networks}

\begin{columns}
	\column{0.5\textwidth}
	
	\begin{itemize}
		\item Many architecture were proposed
		\item Differences in 
		\begin{itemize}
			\item Depth
			\item Resolution
			\item Width
			\item Operators
			\item Connections
			\item ...
		\end{itemize}
	\end{itemize}
	
	\column{0.5\textwidth}
	
	\centering
	\includegraphics[width=0.7\textwidth]{images/overview_achitecture_perf.PNG}\newline
	on Imagenet~\lit{\href{https://arxiv.org/pdf/1605.07678.pdf}{Canzian et al. 2017}}
	
\end{columns}

\pause
\begin{itemize}
	\item Already on a single dataset such as ImageNet,\\ it is not obvious which architeture to choose
	\pause
	\item For other datasets, you might need different architectures\\ to achieve top-performance
	
	\pause
	\begin{itemize}
		\item For similar datasets, you might use scaled versions of known architectures\\ (e.g., CIFAR10 and Imagenet)
	\end{itemize}
\end{itemize}

\end{frame}
%-----------------------------------------------------------------------	
%----------------------------------------------------------------------
\begin{frame}[c]{Neural Architecture Search (NAS)}

\begin{block}{Definition}
	Let 
	\begin{itemize}
		\item $\conf$ be \alert{an architecture for a deep neural network $N$} with domain $\pcs$,
		\item $\dataset_{opt}$ be a dataset which is split into $\dataset_{train}$ and $\dataset_{valid}$ 
		\item $\cost(N_{\conf}, \dataset_{train}, \dataset_{valid})$ denote the cost of $N_{\conf}$ trained on $\dataset_{train}$ and evaluated on $\dataset_{valid}$.
	\end{itemize}
	The \emph{neural architecture search (NAS)} problem is to find an architecture that minimizes this cost:
	\begin{equation}
	\conf^* \in \argmin_{\conf \in \pcs} \cost(N_{\conf}, \dataset_{train}, \dataset_{valid}) \nonumber  
	\end{equation}
\end{block}

\pause
\smallskip
Remark:
\begin{itemize}
	\item very similar to the HPO definition
	\pause
	\item In practice, you want jointly optimize HPO and NAS~\lit{\href{https://ml.informatik.uni-freiburg.de/papers/18-AUTOML-EfficientNAS.pdf}{Zela et al. 2018}}
\end{itemize}

\end{frame}
%-----------------------------------------------------------------------	
%----------------------------------------------------------------------
\begin{frame}[c]{Per-Instance Algorithm Selection}

\begin{itemize}
	\item For solving the HPO, CASH and NAS problem, we use some kind of optimization approach to \alert{search} for a solution
	\pause
	\item A different view: Can we learn from data and thus \alert{predict}\\ an algorithm/ hyperparameter configuration/ neural architecture?
\end{itemize}

\pause
\begin{block}{Definition}
	Let 
	\begin{itemize}
		\item $p(\dataset)$ be a probability \alert{distribution} over datasets $\dataset \in \datasets$,
		\pause
		\item $\portfolio$ a portfolio of algorithms $\algo \in \portfolio$, and
		\pause
		\item $\cost: \portfolio \times \datasets \rightarrow \perf$ be a cost metric   
	\end{itemize}
	
	\pause
	the \emph{per-instance algorithm selection problem} is to obtain a mapping 
	$s: \dataset \mapsto \algo$ 
	such that
	$$\argmin_{s} \int_{\datasets} \cost(s(\dataset),\dataset) p(\dataset) \diff\dataset$$
\end{block}

\end{frame}
%-----------------------------------------------------------------------	
%----------------------------------------------------------------------
\begin{frame}[c]{Dynamic Algorithm Configuration}

\begin{itemize}
	\item So far, we assume that an algorithm runs with some given settings
	\pause
	\item However, some settings, such as learning rate, have to be adapted over time
\end{itemize}

\pause

\begin{block}{Definition}
	Let 
	\begin{itemize}
		\item $\conf$ be a hyperparameter configuration of an algorithm $\algo$,
		\pause
		\item $p(\dataset)$ be a probability distribution over datasets $\dataset \in \datasets$,
		\pause
		\item $\stateRL_t$ be a state description of $\algo$ solving $\dataset$ at time point $t$,
		\pause
		\item $\cost: \policies \times \datasets \to \perf$ be a cost metric assessing the \alert{cost of a conf. policy $\pi \in \Pi$} on $\dataset \in \datasets$
	\end{itemize}
	
	\pause
	the \emph{dynamic algorithm configuration problem (DAC)} is to obtain a configuration policy $\policy^* : \stateRL_t \times \dataset \mapsto \conf$ by optimizing its cost across a distribution of datasets:
	\begin{equation}
	\pi^* \in \argmin_{\policy \in \policies} \int_{\datasets} p(\dataset) c(\policy, \dataset) \diff \dataset \nonumber
	\end{equation}
\end{block}

\end{frame}
%-----------------------------------------------------------------------	
%-----------------------------------------------------------------------	
%----------------------------------------------------------------------
\begin{frame}[c]{Summary}

\begin{description}
	\item[HPO] Search for the best hyperparameter configuration of a ML algorithm
	\item[CASH] Search for the best combination of algorithm and hyperparameter configuration
	\item[NAS] Search for the architecture of neural network
	\bigskip
	\item[Selection] Predict the best algorithm (and its hyperparameter configuration)
	\item[DAC] Predict the best hyperparameter configuration for an algorithm state\\ at a given time point 
\end{description}

\end{frame}
%-----------------------------------------------------------------------	
\end{document}
