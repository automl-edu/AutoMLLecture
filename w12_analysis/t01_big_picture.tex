% !TeX spellcheck = en_US

\pdfminorversion=4 % for acroread
\documentclass[aspectratio=169,t,xcolor={usenames,dvipsnames}]{beamer}
%\documentclass[t,handout,xcolor={usenames,dvipsnames}]{beamer}
\usepackage{../beamerstyle}
\usepackage{dsfont}
\usepackage{bm}
\usepackage[english]{babel}
\usepackage[utf8]{inputenc}
\usepackage{graphicx}
\usepackage{algorithm}
\usepackage[ruled,vlined,algo2e,linesnumbered]{algorithm2e}
%\usepackage[boxed,vlined]{algorithm2e}
\usepackage{hyperref}
\usepackage{booktabs}
\usepackage{mathtools}

\usepackage{amsmath,amssymb}
\usepackage{listings}
\lstset{frame=lines,framesep=3pt,numbers=left,numberblanklines=false,basicstyle=\ttfamily\small}

\usepackage{subfig}
\usepackage{multicol}
%\usepackage{appendixnumberbeamer}
%
\usepackage{tcolorbox}

\usepackage{pgfplots}
\usepackage{tikz}
\usetikzlibrary{trees} 
\usetikzlibrary{shapes.geometric}
\usetikzlibrary{positioning,shapes,shadows,arrows,calc,mindmap}
\usetikzlibrary{positioning,fadings,through}
\usetikzlibrary{decorations.pathreplacing}
\usetikzlibrary{intersections}
\usetikzlibrary{positioning,fit,calc,shadows,backgrounds}
\pgfdeclarelayer{background}
\pgfdeclarelayer{foreground}
\pgfsetlayers{background,main,foreground}
\tikzstyle{activity}=[rectangle, draw=black, rounded corners, text centered, text width=8em]
\tikzstyle{data}=[rectangle, draw=black, text centered, text width=8em]
\tikzstyle{myarrow}=[->, thick, draw=black]

% Define the layers to draw the diagram
\pgfdeclarelayer{background}
\pgfdeclarelayer{foreground}
\pgfsetlayers{background,main,foreground}

%\usepackage{listings}
%\lstset{numbers=left,
%  showstringspaces=false,
%  frame={tb},
%  captionpos=b,
%  lineskip=0pt,
%  basicstyle=\ttfamily,
%%  extendedchars=true,
%  stepnumber=1,
%  numberstyle=\small,
%  xleftmargin=1em,
%  breaklines
%}

 
\definecolor{blue}{RGB}{0, 74, 153}

\usetheme{Boadilla}
%\useinnertheme{rectangles}
\usecolortheme{whale}
\setbeamercolor{alerted text}{fg=blue}
\useoutertheme{infolines}
\setbeamertemplate{navigation symbols}{\vspace{-5pt}} % to lower the logo
\setbeamercolor{date in head/foot}{bg=blue} % blue
\setbeamercolor{date in head/foot}{fg=white}
\setbeamercolor{author in head/foot}{bg=blue} %blue
\setbeamercolor{title in head/foot}{bg=blue} % blue
\setbeamercolor{title}{fg=white, bg=blue}
\setbeamercolor{block title}{fg=white,bg=blue}
\setbeamercolor{block body}{bg=blue!10}
\setbeamercolor{frametitle}{fg=white, bg=blue}
\setbeamercovered{invisible}

\makeatletter
\setbeamertemplate{footline}
{
  \leavevmode%
  \hbox{%
  \begin{beamercolorbox}[wd=.333333\paperwidth,ht=2.25ex,dp=1ex,center]{author in head/foot}%
    \usebeamerfont{author in head/foot}\insertshortauthor
  \end{beamercolorbox}%
  \begin{beamercolorbox}[wd=.333333\paperwidth,ht=2.25ex,dp=1ex,center]{title in head/foot}%
    \usebeamerfont{title in head/foot}\insertshorttitle
  \end{beamercolorbox}%
  \begin{beamercolorbox}[wd=.333333\paperwidth,ht=2.25ex,dp=1ex,right]{date in head/foot}%
    \usebeamerfont{date in head/foot}Week \@week, Topic \@topicnumber, Slide \insertframenumber{}\hspace*{2em}
%    \insertframenumber\hspace*{2ex} 
  \end{beamercolorbox}}%
  \vskip0pt%
}

\newcommand{\@week}{0}
\newcommand{\@topicnumber}{0}
\newcommand{\week}[1]{\renewcommand{\@week}{#1}}
\newcommand{\topicnumber}[1]{\renewcommand{\@topicnumber}{#1}}

\makeatother

%\pgfdeclareimage[height=1.2cm]{automl}{images/logos/automl.png}
%\pgfdeclareimage[height=1.2cm]{freiburg}{images/logos/freiburg}

%\logo{\pgfuseimage{freiburg}}

\input{../latex_main/macros}




\title[AutoML: Interpretability]{AutoML: Interpretability}
\subtitle{Overview: Automated Empirical Analysis}
\author[Marius Lindauer]{Bernd Bischl \and Frank Hutter \and Lars Kotthoff\newline \and \underline{Marius Lindauer} \and Joaquin Vanschoren}
\institute{}
\date{}



% \AtBeginSection[] % Do nothing for \section*
% {
%   \begin{frame}{Outline}
%     \bigskip
%     \vfill
%     \tableofcontents[currentsection]
%   \end{frame}
% }

\begin{document}
	
	\maketitle
	

%----------------------------------------------------------------------
%----------------------------------------------------------------------
\begin{frame}[c]{Idea}


\begin{itemize}
	\item Big challenge of ML: Interpretability
	\begin{itemize}
		\item In some applications, it is required to ''understand`` a prediction 
		\item Users have less trust in systems, they can't understand
	\end{itemize}

	\bigskip
	\pause
	\item AutoML is even worse?
	\begin{itemize}
		\item AutoML  is a black-box that automates the design  of another blackbox (ML)
		\item Also ML-developers have a basic understanding of the design of their ML pipelines
	\end{itemize}

	\bigskip
	\item Automated empirical interpretability helps to
	\begin{itemize}
		\item understand the finally returned ML system
		\item understand the AutoML process
	\end{itemize}

\end{itemize}

\end{frame}
%-----------------------------------------------------------------------

%----------------------------------------------------------------------
\begin{frame}[c]{Approach}


\begin{itemize}
	\item \alert{Insights:}
	\begin{itemize}
		\item AutoML is yet another optimization problem
		\item (Most) AutoML approach are iterative in nature 
    \end{itemize}
	\item[$\leadsto$] AutoML generates a lot of empirical data 
\end{itemize}

\begin{center}
\scalebox{0.9}{
	\input{tikz/automl_analysis_overview.tex}
}
\end{center}

\only<4->
{$\leadsto$ Let's use this data to learn something about our AutoML problem}

\end{frame}
%-----------------------------------------------------------------------
%----------------------------------------------------------------------
\begin{frame}[c]{Basic Examples}

\begin{itemize}
	\item Visualize final incumbent $\finconf$
	\begin{itemize}
		\item ML pipeline with its components
		\item Neural architecture
	\end{itemize}
	\medskip
	\pause
	\item Compare what changed between $\defconf$ and $\finconf$
	\medskip
	\pause		
	\item Show $\finconf$ on different budgets (if you used a multi-fideltiy approach)
\end{itemize}

\end{frame}
%-----------------------------------------------------------------------
%----------------------------------------------------------------------
\begin{frame}[c]{Cost Over Time}

\begin{columns}

\column{0.4\textwidth}
\begin{center}
	\includegraphics[width=1.0\textwidth]{images/overtime_plot.png}\\
	Source: \lit{\href{https://arxiv.org/pdf/1908.06756.pdf}{Lindauer et al. 2019}}
\end{center}

\column{0.6\textwidth}

\begin{itemize}
	\item Study how your AutoML tool improves\\ cost (or loss) over time
	\pause
	\item Allows to identify whether 
	\begin{itemize}
		\item you need less time next time or 
		\pause
		\item the AutoML system is still improving; so you should give it more time
	\end{itemize}
	\pause
	\item \alert{Notes}: 
	\begin{itemize}
		\item Plot on log-scale to see details in the beginning
		\item  If you have done several runs, plot distribution\\ (e.g., median and $25/75\%$-quartiles)
	\end{itemize}
\end{itemize}
	
\end{columns}

\end{frame}
%-----------------------------------------------------------------------
%-----------------------------------------------------------------------
\end{document}
