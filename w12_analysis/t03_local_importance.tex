% !TeX spellcheck = en_US

\pdfminorversion=4 % for acroread
\documentclass[aspectratio=169,t,xcolor={usenames,dvipsnames}]{beamer}
%\documentclass[t,handout,xcolor={usenames,dvipsnames}]{beamer}
\usepackage{../beamerstyle}
\usepackage{dsfont}
\usepackage{bm}
\usepackage[english]{babel}
\usepackage[utf8]{inputenc}
\usepackage{graphicx}
\usepackage{algorithm}
\usepackage[ruled,vlined,algo2e,linesnumbered]{algorithm2e}
%\usepackage[boxed,vlined]{algorithm2e}
\usepackage{hyperref}
\usepackage{booktabs}
\usepackage{mathtools}

\usepackage{amsmath,amssymb}
\usepackage{listings}
\lstset{frame=lines,framesep=3pt,numbers=left,numberblanklines=false,basicstyle=\ttfamily\small}

\usepackage{subfig}
\usepackage{multicol}
%\usepackage{appendixnumberbeamer}
%
\usepackage{tcolorbox}

\usepackage{pgfplots}
\usepackage{tikz}
\usetikzlibrary{trees} 
\usetikzlibrary{shapes.geometric}
\usetikzlibrary{positioning,shapes,shadows,arrows,calc,mindmap}
\usetikzlibrary{positioning,fadings,through}
\usetikzlibrary{decorations.pathreplacing}
\usetikzlibrary{intersections}
\usetikzlibrary{positioning,fit,calc,shadows,backgrounds}
\pgfdeclarelayer{background}
\pgfdeclarelayer{foreground}
\pgfsetlayers{background,main,foreground}
\tikzstyle{activity}=[rectangle, draw=black, rounded corners, text centered, text width=8em]
\tikzstyle{data}=[rectangle, draw=black, text centered, text width=8em]
\tikzstyle{myarrow}=[->, thick, draw=black]

% Define the layers to draw the diagram
\pgfdeclarelayer{background}
\pgfdeclarelayer{foreground}
\pgfsetlayers{background,main,foreground}

%\usepackage{listings}
%\lstset{numbers=left,
%  showstringspaces=false,
%  frame={tb},
%  captionpos=b,
%  lineskip=0pt,
%  basicstyle=\ttfamily,
%%  extendedchars=true,
%  stepnumber=1,
%  numberstyle=\small,
%  xleftmargin=1em,
%  breaklines
%}

 
\definecolor{blue}{RGB}{0, 74, 153}

\usetheme{Boadilla}
%\useinnertheme{rectangles}
\usecolortheme{whale}
\setbeamercolor{alerted text}{fg=blue}
\useoutertheme{infolines}
\setbeamertemplate{navigation symbols}{\vspace{-5pt}} % to lower the logo
\setbeamercolor{date in head/foot}{bg=blue} % blue
\setbeamercolor{date in head/foot}{fg=white}
\setbeamercolor{author in head/foot}{bg=blue} %blue
\setbeamercolor{title in head/foot}{bg=blue} % blue
\setbeamercolor{title}{fg=white, bg=blue}
\setbeamercolor{block title}{fg=white,bg=blue}
\setbeamercolor{block body}{bg=blue!10}
\setbeamercolor{frametitle}{fg=white, bg=blue}
\setbeamercovered{invisible}

\makeatletter
\setbeamertemplate{footline}
{
  \leavevmode%
  \hbox{%
  \begin{beamercolorbox}[wd=.333333\paperwidth,ht=2.25ex,dp=1ex,center]{author in head/foot}%
    \usebeamerfont{author in head/foot}\insertshortauthor
  \end{beamercolorbox}%
  \begin{beamercolorbox}[wd=.333333\paperwidth,ht=2.25ex,dp=1ex,center]{title in head/foot}%
    \usebeamerfont{title in head/foot}\insertshorttitle
  \end{beamercolorbox}%
  \begin{beamercolorbox}[wd=.333333\paperwidth,ht=2.25ex,dp=1ex,right]{date in head/foot}%
    \usebeamerfont{date in head/foot}Week \@week, Topic \@topicnumber, Slide \insertframenumber{}\hspace*{2em}
%    \insertframenumber\hspace*{2ex} 
  \end{beamercolorbox}}%
  \vskip0pt%
}

\newcommand{\@week}{0}
\newcommand{\@topicnumber}{0}
\newcommand{\week}[1]{\renewcommand{\@week}{#1}}
\newcommand{\topicnumber}[1]{\renewcommand{\@topicnumber}{#1}}

\makeatother

%\pgfdeclareimage[height=1.2cm]{automl}{images/logos/automl.png}
%\pgfdeclareimage[height=1.2cm]{freiburg}{images/logos/freiburg}

%\logo{\pgfuseimage{freiburg}}

\input{../latex_main/macros}




\title[AutoML: Local Importance]{AutoML: Interpretability}
\subtitle{Incumbent Analysis and Local Hyperparameter Importance}
\author[Marius Lindauer]{Bernd Bischl \and Frank Hutter \and Lars Kotthoff\newline \and \underline{Marius Lindauer} \and Joaquin Vanschoren}
\institute{}
\date{}


% \AtBeginSection[] % Do nothing for \section*
% {
%   \begin{frame}{Outline}
%     \bigskip
%     \vfill
%     \tableofcontents[currentsection]
%   \end{frame}
% }

\begin{document}
	
	\maketitle
	

%----------------------------------------------------------------------
%----------------------------------------------------------------------
\begin{frame}[c,fragile]{Idea}

\begin{center}
\scalebox{0.9}{
    \let\oldpause=\pause \def\pause{} 
	\input{tikz/automl_analysis_overview.tex}
	\let\pause=\oldpause
}
\end{center}

\begin{itemize}
	\item[$\leadsto$] focus on why is the eventually returned configuration a good choice
\end{itemize}

\end{frame}
%-----------------------------------------------------------------------
%----------------------------------------------------------------------
\begin{frame}[c]{Local Importance \litw{\href{https://www.tnt.uni-hannover.de/papers/data/1410/18-LION12-CAVE.pdf}{Biedenkapp et al. 2018}}}

\begin{columns}
	
	\column{0.4\textwidth}
	\begin{center}
		% Created by Katha as part of the BOBO Paper
		\includegraphics[width=1.0\textwidth]{images/learning_rate_lpi.png}\\
		Source: \lit{\href{https://arxiv.org/pdf/1908.06674.pdf}{Lindauer et al. 2019}}
	\end{center}
	
	\column{0.6\textwidth}
	
	\begin{itemize}
		\item Typical question of users:
		\begin{itemize}
			\item How would the performance change\\
			if we change hyperparameter $\conf_i$?
		\end{itemize}
        \pause 
		\item Problem: Running full study is often too expensive
		\begin{itemize}
			\item Each run of an ML-system is potential expensive
		\end{itemize}
		\pause
		\item \alert{Key Ideas:}
		\begin{itemize}
			\item Re-use probabilistic models as trained in BO
			\item Plot performance change around $\incumbent$\\
			 along each dimension
		\end{itemize} 
	\end{itemize}
	
\end{columns}

\end{frame}
%-----------------------------------------------------------------------
%----------------------------------------------------------------------
\begin{frame}[c]{Quantifying Local Importance \litw{\href{https://www.tnt.uni-hannover.de/papers/data/1410/18-LION12-CAVE.pdf}{Biedenkapp et al. 2018}}}

\begin{eqnarray}
\text{VAR}_{\conf} (i)  &=& \sum_{v \in \pcs_{i}} (\mathbb{E}_{v \sim \pcs_{i}}[\loss(\conf) ] - \loss(\conf[ \conf_{i} := v]))^2\\
\pause
\text{LPI}( i  \mid \conf ) &=& \frac{VAR_{\conf} (i) }{ \sum_{j}  VAR_{\conf} (j)}
\end{eqnarray}

\bigskip
\pause

\begin{itemize}
	\item[$\leadsto$] While fixing all other hyperparameters to the incumbent value,\\ the hyperparameter with the highest variance  is the most important one
\end{itemize}

\end{frame}
%-----------------------------------------------------------------------
%----------------------------------------------------------------------
\begin{frame}[c]{Ablation Study for Importance}

\begin{itemize}
	\item Users often start from some kind of default configuration
	\begin{enumerate}
		\item As given in the documentation 
		\item Or as always used in the last time
	\end{enumerate}
    \pause
	\item \alert{Key Idea}: Going from the default to the automatically optimized configuration,\\
	 which choices were important?
\end{itemize}

\begin{eqnarray}
\confI{\text{start}} &=& [1, 1, 0, 100]  \nonumber\\
\confI{\text{end}} &=& [0.98, 2.42, 1, 42]  \nonumber
\end{eqnarray}

\pause
\begin{itemize}
	\item Cheap approach: Assess $\confI{\text{end}}$ with each hyperparameter value from $\confI{\text{start}}$
	\pause
	\item Expensive approach: Try all mixtures of $\confI{\text{end}}$ and $\confI{\text{start}}$
	\begin{itemize}
		\item  Only feasible for small spaces and fairly cheap ML systems
	\end{itemize}
	\pause
	\item Trade-off: Find a way from $\confI{\text{start}}$ to $\confI{\text{end}}$ in a greedy fashion \lit{\href{https://link.springer.com/article/10.1007\%2Fs10732-014-9275-9}{Fawcett and Hoos. 2016}}
\end{itemize}

\end{frame}
%-----------------------------------------------------------------------
%----------------------------------------------------------------------
\begin{frame}[c]{Greedy Ablation Study}

Given:
\begin{eqnarray}
\begin{array}{lll}
\confI{\text{start}} &= [1, 1, 0, 100] & \loss_{\text{start}} = 20\% \nonumber\\
\confI{\text{end}} &= [0.98, 2.42, 1, 42]  & \loss_{\text{end}} = 4\% \nonumber
\end{array}
\end{eqnarray}

\pause
1st Iteration:
\begin{eqnarray}
\begin{array}{lll}
\confI1& = [\alert{0.98}, 1, 0, 100] & \loss_1 = 19\% \nonumber\\
\pause
\confI2 &= [1, \alert{2.42}, 0, 100] & \loss_2 = 20\% \nonumber\\
\pause
\confI3 &= [1, 1, \alert{1}, 100] & \loss_3 = 7\% \nonumber\\
\pause
\confI4 &= [1, 1, 0, \alert{42}] & \loss_4 = 16\% \nonumber\\
\end{array}
\end{eqnarray}

\pause
$\leadsto$ 1st step: $\conf_2$ -- flipping hyperparameter 3


\end{frame}
%-----------------------------------------------------------------------
%----------------------------------------------------------------------
\begin{frame}[c]{Greedy Ablation Study}

Given:
\begin{eqnarray}
\begin{array}{lll}
\confI{\text{start}} &= [1, 1, 0, 100] & \loss_{\text{start}} = 20\% \nonumber\\
\confI{s1} &= [1, 1, \alert{1}, 100]  & \loss = 7\% \nonumber\\
\confI{\text{end}} &= [0.98, 2.42, 1, 42]  & \loss_{\text{end}} = 4\% \nonumber
\end{array}
\end{eqnarray}

2nd Iteration:
\begin{eqnarray}
\begin{array}{ll}
\confI1 = [\alert{0.98}, 1, 1, 100] & \loss_1 = 6\% \nonumber\\
\pause
\confI2 = [1, \alert{2.42}, 1, 100] & \loss_2 = 7\% \nonumber\\
\pause
\confI3 = [1, 1, 1, \alert{42}] & \loss_3 = 5\% \nonumber\\
\end{array}
\end{eqnarray}

$\leadsto$ 2nd step: $\conf_3$ -- flipping hyperparameter 4

\end{frame}
%-----------------------------------------------------------------------
%----------------------------------------------------------------------
\begin{frame}[c]{Greedy Ablation Study}

Given:
\begin{eqnarray}
\begin{array}{lll}
\confI{\text{start}} &= [1, 1, 0, 100] & \loss_{\text{start}} = 20\% \nonumber\\
\confI{s1} &= [1, 1, \alert{1}, 100]  & \loss = 7\% \nonumber\\
\confI{s2} &= [1, 1, 1, \alert{42}]  & \loss = 5\% \nonumber\\
\confI{\text{end}} &= [0.98, 2.42, 1, 42]  & \loss_{\text{end}} = 4\% \nonumber
\end{array}
\end{eqnarray}

3rd Iteration:
\begin{eqnarray}
\begin{array}{ll}
\confI1 = [\alert{0.98}, 1, 1, 100] & \loss_1 = 4\% \nonumber\\
\confI2 = [1, \alert{2.42}, 1, 100] & \loss_2 = 5\% \nonumber\\
\end{array}
\end{eqnarray}

$\leadsto$ 2nd step: $\conf_3$ -- flipping hyperparameter 1

\end{frame}
%-----------------------------------------------------------------------
%----------------------------------------------------------------------
\begin{frame}[c]{Greedy Ablation Study}

Ablation Path:
\begin{eqnarray}
\begin{array}{ll}
\confI{\text{start}} = [1, 1, 0, 100] & \loss_{\text{start}} = 20\% \nonumber\\
\confI{s1} = [1, 1, \alert{1}, 100]  & \loss = 7\% \nonumber\\
\confI{s2} = [1, 1, 1, \alert{42}]  & \loss = 5\% \nonumber\\
\confI{s3} = [\alert{0.98}, 1, 1, 42] & \loss = 4\% \nonumber\\
\confI{s4} = [0.98, \alert{2.42}, 1, 42] & \loss = 4\% \nonumber\\
\confI{\text{end}} = [0.98, 2.42, 1, 42]  & \loss_{\text{end}} = 4\% \nonumber
\end{array}
\end{eqnarray}

\end{frame}
%-----------------------------------------------------------------------
%----------------------------------------------------------------------
\begin{frame}[c,fragile]{Greedy Ablation Pseudo Code}

\LinesNotNumbered
\begin{algorithm}[H]
	\Input{%
		Algorithm $\algo$ with configuration space $\confs$,
		start configuration $\confI{\text{start}}$,\\
		end configuration $\confI{\text{end}}$, cost metric $\cost$
	}
	\BlankLine
	$\conf \leftarrow  \confI{\text{start}}$; \\
	$P \leftarrow [] $ ;\\ 
	\pause
	\ForEach{$\bocount \in \{1 \ldots |\pcs|\}$}{
		\pause
		\ForEach{$\delta \in \Delta(\conf, \confI{\text{end}})$}{
			$\conf'_\delta$ $\leftarrow$ apply $\delta$ to $\conf$;\\
			evaluate $\cost(\conf'_\delta)$;
		}
	  \pause
		Determine most important change $\delta^* \in \argmin_{\delta \in \Delta(\conf, \confI{\text{end}})} \cost(\conf_\delta)$;\\
		$\conf$ $\leftarrow$ apply $\delta^*$ to $\conf$;\\
		P.append($\delta^*$);
	}
	\pause
	\Return{Ablation path $P$}
	\caption{Greedy Ablation}
	\end{algorithm}
	
\end{frame}
%----------------------------------------------------------------------
%----------------------------------------------------------------------
\begin{frame}[c,fragile]{Remarks on Ablation}

\begin{itemize}
	\item Even this greedy ablation requires $\mathcal{O}(n^2)$ steps 
	\pause
	\item[$\leadsto$] We can also speedup that up by using surrogate models\\
	\lit{\href{https://www.tnt.uni-hannover.de/papers/data/1396/17-AAAI-Surrogate-Ablation.pdf}{Biedenkapp et al. 2017}}
	\medskip
	\pause
	\item Common observations:
	\begin{enumerate}
		\item Some hyperparameters might not matter ($\conf_{2}$ in the example)
		\pause
		\item Often only a few of the hyperparameters have an big impact
		\pause
		\item You have plateaus in your ablation path because of interaction effects
	\end{enumerate}
\end{itemize}

\end{frame}
%----------------------------------------------------------------------


%-----------------------------------------------------------------------
\end{document}
