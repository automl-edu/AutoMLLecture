% !TeX spellcheck = en_US

\pdfminorversion=4 % for acroread
\documentclass[aspectratio=169,t,xcolor={usenames,dvipsnames}]{beamer}
%\documentclass[t,handout,xcolor={usenames,dvipsnames}]{beamer}
\usepackage{../beamerstyle}
\usepackage{dsfont}
\usepackage{bm}
\usepackage[english]{babel}
\usepackage[utf8]{inputenc}
\usepackage{graphicx}
\usepackage{algorithm}
\usepackage[ruled,vlined,algo2e,linesnumbered]{algorithm2e}
%\usepackage[boxed,vlined]{algorithm2e}
\usepackage{hyperref}
\usepackage{booktabs}
\usepackage{mathtools}

\usepackage{amsmath,amssymb}
\usepackage{listings}
\lstset{frame=lines,framesep=3pt,numbers=left,numberblanklines=false,basicstyle=\ttfamily\small}

\usepackage{subfig}
\usepackage{multicol}
%\usepackage{appendixnumberbeamer}
%
\usepackage{tcolorbox}

\usepackage{pgfplots}
\usepackage{tikz}
\usetikzlibrary{trees} 
\usetikzlibrary{shapes.geometric}
\usetikzlibrary{positioning,shapes,shadows,arrows,calc,mindmap}
\usetikzlibrary{positioning,fadings,through}
\usetikzlibrary{decorations.pathreplacing}
\usetikzlibrary{intersections}
\usetikzlibrary{positioning,fit,calc,shadows,backgrounds}
\pgfdeclarelayer{background}
\pgfdeclarelayer{foreground}
\pgfsetlayers{background,main,foreground}
\tikzstyle{activity}=[rectangle, draw=black, rounded corners, text centered, text width=8em]
\tikzstyle{data}=[rectangle, draw=black, text centered, text width=8em]
\tikzstyle{myarrow}=[->, thick, draw=black]

% Define the layers to draw the diagram
\pgfdeclarelayer{background}
\pgfdeclarelayer{foreground}
\pgfsetlayers{background,main,foreground}

%\usepackage{listings}
%\lstset{numbers=left,
%  showstringspaces=false,
%  frame={tb},
%  captionpos=b,
%  lineskip=0pt,
%  basicstyle=\ttfamily,
%%  extendedchars=true,
%  stepnumber=1,
%  numberstyle=\small,
%  xleftmargin=1em,
%  breaklines
%}

 
\definecolor{blue}{RGB}{0, 74, 153}

\usetheme{Boadilla}
%\useinnertheme{rectangles}
\usecolortheme{whale}
\setbeamercolor{alerted text}{fg=blue}
\useoutertheme{infolines}
\setbeamertemplate{navigation symbols}{\vspace{-5pt}} % to lower the logo
\setbeamercolor{date in head/foot}{bg=blue} % blue
\setbeamercolor{date in head/foot}{fg=white}
\setbeamercolor{author in head/foot}{bg=blue} %blue
\setbeamercolor{title in head/foot}{bg=blue} % blue
\setbeamercolor{title}{fg=white, bg=blue}
\setbeamercolor{block title}{fg=white,bg=blue}
\setbeamercolor{block body}{bg=blue!10}
\setbeamercolor{frametitle}{fg=white, bg=blue}
\setbeamercovered{invisible}

\makeatletter
\setbeamertemplate{footline}
{
  \leavevmode%
  \hbox{%
  \begin{beamercolorbox}[wd=.333333\paperwidth,ht=2.25ex,dp=1ex,center]{author in head/foot}%
    \usebeamerfont{author in head/foot}\insertshortauthor
  \end{beamercolorbox}%
  \begin{beamercolorbox}[wd=.333333\paperwidth,ht=2.25ex,dp=1ex,center]{title in head/foot}%
    \usebeamerfont{title in head/foot}\insertshorttitle
  \end{beamercolorbox}%
  \begin{beamercolorbox}[wd=.333333\paperwidth,ht=2.25ex,dp=1ex,right]{date in head/foot}%
    \usebeamerfont{date in head/foot}Week \@week, Topic \@topicnumber, Slide \insertframenumber{}\hspace*{2em}
%    \insertframenumber\hspace*{2ex} 
  \end{beamercolorbox}}%
  \vskip0pt%
}

\newcommand{\@week}{0}
\newcommand{\@topicnumber}{0}
\newcommand{\week}[1]{\renewcommand{\@week}{#1}}
\newcommand{\topicnumber}[1]{\renewcommand{\@topicnumber}{#1}}

\makeatother

%\pgfdeclareimage[height=1.2cm]{automl}{images/logos/automl.png}
%\pgfdeclareimage[height=1.2cm]{freiburg}{images/logos/freiburg}

%\logo{\pgfuseimage{freiburg}}

\input{../latex_main/macros}






\title[AutoML: Global Importance]{AutoML: Interpretability}
\subtitle{Global Hyperparameter Importance}
\author[Marius Lindauer]{Bernd Bischl \and Frank Hutter \and Lars Kotthoff\newline \and \underline{Marius Lindauer} \and Joaquin Vanschoren}
\institute{}
\date{}
\week{12}
\topicnumber{4}

% \AtBeginSection[] % Do nothing for \section*
% {
%   \begin{frame}{Outline}
%     \bigskip
%     \vfill
%     \tableofcontents[currentsection]
%   \end{frame}
% }

\begin{document}
	
	\maketitle
	

%----------------------------------------------------------------------
%----------------------------------------------------------------------
\begin{frame}[c,fragile]{Idea}

\begin{center}
\scalebox{0.9}{
	 \let\oldpause=\pause \def\pause{} 
	\input{tikz/automl_analysis_overview.tex}
	\let\pause=\oldpause
}
\end{center}


\begin{itemize}
	\item[$\leadsto$] focus on which hyperparameters are important across the entire search space
\end{itemize}

\end{frame}
%-----------------------------------------------------------------------
%----------------------------------------------------------------------
\begin{frame}[c,fragile]{Importance Analysis of Surrogate Model}

\begin{itemize}
	\item \alert{Key Idea}: Surrogate models ($\pcs \to \perf$) learn from observations how to predict the performance of a \alert{hyperparameter configuration} $\conf \in \pcs$
	\pause
	\item[$\leadsto$] These models can be used to figure out which hyperparameter was important
	\medskip
	\pause
	\item For example:
	\begin{itemize}
		\item Use forward selection~\lit{\href{https://ml.informatik.uni-freiburg.de/papers/13-LION-FeatureAndParameterImportance.pdf}{Hutter et al. 2013}}
        \item Use automatic feature relevance determination of the model\\
         (e.g., of a surrogate model based on random forest)
	\end{itemize}
	\medskip
	\pause
	\item Advantages:
	\begin{itemize}
		\item Very cheap to do, since we only have to query the surrogate model several times
	\end{itemize}
	\smallskip
	\pause
	\item Potential drawback:
	\begin{itemize}
		\item The surrogate model might overfit to different subsets of the hyperparameters\\
		(if we don't provide sufficient data)
	\end{itemize}

\end{itemize}

\end{frame}
%-----------------------------------------------------------------------
%----------------------------------------------------------------------
\begin{frame}[c,fragile]{Global Importance Analysis}

\begin{itemize}
	\item \alert{Key idea}: What is the importance of a hyperparameter\\
	 by marginalizing over all other hyperparameter effects?
	 \pause
	\item \alert{Key Insight}: We can use a surrogate model to compute these effects
\end{itemize}


\pause
\begin{block}{\fanova{}~\litw{Sobobl. 1993}}
	Write performance predictions as a sum of components:
	\begin{eqnarray}
	\hat{y}(\conf_1,\ldots,\conf_n) =& \hat{f}_0 + \sum_{i=1}^{n} \hat{f}_i(\conf_i) + \sum_{i \neq j} \hat{f}_{ij}(\conf_i,\conf_j) + \ldots \nonumber \\
	\hat{y}(\conf_1,\ldots,\conf_n) =& \text{average response} + \text{main effects} +\nonumber\\ &\text{2-D interaction effects} + \text{higher order effects} \nonumber
	\end{eqnarray}
\end{block}

\pause

\begin{block}{Variance Decomposition}
	\begin{equation}
	\mathrm{V} = \frac{1}{||\confs||} \int_{\conf_1} \ldots \int_{\conf_n} [(\hat{y}(\conf)- \hat{f}_0)^2] d\conf_1 \ldots d\conf_n\nonumber
	\end{equation}
\end{block}

\end{frame}
%-----------------------------------------------------------------------
%----------------------------------------------------------------------
\begin{frame}[c,fragile]{fANOVA Analysis}

\begin{itemize}
	\item The fANOVA and variance decomposition can be done efficiently in linear time\\
	 if the surrogate model is a random forest~\lit{\href{https://ml.informatik.uni-freiburg.de/papers/14-ICML-HyperparameterAssessment.pdf}{Hutter et al. 2014}}
	\pause
\end{itemize}

\begin{columns}
	
	\column{0.5\textwidth}
	\begin{center}
	\includegraphics[width=0.8\textwidth]{images/learning_rate_fanova.png}\\
	Source: \lit{\href{https://arxiv.org/pdf/1908.06756.pdf}{Lindauer et al. 2019}}
	%TODO: Is this effectively the same as a marginal dependency plot?
	\end{center}
	
	\column{0.5\textwidth}
	
	\begin{itemize}
		\item predicted cost is marginalized over all other hyperparameter effects
		\pause
		\item \alert{Warning}: The optimum on these curves does not have to be the global optimum across all hyperparameters
	\end{itemize}
\end{columns}



\end{frame}
%-----------------------------------------------------------------------

%----------------------------------------------------------------------
\begin{frame}[c,fragile]{fANOVA Analysis}

\begin{itemize}
	\item 	How much of the variance can be explained by a hyperparameter\\
	 (or combinations of hyperparamaters) marginalized over all other parameters?
\end{itemize}

\begin{table}
	\caption{Exemplary analysis of PPO on cartpole}
	\begin{tabular}{lc}
		\toprule
		Hyperparameter & Explained Variance \\
		\midrule
		Discount rate & 19.3 \%\\
		Batch size & 15.7 \% \\
		Learning rate & 3.7 \% \\
		Likelihood ration clipping & 3.4\% \\
		... \\
		\midrule
		\pause
		discount rate \& batch size & 10.4\% \\
		discount rate \& likelihood ration clipping & 4.4\% \\
		... \\
		\bottomrule
	\end{tabular}
\end{table}
	

\end{frame}
%-----------------------------------------------------------------------


%----------------------------------------------------------------------
\begin{frame}[c,fragile]{Remarks on fANOVA}

\begin{itemize}
	\item Given compute higher-order interaction effects
	\begin{itemize}
		\item Often too expensive for more than 2 or 3 dimensions
	\end{itemize}
	\medskip
	\pause
	\item Implicit assumption: the surrogate model models the space fairly well
	\medskip
	\pause
	\item Global analysis and local analysis of hyperparameter importance does not always agree \lit{\href{https://www.tnt.uni-hannover.de/papers/data/1410/18-LION12-CAVE.pdf}{Biedenkapp et al. 2018}}
	\smallskip
	\pause
	\item[$\leadsto$] You should run both to get a good understanding of\\
	 why an AutoML tool chose  a configuration
\end{itemize}

\end{frame}
%-----------------------------------------------------------------------


%-----------------------------------------------------------------------
\end{document}
