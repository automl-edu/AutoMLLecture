\pdfminorversion=4 % for acroread
\documentclass[aspectratio=169,t,xcolor={usenames,dvipsnames}]{beamer}
%\documentclass[t,handout,xcolor={usenames,dvipsnames}]{beamer}
\usepackage{../beamerstyle}
\usepackage{dsfont}
\usepackage{bm}
\usepackage[english]{babel}
\usepackage[utf8]{inputenc}
\usepackage{graphicx}
\usepackage{algorithm}
\usepackage[ruled,vlined,algo2e,linesnumbered]{algorithm2e}
%\usepackage[boxed,vlined]{algorithm2e}
\usepackage{hyperref}
\usepackage{booktabs}
\usepackage{mathtools}

\usepackage{amsmath,amssymb}
\usepackage{listings}
\lstset{frame=lines,framesep=3pt,numbers=left,numberblanklines=false,basicstyle=\ttfamily\small}

\usepackage{subfig}
\usepackage{multicol}
%\usepackage{appendixnumberbeamer}
%
\usepackage{tcolorbox}

\usepackage{pgfplots}
\usepackage{tikz}
\usetikzlibrary{trees} 
\usetikzlibrary{shapes.geometric}
\usetikzlibrary{positioning,shapes,shadows,arrows,calc,mindmap}
\usetikzlibrary{positioning,fadings,through}
\usetikzlibrary{decorations.pathreplacing}
\usetikzlibrary{intersections}
\usetikzlibrary{positioning,fit,calc,shadows,backgrounds}
\pgfdeclarelayer{background}
\pgfdeclarelayer{foreground}
\pgfsetlayers{background,main,foreground}
\tikzstyle{activity}=[rectangle, draw=black, rounded corners, text centered, text width=8em]
\tikzstyle{data}=[rectangle, draw=black, text centered, text width=8em]
\tikzstyle{myarrow}=[->, thick, draw=black]

% Define the layers to draw the diagram
\pgfdeclarelayer{background}
\pgfdeclarelayer{foreground}
\pgfsetlayers{background,main,foreground}

%\usepackage{listings}
%\lstset{numbers=left,
%  showstringspaces=false,
%  frame={tb},
%  captionpos=b,
%  lineskip=0pt,
%  basicstyle=\ttfamily,
%%  extendedchars=true,
%  stepnumber=1,
%  numberstyle=\small,
%  xleftmargin=1em,
%  breaklines
%}

 
\definecolor{blue}{RGB}{0, 74, 153}

\usetheme{Boadilla}
%\useinnertheme{rectangles}
\usecolortheme{whale}
\setbeamercolor{alerted text}{fg=blue}
\useoutertheme{infolines}
\setbeamertemplate{navigation symbols}{\vspace{-5pt}} % to lower the logo
\setbeamercolor{date in head/foot}{bg=white} % blue
\setbeamercolor{date in head/foot}{fg=white}
\setbeamercolor{author  in head/foot}{bg=white} %blue
\setbeamercolor{title in head/foot}{bg=white} % blue
\setbeamercolor{title}{fg=white, bg=blue}
\setbeamercolor{block title}{fg=white,bg=blue}
\setbeamercolor{block body}{bg=blue!10}
\setbeamercolor{frametitle}{fg=white, bg=blue}
\setbeamercovered{invisible}

\makeatletter
\setbeamertemplate{footline}
{
  \leavevmode%
  \hbox{%
  \begin{beamercolorbox}[wd=.333333\paperwidth,ht=2.25ex,dp=1ex,center]{author in head/foot}%
%    \usebeamerfont{author in head/foot}\insertshortauthor
  \end{beamercolorbox}%
  \begin{beamercolorbox}[wd=.333333\paperwidth,ht=2.25ex,dp=1ex,center]{title in head/foot}%
    \usebeamerfont{title in head/foot}\insertshorttitle
  \end{beamercolorbox}%
  \begin{beamercolorbox}[wd=.333333\paperwidth,ht=2.25ex,dp=1ex,right]{date in head/foot}%
    \usebeamerfont{date in head/foot}\insertshortdate{}\hspace*{2em}
%    \insertframenumber\hspace*{2ex} 
  \end{beamercolorbox}}%
  \vskip0pt%
}
\makeatother

%\pgfdeclareimage[height=1.2cm]{automl}{images/logos/automl.png}
%\pgfdeclareimage[height=1.2cm]{freiburg}{images/logos/freiburg}

%\logo{\pgfuseimage{freiburg}}

\newcommand{\comment}[1]{
	\noindent
	%\vspace{0.25cm}
	{\color{red}{\textbf{TODO:} #1}}
	%\vspace{0.25cm}
}
\renewcommand{\comment}[1]{}
\newcommand{\hide}[1]{}
\newcommand{\cemph}[2]{\emph{\textcolor{#1}{#2}}}

\newcommand{\lit}[1]{{\footnotesize\color{black!70}[#1]}}

\newcommand{\litw}[1]{{\footnotesize\color{black!20}[#1]}}


\newcommand{\myframe}[2]{\begin{frame}[c]{#1}#2\end{frame}}
\newcommand{\myframetop}[2]{\begin{frame}{#1}#2\end{frame}}
\newcommand{\myit}[1]{\begin{itemize}#1\end{itemize}}
\newcommand{\myblock}[2]{\begin{block}{#1}#2\end{block}}


\newcommand{\votepurple}[1]{\textcolor{Purple}{$\bigstar$}}
\newcommand{\voteyellow}[1]{\textcolor{Goldenrod}{$\bigstar$}}
\newcommand{\voteblue}[1]{\textcolor{RoyalBlue}{$\bigstar$}}
\newcommand{\votepink}[1]{\textcolor{Pink}{$\bigstar$}}

\newcommand{\diff}{\mathop{}\!\mathrm{d}}
\newcommand{\refstyle}[1]{{\small{\textcolor{gray}{#1}}}}
\newcommand{\hands}[0]{\includegraphics[height=1.5em]{images/hands}}
\newcommand{\transpose}[0]{{\textrm{\tiny{\sf{T}}}}}
\newcommand{\norm}{{\mathcal{N}}}
\newcommand{\cutoff}[0]{\kappa}
\newcommand{\instD}[0]{\dataset}
\newcommand{\insts}[0]{\mathcal{I}}
\newcommand{\inst}[0]{i}
\newcommand{\pcs}[0]{\mathbf{\Lambda}}
\newcommand{\bx}[0]{\conf}
\newcommand{\conf}[0]{\mathbf{\lambda}}
\newcommand{\defconf}[0]{\mathbf{\lambda}_{\text{def}}}
\newcommand{\finconf}[0]{\mathbf{\lambda}^*}
\newcommand{\incumbent}[0]{\finconf}
\newcommand{\confs}[0]{\pcs}
%\newcommand{\vlambda}[0]{\bm{\lambda}}
%\newcommand{\vLambda}[0]{\bm{\Lambda}}
\newcommand{\dataset}[0]{\mathcal{D}}
\newcommand{\datasets}[0]{\mathbf{D}}
\newcommand{\loss}[0]{\mathcal{L}}

% \renewcommand{\vec}[1]{\mathbf{#1}}
\newcommand{\hist}[0]{\mathcal{H}}
\newcommand{\param}[0]{p}
\newcommand{\algo}[0]{\mathcal{A}}
\newcommand{\algos}[0]{\mathbf{A}}
%\newcommand{\nn}[0]{N}
\newcommand{\feats}[0]{\mathcal{F}}
\newcommand{\feat}[0]{\vec{f}}
\newcommand{\cluster}[0]{\vec{h}}
\newcommand{\clusters}[0]{\vec{H}}
\newcommand{\perf}[0]{\mathbb{R}}
%\newcommand{\surro}[0]{\mathcal{S}}
\newcommand{\surro}[0]{\hat{f}}
\newcommand{\func}[0]{f}
\newcommand{\epm}[0]{\surro}
\newcommand{\portfolio}[0]{\mathcal{P}}
\newcommand{\schedule}[0]{\mathcal{S}}
\newcommand{\mdata}[0]{\dataset_{\text{meta}}}

% Deep Learning
\newcommand{\weights}[0]{\theta}
\newcommand{\metaweights}[0]{\phi}


% reinforcement learning
\newcommand{\policies}[0]{\Pi}
\newcommand{\policy}[0]{\pi}
\newcommand{\actionRL}[0]{a}
\newcommand{\stateRL}[0]{s}
\newcommand{\statesRL}[0]{\mathcal{S}}
\newcommand{\rewardRL}[0]{r}
\newcommand{\rewardfuncRL}[0]{\mathcal{R}}

\RestyleAlgo{algoruled}
\DontPrintSemicolon
\LinesNumbered
\SetAlgoVlined
\SetFuncSty{textsc}

\SetKwInOut{Input}{Input}
\SetKwInOut{Output}{Output}
\SetKw{Return}{return}

%\newcommand{\changed}[1]{{\color{red}#1}}

%\newcommand{\citeN}[1]{\citeauthor{#1}~(\citeyear{#1})}

\renewcommand{\vec}[1]{\mathbf{#1}}
\DeclareMathOperator*{\argmin}{arg\,min}
\DeclareMathOperator*{\argmax}{arg\,max}

\newcommand{\aqme}{\textit{AQME}}
\newcommand{\aslib}{\textit{ASlib}}
\newcommand{\llama}{\textit{LLAMA}}
\newcommand{\satzilla}{\textit{SATzilla}}
\newcommand{\satzillaY}[1]{\textit{SATzilla'{#1}}}
\newcommand{\snnap}{\textit{SNNAP}}
\newcommand{\claspfolioTwo}{\textit{claspfolio~2}}
\newcommand{\flexfolio}{\textit{FlexFolio}}
\newcommand{\claspfolioOne}{\textit{claspfolio~1}}
\newcommand{\isac}{\textit{ISAC}}
\newcommand{\eisac}{\textit{EISAC}}
\newcommand{\sss}{\textit{3S}}
\newcommand{\sunny}{\textit{Sunny}}
\newcommand{\ssspar}{\textit{3Spar}}
\newcommand{\cshc}{\textit{CSHC}}  
\newcommand{\cshcpar}{\textit{CSHCpar}}  
\newcommand{\measp}{\textit{ME-ASP}} 
\newcommand{\aspeed}{\textit{aspeed}}
\newcommand{\autofolio}{\textit{AutoFolio}}
\newcommand{\cedalion}{\textit{Cedalion}}
\newcommand{\fanova}{\textit{fANOVA}}
\newcommand{\sbs}{\textit{SB}}
\newcommand{\oracle}{\textit{VBS}}

% like approaches
\newcommand{\claspfoliolike}[1]{\texttt{claspfolio-#1-like}}
\newcommand{\satzillalike}[1]{\texttt{SATzilla'#1-like}}
\newcommand{\isaclike}{\texttt{ISAC-like}}
\newcommand{\ssslike}{\texttt{3S-like}}
\newcommand{\measplike}{\texttt{ME-ASP-like}}

\newcommand{\aspCoseal}{\textit{ASP-POTASSCO}}
\newcommand{\cspCoseal}{\textit{CSP-2010}}
\newcommand{\maxsatCoseal}{\textit{MAXSAT12-PMS}}
\newcommand{\premarCoseal}{\textit{PRE\-MARSHALLING}}
\newcommand{\qbfCoseal}{\textit{QBF-2011}}
\newcommand{\satallTwelveCoseal}{\textit{SAT12-ALL}}
\newcommand{\sathandTwelveCoseal}{\textit{SAT12-HAND}}
\newcommand{\satinduTwelveCoseal}{\textit{SAT12-INDU}}
\newcommand{\satrandTwelveCoseal}{\textit{SAT12-RAND}}
\newcommand{\sathandElevenCoseal}{\textit{SAT11-HAND}}
\newcommand{\satinduElevenCoseal}{\textit{SAT11-INDU}}
\newcommand{\satrandElevenCoseal}{\textit{SAT11-RAND}}
\newcommand{\proteusCoseal}{\textit{PROTEUS-2014}}

\newcommand{\irace}{\textit{I/F-race}}
\newcommand{\gga}{\textit{GGA}}
\newcommand{\smac}{\textit{SMAC}}
\newcommand{\paramils}{\textit{ParamILS}}
\newcommand{\spearmint}{\textit{Spearmint}}
\newcommand{\tpe}{\textit{TPE}}

\newcommand{\gringo}{\textit{gringo}}
\newcommand{\clasp}{\textit{clasp}}
\newcommand{\lingeling}{\textit{lingeling}}

\newcommand{\hydra}{\textit{Hydra}}

\newcommand{\plingeling}{\textit{Plingeling}}
\newcommand{\ccasat}{\textit{CCASat}}

\usepackage{pifont}
\newcommand{\itarrow}{\mbox{\Pisymbol{pzd}{229}}}
\newcommand{\ithook}{\mbox{\Pisymbol{pzd}{52}}}
\newcommand{\itcross}{\mbox{\Pisymbol{pzd}{56}}}
\newcommand{\ithand}{\mbox{\raisebox{-1pt}{\Pisymbol{pzd}{43}}}}

%\DeclareMathOperator*{\argmax}{arg\,max}

\newcommand{\ie}{{\it{}i.e.\/}}
\newcommand{\eg}{{\it{}e.g.\/}}
\newcommand{\cf}{{\it{}cf.\/}}
\newcommand{\wrt}{\mbox{w.r.t.}}
\newcommand{\vs}{{\it{}vs\/}}
\newcommand{\vsp}{{\it{}vs\/}}
\newcommand{\etc}{{\copyedit{etc.}}}
\newcommand{\etal}{{\it{}et al.\/}}

\newcommand{\pscProc}{{\bf procedure}}
\newcommand{\pscBegin}{{\bf begin}}
\newcommand{\pscEnd}{{\bf end}}
\newcommand{\pscEndIf}{{\bf endif}}
\newcommand{\pscFor}{{\bf for}}
\newcommand{\pscEach}{{\bf each}}
\newcommand{\pscThen}{{\bf then}}
\newcommand{\pscElse}{{\bf else}}
\newcommand{\pscWhile}{{\bf while}}
\newcommand{\pscIf}{{\bf if}}
\newcommand{\pscRepeat}{{\bf repeat}}
\newcommand{\pscUntil}{{\bf until}}
\newcommand{\pscWithProb}{{\bf with probability}}
\newcommand{\pscOtherwise}{{\bf otherwise}}
\newcommand{\pscDo}{{\bf do}}
\newcommand{\pscTo}{{\bf to}}
\newcommand{\pscOr}{{\bf or}}
\newcommand{\pscAnd}{{\bf and}}
\newcommand{\pscNot}{{\bf not}}
\newcommand{\pscFalse}{{\bf false}}
\newcommand{\pscEachElOf}{{\bf each element of}}
\newcommand{\pscReturn}{{\bf return}}

%\newcommand{\param}[1]{{\sl{}#1}}
\newcommand{\var}[1]{{\it{}#1}}
\newcommand{\cond}[1]{{\sf{}#1}}
%\newcommand{\state}[1]{{\sf{}#1}}
%\newcommand{\func}[1]{{\sl{}#1}}
\newcommand{\set}[1]{{\Bbb #1}}
%\newcommand{\inst}[1]{{\tt{}#1}}
\newcommand{\myurl}[1]{{\small\sf #1}}

\newcommand{\Nats}{{\Bbb N}}
\newcommand{\Reals}{{\Bbb R}}
\newcommand{\extset}[2]{\{#1 \; | \; #2\}}

\newcommand{\vbar}{$\,\;|$\hspace*{-1em}\raisebox{-0.3mm}{$\,\;\;|$}}
\newcommand{\vendbar}{\raisebox{+0.4mm}{$\,\;|$}}
\newcommand{\vend}{$\,\:\lfloor$}


\newcommand{\goleft}[2][.7]{\parbox[t]{#1\linewidth}{\strut\raggedright #2\strut}}
\newcommand{\rightimage}[2][.3]{\mbox{}\hfill\raisebox{1em-\height}[0pt][0pt]{\includegraphics[width=#1\linewidth]{#2}}\vspace*{-\baselineskip}}





\title[AutoML: NAS]{AutoML: Neural Architecture Search (NAS)} 
\subtitle{Practical Recommendations for NAS and HPO}
\author[Marius Lindauer]{Bernd Bischl \and \underline{Frank Hutter} \and Lars Kotthoff\newline \and Marius Lindauer \and Joaquin Vanschoren}
\institute{}
\date{}

\begin{document}
\maketitle

%----------------------------------------------------------------------
\myframe{Maturity of the Fields of NAS and HPO}{
	\myit{
		\item \alert{Hyperparameter optimization is a mature field}
		\myit{
			\item[-] Blackbox HPO has been researched for decades; \\ there are many software packages
			\item[-] Multi-fidelity HPO has also become quite mature
%			\item[-] Gradient-based HPO is still a nascent field
		}
\medskip
\pause
		\item \alert{Neural architecture search is still a very young field}
\pause
		\myit{
			\item[-] Blackbox is quite mature, but slow
			\item[-] Multi-fidelity NAS is also quite mature and faster
\pause
			\item[-] Meta-learning + multi-fidelity NAS is fast, but is still a very young field
\pause
\smallskip
			\item[-] Gradient-based NAS is fast, but can have failure modes with terrible performance
			\item[-] Gradient-based NAS hasn't reached the hands-off AutoML stage yet
		}
\medskip
\pause
		\item NAS is mainly used to \alert{create new architectures that many others can reuse}
		
\medskip
\pause
		\item Given a new dataset, \alert{HPO is crucial for good performance; NAS may not be necessary}
		\myit{
			\item[-] The biggest gains typically come from tuning key hyperparameters (learning rate, etc)
			\item[-] Reusing a previous achitecture often yields competitive results
		}
	}
}
%----------------------------------------------------------------------

%----------------------------------------------------------------------
\myframe{Practical Recommendations for NAS and HPO}{
	\myit{
		\item Recommendations for a new dataset
		\myit{
			\item[$\rightarrow$] Always run HPO
			\item[$\rightarrow$] Try NAS if you can 
		}
\bigskip
\pause
		\item How to combine NAS \& HPO
		\myit{
			\item[-] If the compute budget suffices, \alert{optimize them jointly}, e.g., using BOHB
			\myit{
				\item[+] Auto-PyTorch Tabular \lit{\href{https://github.com/automl/Auto-PyTorch}{Zimmer,\,Lindauer\,\& Hutter, 2020}}
				\item[+] Auto-RL \lit{\href{https://arxiv.org/pdf/1812.11951.pdf}{Runge et al. 2019}}
			}
\medskip
\pause
			\item[-] Else
			\myit{
				\item[+] If you have decent hyperparameters:\\
			\alert{run NAS, followed by HPO for fine-tuning} \lit{\href{https://arxiv.org/pdf/1905.07443.pdf}{Saikat et al. 2019}}
\smallskip
\pause
				\item[+] If you don't have decent hyperparameters: \alert{first run HPO} to get competitive	
			}
%\medskip
%\pause
%			\item[-] To improve robustness of DARTS, \\
%			tune DARTS' own hyperparameters with BOHB \lit{\href{https://openreview.net/forum?id=SJx9ngStPH}{Zela et al, 2020}}
		}
	}
}
%----------------------------------------------------------------------


%----------------------------------------------------------------------
\myframe{Case Study: NAS \& HPO in Auto-PyTorch Tabular \litw{\href{https://github.com/automl/Auto-PyTorch}{Zimmer,\,Lindauer\,\& Hutter, 2020}}}{

\begin{columns}[T]
	\column{0.29\textwidth}

\onslide<1->{
	\myit{
		\item \alert{Joint Architecture Search and Hyperparameter Optimization}
		\myit{
\onslide<2->{
			\item[-] Purely using HPO techniques: very similar methods as in \alert{Auto-sklearn 2.0}
}
\onslide<3->{
			\item[-] \alert{Multi-fidelity optimization} with \alert{BOHB}
			\item[-] \alert{Meta-learning} with \alert{task-independent recommendations}
}
\onslide<4->{
			\item[-] \alert{Ensembling} of neural nets and traditional ML models
}
}
		}
	}
	
	\column{0.29\textwidth}	
\onslide<1->{
	\includegraphics[width=\linewidth]{images/AutoPytorch-ShapedResNet.jpg}\\
}
	\column{0.42\textwidth}	
\vspace*{-0.7cm}
\onslide<1->{
	\begin{center}
	\includegraphics[width=0.9\linewidth]{images/AutoPytorch-ConfigSpace.jpg}
	\end{center}
}
\onslide<5->{
	\vspace*{-0.3cm}\hspace*{-1.1cm}
	\includegraphics[width=1.1\linewidth]{images/AutoPytorch-performance.jpg}
%	\end{center}
}


\end{columns}
}
%----------------------------------------------------------------------


%----------------------------------------------------------------------
\myframe{Case Study: NAS \& HPO in Auto-DispNet}{
	\begin{columns}[T]
		\column{0.64\textwidth}
		\myit{
			\item Problem: disparity estimation
			\myit{
				\item[-] Estimate depth from stereo images
			}
	\onslide<2->{
	\smallskip
				\item Background: U-Net
			\myit{
				\item Skip connections from similar spatial resolution to avoid loosing information
			}
	}\onslide<3->{
	\smallskip
				\item Search space for DARTS
			\myit{
				\item 3 cells: keeping spatial resolution, downsampling, 
				and \alert{new upsampling cell that supports U-Net like skip connections} 
			}
		}
	\smallskip
	}
	\onslide<4->{
		\myit{
			\item Both NAS and HPO improved the state of the art \lit{\href{https://arxiv.org/pdf/1905.07443.pdf}{Saikat et al. 2019}}:
			\myit{
				\item End-point-error (EPE) on Sintel dataset: \alert{2.36 $\rightarrow$ 2.14 (by DARTS)}
				\item Subsequent HPO: \alert{2.14 $\rightarrow$ 1.94 (by BOHB)}
			}
		}
	}

		\column{0.35\textwidth}
		\centering
		\includegraphics[width=\linewidth]{images/disparity_estimation}\\
\bigskip
\bigskip
\onslide<2->
		\includegraphics[width=\linewidth]{images/u-net-architecture}	 
	\end{columns}	
}
%----------------------------------------------------------------------

%----------------------------------------------------------------------
%\myframe{Case Study Details (Auto-DispNet)}{
%	\myit{
%		\item Very important: \alert{warmstarting} of the network weights
%		\myit{
%			\item First half of the training epochs: \\
%			keep one-shot architecture weights fixed to the uniform distribution
%			\item Only afterwards: \\
%			DARTS' alternating updates of weights and architectural parameters
%		}
%\bigskip
%		\item Without warmstarting:\\
%		\alert{DARTS found cells with only parameterless operations}
%	}	
%}
%----------------------------------------------------------------------



%----------------------------------------------------------------------
\myframe{Questions to Answer for Yourself / Discuss with Friends}{

	\myit{
		\item Repetition:\\ \alert{If you want to use both HPO and NAS for your problem, how could you proceed?}
\medskip
		\item Discussion:\\ \alert{Think of a problem of your particular interest. For that problem, which approach would you use to combine HPO and NAS, and why?}
	}	 
}
%-----------------------------------------------------------------------

%----------------------------------------------------------------------
%\myframe{Learning Goals}{
%
%	After this week's videos on HPO and NAS, you should be able to \ldots
%	
%	\begin{itemize}
%		\item describe \alert{several ways of speeding up over blackbox NAS}  %(except via meta-learning)
%		\item describe \alert{several ways of speeding up over blackbox NAS}  %(except via meta-learning)
%		\myit{
%			\item define \alert{network morphisms} \& \alert{explain how to use them to speed up NAS}
%			\item explain various \alert{multi-fidelity Bayesian optimization methods}
%		}
%		\item discuss \alert{when and how to use NAS and HPO in practice} 
%		\item describe \alert{failure modes of DARTS}
%	\end{itemize}
%}

\end{document}