\pdfminorversion=4 % for acroread
\documentclass[aspectratio=169,t,xcolor={usenames,dvipsnames}]{beamer}
%\documentclass[t,handout,xcolor={usenames,dvipsnames}]{beamer}
\usepackage{../beamerstyle}
\usepackage{dsfont}
\usepackage{bm}
\usepackage[english]{babel}
\usepackage[utf8]{inputenc}
\usepackage{graphicx}
\usepackage{algorithm}
\usepackage[ruled,vlined,algo2e,linesnumbered]{algorithm2e}
%\usepackage[boxed,vlined]{algorithm2e}
\usepackage{hyperref}
\usepackage{booktabs}
\usepackage{mathtools}

\usepackage{amsmath,amssymb}
\usepackage{listings}
\lstset{frame=lines,framesep=3pt,numbers=left,numberblanklines=false,basicstyle=\ttfamily\small}

\usepackage{subfig}
\usepackage{multicol}
%\usepackage{appendixnumberbeamer}
%
\usepackage{tcolorbox}

\usepackage{pgfplots}
\usepackage{tikz}
\usetikzlibrary{trees} 
\usetikzlibrary{shapes.geometric}
\usetikzlibrary{positioning,shapes,shadows,arrows,calc,mindmap}
\usetikzlibrary{positioning,fadings,through}
\usetikzlibrary{decorations.pathreplacing}
\usetikzlibrary{intersections}
\usetikzlibrary{positioning,fit,calc,shadows,backgrounds}
\pgfdeclarelayer{background}
\pgfdeclarelayer{foreground}
\pgfsetlayers{background,main,foreground}
\tikzstyle{activity}=[rectangle, draw=black, rounded corners, text centered, text width=8em]
\tikzstyle{data}=[rectangle, draw=black, text centered, text width=8em]
\tikzstyle{myarrow}=[->, thick, draw=black]

% Define the layers to draw the diagram
\pgfdeclarelayer{background}
\pgfdeclarelayer{foreground}
\pgfsetlayers{background,main,foreground}

%\usepackage{listings}
%\lstset{numbers=left,
%  showstringspaces=false,
%  frame={tb},
%  captionpos=b,
%  lineskip=0pt,
%  basicstyle=\ttfamily,
%%  extendedchars=true,
%  stepnumber=1,
%  numberstyle=\small,
%  xleftmargin=1em,
%  breaklines
%}

 
\definecolor{blue}{RGB}{0, 74, 153}

\usetheme{Boadilla}
%\useinnertheme{rectangles}
\usecolortheme{whale}
\setbeamercolor{alerted text}{fg=blue}
\useoutertheme{infolines}
\setbeamertemplate{navigation symbols}{\vspace{-5pt}} % to lower the logo
\setbeamercolor{date in head/foot}{bg=white} % blue
\setbeamercolor{date in head/foot}{fg=white}
\setbeamercolor{author  in head/foot}{bg=white} %blue
\setbeamercolor{title in head/foot}{bg=white} % blue
\setbeamercolor{title}{fg=white, bg=blue}
\setbeamercolor{block title}{fg=white,bg=blue}
\setbeamercolor{block body}{bg=blue!10}
\setbeamercolor{frametitle}{fg=white, bg=blue}
\setbeamercovered{invisible}

\makeatletter
\setbeamertemplate{footline}
{
  \leavevmode%
  \hbox{%
  \begin{beamercolorbox}[wd=.333333\paperwidth,ht=2.25ex,dp=1ex,center]{author in head/foot}%
%    \usebeamerfont{author in head/foot}\insertshortauthor
  \end{beamercolorbox}%
  \begin{beamercolorbox}[wd=.333333\paperwidth,ht=2.25ex,dp=1ex,center]{title in head/foot}%
    \usebeamerfont{title in head/foot}\insertshorttitle
  \end{beamercolorbox}%
  \begin{beamercolorbox}[wd=.333333\paperwidth,ht=2.25ex,dp=1ex,right]{date in head/foot}%
    \usebeamerfont{date in head/foot}\insertshortdate{}\hspace*{2em}
%    \insertframenumber\hspace*{2ex} 
  \end{beamercolorbox}}%
  \vskip0pt%
}
\makeatother

%\pgfdeclareimage[height=1.2cm]{automl}{images/logos/automl.png}
%\pgfdeclareimage[height=1.2cm]{freiburg}{images/logos/freiburg}

%\logo{\pgfuseimage{freiburg}}

\newcommand{\comment}[1]{
	\noindent
	%\vspace{0.25cm}
	{\color{red}{\textbf{TODO:} #1}}
	%\vspace{0.25cm}
}
\renewcommand{\comment}[1]{}
\newcommand{\hide}[1]{}
\newcommand{\cemph}[2]{\emph{\textcolor{#1}{#2}}}

\newcommand{\lit}[1]{{\footnotesize\color{black!70}[#1]}}

\newcommand{\litw}[1]{{\footnotesize\color{black!20}[#1]}}


\newcommand{\myframe}[2]{\begin{frame}[c]{#1}#2\end{frame}}
\newcommand{\myframetop}[2]{\begin{frame}{#1}#2\end{frame}}
\newcommand{\myit}[1]{\begin{itemize}#1\end{itemize}}
\newcommand{\myblock}[2]{\begin{block}{#1}#2\end{block}}


\newcommand{\votepurple}[1]{\textcolor{Purple}{$\bigstar$}}
\newcommand{\voteyellow}[1]{\textcolor{Goldenrod}{$\bigstar$}}
\newcommand{\voteblue}[1]{\textcolor{RoyalBlue}{$\bigstar$}}
\newcommand{\votepink}[1]{\textcolor{Pink}{$\bigstar$}}

\newcommand{\diff}{\mathop{}\!\mathrm{d}}
\newcommand{\refstyle}[1]{{\small{\textcolor{gray}{#1}}}}
\newcommand{\hands}[0]{\includegraphics[height=1.5em]{images/hands}}
\newcommand{\transpose}[0]{{\textrm{\tiny{\sf{T}}}}}
\newcommand{\norm}{{\mathcal{N}}}
\newcommand{\cutoff}[0]{\kappa}
\newcommand{\instD}[0]{\dataset}
\newcommand{\insts}[0]{\mathcal{I}}
\newcommand{\inst}[0]{i}
\newcommand{\pcs}[0]{\mathbf{\Lambda}}
\newcommand{\bx}[0]{\conf}
\newcommand{\conf}[0]{\mathbf{\lambda}}
\newcommand{\defconf}[0]{\mathbf{\lambda}_{\text{def}}}
\newcommand{\finconf}[0]{\mathbf{\lambda}^*}
\newcommand{\incumbent}[0]{\finconf}
\newcommand{\confs}[0]{\pcs}
%\newcommand{\vlambda}[0]{\bm{\lambda}}
%\newcommand{\vLambda}[0]{\bm{\Lambda}}
\newcommand{\dataset}[0]{\mathcal{D}}
\newcommand{\datasets}[0]{\mathbf{D}}
\newcommand{\loss}[0]{\mathcal{L}}

% \renewcommand{\vec}[1]{\mathbf{#1}}
\newcommand{\hist}[0]{\mathcal{H}}
\newcommand{\param}[0]{p}
\newcommand{\algo}[0]{\mathcal{A}}
\newcommand{\algos}[0]{\mathbf{A}}
%\newcommand{\nn}[0]{N}
\newcommand{\feats}[0]{\mathcal{F}}
\newcommand{\feat}[0]{\vec{f}}
\newcommand{\cluster}[0]{\vec{h}}
\newcommand{\clusters}[0]{\vec{H}}
\newcommand{\perf}[0]{\mathbb{R}}
%\newcommand{\surro}[0]{\mathcal{S}}
\newcommand{\surro}[0]{\hat{f}}
\newcommand{\func}[0]{f}
\newcommand{\epm}[0]{\surro}
\newcommand{\portfolio}[0]{\mathcal{P}}
\newcommand{\schedule}[0]{\mathcal{S}}
\newcommand{\mdata}[0]{\dataset_{\text{meta}}}

% Deep Learning
\newcommand{\weights}[0]{\theta}
\newcommand{\metaweights}[0]{\phi}


% reinforcement learning
\newcommand{\policies}[0]{\Pi}
\newcommand{\policy}[0]{\pi}
\newcommand{\actionRL}[0]{a}
\newcommand{\stateRL}[0]{s}
\newcommand{\statesRL}[0]{\mathcal{S}}
\newcommand{\rewardRL}[0]{r}
\newcommand{\rewardfuncRL}[0]{\mathcal{R}}

\RestyleAlgo{algoruled}
\DontPrintSemicolon
\LinesNumbered
\SetAlgoVlined
\SetFuncSty{textsc}

\SetKwInOut{Input}{Input}
\SetKwInOut{Output}{Output}
\SetKw{Return}{return}

%\newcommand{\changed}[1]{{\color{red}#1}}

%\newcommand{\citeN}[1]{\citeauthor{#1}~(\citeyear{#1})}

\renewcommand{\vec}[1]{\mathbf{#1}}
\DeclareMathOperator*{\argmin}{arg\,min}
\DeclareMathOperator*{\argmax}{arg\,max}

\newcommand{\aqme}{\textit{AQME}}
\newcommand{\aslib}{\textit{ASlib}}
\newcommand{\llama}{\textit{LLAMA}}
\newcommand{\satzilla}{\textit{SATzilla}}
\newcommand{\satzillaY}[1]{\textit{SATzilla'{#1}}}
\newcommand{\snnap}{\textit{SNNAP}}
\newcommand{\claspfolioTwo}{\textit{claspfolio~2}}
\newcommand{\flexfolio}{\textit{FlexFolio}}
\newcommand{\claspfolioOne}{\textit{claspfolio~1}}
\newcommand{\isac}{\textit{ISAC}}
\newcommand{\eisac}{\textit{EISAC}}
\newcommand{\sss}{\textit{3S}}
\newcommand{\sunny}{\textit{Sunny}}
\newcommand{\ssspar}{\textit{3Spar}}
\newcommand{\cshc}{\textit{CSHC}}  
\newcommand{\cshcpar}{\textit{CSHCpar}}  
\newcommand{\measp}{\textit{ME-ASP}} 
\newcommand{\aspeed}{\textit{aspeed}}
\newcommand{\autofolio}{\textit{AutoFolio}}
\newcommand{\cedalion}{\textit{Cedalion}}
\newcommand{\fanova}{\textit{fANOVA}}
\newcommand{\sbs}{\textit{SB}}
\newcommand{\oracle}{\textit{VBS}}

% like approaches
\newcommand{\claspfoliolike}[1]{\texttt{claspfolio-#1-like}}
\newcommand{\satzillalike}[1]{\texttt{SATzilla'#1-like}}
\newcommand{\isaclike}{\texttt{ISAC-like}}
\newcommand{\ssslike}{\texttt{3S-like}}
\newcommand{\measplike}{\texttt{ME-ASP-like}}

\newcommand{\aspCoseal}{\textit{ASP-POTASSCO}}
\newcommand{\cspCoseal}{\textit{CSP-2010}}
\newcommand{\maxsatCoseal}{\textit{MAXSAT12-PMS}}
\newcommand{\premarCoseal}{\textit{PRE\-MARSHALLING}}
\newcommand{\qbfCoseal}{\textit{QBF-2011}}
\newcommand{\satallTwelveCoseal}{\textit{SAT12-ALL}}
\newcommand{\sathandTwelveCoseal}{\textit{SAT12-HAND}}
\newcommand{\satinduTwelveCoseal}{\textit{SAT12-INDU}}
\newcommand{\satrandTwelveCoseal}{\textit{SAT12-RAND}}
\newcommand{\sathandElevenCoseal}{\textit{SAT11-HAND}}
\newcommand{\satinduElevenCoseal}{\textit{SAT11-INDU}}
\newcommand{\satrandElevenCoseal}{\textit{SAT11-RAND}}
\newcommand{\proteusCoseal}{\textit{PROTEUS-2014}}

\newcommand{\irace}{\textit{I/F-race}}
\newcommand{\gga}{\textit{GGA}}
\newcommand{\smac}{\textit{SMAC}}
\newcommand{\paramils}{\textit{ParamILS}}
\newcommand{\spearmint}{\textit{Spearmint}}
\newcommand{\tpe}{\textit{TPE}}

\newcommand{\gringo}{\textit{gringo}}
\newcommand{\clasp}{\textit{clasp}}
\newcommand{\lingeling}{\textit{lingeling}}

\newcommand{\hydra}{\textit{Hydra}}

\newcommand{\plingeling}{\textit{Plingeling}}
\newcommand{\ccasat}{\textit{CCASat}}

\usepackage{pifont}
\newcommand{\itarrow}{\mbox{\Pisymbol{pzd}{229}}}
\newcommand{\ithook}{\mbox{\Pisymbol{pzd}{52}}}
\newcommand{\itcross}{\mbox{\Pisymbol{pzd}{56}}}
\newcommand{\ithand}{\mbox{\raisebox{-1pt}{\Pisymbol{pzd}{43}}}}

%\DeclareMathOperator*{\argmax}{arg\,max}

\newcommand{\ie}{{\it{}i.e.\/}}
\newcommand{\eg}{{\it{}e.g.\/}}
\newcommand{\cf}{{\it{}cf.\/}}
\newcommand{\wrt}{\mbox{w.r.t.}}
\newcommand{\vs}{{\it{}vs\/}}
\newcommand{\vsp}{{\it{}vs\/}}
\newcommand{\etc}{{\copyedit{etc.}}}
\newcommand{\etal}{{\it{}et al.\/}}

\newcommand{\pscProc}{{\bf procedure}}
\newcommand{\pscBegin}{{\bf begin}}
\newcommand{\pscEnd}{{\bf end}}
\newcommand{\pscEndIf}{{\bf endif}}
\newcommand{\pscFor}{{\bf for}}
\newcommand{\pscEach}{{\bf each}}
\newcommand{\pscThen}{{\bf then}}
\newcommand{\pscElse}{{\bf else}}
\newcommand{\pscWhile}{{\bf while}}
\newcommand{\pscIf}{{\bf if}}
\newcommand{\pscRepeat}{{\bf repeat}}
\newcommand{\pscUntil}{{\bf until}}
\newcommand{\pscWithProb}{{\bf with probability}}
\newcommand{\pscOtherwise}{{\bf otherwise}}
\newcommand{\pscDo}{{\bf do}}
\newcommand{\pscTo}{{\bf to}}
\newcommand{\pscOr}{{\bf or}}
\newcommand{\pscAnd}{{\bf and}}
\newcommand{\pscNot}{{\bf not}}
\newcommand{\pscFalse}{{\bf false}}
\newcommand{\pscEachElOf}{{\bf each element of}}
\newcommand{\pscReturn}{{\bf return}}

%\newcommand{\param}[1]{{\sl{}#1}}
\newcommand{\var}[1]{{\it{}#1}}
\newcommand{\cond}[1]{{\sf{}#1}}
%\newcommand{\state}[1]{{\sf{}#1}}
%\newcommand{\func}[1]{{\sl{}#1}}
\newcommand{\set}[1]{{\Bbb #1}}
%\newcommand{\inst}[1]{{\tt{}#1}}
\newcommand{\myurl}[1]{{\small\sf #1}}

\newcommand{\Nats}{{\Bbb N}}
\newcommand{\Reals}{{\Bbb R}}
\newcommand{\extset}[2]{\{#1 \; | \; #2\}}

\newcommand{\vbar}{$\,\;|$\hspace*{-1em}\raisebox{-0.3mm}{$\,\;\;|$}}
\newcommand{\vendbar}{\raisebox{+0.4mm}{$\,\;|$}}
\newcommand{\vend}{$\,\:\lfloor$}


\newcommand{\goleft}[2][.7]{\parbox[t]{#1\linewidth}{\strut\raggedright #2\strut}}
\newcommand{\rightimage}[2][.3]{\mbox{}\hfill\raisebox{1em-\height}[0pt][0pt]{\includegraphics[width=#1\linewidth]{#2}}\vspace*{-\baselineskip}}





\newcommand{\inducer}{\mathcal{I}}
\newcommand{\R}{\mathds{R}}

%The following might look confusing but allows us to switch the notation of the optimization problem independently from the notation of the hyper parameter optimization
\newcommand{\xx}{\conf} %x of the optimizer
\newcommand{\xxi}[1][i]{\conf_{#1}} %i-th component of xx (not confuse with i-th individual)
\newcommand{\XX}{\pcs} %search space / domain of f
\newcommand{\f}{\cost} %objective function

\newenvironment{blocki}[1] % itemize block
{
 \begin{block}{#1}\begin{itemize}
}
{
\end{itemize}\end{block}
}

\title[AutoML: Hyperparameter Optimization]{AutoML: Hyperparameter Optimization}
%\subtitle{Overview for this Week} %To be defined in source!
%TODO: change authors!
\author[Marius Lindauer]{\underline{Bernd Bischl} \and Frank Hutter \and Lars Kotthoff\newline \and Marius Lindauer \and Joaquin Vanschoren}
\institute{}
\date{}


\usepackage[normalem]{ulem}
\usepackage{pifont}
\usepackage{relsize}
\renewcommand{\lit}[1]{{\smaller\color{black!60}[#1]}}
\subtitle{Wrap Up}


\begin{document}

\maketitle


%----------------------------------------------------------------------
%----------------------------------------------------------------------

\begin{frame}{From HPO to AutoML}
  So far we covered
  \begin{itemize}
    \item Mechanisms to select ML algorithms for a data set (algorithm selection)
    \item HPO as black-box optimization
    \begin{itemize}
      \item Grid- and random search, EAs, BO
    \end{itemize}
    \item HPO as a grey box problem
    \begin{itemize}
      \item Hyperband, BOHB
    \end{itemize}
    \item Neural Architecture Search (NAS)
    \begin{itemize}
      \item One-Shot approaches, DART
    \end{itemize}
    \item Dynamic algorithm configuration (learning to learn)
    \begin{itemize}
      \item Adapt configuration during training
    \end{itemize}
  \end{itemize}  
\end{frame}

\begin{frame}{From HPO to AutoML}
    \begin{center}
      \includegraphics[width = 0.9\linewidth]{images/18_AutoML-Components-Overview-Infographic_corrected.png}  
    \end{center}
\end{frame}

\begin{frame}{What is missing?}
  \begin{columns}
    \begin{column}{0.5\textwidth}
        What do I need to know as an AutoML user?
        \begin{itemize}
          \item \sout{Nothing, because it is automatic.}
          \item Understand limitations of AutoML and framework.
          \item Know how to interpret the results.
          \item Maybe: Preprocessing and feature extraction.
        \end{itemize}

        \vspace{1em}

        Ingredients to implement an AutoML?
        \begin{itemize}
          \item HPO algorithm
          \item ML / Pipeline framework 
          \item Parallelization / Multifidelity
          \item Process encapsulation and time capping 
          % \item (Preprocessing)
        \end{itemize}
    \end{column}%
    \begin{column}{0.5\textwidth}
      \begin{center}
        Academic view:
        \scalebox{0.45}{
          \begin{tikzpicture}[node distance=4cm, thick]
	\node (function) [data] {Cost $\cost$};
	\node (budgets) [data, below of=function, node distance=1cm] {Budgets};
	\node (space) [data, below of=budgets, node distance=1cm] {Design Space $\pcs$};
  \node (resampling) [data, below of=space, node distance=1cm] {Resampling};
	
	\node (hb) [activity, right of=space, node distance=6cm, yshift=-.5cm] {ML System};
	\node (kde) [activity, above of=hb, node distance=2cm] {AutoML Optimizer};
	
	\draw[myarrow] ($(kde.south)+(-0.3,0.0)$) -- ++(0.0,-0.6) node[left] {$\conf \in \pcs$} -- ($(hb.north)+(-0.3,+0.0)$);
	\draw[myarrow] ($(hb.north)+(0.3,+0.0)$) -- ++(0.0,0.6) node[right] {$\cost(\conf)$} -- ($(kde.south)+(0.3,0.0)$);
	
	\draw[myarrow] (function.east) -- ($(kde.west)+(-0.3,0.5)$);
	\draw[myarrow] (budgets.east) -- ($(kde.west)+(-0.3,-0.5)$);
	\draw[myarrow] (space.east) -- ($(kde.west)+(-0.3,-1.5)$);
  \draw[myarrow] (resampling.east) -- ($(kde.west)+(-0.3,-2.5)$);
	
	\node (perf) [activity, right of=kde, node distance=6.3cm] {Performance Analysis};
	\node (budget) [activity, below of=perf, node distance=1.2cm] {Incumbent Analysis};
	\node (imp) [activity, below of=budget, node distance=1cm] {Space Analysis};
	
	\draw[myarrow] ($(kde.east)+(0.3,-1.)$) -- node[above] {$\langle \conf^{(i)}, \cost(\confI{i}) \rangle_i$} ($(perf.west)+(-0.3,-1.)$);
	\draw[myarrow] ($(kde.east)+(0.3,-1.)$) -- node[below] {$\incumbent$} ($(perf.west)+(-0.3,-1.)$);
	
	\begin{pgfonlayer}{background}
	
	% Configuration Process
	\path (kde -| kde.west)+(-0.25,0.85) node (resUL) {};
	\path (hb.east |- hb.south)+(0.25,-0.5) node(resBR) {};
	\path [rounded corners, draw=black!60, dashed] (resUL) rectangle (resBR);
	\path (hb.east |- hb.south)+(-.5,-.1) node [text=black!60] {};
	
	\path (perf -| perf.west)+(-0.25,0.85) node (resUL) {};
	\path (imp.east |- imp.south)+(0.25,-0.5) node(resBR) {};
	\path [rounded corners, draw=black!60, dashed] (resUL) rectangle (resBR);
	\path (imp.east |- imp.south)+(-.5,-.2) node [text=black!60] {};
	
	\end{pgfonlayer}

\end{tikzpicture}
        }

        \vspace{1em}

        Practitioners view:
        \scalebox{0.45}{
          \begin{tikzpicture}[node distance=4cm, thick]
  \node (function) [data] {Cost $\cost$};
  \node (budget) [data, below of=function, node distance=1cm] {Budget};
  \node (dataset) [data, below of=budget, node distance=1cm] {Dataset};
  
  \node (hb) [activity, right of=dataset, node distance=6cm, yshift=-0.0cm] {ML System};
  \node (kde) [activity, above of=hb, node distance=2cm] {AutoML Optimizer};
  \node (resampling) [data, below of=hb, node distance=1cm] {Resampling};
  \node (space) [data, above of=kde, node distance=1cm] {Design Space $\pcs$};
  
  \draw[myarrow] ($(kde.south)+(-0.3,0.0)$) -- ++(0.0,-0.6) node[left] {$\conf \in \pcs$} -- ($(hb.north)+(-0.3,+0.0)$);
  \draw[myarrow] ($(hb.north)+(0.3,+0.0)$) -- ++(0.0,0.6) node[right] {$\cost(\conf)$} -- ($(kde.south)+(0.3,0.0)$);
  
  \draw[myarrow] (function.east) -- ($(kde.west)+(-0.3,0.0)$);
  \draw[myarrow] (budget.east) -- ($(kde.west)+(-0.3,-1.)$);
  \draw[myarrow] (dataset.east) -- ($(kde.west)+(-0.3,-2.)$);

  \draw[myarrow] (space.south) -- ($(kde.north)+(-0.0,0.0)$);
  \draw[myarrow] (resampling.north) -- ($(hb.south)+(-0.0,0.0)$);
  
  \node (perf) [activity, right of=kde, node distance=6.3cm] {Analysis};
  \node (budget) [activity, below of=perf, node distance=1.2cm] {Deployment};
  %\node (imp) [activity, below of=budget, node distance=1cm] {Space Analysis};
  
  \draw[myarrow] ($(kde.east)+(0.3,-1.)$) -- node[above] {$\langle \conf^{(i)}, \cost(\confI{i}) \rangle_i$} ($(perf.west)+(-0.3,-1.)$);
  \draw[myarrow] ($(kde.east)+(0.3,-1.)$) -- node[below] {$\finconf$} ($(perf.west)+(-0.3,-1.)$);
  
  \begin{pgfonlayer}{background}
  
  % Configuration Process
  \path (space -| space.west)+(-0.25,0.5) node (resUL) {};
  \path (resampling.east |- resampling.south)+(0.25,-0.3) node(resBR) {};
  \path [rounded corners, draw=black!60, dashed] (resUL) rectangle (resBR);
  \path (resampling.east |- resampling.south)+(-.5,-.15) node [text=black!60] {};
  
  \path (perf -| perf.west)+(-0.25,0.85) node (resUL) {};
  \path (budget.east |- budget.south)+(0.25,-0.5) node(resBR) {};
  \path [rounded corners, draw=black!60, dashed] (resUL) rectangle (resBR);
  \path (budget.east |- budget.south)+(-.5,-.3) node [text=black!60] {};
  
  \end{pgfonlayer}

\end{tikzpicture}
        }
      \end{center}
    \end{column}
  \end{columns}
\end{frame}

\begin{frame}{Choice of Learning Algorithm}
  \begin{itemize}
    % \item A good AutoML System should consider more than one learning algorithm. More on that later.
    \item A plethora of learners exists, for different data sets different models
        are likely needed.

        
    \item Studies and experience show:\\

        One these is often good -- on tabular data:
    \begin{itemize}
      \item Penalized regression, e.g. elastic net
      \item Support vector machines
      \item Gradient boosting
      \item Random forests
      \item Neural networks
  \end{itemize}
      % \item Random forests only beaten on few tabular datasets by current AutoML frameworks~\lit{\href{https://arxiv.org/abs/1907.00909}{Gijsbers et al., 2019}}.

      \item Example: Auto-Sklearn 2.0~\lit{{\href{https://arxiv.org/pdf/2007.04074.pdf}{Feurer et al. 2020}}} uses: 

    \begin{itemize}
        \item Extra trees 
        \item Gradient boosting 
        \item Passive aggressive 
        \item Random forest 
        \item Linear regression with SGD
        \item Multi-layer perceptron
  \end{itemize}
    \end{itemize}
  % \end{itemize}
\end{frame}

\begin{frame}{Choice of Search Space for a Learning Algorithm}
  \begin{columns}
  	
  	\begin{column}{0.4\textwidth}
  		\begin{center}
  			\only<1>{
  				\includegraphics[width = 0.8\linewidth]{images/probst2019jmlr_tab1.pdf}
  			}
  			\only<2->{
  				\includegraphics[width = 0.8\linewidth]{images/probst2019jmlr_tab3.pdf}   
  			}
  			
  			{\tiny Source: \lit{\href{https://www.jmlr.org/papers/volume20/18-444/18-444.pdf}{Probst et al. 2019 }}.}
  			
  		\end{center}
  	\end{column}
  	
    \begin{column}{0.6\textwidth}
    
    Ranges often selected based on experience
    \begin{itemize}

      \item See other AutoML frameworks: e.g.\ Auto-Sklearn~2.0 \lit{\href{https://arxiv.org/pdf/2007.04074.pdf}{Feurer et al. 2020}}

      \item Sensitivity analysis often does not exist for ML~algorithms
      \item Check literature on specific ML algorithm
    \end{itemize}
	\pause
    Options for automation:
    \begin{enumerate}
      \item Use huge search space to cover all possibilities \\ 
            (combine with meta-learning for good initial design for Bayesian optimization)
      \begin{itemize} 
      		\item Use results of meta-experiments to obtain smaller search space that is estimated to work well.
       \end{itemize}
   		\pause
   	  \item Start with a small space and increase bit by bit
    \end{enumerate}
    \end{column}%

  \end{columns}
\end{frame}

\begin{frame}{Choice of Resampling Strategy}
  
  For computation of generalization error / cost:
  \begin{equation*}
    \cost(\conf) = \frac{1}{k}\sum_{i = 1}^k \widehat{GE}_{\dataset_{\text{val}}^i}\left(\inducer(\dataset_{\text{train}}^i, \conf)\right)
  \end{equation*}
  % that defines the objective of the black-box optimization we need a resampling strategy.

  \vspace{1em}
  \begin{columns}
    \begin{column}{0.5\textwidth}
    Rules of thumb:
    \begin{itemize}
      \item Default: 10-fold CV ($k=10$)
      \item Huge datasets: holdout
      \item Tiny datasets: 10x10 repeated CV
      \item Stratification for imbalanced classes
    \end{itemize}
    \end{column}
    
    \begin{column}{0.5\textwidth}
    	\pause
 Watch out for this:       
    \begin{itemize}
      \item Small sample size because of imbalances    
      \item Repeated mesurements (leave-one-object out)
      \item Time dependencies
      \item A good AutoML system should let you customize resampling
          % errors here can mean: garbage in, garbvgae out
      \item Meta-learn good resampling strategy~\lit{\href{https://arxiv.org/pdf/2007.04074.pdf}{Feurer et al. 2020}}
    \end{itemize}
    \end{column}
    \end{columns}
    
\end{frame}

\begin{frame}{Choice of Optimization Algorithm}
  Choose optimization algorithm based on \ldots
  \begin{itemize}
    \item complexity of search space / budget
    \item time-costs of evaluations
  \end{itemize}

  \vspace{0.5em}

  Complex search space
  \begin{itemize}
    \item[$\rightarrow$] BO with RF surrogate, EA with exploratory character, TPE 
    % \item[$\rightarrow$] Make use of Grey-Box Optimizers: Hyperband, BOHB
  \end{itemize}
  \pause
  Numerical (lower-dim) search space and tight budget
  \begin{itemize}
    \item[$\rightarrow$] BO with GP surrogate\footnote{Still has its own hyperparameters \lit{\href{https://arxiv.org/abs/1908.06674}{Lindauer et al. 2019}}}
  \end{itemize}
  \pause
  Expensive evaluations
  \begin{itemize}
    \item[$\rightarrow$] Hyperband, BOHB, DEHB
  \end{itemize}
  \pause
  Deep learning 
  \begin{itemize}
    \item[$\rightarrow$] common practice: Parameterize architectures, then HPO -- better do it jointly!
    \item[$\rightarrow$] one-shot models and gradient-based optimization
  \end{itemize}

\end{frame}


\end{document}
