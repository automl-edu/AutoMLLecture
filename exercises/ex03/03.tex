\documentclass{exam}
\usepackage{amsmath, amsfonts}
\usepackage{verbatim}
\usepackage{graphicx}
\usepackage[super]{nth}

\DeclareMathOperator*{\argmin}{argmin}

\usepackage[hyperfootnotes=false]{hyperref}

\usepackage[usenames,dvipsnames]{color}
\newcommand{\note}[1]{
	\noindent~\\
	\vspace{0.25cm}
	\fcolorbox{Red}{Orange}{\parbox{0.99\textwidth}{#1\\}}
	%{\parbox{0.99\textwidth}{#1\\}}
	\vspace{0.25cm}
}


\renewcommand{\vec}[1]{\mathbf{#1}}
\newcommand{\lecture}{ML4AAD}
\newcommand{\lecturelong}{Machine Learning for Automated Algorithm Design}
\newcommand{\semester}{WS 2018/19}
\newcommand{\assignment}[1]{\nth{#1} Assignment}
\newcommand{\lectors}{M. Lindauer \& A. Biedenkapp}
\newcommand{\hide}[1]{}


\newcommand{\gccs}{\paragraph{General constraints for code submissions}
    
    \begin{itemize}
        \item The program can be called as stated on the exercise sheet.
        \item The program exactly returns the required output (neither less nor more) -- please use a \texttt{--verbose} option to increase the verbosity level for debugging\footnote{You might want to use \texttt{argparse} for simplicity's sake.}.
        \item Your scripts should be commented to be readable for the tutors. All functions and classes are documented with a docstring. 
        \item Provide a README ($\to$ how to install requirements and run your program(s)) and (if necessary) an installation script if your program requires any other packages.
        \item Programs are to be submitted in python $3.5$ or newer.
        \item Adding new packages to the requirements.txt is fine. If you do this however, you'll have to give a brief description why you use that package and a link to it's documentation or github page.
        \item All prepared unittests have to pass.
        \item We don't accept ipython notebook submissions.
        \item Points will be deducted if you don't fullfill these constraints.
        \item You are allowed (sometimes required) to reuse code from previous exercises.
    \end{itemize}
    \rule{\textwidth}{.5pt}
    \smallskip\\
    \noindent}
%\renewcommand{\hide}[1]{#1}

\qformat{\thequestion. \textbf{\thequestiontitle}\hfill[\thepoints]}
\bonusqformat{\thequestion. \textbf{\thequestiontitle}\hfill[\thepoints]}

\pagestyle{headandfoot}

%%%%%% MODIFY FOR EACH SHEET!!!! %%%%%%
\newcommand{\duedate}{17.05.19 (10:00)}
\newcommand{\due}{{\bf This assignment is due on \duedate.} }
\firstpageheader
{Due: \duedate \\ Points: 4}
{{\bf\lecture}\\ \assignment{3}}
{\lectors\\ \semester}

\runningheader
{Due: \duedate}
{\assignment{5}}
{\semester}
%%%%%% MODIFY FOR EACH SHEET!!!! %%%%%%

\firstpagefooter
{}
{\thepage}
{}

\runningfooter
{}
{\thepage}
{}

\headrule
\pointsinrightmargin
\bracketedpoints
\marginpointname{pt.}


\begin{document}
	\gccs
	Having learned about different ways to empirically evaluate the performances of algorithms and automl systems in this exercise you will now implement some of these techniques. Add your code creating plots and outputting statistics to \texttt{main.py} (callable as \verb|python main.py|). Furthermore, combine all plots and answers to the questions into one PDF. Needless to say, that we expect all plots to have axes labels and a legend.
	\begin{questions}
		\titledquestion{Visualization and Evaluation}[3]
		\verb|data.csv| contains \textit{RMSE} score of two algorithms $A$ and $B$ on $n=419$ datasets. The method \verb|load_data()| loads this dataset as an $n \times 2$ \textit{numpy array}.
		\begin{parts}
			\part[1] Create a scatterplot comparing the performance of $A$ and $B$. Use different markers for the following three categories of datasets:
			\begin{itemize}
				\item algorithm $A$ achieved an \textit{RMSE} value that is $0.1$ lower than that of $B$
				\item algorithm $B$ achieved an \textit{RMSE} value that is $0.1$ lower than that of $A$
				\item other
			\end{itemize}
			How many datasets are in each category? Report overall mean and standard deviation as well as median and quartile values for $A$ and $B$.
			\part[0.5] Plot the empirical cumulative distribution function (eCDF) of $A$ and $B$ as shown in the lecture. What are the probabilities of $A$ and $B$ to achieve an \textit{RMSE} value of $0.4$ and lower (it is okay to estimate the value from the plot).
			\part[0.5] Create a boxplot\footnote{using e.g. \url{https://matplotlib.org/api/_as_gen/matplotlib.axes.Axes.violinplot.html}} and a violin plot\footnote{using e.g. \url{https://matplotlib.org/api/_as_gen/matplotlib.pyplot.boxplot.html}} for $A$ and $B$. Write one sentence to describe the difference between these two kinds of plots.
			\part[1] Implement a paired permutation test to determine whether $A$ and $B$ have equal means. In your solution state what is $H_0$, $H_1$ and which $\alpha$-value you used?
		\end{parts}
	
	\titledquestion{Feedback}[Bonus: 1]
	For each question in this assignment, state:
	\begin{itemize}
	\item How long you worked on it.
	\item What you learned.
	\item Anything you would improve in this question if you were teaching the course.
	\end{itemize}
	\end{questions}
	
	\noindent
	\due Submit your solution for the tasks by uploading a PDF to your groups BitBucket repository. The PDF has to include the name of the submitter(s). Teams of at most $2$ students are allowed.
\end{document}