\documentclass[10pt,a4paper]{article}
\usepackage{a4wide}
\usepackage[hyperfootnotes=false]{hyperref}


\renewcommand{\vec}[1]{\mathbf{#1}}
\newcommand{\lecture}{ML4AAD}
\newcommand{\lecturelong}{Machine Learning for Automated Algorithm Design}
\newcommand{\semester}{WS 2018/19}
\newcommand{\assignment}[1]{\nth{#1} Assignment}
\newcommand{\lectors}{M. Lindauer \& A. Biedenkapp}
\newcommand{\hide}[1]{}


\newcommand{\gccs}{\paragraph{General constraints for code submissions}
    
    \begin{itemize}
        \item The program can be called as stated on the exercise sheet.
        \item The program exactly returns the required output (neither less nor more) -- please use a \texttt{--verbose} option to increase the verbosity level for debugging\footnote{You might want to use \texttt{argparse} for simplicity's sake.}.
        \item Your scripts should be commented to be readable for the tutors. All functions and classes are documented with a docstring. 
        \item Provide a README ($\to$ how to install requirements and run your program(s)) and (if necessary) an installation script if your program requires any other packages.
        \item Programs are to be submitted in python $3.5$ or newer.
        \item Adding new packages to the requirements.txt is fine. If you do this however, you'll have to give a brief description why you use that package and a link to it's documentation or github page.
        \item All prepared unittests have to pass.
        \item We don't accept ipython notebook submissions.
        \item Points will be deducted if you don't fullfill these constraints.
        \item You are allowed (sometimes required) to reuse code from previous exercises.
    \end{itemize}
    \rule{\textwidth}{.5pt}
    \smallskip\\
    \noindent}

\newcommand{\duedate}{20.09.19 (23:59)}
\newcommand{\due}{{\bf This project is due on \duedate.} }

\usepackage{fancyhdr}
\pagestyle{fancy}

\fancyhf{}
\lhead{Due: \duedate}
\chead{{\bf AutoML}\\Final Project}
\rhead{\lectors\\ \semester}
\cfoot{Page \thepage}

\begin{document}
	\paragraph{The final project} is part of the oral exam, which means you are \textit{not allowed to work in groups}.
	The purpose of this project is that you get hands-on experience on most topics of the course and to show that you can present and explain the results of your work. 
	To this end, please submit your code, plots, tables etc. to a bitbucket repository to which at least one organizer of the lecture has access.\footnote{
	For bitbucket invite user \texttt{biedenka}.}

	In the first $15$ minutes of the exam, you will present your approach and the results such that we can discuss it together.
	It is important that \textit{your evaluation builds the basis for discussion and scientifically analyzes which are the important aspects and characteristics of your approach}---you should present your findings in a convincing manner.
	
	To give the project presentation some structure, you will have to prepare a few presentation slides.
	Your slides should consist of a motivation slide, slides detailing your approach (2-3) as well as slides for your results (2-3).
	You are allowed to submit at most 5 slides.
	\textit{Don't go overboard with your slides.}
	They are intended to make your presentation coherent.
	
	\section*{Optimization of a Convolutional Neural Network}
		
		You are tasked with the improve and analysis of the performance of a neural network on the \emph{Kuzushiji-49}\footnote{\url{https://github.com/rois-codh/kmnist}} dataset.
		How you improve the performance of the given network by means of AutoML is up to you. 
		For example, you could optimize the hyperparameter of the networks optimizer, apply neural architecture search or a joint approach.
		In the end, you should convince us that you indeed improved the performance of the network when compared to a provided default configuration.
		To this end, you could consider the following tasks:
		
		\begin{itemize}
			\item Measure and compare against the default performance of the given network;
			\item Apply HPO to obtain a well-performing hyperparameter configuration;
			\item Apply NAS (e.g., BOHB, Lemonade, DARTS) to improve the architecture of the network;
			\item Optimize the preprocessing and regularization;
			\item Apply learning curve predictions to speed up HPO/NAS;
			\item Determine the importance of the algorithm's hyperparameters;
			\item Apply algorithm selection to select a well-performing architecture;
			\item Apply transfer-learning from similar datasets (e.g., CIFAR-10, Omniglot or Kuzushiji-MNIST)
			\item Apply HPO/NAS warmstarting trained on similar datasets 
			\item Apply a learning to learn approach to learn how to optimize the network;
%			\item Extend the configuration space to include preprocessing steps/dropout/skip connections/...;
			\item Plot a confusion matrix;
			\item Plot the performance of your AutoML approach over time.
		\end{itemize}
		\noindent
		Please note that you do not have to necessarily apply all these methods -- pick the ones that you think are most appropriate.
		We provide a repository (\url{TODO_Arber}) for you to fork\footnote{\url{https://help.github.com/articles/fork-a-repo/}}, in which we will upload the following:
		\begin{itemize}
			\item A pytorch implementation to access the \textit{K49} dataset
			\item A baseline parameterized network to optimize (also written in pytorch)
			\item An example script to show you how to train and evaluate an network based on the default configuration
		\end{itemize}

		\noindent		
		You are allowed to use all scripts and tools you already know from the exercises; however, you are not limited to them.
		Overall, you should respect the following constraints:
		\begin{itemize}
			\item \textbf{Metric:}
			\begin{itemize}
				\item The final performance has to be measured in terms of classification accuracy.\\
			\end{itemize}
			\item \textbf{Experimental Constraints:}
			\begin{itemize}
				\item Your code for making design decisions should run no longer than $86\,400$ seconds (without validation).
				\item Training a network can take at most 20 epochs.
				\item You can use at most 4 GPUs on the provided remote machines with at most $\frac{86\,400}{4}$ seconds per GPU.
			\end{itemize}
			\item \textbf{Implementation Constraints:}
			\begin{itemize}
			  \item You can freely extend the baseline implementation provided to you. However, your search space should always include the given default network.
			  \item You are not allowed to use manual tuning to achieve better performance. All improved design decisions have to be made (somehow) automatically.
			\end{itemize}
			\item \textbf{Grading Guidelines:}
			\begin{itemize}
			  \item Only applying tools presented in the lectures (such as BOHB) is a valid, but not a very interesting approach and will only earn you, at best, a 2.0 as a grade for the project part. To obtain a better grade, you have to come up some own ideas.
			  \item The \textit{K49} website\footnote{\url{https://github.com/rois-codh/kmnist}} provides some results of baseline implementations. We expect that you achieve an accuracy of at least $90\%$. You get bonus points by achieving better results close to the results on the homepage.
			\end{itemize}
		\end{itemize}
		
		We provide a Google spreadsheet\footnote{\url{https://docs.google.com/spreadsheets/d/1dkoSwDr8qzuhyqhTs3_DC6_pj8Wok9kXWcQqy3Tjkuo/edit?usp=sharing}} in which you can upload your current progress. This sheet will not be monitored by us but gives you the opportunity to compare your results. This might help you identify early if your approach is working well or not.
\vspace*{\fill}\\
\noindent
\due Submit your presentation for the exam by sending a PDF of your slides to \texttt{biedenka@cs.uni-freiburg.de}. \textbf{No teams are allowed for the final project.}
\end{document}