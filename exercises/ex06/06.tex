\documentclass{exam}
\usepackage{amsmath, amsfonts}
\usepackage{bm}
\usepackage{verbatim}
\usepackage{graphicx}
\usepackage[super]{nth}
\usepackage{booktabs}

\DeclareMathOperator*{\argmin}{argmin}

\usepackage[hyperfootnotes=false]{hyperref}

\usepackage[usenames,dvipsnames]{color}
\newcommand{\note}[1]{
	\noindent~\\
	\vspace{0.25cm}
	\fcolorbox{Red}{Orange}{\parbox{0.99\textwidth}{#1\\}}
	%{\parbox{0.99\textwidth}{#1\\}}
	\vspace{0.25cm}
}


\renewcommand{\vec}[1]{\mathbf{#1}}
\newcommand{\lecture}{ML4AAD}
\newcommand{\lecturelong}{Machine Learning for Automated Algorithm Design}
\newcommand{\semester}{WS 2018/19}
\newcommand{\assignment}[1]{\nth{#1} Assignment}
\newcommand{\lectors}{M. Lindauer \& A. Biedenkapp}
\newcommand{\hide}[1]{}


\newcommand{\gccs}{\paragraph{General constraints for code submissions}
    
    \begin{itemize}
        \item The program can be called as stated on the exercise sheet.
        \item The program exactly returns the required output (neither less nor more) -- please use a \texttt{--verbose} option to increase the verbosity level for debugging\footnote{You might want to use \texttt{argparse} for simplicity's sake.}.
        \item Your scripts should be commented to be readable for the tutors. All functions and classes are documented with a docstring. 
        \item Provide a README ($\to$ how to install requirements and run your program(s)) and (if necessary) an installation script if your program requires any other packages.
        \item Programs are to be submitted in python $3.5$ or newer.
        \item Adding new packages to the requirements.txt is fine. If you do this however, you'll have to give a brief description why you use that package and a link to it's documentation or github page.
        \item All prepared unittests have to pass.
        \item We don't accept ipython notebook submissions.
        \item Points will be deducted if you don't fullfill these constraints.
        \item You are allowed (sometimes required) to reuse code from previous exercises.
    \end{itemize}
    \rule{\textwidth}{.5pt}
    \smallskip\\
    \noindent}
%\renewcommand{\hide}[1]{#1}

\qformat{\thequestion. \textbf{\thequestiontitle}\hfill[\thepoints]}
\bonusqformat{\thequestion. \textbf{\thequestiontitle}\hfill[\thepoints]}

\pagestyle{headandfoot}

%%%%%% MODIFY FOR EACH SHEET!!!! %%%%%%
\newcommand{\duedate}{07.06.19 (10:00)}
\newcommand{\due}{{\bf This assignment is due on \duedate.} }
\firstpageheader
{Due: \duedate \\ Points: 6}
{{\bf\lecture}\\ \assignment{6}}
{\lectors\\ \semester}

\runningheader
{Due: \duedate}
{\assignment{6}}
{\semester}
%%%%%% MODIFY FOR EACH SHEET!!!! %%%%%%

\firstpagefooter
{}
{\thepage}
{}

\runningfooter
{}
{\thepage}
{}

\headrule
\pointsinrightmargin
\bracketedpoints
\marginpointname{pt.}




\newcommand{\parents}{p}
\newcommand{\negation}[1]{\overline{#1}}
%\newcommand{\tuple}[1]{\langle #1 \rangle}
\newcommand{\tuple}[1]{\left<#1\right>}
\newcommand{\dom}[1]{dom(#1)}              % domain

\newcommand{\false}{false}
\newcommand{\true}{true}
\newcommand{\TRUE}{{\mbox{\scriptsize \em TRUE}}}
\newcommand{\FALSE}{{\mbox{\scriptsize \em FALSE}}}

\newcommand{\bSigma}{\bm{\Sigma}}
\newcommand{\bmu}{\bm{\mu}}
\newcommand{\bx}{\bm{x}}
\newcommand{\by}{\bm{y}}
\newcommand{\bX}{\bm{X}}
\newcommand{\bI}{\bm{I}}
\newcommand{\bw}{\bm{w}}
\newcommand{\ba}{\bm{a}}
\newcommand{\bb}{\bm{b}}
\newcommand{\bk}{\bm{k}}  
\newcommand{\inv}{^{-1}}

\newcommand{\norm}{{\mathcal{N}}}

\newcommand\transpose{^{\textrm{\tiny{\sf{T}}}}}

\begin{document}
	\gccs
	Now that you have learned about the theory behind GPs, you will have to use that theory to implement GPs yourself.
	\begin{questions}
		\titledquestion{Gaussian Processes}[6]
		The exercise is mostly concerned with equations 2.11 and 2.12 in chapter 2 of "Gaussian Processes for Machine Learning\footnote{\url{http://www.gaussianprocess.org/gpml/chapters/RW2.pdf}}"
		\begin{parts}
		\part[2] Equation 2.11 gives the predictive distribution using an explicit feature space formulation as:
		\begin{equation*}
		f_{\star} | \bx_{\star}, \bX, \by \sim \norm(\frac{1}{\sigma_n^2} \phi(\bx_{\star})\transpose A\inv \Phi \by,
		\phi(\bx_{\star})\transpose A\inv \phi(\bx_{\star}))
		\end{equation*}
		where  $\Phi = \phi(\bX) \text{ and } A = \sigma_n^{-2} \Phi \Phi\transpose + \Sigma_p\inv$.\\
		Equation 2.12 is an alternative formulation that requires only an inversion of an $n\times n$ matrix instead of an $N\times N$ one, where $n$ is the number of data points and $N$ the number
		of features. Equation 2.12 is given as:
		\begin{equation*}
		f_{\star} | \bx_{\star}, \bX, \by \sim \norm (\phi_{\star}\transpose \Sigma_p \Phi(\Phi\transpose \Sigma_p \Phi + \sigma_n^2I)^{-1}\by, \\
		\nonumber{}  \phi_{\star}\transpose \Sigma_p\phi_{\star} - \phi_{\star}\transpose \Sigma_p \Phi(\Phi\transpose \Sigma_p \Phi + \sigma_n^2I)\inv
		\Phi\transpose \Sigma_p \phi_{\star})
		\end{equation*}
		Show the equivalence of 2.11 and 2.12.\\
		(\textit{It suffices to show that $\frac{1}{\sigma_n^2} \phi(\bx_{\star})\transpose A\inv \Phi \by = \phi_\star\transpose \Sigma_p\Phi (\Phi\transpose\Sigma_{p}\Phi + \sigma_n^2I )\inv \by$}).
		
		Try to apply the following rules wherever possible and clearly state at each step what you did.
		\begin{table}[h]\centering
			\begin{tabular}{cl}
				Rules & \\
				\toprule
				(1) & $(AB)\inv = B\inv A\inv$ \\
				(2) & $A(BC) = (AB)C$ \\
				(3) & $I = BB\inv = \Sigma_p\Sigma_p\inv = (\Phi \Phi\transpose)(\Phi \Phi\transpose)\inv$ \\
				\bottomrule
			\end{tabular}
		\end{table}
	
		With $I$ the identity matrix. \textit{Please tex your solution.}
		
		\part[2] Your second task is to implement 2.11 to predict the mean and variance given some observations.
		We provide a function prototype that takes the observations, a function $\phi$, and an array of points, $[\vec x_{1}, \vec x_{2}, \dots] $, as arguments. Compute the mean $\mu$ and the variance $\sigma^2$ at all $\vec x$. For matrix inversion you can use \texttt{numpy.linalg.inv}. Use the data provided in your source folder to create a plot that shows the mean and the $2\sigma$ confidence interval around it\footnote{You can use \texttt{matplotlib.pyplot.fill\_between} to generate the confidence interval.} for a feature space of size $N=2$. Use  $\sigma_n = 1$ and $\Sigma_p = I$ and
		\begin{equation*}
		\phi(x) = (1,x)^T
		\label{eq:phi_of_x}
		\end{equation*}
		which corresponds to Bayesian \textit{linear} regression for one dimensional input.
		
		\part[2] Finally, implement a second function based on 2.12. Convince yourself, that both yield the same values, by checking the output for the given input\footnote{By the nature of numerical calculations, the results will not be identical, but the difference will be very small. Use \texttt{numpy.allclose} to test for equality.}. Compare the time it takes for both implementations to compute the output for the given data, and points $\vec x_{i}$, for a growing number of features. Again use $\sigma_n = 1$ and $\Sigma_p = I$ and
		\begin{equation}
		\phi_n(x) = (1,x,x^2,\dots,x^{n-1})^T
		\label{eq:phi_n_of_x}
		\end{equation}
		to plot the computing time for $n=2,4,8,\dots,2048, 4096$. 
	\end{parts}
		
		\titledquestion{Feedback}[Bonus: 0.5]
		For each question in this assignment, state:
		\begin{itemize}
			\item How long you worked on it.
			\item What you learned.
			\item Anything you would improve in this question if you were teaching the course.
		\end{itemize}
	\end{questions}
	
	\noindent
	\due Submit your solution for the tasks by uploading a PDF to your groups BitBucket repository. The PDF has to include the name of the submitter(s).
\end{document}