\documentclass{exam}
\usepackage{amsmath, amsfonts}
\usepackage{bm}
\usepackage{verbatim}
\usepackage{graphicx}
\usepackage[super]{nth}
\usepackage{booktabs}

\DeclareMathOperator*{\argmin}{argmin}

\usepackage[hyperfootnotes=false]{hyperref}

\usepackage[usenames,dvipsnames]{color}
\newcommand{\note}[1]{
	\noindent~\\
	\vspace{0.25cm}
	\fcolorbox{Red}{Orange}{\parbox{0.99\textwidth}{#1\\}}
	%{\parbox{0.99\textwidth}{#1\\}}
	\vspace{0.25cm}
}


\renewcommand{\vec}[1]{\mathbf{#1}}
\newcommand{\lecture}{ML4AAD}
\newcommand{\lecturelong}{Machine Learning for Automated Algorithm Design}
\newcommand{\semester}{WS 2018/19}
\newcommand{\assignment}[1]{\nth{#1} Assignment}
\newcommand{\lectors}{M. Lindauer \& A. Biedenkapp}
\newcommand{\hide}[1]{}


\newcommand{\gccs}{\paragraph{General constraints for code submissions}
    
    \begin{itemize}
        \item The program can be called as stated on the exercise sheet.
        \item The program exactly returns the required output (neither less nor more) -- please use a \texttt{--verbose} option to increase the verbosity level for debugging\footnote{You might want to use \texttt{argparse} for simplicity's sake.}.
        \item Your scripts should be commented to be readable for the tutors. All functions and classes are documented with a docstring. 
        \item Provide a README ($\to$ how to install requirements and run your program(s)) and (if necessary) an installation script if your program requires any other packages.
        \item Programs are to be submitted in python $3.5$ or newer.
        \item Adding new packages to the requirements.txt is fine. If you do this however, you'll have to give a brief description why you use that package and a link to it's documentation or github page.
        \item All prepared unittests have to pass.
        \item We don't accept ipython notebook submissions.
        \item Points will be deducted if you don't fullfill these constraints.
        \item You are allowed (sometimes required) to reuse code from previous exercises.
    \end{itemize}
    \rule{\textwidth}{.5pt}
    \smallskip\\
    \noindent}
%\renewcommand{\hide}[1]{#1}

\qformat{\thequestion. \textbf{\thequestiontitle}\hfill[\thepoints]}
\bonusqformat{\thequestion. \textbf{\thequestiontitle}\hfill[\thepoints]}

\pagestyle{headandfoot}

%%%%%% MODIFY FOR EACH SHEET!!!! %%%%%%
\newcommand{\duedate}{07.06.19 (10:00)}
\newcommand{\due}{{\bf This assignment is due on \duedate.} }
\firstpageheader
{Due: \duedate \\ Points: 6}
{{\bf\lecture}\\ \assignment{6}}
{\lectors\\ \semester}

\runningheader
{Due: \duedate}
{\assignment{6}}
{\semester}
%%%%%% MODIFY FOR EACH SHEET!!!! %%%%%%

\firstpagefooter
{}
{\thepage}
{}

\runningfooter
{}
{\thepage}
{}

\headrule
\pointsinrightmargin
\bracketedpoints
\marginpointname{pt.}




\newcommand{\parents}{p}
\newcommand{\negation}[1]{\overline{#1}}
%\newcommand{\tuple}[1]{\langle #1 \rangle}
\newcommand{\tuple}[1]{\left<#1\right>}
\newcommand{\dom}[1]{dom(#1)}              % domain

\newcommand{\false}{false}
\newcommand{\true}{true}
\newcommand{\TRUE}{{\mbox{\scriptsize \em TRUE}}}
\newcommand{\FALSE}{{\mbox{\scriptsize \em FALSE}}}

\newcommand{\bSigma}{\bm{\Sigma}}
\newcommand{\bmu}{\bm{\mu}}
\newcommand{\bx}{\bm{x}}
\newcommand{\by}{\bm{y}}
\newcommand{\bX}{\bm{X}}
\newcommand{\bI}{\bm{I}}
\newcommand{\bw}{\bm{w}}
\newcommand{\ba}{\bm{a}}
\newcommand{\bb}{\bm{b}}
\newcommand{\bk}{\bm{k}}  
\newcommand{\inv}{^{-1}}

\newcommand{\norm}{{\mathcal{N}}}

\newcommand\transpose{^{\textrm{\tiny{\sf{T}}}}}

\begin{document}
	\gccs
	Now that you have learned about the theory behind GPs, you will have to use that theory to implement GPs yourself.
	\begin{questions}
		\titledquestion{Differentiable Architecture Search}[5]
		In this second part of the exercise you will run DARTS for finding an optimal CNN architecture on MNIST.
		The search model is defined in \texttt{model\_search.py}. It contains three stacked cells: 
		reduction-normal-reduction. The architecture search problem is to find an optimal operation out of
		$\mathcal{O} = \{conv\_3x3,\\ max\_pool\_3x3, avg\_pool\_3x3, Identity\}$ in each edge
		of these cells. The number of intermediate nodes is 2. 
		
		\begin{parts}
		\part[3] In order to create the architecture continuous relaxation,
		we need to define a \textit{MixedOp}, which is a convex combination of the operations outputs connecting two nodes
		in the cells. It is defined as follows: $$x^{(j)} = \sum_{i<j}\tilde{o}^{(i,j)}(x^{(i)}) = \sum_{i<j}\sum_{o\in\mathcal{O}}\frac{e^{\alpha_{o}^{(i,j)}}}{\sum_{o^{\prime}\in\mathcal{O}}e^{\alpha_{o^{\prime}}^{(i,j)}}}o(x^{(i)})$$

		Based on this formulation you have to fill in the code in lines 46-57 of \texttt{model\_search.py} in order to 
		compute the output tensor $x^{(j)}$ of the MixedOp.

		\part[2] Having defined the search model, we now need to run the DARTS optimization loop (Algorithmi 1, Slide 31 of Lecture 7). We will use the first-order approximation. Your task is going to be only to write the lines of code that compute the architectural updates in lines 43-49 of \texttt{train\_search.py}. Afterwards, you should be able to run \texttt{python train\_search.py} without any errors. This will conduct the search for 5 epochs and write in a directory named \texttt{logs/} the output logs and a file with the optimal architecture configuration.\\

		In the end, to generate a visualization of the found cells run \texttt{python visualize.py}. This should generate two \texttt{.pdf} files named \texttt{normal.pdf} and \texttt{reduction.pdf}.
		Push the contents written in \texttt{logs/} together with the two \texttt{.pdf} files generated by the visualization script to your Bitbucket repository.\\

		\textbf{NOTE}: Running \texttt{train\_search.py} on a GPU machine takes a few minutes. This might scale to more than 1h when running on a CPU machine.

		\end{parts}

		\titledquestion{Feedback}[Bonus: 0.5]
		For each question in this assignment, state:
		\begin{itemize}
			\item How long you worked on it.
			\item What you learned.
			\item Anything you would improve in this question if you were teaching the course.
		\end{itemize}
	\end{questions}
	
	\noindent
	\due Submit your solution for the tasks by uploading a PDF to your groups BitBucket repository. The PDF has to include the name of the submitter(s).
\end{document}
