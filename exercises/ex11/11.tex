\documentclass{exam}
\usepackage{amsmath, amsfonts}
\usepackage{bm}
\usepackage{verbatim}
\usepackage{graphicx}
\usepackage[super]{nth}
\usepackage{booktabs}

\DeclareMathOperator*{\argmin}{argmin}

\usepackage[hyperfootnotes=false]{hyperref}

\usepackage[usenames,dvipsnames]{color}
\newcommand{\note}[1]{
	\noindent~\\
	\vspace{0.25cm}
	\fcolorbox{Red}{Orange}{\parbox{0.99\textwidth}{#1\\}}
	%{\parbox{0.99\textwidth}{#1\\}}
	\vspace{0.25cm}
}


\renewcommand{\vec}[1]{\mathbf{#1}}
\newcommand{\lecture}{ML4AAD}
\newcommand{\lecturelong}{Machine Learning for Automated Algorithm Design}
\newcommand{\semester}{WS 2018/19}
\newcommand{\assignment}[1]{\nth{#1} Assignment}
\newcommand{\lectors}{M. Lindauer \& A. Biedenkapp}
\newcommand{\hide}[1]{}


\newcommand{\gccs}{\paragraph{General constraints for code submissions}
    
    \begin{itemize}
        \item The program can be called as stated on the exercise sheet.
        \item The program exactly returns the required output (neither less nor more) -- please use a \texttt{--verbose} option to increase the verbosity level for debugging\footnote{You might want to use \texttt{argparse} for simplicity's sake.}.
        \item Your scripts should be commented to be readable for the tutors. All functions and classes are documented with a docstring. 
        \item Provide a README ($\to$ how to install requirements and run your program(s)) and (if necessary) an installation script if your program requires any other packages.
        \item Programs are to be submitted in python $3.5$ or newer.
        \item Adding new packages to the requirements.txt is fine. If you do this however, you'll have to give a brief description why you use that package and a link to it's documentation or github page.
        \item All prepared unittests have to pass.
        \item We don't accept ipython notebook submissions.
        \item Points will be deducted if you don't fullfill these constraints.
        \item You are allowed (sometimes required) to reuse code from previous exercises.
    \end{itemize}
    \rule{\textwidth}{.5pt}
    \smallskip\\
    \noindent}
%\renewcommand{\hide}[1]{#1}

\qformat{\thequestion. \textbf{\thequestiontitle}\hfill[\thepoints]}
\bonusqformat{\thequestion. \textbf{\thequestiontitle}\hfill[\thepoints]}

\pagestyle{headandfoot}

%%%%%% MODIFY FOR EACH SHEET!!!! %%%%%%
\newcommand{\duedate}{19.07.19 (10:00)}
\newcommand{\due}{{\bf This assignment is due on \duedate.} }
\firstpageheader
{Due: \duedate \\ Points: 11}
{{\bf\lecture}\\ \assignment{11}}
{\lectors\\ \semester}

\runningheader
{Due: \duedate}
{\assignment{6}}
{\semester}
%%%%%% MODIFY FOR EACH SHEET!!!! %%%%%%

\firstpagefooter
{}
{\thepage}
{}

\runningfooter
{}
{\thepage}
{}

\headrule
\pointsinrightmargin
\bracketedpoints
\marginpointname{pt.}




\newcommand{\parents}{p}
\newcommand{\negation}[1]{\overline{#1}}
%\newcommand{\tuple}[1]{\langle #1 \rangle}
\newcommand{\tuple}[1]{\left<#1\right>}
\newcommand{\dom}[1]{dom(#1)}              % domain

\newcommand{\false}{false}
\newcommand{\true}{true}
\newcommand{\TRUE}{{\mbox{\scriptsize \em TRUE}}}
\newcommand{\FALSE}{{\mbox{\scriptsize \em FALSE}}}

\newcommand{\bSigma}{\bm{\Sigma}}
\newcommand{\bmu}{\bm{\mu}}
\newcommand{\bx}{\bm{x}}
\newcommand{\by}{\bm{y}}
\newcommand{\bX}{\bm{X}}
\newcommand{\bI}{\bm{I}}
\newcommand{\bw}{\bm{w}}
\newcommand{\ba}{\bm{a}}
\newcommand{\bb}{\bm{b}}
\newcommand{\bk}{\bm{k}}  
\newcommand{\inv}{^{-1}}

\newcommand{\norm}{{\mathcal{N}}}

\newcommand\transpose{^{\textrm{\tiny{\sf{T}}}}}

\begin{document}
	\gccs
	After you now know how to use algorithm configuration,
	your next task is to configure the SAT solver \textit{SATenstein} to optimize its performance using SMAC.
	
	\bigskip
	
	The scenario consists of the following directories
	\begin{itemize}
		\item \texttt{SATenstein}: a directory with the binary of \textit{SATenstein} and a basic framework for an algorithm wrapper that uses \texttt{genericWrapper4AC}\footnote{\url{https://github.com/mlindauer/GenericWrapper4AC}} (to install the wrapper, clone the repo and run \texttt{python setup.py install})
		\item \texttt{indu}: a directory with industrial instances from various SAT competitions. All instances are satisfiable!
	\end{itemize}
	
	\begin{questions}
		
		\titledquestion{Configuration of \textit{SATenstein}}[11]
		
		Given the above mentioned files, your task is to optimize the performance of \textit{SATenstein} on the provided instances with \textit{SMACv3}\footnote{\url{https://github.com/automl/SMAC3}}. You will have to clone the SMAC repository to use its command line interface\footnote{\url{https://automl.github.io/SMAC3/stable/quickstart.html\#command-line}}. You can however choose to use SMAC as a python package instead\footnote{\url{https://github.com/automl/ParameterImportance/blob/master/notebooks/interface_example.ipynb} (see lines 2-6)}.
		To use either method of running SMAC, please work through the following steps:
		
		\begin{itemize}
			\item Split the instances in a training (\texttt{training.txt}) and test set (\texttt{test.txt}) -- write a bash or Python script to do so and upload it into your \texttt{src} folder. The instances should be split evenly into training and test sets. The given instances are \textbf{heterogeneous} and consist of instances from two domains. You have to be mindful how you split the instances. The instance names indicate if they are from the same domain.
			\item Complete the Python script \texttt{wrapper.py}; the call to \textit{SATenstein} should look like:\\
			\texttt{\small satenstein/ubcsat -param1 value1 $\ldots$ -paramN valueN -seed <int> -inst <instance> -target 0 -r satcomp -cutoff <max search steps> -timeout <max running time>}\\
			e.g. \texttt{\small satenstein/ubcsat -alg satenstein -adaptive 0 -inst indu/factor-3023-3607.cnf -target 0 -seed 0 -r satcomp -cutoff -1 -timeout 5}\\
			\textit{The cutoff parameter is used to limit the number of search steps whereas the timeout parameter limits the runtime!} So you have to set cutoff to -1 for unlimited search steps.\\
			\\
			Verify the functionality of the completed wrapper with the following call (in the scenario folder):\\
			\texttt{\small python satenstein/wrapper.py indu/factor-3023-3607.cnf 0 5 0 1 -adaptive 1}\\
			This calls the wrapper to run SATenstein on instance \texttt{indu/factor-3023-3607.cnf}, with 0 as \texttt{instance specifics}, a  \texttt{cutoff} of 5 seconds and a  \texttt{runlength} of 0 (which should be ignored by your wrapper and instead set to -1),  \texttt{seed} 1 and the parameter \texttt{adaptive} set to 1.\\
			\\ The final printed line will look something like this:\\
			\texttt{\small Result for ParamILS: <Status>, <runningtime>, <runlenght>, <quality>, <seed>}
			\\
			All the paramters for the call to the wrapper will be automatically set by SMAC during optimization, so your wrapper only needs to construct the cmdline call to SATenstein.
			
			\item Complete the scenario file (\texttt{scenario.txt}) that defines the following characteristics of the configuration scenario
			\begin{itemize}
				\item use your generated training and test files (\texttt{train.txt} and \texttt{test.txt})
				\item the algorithm is non-deterministic
				\item optimize runtime
				\item the overall objective is the mean10 (PAR10) score
				\item the cutoff time time will be $5$ seconds
				\item the configuration budget will be $900$ seconds
			\end{itemize}
			\item Run \textit{SMAC} with the above defined scenario, report the hardware you used and the performance of the configured \textit{SATenstein} on the test instances.
			\item Validate the default configuration of \textit{SATenstein} on the test instances -- you can use SMACs validation script and the script provided in \texttt{src/read\_validation\_data.py}. This will print some details about all validated configurations.
			Report the performance of the default configuration as well as the incumbent.
			\item Upload all files you created, i.e. training and test files, SMAC output as well as the validated runhistory.
		\end{itemize}

		\titledquestion{Feedback}[Bonus: 0.5]
		For each question in this assignment, state:
		\begin{itemize}
			\item How long you worked on it.
			\item What you learned.
			\item Anything you would improve in this question if you were teaching the course.
		\end{itemize}
	\end{questions}
	
	\noindent
	\due Submit your solution for the tasks by uploading a PDF to your groups BitBucket repository. The PDF has to include the name of the submitter(s).
\end{document}
