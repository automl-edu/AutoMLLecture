\documentclass{exam}
\usepackage{amsmath, amsfonts}
\usepackage{verbatim}
\usepackage{graphicx}
\usepackage[super]{nth}

\DeclareMathOperator*{\argmin}{argmin}

\usepackage[hyperfootnotes=false]{hyperref}

\usepackage[usenames,dvipsnames]{color}
\newcommand{\note}[1]{
	\noindent~\\
	\vspace{0.25cm}
	\fcolorbox{Red}{Orange}{\parbox{0.99\textwidth}{#1\\}}
	%{\parbox{0.99\textwidth}{#1\\}}
	\vspace{0.25cm}
}


\renewcommand{\vec}[1]{\mathbf{#1}}
\newcommand{\lecture}{ML4AAD}
\newcommand{\lecturelong}{Machine Learning for Automated Algorithm Design}
\newcommand{\semester}{WS 2018/19}
\newcommand{\assignment}[1]{\nth{#1} Assignment}
\newcommand{\lectors}{M. Lindauer \& A. Biedenkapp}
\newcommand{\hide}[1]{}


\newcommand{\gccs}{\paragraph{General constraints for code submissions}
    
    \begin{itemize}
        \item The program can be called as stated on the exercise sheet.
        \item The program exactly returns the required output (neither less nor more) -- please use a \texttt{--verbose} option to increase the verbosity level for debugging\footnote{You might want to use \texttt{argparse} for simplicity's sake.}.
        \item Your scripts should be commented to be readable for the tutors. All functions and classes are documented with a docstring. 
        \item Provide a README ($\to$ how to install requirements and run your program(s)) and (if necessary) an installation script if your program requires any other packages.
        \item Programs are to be submitted in python $3.5$ or newer.
        \item Adding new packages to the requirements.txt is fine. If you do this however, you'll have to give a brief description why you use that package and a link to it's documentation or github page.
        \item All prepared unittests have to pass.
        \item We don't accept ipython notebook submissions.
        \item Points will be deducted if you don't fullfill these constraints.
        \item You are allowed (sometimes required) to reuse code from previous exercises.
    \end{itemize}
    \rule{\textwidth}{.5pt}
    \smallskip\\
    \noindent}
%\renewcommand{\hide}[1]{#1}

\qformat{\thequestion. \textbf{\thequestiontitle}\hfill[\thepoints]}
\bonusqformat{\thequestion. \textbf{\thequestiontitle}\hfill[\thepoints]}

\pagestyle{headandfoot}

%%%%%% MODIFY FOR EACH SHEET!!!! %%%%%%
\newcommand{\duedate}{09.05.19 (14:00)}
\newcommand{\due}{{\bf This assignment is due on \duedate.} }
\firstpageheader
{Due: \duedate \\ Points: 1}
{{\bf\lecture}\\ \assignment{1}}
{\lectors\\ \semester}

\runningheader
{Due: \duedate}
{\assignment{1}}
{\semester}
%%%%%% MODIFY FOR EACH SHEET!!!! %%%%%%

\firstpagefooter
{}
{\thepage}
{}

\runningfooter
{}
{\thepage}
{}

\headrule
\pointsinrightmargin
\bracketedpoints
\marginpointname{pt.}


\begin{document}

\noindent
The automated machine learning methods you will learn about in this course help you to improve your skill and knowledge for applying machine learning in practice. The goal of this first exercise is to set up teams and learn about git and the workflow for future exercises.\vspace*{5pt}

\begin{questions}
  \titledquestion{Form teams of $2$ students and get familiar with \emph{git}} [$0.5$]
    Exercises have to be handed in \emph{teams of $2$} students. Send an email with both names and email addresses to \texttt{automl-lecture@informatik.uni-freiburg.de}. We will then create a BitBucket\footnote{bitbucket.org} repository for each group and send an invitation. Use this repository to upload solutions for exercises.
  
  \emph{Note 1:} You must send this email before \textbf{Wednesday, 08/05/2019, 13:00}. \\
  \emph{Note 2:} To work with BitBucket you will have to use \emph{git}. If you have never worked with \emph{git} before we suggest you take a look at this simple guide \url{http://rogerdudler.github.io/git-guide/}. 
    \\\vspace*{-5pt}\hspace*{-2pt}
\\

	\titledquestion{Algorithms with hyperparameters}[$0.5$]
		Identify two algorithms that have tunable hyperparameters. Describe each algorithm briefly (1-2 sentences, without code) and explain its most important hyperparameters; this includes:
		\begin{itemize}
		  \item Why does the hyperparameter influence the performance of the algorithm? 
		  \item What could be a good range of possible values of the parameter (e.g., $[0,1]$, $[1,1024]$ or $\{yes,no\}$)? Briefly explain your choices.
		\end{itemize}
		Submit your solution by uploading a PDF to your BitBucket repository. The PDF has to include the name of the submitter(s).
		
\end{questions}

\end{document}
