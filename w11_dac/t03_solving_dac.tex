
\pdfminorversion=4 % for acroread
\documentclass[aspectratio=169,t,xcolor={usenames,dvipsnames}]{beamer}
%\documentclass[t,handout,xcolor={usenames,dvipsnames}]{beamer}
\usepackage{../beamerstyle}
\usepackage{dsfont}
\usepackage{bm}
\usepackage[english]{babel}
\usepackage[utf8]{inputenc}
\usepackage{graphicx}
\usepackage{algorithm}
\usepackage[ruled,vlined,algo2e,linesnumbered]{algorithm2e}
%\usepackage[boxed,vlined]{algorithm2e}
\usepackage{hyperref}
\usepackage{booktabs}
\usepackage{mathtools}

\usepackage{amsmath,amssymb}
\usepackage{listings}
\lstset{frame=lines,framesep=3pt,numbers=left,numberblanklines=false,basicstyle=\ttfamily\small}

\usepackage{subfig}
\usepackage{multicol}
%\usepackage{appendixnumberbeamer}
%
\usepackage{tcolorbox}

\usepackage{pgfplots}
\usepackage{tikz}
\usetikzlibrary{trees} 
\usetikzlibrary{shapes.geometric}
\usetikzlibrary{positioning,shapes,shadows,arrows,calc,mindmap}
\usetikzlibrary{positioning,fadings,through}
\usetikzlibrary{decorations.pathreplacing}
\usetikzlibrary{intersections}
\usetikzlibrary{positioning,fit,calc,shadows,backgrounds}
\pgfdeclarelayer{background}
\pgfdeclarelayer{foreground}
\pgfsetlayers{background,main,foreground}
\tikzstyle{activity}=[rectangle, draw=black, rounded corners, text centered, text width=8em]
\tikzstyle{data}=[rectangle, draw=black, text centered, text width=8em]
\tikzstyle{myarrow}=[->, thick, draw=black]

% Define the layers to draw the diagram
\pgfdeclarelayer{background}
\pgfdeclarelayer{foreground}
\pgfsetlayers{background,main,foreground}

%\usepackage{listings}
%\lstset{numbers=left,
%  showstringspaces=false,
%  frame={tb},
%  captionpos=b,
%  lineskip=0pt,
%  basicstyle=\ttfamily,
%%  extendedchars=true,
%  stepnumber=1,
%  numberstyle=\small,
%  xleftmargin=1em,
%  breaklines
%}

 
\definecolor{blue}{RGB}{0, 74, 153}

\usetheme{Boadilla}
%\useinnertheme{rectangles}
\usecolortheme{whale}
\setbeamercolor{alerted text}{fg=blue}
\useoutertheme{infolines}
\setbeamertemplate{navigation symbols}{\vspace{-5pt}} % to lower the logo
\setbeamercolor{date in head/foot}{bg=white} % blue
\setbeamercolor{date in head/foot}{fg=white}
\setbeamercolor{author  in head/foot}{bg=white} %blue
\setbeamercolor{title in head/foot}{bg=white} % blue
\setbeamercolor{title}{fg=white, bg=blue}
\setbeamercolor{block title}{fg=white,bg=blue}
\setbeamercolor{block body}{bg=blue!10}
\setbeamercolor{frametitle}{fg=white, bg=blue}
\setbeamercovered{invisible}

\makeatletter
\setbeamertemplate{footline}
{
  \leavevmode%
  \hbox{%
  \begin{beamercolorbox}[wd=.333333\paperwidth,ht=2.25ex,dp=1ex,center]{author in head/foot}%
%    \usebeamerfont{author in head/foot}\insertshortauthor
  \end{beamercolorbox}%
  \begin{beamercolorbox}[wd=.333333\paperwidth,ht=2.25ex,dp=1ex,center]{title in head/foot}%
    \usebeamerfont{title in head/foot}\insertshorttitle
  \end{beamercolorbox}%
  \begin{beamercolorbox}[wd=.333333\paperwidth,ht=2.25ex,dp=1ex,right]{date in head/foot}%
    \usebeamerfont{date in head/foot}\insertshortdate{}\hspace*{2em}
%    \insertframenumber\hspace*{2ex} 
  \end{beamercolorbox}}%
  \vskip0pt%
}
\makeatother

%\pgfdeclareimage[height=1.2cm]{automl}{images/logos/automl.png}
%\pgfdeclareimage[height=1.2cm]{freiburg}{images/logos/freiburg}

%\logo{\pgfuseimage{freiburg}}

\newcommand{\comment}[1]{
	\noindent
	%\vspace{0.25cm}
	{\color{red}{\textbf{TODO:} #1}}
	%\vspace{0.25cm}
}
\renewcommand{\comment}[1]{}
\newcommand{\hide}[1]{}
\newcommand{\cemph}[2]{\emph{\textcolor{#1}{#2}}}

\newcommand{\lit}[1]{{\footnotesize\color{black!70}[#1]}}

\newcommand{\litw}[1]{{\footnotesize\color{black!20}[#1]}}


\newcommand{\myframe}[2]{\begin{frame}[c]{#1}#2\end{frame}}
\newcommand{\myframetop}[2]{\begin{frame}{#1}#2\end{frame}}
\newcommand{\myit}[1]{\begin{itemize}#1\end{itemize}}
\newcommand{\myblock}[2]{\begin{block}{#1}#2\end{block}}


\newcommand{\votepurple}[1]{\textcolor{Purple}{$\bigstar$}}
\newcommand{\voteyellow}[1]{\textcolor{Goldenrod}{$\bigstar$}}
\newcommand{\voteblue}[1]{\textcolor{RoyalBlue}{$\bigstar$}}
\newcommand{\votepink}[1]{\textcolor{Pink}{$\bigstar$}}

\newcommand{\diff}{\mathop{}\!\mathrm{d}}
\newcommand{\refstyle}[1]{{\small{\textcolor{gray}{#1}}}}
\newcommand{\hands}[0]{\includegraphics[height=1.5em]{images/hands}}
\newcommand{\transpose}[0]{{\textrm{\tiny{\sf{T}}}}}
\newcommand{\norm}{{\mathcal{N}}}
\newcommand{\cutoff}[0]{\kappa}
\newcommand{\instD}[0]{\dataset}
\newcommand{\insts}[0]{\mathcal{I}}
\newcommand{\inst}[0]{i}
\newcommand{\pcs}[0]{\mathbf{\Lambda}}
\newcommand{\bx}[0]{\conf}
\newcommand{\conf}[0]{\mathbf{\lambda}}
\newcommand{\defconf}[0]{\mathbf{\lambda}_{\text{def}}}
\newcommand{\finconf}[0]{\mathbf{\lambda}^*}
\newcommand{\incumbent}[0]{\finconf}
\newcommand{\confs}[0]{\pcs}
%\newcommand{\vlambda}[0]{\bm{\lambda}}
%\newcommand{\vLambda}[0]{\bm{\Lambda}}
\newcommand{\dataset}[0]{\mathcal{D}}
\newcommand{\datasets}[0]{\mathbf{D}}
\newcommand{\loss}[0]{\mathcal{L}}

% \renewcommand{\vec}[1]{\mathbf{#1}}
\newcommand{\hist}[0]{\mathcal{H}}
\newcommand{\param}[0]{p}
\newcommand{\algo}[0]{\mathcal{A}}
\newcommand{\algos}[0]{\mathbf{A}}
%\newcommand{\nn}[0]{N}
\newcommand{\feats}[0]{\mathcal{F}}
\newcommand{\feat}[0]{\vec{f}}
\newcommand{\cluster}[0]{\vec{h}}
\newcommand{\clusters}[0]{\vec{H}}
\newcommand{\perf}[0]{\mathbb{R}}
%\newcommand{\surro}[0]{\mathcal{S}}
\newcommand{\surro}[0]{\hat{f}}
\newcommand{\func}[0]{f}
\newcommand{\epm}[0]{\surro}
\newcommand{\portfolio}[0]{\mathcal{P}}
\newcommand{\schedule}[0]{\mathcal{S}}
\newcommand{\mdata}[0]{\dataset_{\text{meta}}}

% Deep Learning
\newcommand{\weights}[0]{\theta}
\newcommand{\metaweights}[0]{\phi}


% reinforcement learning
\newcommand{\policies}[0]{\Pi}
\newcommand{\policy}[0]{\pi}
\newcommand{\actionRL}[0]{a}
\newcommand{\stateRL}[0]{s}
\newcommand{\statesRL}[0]{\mathcal{S}}
\newcommand{\rewardRL}[0]{r}
\newcommand{\rewardfuncRL}[0]{\mathcal{R}}

\RestyleAlgo{algoruled}
\DontPrintSemicolon
\LinesNumbered
\SetAlgoVlined
\SetFuncSty{textsc}

\SetKwInOut{Input}{Input}
\SetKwInOut{Output}{Output}
\SetKw{Return}{return}

%\newcommand{\changed}[1]{{\color{red}#1}}

%\newcommand{\citeN}[1]{\citeauthor{#1}~(\citeyear{#1})}

\renewcommand{\vec}[1]{\mathbf{#1}}
\DeclareMathOperator*{\argmin}{arg\,min}
\DeclareMathOperator*{\argmax}{arg\,max}

\newcommand{\aqme}{\textit{AQME}}
\newcommand{\aslib}{\textit{ASlib}}
\newcommand{\llama}{\textit{LLAMA}}
\newcommand{\satzilla}{\textit{SATzilla}}
\newcommand{\satzillaY}[1]{\textit{SATzilla'{#1}}}
\newcommand{\snnap}{\textit{SNNAP}}
\newcommand{\claspfolioTwo}{\textit{claspfolio~2}}
\newcommand{\flexfolio}{\textit{FlexFolio}}
\newcommand{\claspfolioOne}{\textit{claspfolio~1}}
\newcommand{\isac}{\textit{ISAC}}
\newcommand{\eisac}{\textit{EISAC}}
\newcommand{\sss}{\textit{3S}}
\newcommand{\sunny}{\textit{Sunny}}
\newcommand{\ssspar}{\textit{3Spar}}
\newcommand{\cshc}{\textit{CSHC}}  
\newcommand{\cshcpar}{\textit{CSHCpar}}  
\newcommand{\measp}{\textit{ME-ASP}} 
\newcommand{\aspeed}{\textit{aspeed}}
\newcommand{\autofolio}{\textit{AutoFolio}}
\newcommand{\cedalion}{\textit{Cedalion}}
\newcommand{\fanova}{\textit{fANOVA}}
\newcommand{\sbs}{\textit{SB}}
\newcommand{\oracle}{\textit{VBS}}

% like approaches
\newcommand{\claspfoliolike}[1]{\texttt{claspfolio-#1-like}}
\newcommand{\satzillalike}[1]{\texttt{SATzilla'#1-like}}
\newcommand{\isaclike}{\texttt{ISAC-like}}
\newcommand{\ssslike}{\texttt{3S-like}}
\newcommand{\measplike}{\texttt{ME-ASP-like}}

\newcommand{\aspCoseal}{\textit{ASP-POTASSCO}}
\newcommand{\cspCoseal}{\textit{CSP-2010}}
\newcommand{\maxsatCoseal}{\textit{MAXSAT12-PMS}}
\newcommand{\premarCoseal}{\textit{PRE\-MARSHALLING}}
\newcommand{\qbfCoseal}{\textit{QBF-2011}}
\newcommand{\satallTwelveCoseal}{\textit{SAT12-ALL}}
\newcommand{\sathandTwelveCoseal}{\textit{SAT12-HAND}}
\newcommand{\satinduTwelveCoseal}{\textit{SAT12-INDU}}
\newcommand{\satrandTwelveCoseal}{\textit{SAT12-RAND}}
\newcommand{\sathandElevenCoseal}{\textit{SAT11-HAND}}
\newcommand{\satinduElevenCoseal}{\textit{SAT11-INDU}}
\newcommand{\satrandElevenCoseal}{\textit{SAT11-RAND}}
\newcommand{\proteusCoseal}{\textit{PROTEUS-2014}}

\newcommand{\irace}{\textit{I/F-race}}
\newcommand{\gga}{\textit{GGA}}
\newcommand{\smac}{\textit{SMAC}}
\newcommand{\paramils}{\textit{ParamILS}}
\newcommand{\spearmint}{\textit{Spearmint}}
\newcommand{\tpe}{\textit{TPE}}

\newcommand{\gringo}{\textit{gringo}}
\newcommand{\clasp}{\textit{clasp}}
\newcommand{\lingeling}{\textit{lingeling}}

\newcommand{\hydra}{\textit{Hydra}}

\newcommand{\plingeling}{\textit{Plingeling}}
\newcommand{\ccasat}{\textit{CCASat}}

\usepackage{pifont}
\newcommand{\itarrow}{\mbox{\Pisymbol{pzd}{229}}}
\newcommand{\ithook}{\mbox{\Pisymbol{pzd}{52}}}
\newcommand{\itcross}{\mbox{\Pisymbol{pzd}{56}}}
\newcommand{\ithand}{\mbox{\raisebox{-1pt}{\Pisymbol{pzd}{43}}}}

%\DeclareMathOperator*{\argmax}{arg\,max}

\newcommand{\ie}{{\it{}i.e.\/}}
\newcommand{\eg}{{\it{}e.g.\/}}
\newcommand{\cf}{{\it{}cf.\/}}
\newcommand{\wrt}{\mbox{w.r.t.}}
\newcommand{\vs}{{\it{}vs\/}}
\newcommand{\vsp}{{\it{}vs\/}}
\newcommand{\etc}{{\copyedit{etc.}}}
\newcommand{\etal}{{\it{}et al.\/}}

\newcommand{\pscProc}{{\bf procedure}}
\newcommand{\pscBegin}{{\bf begin}}
\newcommand{\pscEnd}{{\bf end}}
\newcommand{\pscEndIf}{{\bf endif}}
\newcommand{\pscFor}{{\bf for}}
\newcommand{\pscEach}{{\bf each}}
\newcommand{\pscThen}{{\bf then}}
\newcommand{\pscElse}{{\bf else}}
\newcommand{\pscWhile}{{\bf while}}
\newcommand{\pscIf}{{\bf if}}
\newcommand{\pscRepeat}{{\bf repeat}}
\newcommand{\pscUntil}{{\bf until}}
\newcommand{\pscWithProb}{{\bf with probability}}
\newcommand{\pscOtherwise}{{\bf otherwise}}
\newcommand{\pscDo}{{\bf do}}
\newcommand{\pscTo}{{\bf to}}
\newcommand{\pscOr}{{\bf or}}
\newcommand{\pscAnd}{{\bf and}}
\newcommand{\pscNot}{{\bf not}}
\newcommand{\pscFalse}{{\bf false}}
\newcommand{\pscEachElOf}{{\bf each element of}}
\newcommand{\pscReturn}{{\bf return}}

%\newcommand{\param}[1]{{\sl{}#1}}
\newcommand{\var}[1]{{\it{}#1}}
\newcommand{\cond}[1]{{\sf{}#1}}
%\newcommand{\state}[1]{{\sf{}#1}}
%\newcommand{\func}[1]{{\sl{}#1}}
\newcommand{\set}[1]{{\Bbb #1}}
%\newcommand{\inst}[1]{{\tt{}#1}}
\newcommand{\myurl}[1]{{\small\sf #1}}

\newcommand{\Nats}{{\Bbb N}}
\newcommand{\Reals}{{\Bbb R}}
\newcommand{\extset}[2]{\{#1 \; | \; #2\}}

\newcommand{\vbar}{$\,\;|$\hspace*{-1em}\raisebox{-0.3mm}{$\,\;\;|$}}
\newcommand{\vendbar}{\raisebox{+0.4mm}{$\,\;|$}}
\newcommand{\vend}{$\,\:\lfloor$}


\newcommand{\goleft}[2][.7]{\parbox[t]{#1\linewidth}{\strut\raggedright #2\strut}}
\newcommand{\rightimage}[2][.3]{\mbox{}\hfill\raisebox{1em-\height}[0pt][0pt]{\includegraphics[width=#1\linewidth]{#2}}\vspace*{-\baselineskip}}






\title[AutoML: DAC]{AutoML: Dynamic Algorithm Configuration}
\subtitle{Approaches for Solving Dynamic Algorithm Configuration}
\author[Marius Lindauer]{Bernd Bischl \and Frank Hutter \and Lars Kotthoff\newline \and \underline{Marius Lindauer} \and Joaquin Vanschoren}
\institute{}
\date{}



% \AtBeginSection[] % Do nothing for \section*
% {
%   \begin{frame}{Outline}
%     \bigskip
%     \vfill
%     \tableofcontents[currentsection]
%   \end{frame}
% }

\begin{document}
	
	\maketitle
	

%----------------------------------------------------------------------
%----------------------------------------------------------------------
\begin{frame}[c]{Recap: Setup}
	
	Objective:
	\begin{equation}
	\policy^* \in \argmin_{\policy \in \policies} \int_{\datasets} p(\dataset) \cost(\policy, \dataset) \diff \dataset \nonumber
	\end{equation}
	
	\bigskip
	
	Task: Determine a dynamic policy
	
	$$\policy^* : \stateRL_t \times \dataset \mapsto \conf$$
	
	$\leadsto$ let's ignore generalization across datasets for now -- will be covered in a later video
	\bigskip
	
	Running example: learning rates
	
	\centering
		\includegraphics[width=0.5\textwidth]{images/learning_rates}
	
\end{frame}
%----------------------------------------------------------------------
%----------------------------------------------------------------------
\begin{frame}[c]{DAC by Configuring Hand-Designed Heuristics}
	
	
	\begin{itemize}
	    \item Adaptively changing hyperparameters itself is not a new idea
	    \item Often heuristics are designed manually
	    \item Example: cosine annealing learning rates \lit{\href{https://arxiv.org/abs/1608.03983}{Loshchilov and Hutter. 2016}}
	\end{itemize}
	
	\begin{eqnarray}
	         \eta_t = \eta_{min} + \frac{1}{2}(\eta_{max} - \eta_{min})\left(1 +
        \cos\left(\frac{T_{cur}}{T_{max}} \cdot 3.141592 \right)\right)
	\end{eqnarray}
	
	Hyperparameters:
	\begin{description}
	    \item[$\eta_{max}$] initial (maximal) learning rate
	    \item[$\eta_{min}$] Minimal learning rate
	    \item[$T_{max}$] Maximum number of iterations (i.e. restart of learning rate)
	\end{description}
	
	$\leadsto$ Use HPO to optimize these three continuous hyperparameters
	
\end{frame}
%----------------------------------------------------------------------
%----------------------------------------------------------------------
\begin{frame}[c,fragile]{DAC by Optimization (Direct Schedule)}
	
	
	\begin{itemize}
	    \item Maybe we don't know a well-performing analytic expression
	    \item We can also optimize directly for a discrete schedule of learning rates
	\end{itemize}
	
	On a linear schedule:
	\begin{verbatim}
	    lr_after_0_epochs [0.00001, 1] float log
	    lr_after_10_epochs [0.00001, 1] float log
	    lr_after_20_epochs [0.00001, 1] float log
	    lr_after_30_epochs [0.00001, 1] float log
	    ...
	\end{verbatim}
	
	Or exponential schedule:
	\begin{verbatim}
	    lr_after_0_epochs [0.00001, 1] float log
	    lr_after_10_epochs [0.00001, 1] float log
	    lr_after_30_epochs [0.00001, 1] float log
	    lr_after_90_epochs [0.00001, 1] float log
	    ...
	\end{verbatim}
	
\end{frame}
%----------------------------------------------------------------------
%----------------------------------------------------------------------
\begin{frame}[c,fragile]{DAC by Optimization (Functional Schedule)}
	
	
	\begin{itemize}
	    \item The previous approach has the disadvantages
	    \begin{itemize}
	        \item Discretization is important
	        \item Hard to optimize for a reasonable pattern in the learning rate schedule
	    \end{itemize}
	    \smallskip
	    \pause
	    \item Directly learn a function of a schedule:\\ $\pi(s; \theta)$ where $\theta$ are the implicit parameters of your schedule\\ (e.g., the weight of a neural network)
	    \pause
	    \item Since we can observe the cost induced by our learning rate schedule (i.e., validation loss), we can apply any black-box optimizer to optimize $\theta$
	\end{itemize}
	
	\begin{equation}
	    \bm{\theta}^* \in \argmin_{\bm{\theta} \in \mathbb{R}^{(n\times m)}}  \cost(\policy_{\bm{\theta}}) = \argmin_{\bm{\theta} \in \mathbb{R}^{(n\times m)}} \cost(s_T,\policy_{\bm{\theta}})\nonumber
	\end{equation}
	
	\begin{itemize}
	    \item For example, you could use an evolutionary strategy~\lit{\href{https://arxiv.org/abs/1802.08842}{Chrabaszcz et al. 2018}}
	\end{itemize}
	
	
\end{frame}
%----------------------------------------------------------------------
%----------------------------------------------------------------------
\begin{frame}[c,fragile]{DAC by Reinforcement Learning}
	
	\centering
    \begin{tikzpicture}[node distance=2.1cm]
        			%PreProcessing
        			
        			\node (Agent) [activity] {Policy $\policy$};
        			
        			\node (Algo) [activity, right of=Agent, xshift=3cm] {Algorithm};
        			
        			\begin{pgfonlayer}{background}
        			\path (Agent -| Agent.west)+(-0.12,1.125) node (resUL) {};
        			\path (Algo.east |- Algo.south)+(0.125,-1.125) node (resBR) {};
        			
        			% Context
        			\path [rounded corners, draw=black!50, fill=white] ($(resUL)+(0.5, -0.5)$) rectangle ($(resBR)+(0.5, -0.5)$);
        			\path [rounded corners, draw=black!50, fill=white] ($(resUL)+(0.375, -0.375)$) rectangle ($(resBR)+(0.375, -0.375)$);
        			\path [rounded corners, draw=black!50, fill=white] ($(resUL)+(0.25, -0.25)$) rectangle ($(resBR)+(0.25, -0.25)$);
        			\path [rounded corners, draw=black!50, fill=white] ($(resUL)+(0.125, -0.125)$) rectangle ($(resBR)+(0.125, -0.125)$);
        			
        			% Top level
        			\path [rounded corners, draw=black!50, fill=white] (resUL) rectangle (resBR);
        			\path (resBR)+(-1.3,0.175) node [text=black!75] {instance $\inst \in \insts$};
        			\path (resUL.east |- resBR.north)+(+.9,0.075) node [text=black!75] {control of $h$};
        			
        			\end{pgfonlayer}
        			
        			%        \draw[myarrow] (feat.south) -- ($(feat.south |- Agent)+(0,1.125)$);
        			
        			\draw[myarrow] (Agent.north) -- ($(Agent.north)+(0.0,+0.35)$) -- ($(Algo.north)+(0.0,+0.35)$) node [above,pos=0.5] {apply action $a_t$} node [below,pos=0.5] {$\lambda_h = v$} -- (Algo.north);
        			\draw[myarrow, dashed] ($(Algo.south)+(-0.25, 0)$) -- ($(Algo.south)+(-0.25, -0.35)$) -- ($(Agent.south)+(0.25, -0.35)$) node [above,pos=0.5] {state $s_{t+1}$} -- ($(Agent.south)+(0.25, 0)$);
        			\draw[myarrow] ($(Algo.south)+(0.25, 0)$) -- ($(Algo.south)+(0.25, -0.55)$) -- ($(Agent.south)+(-0.25, -0.55)$) node [below,pos=0.5] {reward $r_{t+1}$} -- ($(Agent.south)+(-0.25, 0)$);
        			
        			\draw[<->, thick, draw=black!32.5] (resBR.east |- resUL.center) -- ($(resBR.east |- resUL.center)+(0.5, -0.5)$);
        			\path (resBR.east |- resUL.center)+(.5, -0.175) node [text=black!32.5] {$\insts$};
        			% This path is only needed so the figure stays centered (the text is not visible)
        			\path (resUL.west |- resUL.center)+(-.5, -0.175) node [text=black!0] {$\insts$}; 
        			
        			\end{tikzpicture}
	
	\begin{itemize}
	    \item The way we defined the problem already hinted at using reinforcement learning ($\rightarrow$ more in the next video)
	    \item The policy agent is trained by means of RL in a purely data-driven way
	    \item Difference to DAC by Optimziation with a functional schedule:
	    \begin{itemize}
	        \item DAC by RL can make use of sequential structure of the problem
	        \item Allows to find generalizing policies conditioned to the dataset at hand
	    \end{itemize}
	\end{itemize}
	
\end{frame}
%----------------------------------------------------------------------
%----------------------------------------------------------------------
\begin{frame}[c,fragile]{DAC by Optimization (Online)}
	
	
	\begin{itemize}
	    \item Sometimes we cannot afford evaluations of different learning rates in a sequential manner
	    \item Assume we have $n$ parallel compute units (e.g., CPUs or GPUs),\\ but we can use them only once in parallel
	    \item Main idea: 
	    \begin{enumerate}
	        \item Initialize a your learning rates randomly on each compute unit\\
	        $\leadsto$ Population of learning rates (and partially trained models)
	        \item Train all models for a bit (e.g., 1 epoch), and observe $\stateRL^{(t+1)}$ and $\rewardRL^{(t+1)}$
	        \item Update your population of learning rates and paritally trained models
	        \item Go back to 2.
	    \end{enumerate}
	    \item The challenge is how to do Step 3.
	    
	\end{itemize}
	
\end{frame}
%----------------------------------------------------------------------
%----------------------------------------------------------------------
\begin{frame}[c,fragile]{Summary}
	
	
	\begin{itemize}
	    \item DAC can be solved in different ways, incl.
	    \begin{itemize}
	        \item Configuration of hand-designed heuristics
	        \item Direct configuration of a schedule
	        \item Optimization of a functional schedule
	        \item Using Reinforcement Learning
	        \item On-the-fly online optimization
	    \end{itemize}
	    
	\end{itemize}
	
\end{frame}
%----------------------------------------------------------------------
%----------------------------------------------------------------------
\begin{frame}[c]{Questions to Answer for Yourself / Discuss with Friends}
    
    \begin{enumerate}
        \item Under which assumptions would you choose which of the approaches discussed in this video?
        \item What are the advantages and disadvantages of using reinforcement learning for DAC or online DAC?
        \item How would you optimize a policy that can generalize to more than one dataset / task?
    \end{enumerate}
    
\end{frame}


\end{document}
