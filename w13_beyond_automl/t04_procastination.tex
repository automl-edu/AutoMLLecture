% !TeX spellcheck = en_US

\pdfminorversion=4 % for acroread
\documentclass[aspectratio=169,t,xcolor={usenames,dvipsnames}]{beamer}
%\documentclass[t,handout,xcolor={usenames,dvipsnames}]{beamer}
\usepackage{../beamerstyle}
\usepackage{dsfont}
\usepackage{bm}
\usepackage[english]{babel}
\usepackage[utf8]{inputenc}
\usepackage{graphicx}
\usepackage{algorithm}
\usepackage[ruled,vlined,algo2e,linesnumbered]{algorithm2e}
%\usepackage[boxed,vlined]{algorithm2e}
\usepackage{hyperref}
\usepackage{booktabs}
\usepackage{mathtools}

\usepackage{amsmath,amssymb}
\usepackage{listings}
\lstset{frame=lines,framesep=3pt,numbers=left,numberblanklines=false,basicstyle=\ttfamily\small}

\usepackage{subfig}
\usepackage{multicol}
%\usepackage{appendixnumberbeamer}
%
\usepackage{tcolorbox}

\usepackage{pgfplots}
\usepackage{tikz}
\usetikzlibrary{trees} 
\usetikzlibrary{shapes.geometric}
\usetikzlibrary{positioning,shapes,shadows,arrows,calc,mindmap}
\usetikzlibrary{positioning,fadings,through}
\usetikzlibrary{decorations.pathreplacing}
\usetikzlibrary{intersections}
\usetikzlibrary{positioning,fit,calc,shadows,backgrounds}
\pgfdeclarelayer{background}
\pgfdeclarelayer{foreground}
\pgfsetlayers{background,main,foreground}
\tikzstyle{activity}=[rectangle, draw=black, rounded corners, text centered, text width=8em]
\tikzstyle{data}=[rectangle, draw=black, text centered, text width=8em]
\tikzstyle{myarrow}=[->, thick, draw=black]

% Define the layers to draw the diagram
\pgfdeclarelayer{background}
\pgfdeclarelayer{foreground}
\pgfsetlayers{background,main,foreground}

%\usepackage{listings}
%\lstset{numbers=left,
%  showstringspaces=false,
%  frame={tb},
%  captionpos=b,
%  lineskip=0pt,
%  basicstyle=\ttfamily,
%%  extendedchars=true,
%  stepnumber=1,
%  numberstyle=\small,
%  xleftmargin=1em,
%  breaklines
%}

 
\definecolor{blue}{RGB}{0, 74, 153}

\usetheme{Boadilla}
%\useinnertheme{rectangles}
\usecolortheme{whale}
\setbeamercolor{alerted text}{fg=blue}
\useoutertheme{infolines}
\setbeamertemplate{navigation symbols}{\vspace{-5pt}} % to lower the logo
\setbeamercolor{date in head/foot}{bg=blue} % blue
\setbeamercolor{date in head/foot}{fg=white}
\setbeamercolor{author in head/foot}{bg=blue} %blue
\setbeamercolor{title in head/foot}{bg=blue} % blue
\setbeamercolor{title}{fg=white, bg=blue}
\setbeamercolor{block title}{fg=white,bg=blue}
\setbeamercolor{block body}{bg=blue!10}
\setbeamercolor{frametitle}{fg=white, bg=blue}
\setbeamercovered{invisible}

\makeatletter
\setbeamertemplate{footline}
{
  \leavevmode%
  \hbox{%
  \begin{beamercolorbox}[wd=.333333\paperwidth,ht=2.25ex,dp=1ex,center]{author in head/foot}%
    \usebeamerfont{author in head/foot}\insertshortauthor
  \end{beamercolorbox}%
  \begin{beamercolorbox}[wd=.333333\paperwidth,ht=2.25ex,dp=1ex,center]{title in head/foot}%
    \usebeamerfont{title in head/foot}\insertshorttitle
  \end{beamercolorbox}%
  \begin{beamercolorbox}[wd=.333333\paperwidth,ht=2.25ex,dp=1ex,right]{date in head/foot}%
    \usebeamerfont{date in head/foot}Week \@week, Topic \@topicnumber, Slide \insertframenumber{}\hspace*{2em}
%    \insertframenumber\hspace*{2ex} 
  \end{beamercolorbox}}%
  \vskip0pt%
}

\newcommand{\@week}{0}
\newcommand{\@topicnumber}{0}
\newcommand{\week}[1]{\renewcommand{\@week}{#1}}
\newcommand{\topicnumber}[1]{\renewcommand{\@topicnumber}{#1}}

\makeatother

%\pgfdeclareimage[height=1.2cm]{automl}{images/logos/automl.png}
%\pgfdeclareimage[height=1.2cm]{freiburg}{images/logos/freiburg}

%\logo{\pgfuseimage{freiburg}}

\input{../latex_main/macros}




\title[AutoML: Capping]{AutoML: Beyond AutoML}
\subtitle{Structured Procastination}
\author[Marius Lindauer]{Bernd Bischl \and Frank Hutter \and Lars Kotthoff\newline \and \underline{Marius Lindauer} \and Joaquin Vanschoren}
\institute{}
\date{}
\week{13}
\topicnumber{4}

% \AtBeginSection[] % Do nothing for \section*
% {
%   \begin{frame}{Outline}
%     \bigskip
%     \vfill
%     \tableofcontents[currentsection]
%   \end{frame}
% }

\begin{document}
	
	\maketitle


%-----------------------------------------------------------------------
\begin{frame}[c,fragile]{Structured Procrastination \litw{\href{https://www.ijcai.org/proceedings/2017/0281.pdf}{Kleinberg et al. 2017}}}

\begin{block}{Idea}
	\begin{itemize}
		\item incumbent driven methods (such as aggressive racing with adaptive capping) provide no theoretical guarantees about runtime
		\pause
		\item task: for a fix set of configuration, identify the one with the best average runtime
		
		\item instead of top-down capping, use bottom up capping
		\pause
		\item start with a minimal cap-time and increase it step by step
		\pause
		\item unsuccessful runs (with too small cap-time) are procrastinated to later
		\item[$\leadsto$] worst-case runtime guarantees
	\end{itemize}
\end{block}

\end{frame}
%-----------------------------------------------------------------------

%-----------------------------------------------------------------------
\begin{frame}[c,fragile]{Structured Procrastination: Outline \litw{\href{https://www.ijcai.org/proceedings/2017/0281.pdf}{Kleinberg et al. 2017}}}


\LinesNotNumbered
\begin{algorithm}[H]
\begin{footnotesize}
	\Input{%
		finite (small) set of configurations $\confs$,
		minimal cap-time $\kappa_0$,
		sequence of instances $\inst^{(1)}, \ldots, \inst^{(N)}$
	}
	\Output{best incumbent configuration $\finconf$}
	\BlankLine
	for each $\conf \in \confs$ initialize a queue $Q_{\conf}$ with entries $(\inst^{(k)}, \kappa_0)$; \tcp*{small queue in the beginning}
	initialize a look-up table $R(\conf, \inst) = 0$; \tcp*{optimistic runtime estimate}
	\pause
	\While{$b$ remains} {
		\pause
		determine the best $\finconf$ according to $R(\conf, \cdot)$;\\
		\pause
		get first element $(\inst^{(k)}, \kappa)$ from $Q_{\finconf}$;\\
		\pause
		Run $\finconf$ on $\inst^{(k)}$ capped at $\kappa$;\\
		\If{terminates}{$R(\finconf, \inst^{(k)})$ := t;}
		\pause
		\Else{
		$R(\finconf, \inst^{(k)}) := \kappa$;\\
		Insert $(\inst^{(k)}, 2\cdot \kappa)$ at the end of $Q_{\finconf}$;
		}
		\pause
		Replenish queue $Q_{\finconf}$ if too small;	
	}
	\pause
	\Return{$\finconf:= \argmin_{\conf \in \confs} \sum_{k=1}^N R(\conf, \inst^{(k)})$}
	\caption{Structured Procrastination}
\end{footnotesize}
\end{algorithm}
\end{frame}
%------------

%-----------------------------------------------------------------------
\begin{frame}[c,fragile]{Extensions}

\begin{itemize}
	\item We can derive theoretical optimality guarantees with structured procrastination (SP)
	\pause
	\item In practice, plain SP is rather slow and requires the setting of some hyperparameters
	\pause
	\item Several extensions and similar ideas:
	\begin{itemize}
		\item \lit{\href{https://arxiv.org/pdf/1902.05454.pdf}{Kleinberg et al. 2019%Procrastinating with Confidence: Near-Optimal, Anytime, Adaptive Algorithm Configuration
		}}
		\item \lit{\href{https://arxiv.org/pdf/1807.00755.pdf}{Weisz et al. 2018%LeapsAndBounds: A Method for Approximately Optimal Algorithm Configuration
		}}
		\item \lit{\href{http://proceedings.mlr.press/v97/weisz19a/weisz19a.pdf}{Weisz et al. 2019%CapsAndRuns: An Improved Method for Approximately Optimal Algorithm Configuration
		}}
	\end{itemize}
\end{itemize}
\end{frame}
%------------


\end{document}
