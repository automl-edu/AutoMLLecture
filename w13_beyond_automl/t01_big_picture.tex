% !TeX spellcheck = en_US

\pdfminorversion=4 % for acroread
\documentclass[aspectratio=169,t,xcolor={usenames,dvipsnames}]{beamer}
%\documentclass[t,handout,xcolor={usenames,dvipsnames}]{beamer}
\usepackage{../beamerstyle}
\usepackage{dsfont}
\usepackage{bm}
\usepackage[english]{babel}
\usepackage[utf8]{inputenc}
\usepackage{graphicx}
\usepackage{algorithm}
\usepackage[ruled,vlined,algo2e,linesnumbered]{algorithm2e}
%\usepackage[boxed,vlined]{algorithm2e}
\usepackage{hyperref}
\usepackage{booktabs}
\usepackage{mathtools}

\usepackage{amsmath,amssymb}
\usepackage{listings}
\lstset{frame=lines,framesep=3pt,numbers=left,numberblanklines=false,basicstyle=\ttfamily\small}

\usepackage{subfig}
\usepackage{multicol}
%\usepackage{appendixnumberbeamer}
%
\usepackage{tcolorbox}

\usepackage{pgfplots}
\usepackage{tikz}
\usetikzlibrary{trees} 
\usetikzlibrary{shapes.geometric}
\usetikzlibrary{positioning,shapes,shadows,arrows,calc,mindmap}
\usetikzlibrary{positioning,fadings,through}
\usetikzlibrary{decorations.pathreplacing}
\usetikzlibrary{intersections}
\usetikzlibrary{positioning,fit,calc,shadows,backgrounds}
\pgfdeclarelayer{background}
\pgfdeclarelayer{foreground}
\pgfsetlayers{background,main,foreground}
\tikzstyle{activity}=[rectangle, draw=black, rounded corners, text centered, text width=8em]
\tikzstyle{data}=[rectangle, draw=black, text centered, text width=8em]
\tikzstyle{myarrow}=[->, thick, draw=black]

% Define the layers to draw the diagram
\pgfdeclarelayer{background}
\pgfdeclarelayer{foreground}
\pgfsetlayers{background,main,foreground}

%\usepackage{listings}
%\lstset{numbers=left,
%  showstringspaces=false,
%  frame={tb},
%  captionpos=b,
%  lineskip=0pt,
%  basicstyle=\ttfamily,
%%  extendedchars=true,
%  stepnumber=1,
%  numberstyle=\small,
%  xleftmargin=1em,
%  breaklines
%}

 
\definecolor{blue}{RGB}{0, 74, 153}

\usetheme{Boadilla}
%\useinnertheme{rectangles}
\usecolortheme{whale}
\setbeamercolor{alerted text}{fg=blue}
\useoutertheme{infolines}
\setbeamertemplate{navigation symbols}{\vspace{-5pt}} % to lower the logo
\setbeamercolor{date in head/foot}{bg=blue} % blue
\setbeamercolor{date in head/foot}{fg=white}
\setbeamercolor{author in head/foot}{bg=blue} %blue
\setbeamercolor{title in head/foot}{bg=blue} % blue
\setbeamercolor{title}{fg=white, bg=blue}
\setbeamercolor{block title}{fg=white,bg=blue}
\setbeamercolor{block body}{bg=blue!10}
\setbeamercolor{frametitle}{fg=white, bg=blue}
\setbeamercovered{invisible}

\makeatletter
\setbeamertemplate{footline}
{
  \leavevmode%
  \hbox{%
  \begin{beamercolorbox}[wd=.333333\paperwidth,ht=2.25ex,dp=1ex,center]{author in head/foot}%
    \usebeamerfont{author in head/foot}\insertshortauthor
  \end{beamercolorbox}%
  \begin{beamercolorbox}[wd=.333333\paperwidth,ht=2.25ex,dp=1ex,center]{title in head/foot}%
    \usebeamerfont{title in head/foot}\insertshorttitle
  \end{beamercolorbox}%
  \begin{beamercolorbox}[wd=.333333\paperwidth,ht=2.25ex,dp=1ex,right]{date in head/foot}%
    \usebeamerfont{date in head/foot}Week \@week, Topic \@topicnumber, Slide \insertframenumber{}\hspace*{2em}
%    \insertframenumber\hspace*{2ex} 
  \end{beamercolorbox}}%
  \vskip0pt%
}

\newcommand{\@week}{0}
\newcommand{\@topicnumber}{0}
\newcommand{\week}[1]{\renewcommand{\@week}{#1}}
\newcommand{\topicnumber}[1]{\renewcommand{\@topicnumber}{#1}}

\makeatother

%\pgfdeclareimage[height=1.2cm]{automl}{images/logos/automl.png}
%\pgfdeclareimage[height=1.2cm]{freiburg}{images/logos/freiburg}

%\logo{\pgfuseimage{freiburg}}

\input{../latex_main/macros}




\title[AutoML: AC]{AutoML: Beyond AutoML}
\subtitle{Overview: Algorithm Configuration}
\author[Marius Lindauer]{Bernd Bischl \and Frank Hutter \and Lars Kotthoff\newline \and \underline{Marius Lindauer} \and Joaquin Vanschoren}
\institute{}
\date{}
\week{13}
\topicnumber{1}


% \AtBeginSection[] % Do nothing for \section*
% {
%   \begin{frame}{Outline}
%     \bigskip
%     \vfill
%     \tableofcontents[currentsection]
%   \end{frame}
% }

\begin{document}
	
	\maketitle
	

%----------------------------------------------------------------------
\begin{frame}[c]{Generalization of HPO}

\begin{itemize}
	\item hyperparameter optimization (HPO) is not limited to ML
	\pause
	\item in fact, you can optimize the performance of any algorithm by means of HPO if
	\begin{enumerate}
		\pause
		\item the algorithm at hand has parameters that influence its performance
		\pause
		\item you care about the empirical performance of an algorithm
	\end{enumerate}
	\pause
	\smallskip
	\item a limitation of HPO is that we assume that we care only about a single task\\ (i.e., dataset or input to the algorithm)
	\smallskip
	\pause
	\item[$\leadsto$] \alert{Can we find an algorithm's configuration that performs well and\\ robustly across a set of tasks?}
	\begin{itemize}
		\pause
		\item A hyperparameter configuration for a set of datasets
		\pause
		\item A parameter configuration of a SAT solver for a set of SAT instances
		\pause
		\item A parameter configuration of an AI planning solver for a set of planning problems
		\item \ldots
	\end{itemize}
	\pause
	\item[$\leadsto$] \alert{Algorithm configuration}
\end{itemize}


\end{frame}
%----------------------------------------------------------------------
%----------------------------------------------------------------------
\begin{frame}[c]{Algorithm Configuration Visualized}

\centering
\scalebox{0.5}{
\includegraphics{images/Configuration-Process.pdf}
}

\end{frame}
%-----------------------------------------------------------------------
%----------------------------------------------------------------------
\begin{frame}[c]{Algorithm Configuration -- in More Detail}

\bigskip

\centering
\scalebox{0.75}{
\input{tikz/ac}
}

\bigskip

\begin{block}{Definition}
Given a parameterized algorithm $\algo$ with possible (hyper-)parameter settings $\confs$, \pause 
a set of training problem instances $\insts$, \pause 
and a cost metric $c: \confs \times \insts \rightarrow \perf$, \pause 
the algorithm configuration problem is 
to \alert{find a parameter configuration $\conf^* \in \confs$ 
that minimizes $c$ across the instances in $\insts$}.
\end{block}

\end{frame}
%-----------------------------------------------------------------------


%----------------------------------------------------------------------
\begin{frame}[c]{Algorithm Configuration -- Full Formal Definition}

\begin{block}{Definition}
An instance of the algorithm configuration problem
is a 5-tuple $(\algo, \pcs, \instD, \cutoff, c)$ where:
\begin{itemize}
\item $\algo$ is a parameterized algorithm;
\item $\pcs$ is the (hyper-)parameter configuration space of $\algo$;
\item $\instD$ is a \alert{distribution over problem instances} with domain $\insts$;
\pause
\item $\cutoff < \infty$ is a \alert{cutoff time}, after which each run of $\algo$ will be terminated if still running
\pause
\item $c: \confs \times \insts \rightarrow \mathds{R}$ is a function that
measures the observed cost of running $\algo(\conf)$ on an instance $\inst \in
\insts$ with cutoff time $\cutoff$ 
%  \item $s$ is a statistical population parameter\\ (e.g., expectation, median,  or variance)
\end{itemize}
\pause
The cost of a candidate solution $\conf\in\confs$ is
%\begin{equation}
%\finconf \in \argmin_{\conf \in \pcs}
\alert{$f(\conf) = \mathds{E}_{\inst \sim \instD} (c(\conf,\inst))$}.\\
The goal is to find \alert{$\conf^* \in \argmin_{\conf \in \pcs} f(\conf)$}.
%\end{equation}

\end{block}

\end{frame}
%-----------------------------------------------------------------------

%----------------------------------------------------------------------
\begin{frame}[c]{Distribution of Instances}

We usually have a finite number of instances from a given application
\begin{itemize}
\item We want to do well on that type of instances
\item Future instances of this type should be solved well 
\end{itemize}

\pause
\bigskip

Like in machine learning
\begin{itemize}
\item We split the instances into a \alert{training set} and a \alert{test set}
\item We configure algorithms on the training instances
\item We only use the test instances afterwards
\begin{itemize}
\item[$\to$] unbiased estimate of generalization performance for unseen instances
\end{itemize}  
\end{itemize}


\end{frame}
%-----------------------------------------------------------------------
\begin{frame}[c]{Challenges of Algorithm Configuration}

\begin{itemize}
\item \alert{Structured high-dimensional parameter space}
\begin{itemize}
\item categorical vs. continuous parameters
\item conditionals between parameters
\end{itemize}
\pause
\medskip
\item \alert{Stochastic optimization}
\begin{itemize}
\item Randomized algorithms: optimization across various seeds
\item Distribution of benchmark instances (often wide range of hardness)
\item Subsumes so-called \emph{multi-armed bandit problem}
\end{itemize}
\pause
\medskip
\item \alert{Generalization across instances}
\begin{itemize}
\item apply algorithm configuration to \alert{homogeneous} instance sets
\item Instance sets can also be \alert{heterogeneous},\\i.e., no single configuration performs well on all instances\\ 
\item[$\leadsto$] combination of algorithm configuration and selection
\end{itemize}

\end{itemize}

\pause
\medskip
$\leadsto$ Hyperparameter optimization is a subproblem of algorithm configuration\newline \lit{\href{https://arxiv.org/pdf/1705.06058.pdf}{Eggensperger et al. 2019}}

\end{frame}
%-----------------------------------------------------------------------

\end{document}
