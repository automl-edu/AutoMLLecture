
\pdfminorversion=4 % for acroread
\documentclass[aspectratio=169,t,xcolor={usenames,dvipsnames}]{beamer}
%\documentclass[t,handout,xcolor={usenames,dvipsnames}]{beamer}
\usepackage{../beamerstyle}
\usepackage{dsfont}
\usepackage{bm}
\usepackage[english]{babel}
\usepackage[utf8]{inputenc}
\usepackage{graphicx}
\usepackage{algorithm}
\usepackage[ruled,vlined,algo2e,linesnumbered]{algorithm2e}
%\usepackage[boxed,vlined]{algorithm2e}
\usepackage{hyperref}
\usepackage{booktabs}
\usepackage{mathtools}

\usepackage{amsmath,amssymb}
\usepackage{listings}
\lstset{frame=lines,framesep=3pt,numbers=left,numberblanklines=false,basicstyle=\ttfamily\small}

\usepackage{subfig}
\usepackage{multicol}
%\usepackage{appendixnumberbeamer}
%
\usepackage{tcolorbox}

\usepackage{pgfplots}
\usepackage{tikz}
\usetikzlibrary{trees} 
\usetikzlibrary{shapes.geometric}
\usetikzlibrary{positioning,shapes,shadows,arrows,calc,mindmap}
\usetikzlibrary{positioning,fadings,through}
\usetikzlibrary{decorations.pathreplacing}
\usetikzlibrary{intersections}
\usetikzlibrary{positioning,fit,calc,shadows,backgrounds}
\pgfdeclarelayer{background}
\pgfdeclarelayer{foreground}
\pgfsetlayers{background,main,foreground}
\tikzstyle{activity}=[rectangle, draw=black, rounded corners, text centered, text width=8em]
\tikzstyle{data}=[rectangle, draw=black, text centered, text width=8em]
\tikzstyle{myarrow}=[->, thick, draw=black]

% Define the layers to draw the diagram
\pgfdeclarelayer{background}
\pgfdeclarelayer{foreground}
\pgfsetlayers{background,main,foreground}

%\usepackage{listings}
%\lstset{numbers=left,
%  showstringspaces=false,
%  frame={tb},
%  captionpos=b,
%  lineskip=0pt,
%  basicstyle=\ttfamily,
%%  extendedchars=true,
%  stepnumber=1,
%  numberstyle=\small,
%  xleftmargin=1em,
%  breaklines
%}

 
\definecolor{blue}{RGB}{0, 74, 153}

\usetheme{Boadilla}
%\useinnertheme{rectangles}
\usecolortheme{whale}
\setbeamercolor{alerted text}{fg=blue}
\useoutertheme{infolines}
\setbeamertemplate{navigation symbols}{\vspace{-5pt}} % to lower the logo
\setbeamercolor{date in head/foot}{bg=white} % blue
\setbeamercolor{date in head/foot}{fg=white}
\setbeamercolor{author  in head/foot}{bg=white} %blue
\setbeamercolor{title in head/foot}{bg=white} % blue
\setbeamercolor{title}{fg=white, bg=blue}
\setbeamercolor{block title}{fg=white,bg=blue}
\setbeamercolor{block body}{bg=blue!10}
\setbeamercolor{frametitle}{fg=white, bg=blue}
\setbeamercovered{invisible}

\makeatletter
\setbeamertemplate{footline}
{
  \leavevmode%
  \hbox{%
  \begin{beamercolorbox}[wd=.333333\paperwidth,ht=2.25ex,dp=1ex,center]{author in head/foot}%
%    \usebeamerfont{author in head/foot}\insertshortauthor
  \end{beamercolorbox}%
  \begin{beamercolorbox}[wd=.333333\paperwidth,ht=2.25ex,dp=1ex,center]{title in head/foot}%
    \usebeamerfont{title in head/foot}\insertshorttitle
  \end{beamercolorbox}%
  \begin{beamercolorbox}[wd=.333333\paperwidth,ht=2.25ex,dp=1ex,right]{date in head/foot}%
    \usebeamerfont{date in head/foot}\insertshortdate{}\hspace*{2em}
%    \insertframenumber\hspace*{2ex} 
  \end{beamercolorbox}}%
  \vskip0pt%
}
\makeatother

%\pgfdeclareimage[height=1.2cm]{automl}{images/logos/automl.png}
%\pgfdeclareimage[height=1.2cm]{freiburg}{images/logos/freiburg}

%\logo{\pgfuseimage{freiburg}}

\newcommand{\comment}[1]{
	\noindent
	%\vspace{0.25cm}
	{\color{red}{\textbf{TODO:} #1}}
	%\vspace{0.25cm}
}
\renewcommand{\comment}[1]{}
\newcommand{\hide}[1]{}
\newcommand{\cemph}[2]{\emph{\textcolor{#1}{#2}}}

\newcommand{\lit}[1]{{\footnotesize\color{black!70}[#1]}}

\newcommand{\litw}[1]{{\footnotesize\color{black!20}[#1]}}


\newcommand{\myframe}[2]{\begin{frame}[c]{#1}#2\end{frame}}
\newcommand{\myframetop}[2]{\begin{frame}{#1}#2\end{frame}}
\newcommand{\myit}[1]{\begin{itemize}#1\end{itemize}}
\newcommand{\myblock}[2]{\begin{block}{#1}#2\end{block}}


\newcommand{\votepurple}[1]{\textcolor{Purple}{$\bigstar$}}
\newcommand{\voteyellow}[1]{\textcolor{Goldenrod}{$\bigstar$}}
\newcommand{\voteblue}[1]{\textcolor{RoyalBlue}{$\bigstar$}}
\newcommand{\votepink}[1]{\textcolor{Pink}{$\bigstar$}}

\newcommand{\diff}{\mathop{}\!\mathrm{d}}
\newcommand{\refstyle}[1]{{\small{\textcolor{gray}{#1}}}}
\newcommand{\hands}[0]{\includegraphics[height=1.5em]{images/hands}}
\newcommand{\transpose}[0]{{\textrm{\tiny{\sf{T}}}}}
\newcommand{\norm}{{\mathcal{N}}}
\newcommand{\cutoff}[0]{\kappa}
\newcommand{\instD}[0]{\dataset}
\newcommand{\insts}[0]{\mathcal{I}}
\newcommand{\inst}[0]{i}
\newcommand{\pcs}[0]{\mathbf{\Lambda}}
\newcommand{\bx}[0]{\conf}
\newcommand{\conf}[0]{\mathbf{\lambda}}
\newcommand{\defconf}[0]{\mathbf{\lambda}_{\text{def}}}
\newcommand{\finconf}[0]{\mathbf{\lambda}^*}
\newcommand{\incumbent}[0]{\finconf}
\newcommand{\confs}[0]{\pcs}
%\newcommand{\vlambda}[0]{\bm{\lambda}}
%\newcommand{\vLambda}[0]{\bm{\Lambda}}
\newcommand{\dataset}[0]{\mathcal{D}}
\newcommand{\datasets}[0]{\mathbf{D}}
\newcommand{\loss}[0]{\mathcal{L}}

% \renewcommand{\vec}[1]{\mathbf{#1}}
\newcommand{\hist}[0]{\mathcal{H}}
\newcommand{\param}[0]{p}
\newcommand{\algo}[0]{\mathcal{A}}
\newcommand{\algos}[0]{\mathbf{A}}
%\newcommand{\nn}[0]{N}
\newcommand{\feats}[0]{\mathcal{F}}
\newcommand{\feat}[0]{\vec{f}}
\newcommand{\cluster}[0]{\vec{h}}
\newcommand{\clusters}[0]{\vec{H}}
\newcommand{\perf}[0]{\mathbb{R}}
%\newcommand{\surro}[0]{\mathcal{S}}
\newcommand{\surro}[0]{\hat{f}}
\newcommand{\func}[0]{f}
\newcommand{\epm}[0]{\surro}
\newcommand{\portfolio}[0]{\mathcal{P}}
\newcommand{\schedule}[0]{\mathcal{S}}
\newcommand{\mdata}[0]{\dataset_{\text{meta}}}

% Deep Learning
\newcommand{\weights}[0]{\theta}
\newcommand{\metaweights}[0]{\phi}


% reinforcement learning
\newcommand{\policies}[0]{\Pi}
\newcommand{\policy}[0]{\pi}
\newcommand{\actionRL}[0]{a}
\newcommand{\stateRL}[0]{s}
\newcommand{\statesRL}[0]{\mathcal{S}}
\newcommand{\rewardRL}[0]{r}
\newcommand{\rewardfuncRL}[0]{\mathcal{R}}

\RestyleAlgo{algoruled}
\DontPrintSemicolon
\LinesNumbered
\SetAlgoVlined
\SetFuncSty{textsc}

\SetKwInOut{Input}{Input}
\SetKwInOut{Output}{Output}
\SetKw{Return}{return}

%\newcommand{\changed}[1]{{\color{red}#1}}

%\newcommand{\citeN}[1]{\citeauthor{#1}~(\citeyear{#1})}

\renewcommand{\vec}[1]{\mathbf{#1}}
\DeclareMathOperator*{\argmin}{arg\,min}
\DeclareMathOperator*{\argmax}{arg\,max}

\newcommand{\aqme}{\textit{AQME}}
\newcommand{\aslib}{\textit{ASlib}}
\newcommand{\llama}{\textit{LLAMA}}
\newcommand{\satzilla}{\textit{SATzilla}}
\newcommand{\satzillaY}[1]{\textit{SATzilla'{#1}}}
\newcommand{\snnap}{\textit{SNNAP}}
\newcommand{\claspfolioTwo}{\textit{claspfolio~2}}
\newcommand{\flexfolio}{\textit{FlexFolio}}
\newcommand{\claspfolioOne}{\textit{claspfolio~1}}
\newcommand{\isac}{\textit{ISAC}}
\newcommand{\eisac}{\textit{EISAC}}
\newcommand{\sss}{\textit{3S}}
\newcommand{\sunny}{\textit{Sunny}}
\newcommand{\ssspar}{\textit{3Spar}}
\newcommand{\cshc}{\textit{CSHC}}  
\newcommand{\cshcpar}{\textit{CSHCpar}}  
\newcommand{\measp}{\textit{ME-ASP}} 
\newcommand{\aspeed}{\textit{aspeed}}
\newcommand{\autofolio}{\textit{AutoFolio}}
\newcommand{\cedalion}{\textit{Cedalion}}
\newcommand{\fanova}{\textit{fANOVA}}
\newcommand{\sbs}{\textit{SB}}
\newcommand{\oracle}{\textit{VBS}}

% like approaches
\newcommand{\claspfoliolike}[1]{\texttt{claspfolio-#1-like}}
\newcommand{\satzillalike}[1]{\texttt{SATzilla'#1-like}}
\newcommand{\isaclike}{\texttt{ISAC-like}}
\newcommand{\ssslike}{\texttt{3S-like}}
\newcommand{\measplike}{\texttt{ME-ASP-like}}

\newcommand{\aspCoseal}{\textit{ASP-POTASSCO}}
\newcommand{\cspCoseal}{\textit{CSP-2010}}
\newcommand{\maxsatCoseal}{\textit{MAXSAT12-PMS}}
\newcommand{\premarCoseal}{\textit{PRE\-MARSHALLING}}
\newcommand{\qbfCoseal}{\textit{QBF-2011}}
\newcommand{\satallTwelveCoseal}{\textit{SAT12-ALL}}
\newcommand{\sathandTwelveCoseal}{\textit{SAT12-HAND}}
\newcommand{\satinduTwelveCoseal}{\textit{SAT12-INDU}}
\newcommand{\satrandTwelveCoseal}{\textit{SAT12-RAND}}
\newcommand{\sathandElevenCoseal}{\textit{SAT11-HAND}}
\newcommand{\satinduElevenCoseal}{\textit{SAT11-INDU}}
\newcommand{\satrandElevenCoseal}{\textit{SAT11-RAND}}
\newcommand{\proteusCoseal}{\textit{PROTEUS-2014}}

\newcommand{\irace}{\textit{I/F-race}}
\newcommand{\gga}{\textit{GGA}}
\newcommand{\smac}{\textit{SMAC}}
\newcommand{\paramils}{\textit{ParamILS}}
\newcommand{\spearmint}{\textit{Spearmint}}
\newcommand{\tpe}{\textit{TPE}}

\newcommand{\gringo}{\textit{gringo}}
\newcommand{\clasp}{\textit{clasp}}
\newcommand{\lingeling}{\textit{lingeling}}

\newcommand{\hydra}{\textit{Hydra}}

\newcommand{\plingeling}{\textit{Plingeling}}
\newcommand{\ccasat}{\textit{CCASat}}

\usepackage{pifont}
\newcommand{\itarrow}{\mbox{\Pisymbol{pzd}{229}}}
\newcommand{\ithook}{\mbox{\Pisymbol{pzd}{52}}}
\newcommand{\itcross}{\mbox{\Pisymbol{pzd}{56}}}
\newcommand{\ithand}{\mbox{\raisebox{-1pt}{\Pisymbol{pzd}{43}}}}

%\DeclareMathOperator*{\argmax}{arg\,max}

\newcommand{\ie}{{\it{}i.e.\/}}
\newcommand{\eg}{{\it{}e.g.\/}}
\newcommand{\cf}{{\it{}cf.\/}}
\newcommand{\wrt}{\mbox{w.r.t.}}
\newcommand{\vs}{{\it{}vs\/}}
\newcommand{\vsp}{{\it{}vs\/}}
\newcommand{\etc}{{\copyedit{etc.}}}
\newcommand{\etal}{{\it{}et al.\/}}

\newcommand{\pscProc}{{\bf procedure}}
\newcommand{\pscBegin}{{\bf begin}}
\newcommand{\pscEnd}{{\bf end}}
\newcommand{\pscEndIf}{{\bf endif}}
\newcommand{\pscFor}{{\bf for}}
\newcommand{\pscEach}{{\bf each}}
\newcommand{\pscThen}{{\bf then}}
\newcommand{\pscElse}{{\bf else}}
\newcommand{\pscWhile}{{\bf while}}
\newcommand{\pscIf}{{\bf if}}
\newcommand{\pscRepeat}{{\bf repeat}}
\newcommand{\pscUntil}{{\bf until}}
\newcommand{\pscWithProb}{{\bf with probability}}
\newcommand{\pscOtherwise}{{\bf otherwise}}
\newcommand{\pscDo}{{\bf do}}
\newcommand{\pscTo}{{\bf to}}
\newcommand{\pscOr}{{\bf or}}
\newcommand{\pscAnd}{{\bf and}}
\newcommand{\pscNot}{{\bf not}}
\newcommand{\pscFalse}{{\bf false}}
\newcommand{\pscEachElOf}{{\bf each element of}}
\newcommand{\pscReturn}{{\bf return}}

%\newcommand{\param}[1]{{\sl{}#1}}
\newcommand{\var}[1]{{\it{}#1}}
\newcommand{\cond}[1]{{\sf{}#1}}
%\newcommand{\state}[1]{{\sf{}#1}}
%\newcommand{\func}[1]{{\sl{}#1}}
\newcommand{\set}[1]{{\Bbb #1}}
%\newcommand{\inst}[1]{{\tt{}#1}}
\newcommand{\myurl}[1]{{\small\sf #1}}

\newcommand{\Nats}{{\Bbb N}}
\newcommand{\Reals}{{\Bbb R}}
\newcommand{\extset}[2]{\{#1 \; | \; #2\}}

\newcommand{\vbar}{$\,\;|$\hspace*{-1em}\raisebox{-0.3mm}{$\,\;\;|$}}
\newcommand{\vendbar}{\raisebox{+0.4mm}{$\,\;|$}}
\newcommand{\vend}{$\,\:\lfloor$}


\newcommand{\goleft}[2][.7]{\parbox[t]{#1\linewidth}{\strut\raggedright #2\strut}}
\newcommand{\rightimage}[2][.3]{\mbox{}\hfill\raisebox{1em-\height}[0pt][0pt]{\includegraphics[width=#1\linewidth]{#2}}\vspace*{-\baselineskip}}







\title[AutoML: Overview]{AutoML: Automated Machine Learning}
\subtitle{Overview: Algorithm Configuration -- Beyond AutoML}
%TODO: change authors!
\author[Marius Lindauer]{Bernd Bischl \and Frank Hutter \and Lars Kotthoff \and \underline{Marius Lindauer}}
\institute{}
\date{}



% \AtBeginSection[] % Do nothing for \section*
% {
%   \begin{frame}{Outline}
%     \bigskip
%     \vfill
%     \tableofcontents[currentsection]
%   \end{frame}
% }

\begin{document}
	
	\maketitle
	

%----------------------------------------------------------------------
\begin{frame}[c]{Generalization of HPO}

\begin{itemize}
	\item hyperparameter optimization (HPO) is not limited to ML
	\pause
	\item in fact, you can optimize the performance of any algorithm by means of HPO if
	\begin{enumerate}
		\pause
		\item the algorithm at hand has parameters that influence its performance
		\pause
		\item you care about the empirical performance of an algorithm
	\end{enumerate}
	\pause
	\smallskip
	\item a limitation of HPO is that we assume that we care only about a single task\\ (i.e., dataset or input to the algorithm)
	\smallskip
	\pause
	\item[$\leadsto$] \alert{Can we find an algorithm's configuration that performs well and robustly across a set of tasks?}
	\begin{itemize}
		\pause
		\item An hyperparameter configuration for a set of datasets
		\pause
		\item A parameter configuration of a SAT solver for a set of SAT instances
		\pause
		\item A parameter configuration of a AI planning solver for a set of planning problems
		\item \ldots
	\end{itemize}
	\pause
	\item[$\leadsto$] \alert{Algorithm configuration}
\end{itemize}


\end{frame}
%----------------------------------------------------------------------
%----------------------------------------------------------------------
\begin{frame}[c]{Algorithm Configuration Visualized}

\centering
\scalebox{0.5}{
\includegraphics{images/Configuration-Process.pdf}
}

\end{frame}
%-----------------------------------------------------------------------
%----------------------------------------------------------------------
\begin{frame}[c]{Algorithm Configuration -- in More Detail}

\bigskip

\centering
\scalebox{0.75}{
\tikzstyle{activity}=[rectangle, draw=black, rounded corners, text centered, text width=8em, fill=white, drop shadow]
\tikzstyle{wideactivity}=[rectangle, draw=black, rounded corners, text centered, text width=10em, fill=white, drop shadow]
\tikzstyle{data}=[rectangle, draw=black, text centered, fill=black!10, text width=8em, drop shadow]
\tikzstyle{myarrow}=[->, thick]
\begin{tikzpicture}[node distance=5cm,thick]
	%PreProcessing
	%\node (Algo) [data] {Algorithm $A$};
	\node (Data) [data] {Instances $\insts$};
	\node (CS) [data, right of=Data, xshift=-0.5cm] {Algorithm $\algo$ and\\ its Configuration\\ Space $\pcs$};
	\node (Select) [activity, below of=Data, node distance=2.0cm] {Select $\conf \in \pcs$\\ and $\inst \in \insts$};
	\node (Run) [wideactivity, right of=Select, xshift=-0.5cm] {Run $\algo(\conf)$ on $\inst$ to measure $c(\conf,\inst)$};
	%\node (Return) [activity, right of=Run, text width=9em] {Return Performance\\ of $A(c)$ on $I'$}; 
	%\node (Data) [data, left of=Select] {Instances $I$};	
	\node (Result) [activity, right of=Run, node distance=4.6cm] {Returns Best\\ Configuration $\hat{\conf}$}; 
	
	\draw[myarrow] (Data) -- ($(Select)+(-0.0,+0.8)$);
	\draw[myarrow] (CS) -- ($(Run)+(-0.0,+0.8)$);
	%\draw[myarrow] (Data) -- ($(Select)+(-2.1,+0.0)$);
	
	%\draw[thick, dashed] (Algo) -- (CS);
	\draw[myarrow] ($(Run.east)+(0.25,0)$) -- (Result);
	\draw[myarrow] (Select) -- (Run);
	%\draw[myarrow] (Run) -- (Return);
	\draw[myarrow] (Run.south) |- ++(0.0,-0.8)  node[above, xshift=-2.2cm] {Return Cost} -| (Select.south);
	
	\begin{pgfonlayer}{background}
    
        % Configuration Process
    	\path (Select -| Select.west)+(-0.25,0.85) node (resUL) {};
    	\path (Run.east |- Run.south)+(0.25,-1.3) node(resBR) {};
    	\path [rounded corners, draw=black!60, dashed] (resUL) rectangle (resBR);
		\path (Run.east |- Run.south)+(-1.5,-1.1) node [text=black!60] {Configuration Task};
    	
    \end{pgfonlayer}
	
\end{tikzpicture}

}

\bigskip

\begin{block}{Definition}
Given a parameterized algorithm $\algo$ with possible (hyper-)parameter settings $\confs$, \pause 
a set of training problem instances $\insts$, \pause 
and a cost metric $c: \confs \times \insts \rightarrow \perf$, \pause 
the algorithm configuration problem is 
to \alert{find a parameter configuration $\conf^* \in \confs$ 
that minimizes $c$ across the instances in $\insts$}.
\end{block}

\end{frame}
%-----------------------------------------------------------------------


%----------------------------------------------------------------------
\begin{frame}[c]{Algorithm Configuration -- Full Formal Definition}

\begin{block}{Definition}
An instance of the algorithm configuration problem
is a 5-tuple $(\algo, \pcs, \instD, \cutoff, c)$ where:
\begin{itemize}
\item $\algo$ is a parameterized algorithm;
\item $\pcs$ is the (hyper-)parameter configuration space of $\algo$;
\item $\instD$ is a \alert{distribution over problem instances} with domain $\insts$;
\pause
\item $\cutoff < \infty$ is a \alert{cutoff time}, after which each run of $\algo$ will be terminated if still running
\pause
\item $c: \confs \times \insts \rightarrow \mathds{R}$ is a function that
measures the observed cost of running $\algo(\conf)$ on an instance $\inst \in
\insts$ with cutoff time $\cutoff$ 
%  \item $s$ is a statistical population parameter\\ (e.g., expectation, median,  or variance)
\end{itemize}
\pause
The cost of a candidate solution $\conf\in\confs$ is
%\begin{equation}
%\hat{\conf} \in \argmin_{\conf \in \pcs}
\alert{$f(\conf) = \mathds{E}_{\inst \sim \instD} (c(\conf,\inst))$}.\\
The goal is to find \alert{$\conf^* \in \argmin_{\conf \in \pcs} f(\conf)$}.
%\end{equation}

\end{block}

\end{frame}
%-----------------------------------------------------------------------

%----------------------------------------------------------------------
\begin{frame}[c]{Distribution of Instances}

We usually have a finite number of instances from a given application
\begin{itemize}
\item We want to do well on that type of instances
\item Future instances of this type should be solved well 
\end{itemize}

\pause
\bigskip

Like in machine learning
\begin{itemize}
\item We split the instances into a \alert{training set} and a \alert{test set}
\item We configure algorithms on the training instances
\item We only use the test instances afterwards
\begin{itemize}
\item[$\to$] unbiased estimate of generalization performance for unseen instances
\end{itemize}  
\end{itemize}


\end{frame}
%-----------------------------------------------------------------------
\begin{frame}[c]{Challenges of Algorithm Configuration}

\begin{itemize}
\item \alert{Structured high-dimensional parameter space}
\begin{itemize}
\item categorical vs. continuous parameters
\item conditionals between parameters
\end{itemize}
\pause
\medskip
\item \alert{Stochastic optimization}
\begin{itemize}
\item Randomized algorithms: optimization across various seeds
\item Distribution of benchmark instances (often wide range of hardness)
\item Subsumes so-called \emph{multi-armed bandit problem}
\end{itemize}
\pause
\medskip
\item \alert{Generalization across instances}
\begin{itemize}
\item apply algorithm configuration to \alert{homogeneous} instance sets
\item Instance sets can also be \alert{heterogeneous},\\i.e., no single configuration performs well on all instances\\ 
\item[$\leadsto$] combination of algorithm configuration and selection
\end{itemize}

\end{itemize}

\pause
\medskip
$\leadsto$ Hyperparameter optimization is a subproblem of algorithm configuration\newline \lit{\href{https://arxiv.org/abs/1705.06058}{Eggensperger et al. 2019}}

\end{frame}
%-----------------------------------------------------------------------

\end{document}
