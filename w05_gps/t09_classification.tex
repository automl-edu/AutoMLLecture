
\pdfminorversion=4 % for acroread
\documentclass[aspectratio=169,t,xcolor={usenames,dvipsnames}]{beamer}
%\documentclass[t,handout,xcolor={usenames,dvipsnames}]{beamer}
\usepackage{../beamerstyle}
\usepackage{dsfont}
\usepackage{bm}
\usepackage[english]{babel}
\usepackage[utf8]{inputenc}
\usepackage{graphicx}
\usepackage{algorithm}
\usepackage[ruled,vlined,algo2e,linesnumbered]{algorithm2e}
%\usepackage[boxed,vlined]{algorithm2e}
\usepackage{hyperref}
\usepackage{booktabs}
\usepackage{mathtools}

\usepackage{amsmath,amssymb}
\usepackage{listings}
\lstset{frame=lines,framesep=3pt,numbers=left,numberblanklines=false,basicstyle=\ttfamily\small}

\usepackage{subfig}
\usepackage{multicol}
%\usepackage{appendixnumberbeamer}
%
\usepackage{tcolorbox}

\usepackage{pgfplots}
\usepackage{tikz}
\usetikzlibrary{trees} 
\usetikzlibrary{shapes.geometric}
\usetikzlibrary{positioning,shapes,shadows,arrows,calc,mindmap}
\usetikzlibrary{positioning,fadings,through}
\usetikzlibrary{decorations.pathreplacing}
\usetikzlibrary{intersections}
\usetikzlibrary{positioning,fit,calc,shadows,backgrounds}
\pgfdeclarelayer{background}
\pgfdeclarelayer{foreground}
\pgfsetlayers{background,main,foreground}
\tikzstyle{activity}=[rectangle, draw=black, rounded corners, text centered, text width=8em]
\tikzstyle{data}=[rectangle, draw=black, text centered, text width=8em]
\tikzstyle{myarrow}=[->, thick, draw=black]

% Define the layers to draw the diagram
\pgfdeclarelayer{background}
\pgfdeclarelayer{foreground}
\pgfsetlayers{background,main,foreground}

%\usepackage{listings}
%\lstset{numbers=left,
%  showstringspaces=false,
%  frame={tb},
%  captionpos=b,
%  lineskip=0pt,
%  basicstyle=\ttfamily,
%%  extendedchars=true,
%  stepnumber=1,
%  numberstyle=\small,
%  xleftmargin=1em,
%  breaklines
%}

 
\definecolor{blue}{RGB}{0, 74, 153}

\usetheme{Boadilla}
%\useinnertheme{rectangles}
\usecolortheme{whale}
\setbeamercolor{alerted text}{fg=blue}
\useoutertheme{infolines}
\setbeamertemplate{navigation symbols}{\vspace{-5pt}} % to lower the logo
\setbeamercolor{date in head/foot}{bg=blue} % blue
\setbeamercolor{date in head/foot}{fg=white}
\setbeamercolor{author in head/foot}{bg=blue} %blue
\setbeamercolor{title in head/foot}{bg=blue} % blue
\setbeamercolor{title}{fg=white, bg=blue}
\setbeamercolor{block title}{fg=white,bg=blue}
\setbeamercolor{block body}{bg=blue!10}
\setbeamercolor{frametitle}{fg=white, bg=blue}
\setbeamercovered{invisible}

\makeatletter
\setbeamertemplate{footline}
{
  \leavevmode%
  \hbox{%
  \begin{beamercolorbox}[wd=.333333\paperwidth,ht=2.25ex,dp=1ex,center]{author in head/foot}%
    \usebeamerfont{author in head/foot}\insertshortauthor
  \end{beamercolorbox}%
  \begin{beamercolorbox}[wd=.333333\paperwidth,ht=2.25ex,dp=1ex,center]{title in head/foot}%
    \usebeamerfont{title in head/foot}\insertshorttitle
  \end{beamercolorbox}%
  \begin{beamercolorbox}[wd=.333333\paperwidth,ht=2.25ex,dp=1ex,right]{date in head/foot}%
    \usebeamerfont{date in head/foot}Week \@week, Topic \@topicnumber, Slide \insertframenumber{}\hspace*{2em}
%    \insertframenumber\hspace*{2ex} 
  \end{beamercolorbox}}%
  \vskip0pt%
}

\newcommand{\@week}{0}
\newcommand{\@topicnumber}{0}
\newcommand{\week}[1]{\renewcommand{\@week}{#1}}
\newcommand{\topicnumber}[1]{\renewcommand{\@topicnumber}{#1}}

\makeatother

%\pgfdeclareimage[height=1.2cm]{automl}{images/logos/automl.png}
%\pgfdeclareimage[height=1.2cm]{freiburg}{images/logos/freiburg}

%\logo{\pgfuseimage{freiburg}}

\input{../latex_main/macros}





\newcommand{\lz}{\vspace{0.5cm}}
\newcommand{\thetab}{\bm{\weights}}
\newcommand{\zero}{\mathbf{0}}
\newcommand{\Xmat}{\mathbf{X}}
\newcommand{\ydat}{\mathbf{y}}
\newcommand{\id}{\boldsymbol{I}}
\newcommand{\Amat}{\mathbf{A}}
\newcommand{\Xspace}{\mathcal{X}}                                         
\newcommand{\Yspace}{\mathcal{Y}}
\newcommand{\ls}{\ell}
\newcommand{\natnum}{\mathbb{N}}
\newcommand{\intnum}{\mathbb{Z}}
\newcommand{\sumin}{\sum\limits_{i=1}^n}	

\usepackage{fontawesome}
\usepackage{dirtytalk}
\usepackage{csquotes}

%\begin{frame}[c]{}
%\centering
%\huge
%\textbf{}
%\end{frame}


%\item[\faLightbulbO]
\title[AutoML: GPs]{AutoML: Gaussian Processes} % week title
\subtitle{Gaussian Process Classification} % video title
\author[Marius Lindauer]{\underline{Bernd Bischl} \and Frank Hutter \and Lars Kotthoff\newline \and Marius Lindauer \and Joaquin Vanschoren}
\institute{}
\date{}
\week{5}
\topicnumber{9}




\begin{document}
\maketitle


%%%%%%%%%%%%%%%%%%%%%%%%%%%%%%%%%%%%%%%%%%%%%%%%%%%%%%%%%%%%%%%%%%%%%%%%%%%%%%%%%%%%

\begin{frame}[c,allowframebreaks]{Training a GP via the Maximum Likelihood}

\begin{itemize}
\item Consider a binary classification problem, in which we want to learn $h: \Xspace \to \Yspace$, where $\Yspace = \{0, 1\}$. 

\lz
\item The idea behind Gaussian process classification is straightforward: a GP prior is placed over the score function $f(\x)$ and then transformed to a class probability via a sigmoid function $s(t)$:
\vspace{-3mm}
$$p(y = 1 \mid f(\x) ) = s(f(\x)).$$

\vspace{3mm}
\item Since this is a non-Gaussian likelihood, we need to use approximate inference methods, such as Laplace approximation, expecation propagation, MCMC.

\lz
\item For more details see \lit{\href{http://www.gaussianprocess.org/gpml/chapters/RW.pdf\#page=51&zoom=auto,-17,649}{Rasmussen and Williams. 2006: Chapter 3}} 
%\emph{Rasmussen, Gaussian Processes for Machine Learning, Chapter 3}.
\end{itemize}


\framebreak
%%%%%%%%%%%%%%%%%%%%%%%%%%%%%%%%%%%%%%%%%%%%%%%%%%%%%%%%%%%%%%%%%%%%%%%%%%%%%%%%%%%%

%\vspace*{1cm}

\begin{figure}
\includegraphics[width=0.9\textwidth]{figure_man/gp-classif-1-1.pdf}
\end{figure}



\framebreak
%%%%%%%%%%%%%%%%%%%%%%%%%%%%%%%%%%%%%%%%%%%%%%%%%%%%%%%%%%%%%%%%%%%%%%%%%%%%%%%%%%%%


\begin{itemize}

\item According to Bayes' rule, the posterior of the score function $\bm{f}$ takes the following form:

\vspace{-0.5cm}
\begin{eqnarray*}p(\bm{f}\mid\Xmat, \ydat) &=&  \frac{p(\ydat\mid\bm{f}, \Xmat) \cdot p(\bm{f}\mid\Xmat)}{p(\ydat\mid\Xmat)} \propto p(\ydat\mid\bm{f}) \cdot p(\bm{f}\mid\Xmat),\end{eqnarray*}
\vspace{-2mm}
where, the denominator is independent of $\bm{f}$ and hence has been dropped.

\vspace{15mm}

\item From the GP assumption, we can assert that $p(\bm{f}\mid\Xmat) \sim \mathcal{N}\left(0, \bm{K}\right)$. Hence, we have:
\vspace{-2mm}
$$\log p(\bm{f}\mid\Xmat, y) \propto \log p(\ydat\mid\bm{f}) - \frac{1}{2} \bm{f}^\top \bm{K}^{-1} \bm{f} - \frac{1}{2} \log |\bm{K}| - \frac{n}{2} \log 2 \pi.$$

\end{itemize}


\framebreak
%%%%%%%%%%%%%%%%%%%%%%%%%%%%%%%%%%%%%%%%%%%%%%%%%%%%%%%%%%%%%%%%%%%%%%%%%%%%%%%%%%%%

\begin{itemize}

\item If the kernel is fixed, the last two terms will be fixed. To obtain the maximum a-posteriori estimate (MAP), we should minimize:

$$\frac{1}{2} \bm{f}^\top \bm{K}^{-1}\bm{f} - \sumin\, \log\, p(\yI{i}\mid f^{(i)}) + C.$$

\vspace{0.5cm}
\item Note that $-\sumin\,\log \,p(\yI{i}\mid f^{(i)})$ is the logistic loss.

\vspace{0.7cm}
\item It can be seen that the Gaussian process classification corresponds to the \textbf{kernel Bayesian logistic regression}!
\end{itemize}

\end{frame}




\begin{frame}[c,allowframebreaks]{Comparison: GP vs. SVM}


\begin{itemize}
\item For the SVM, we have: 
\vspace{-2mm}
$$\frac{1}{2} \|\thetab\|^2 + C \sumin \loss(\yI{i},f(\xI{i})),$$

\vspace{-2mm}
where $\loss(\y,f(\x))  = \max\{0, 1-f(\x)\cdot\y\}$ is the Hinge loss.


\vspace{.5cm}
\item Plugging that in, the optimization objective would be:
\vspace{-2mm}
$$\frac{1}{2} \bm{f}^\top \bm{K}^{-1} \bm{f} + C \sumin \loss(\yI{i},f(\xI{i})).$$


\vspace{.1cm}
\item From the representer theorem: $\thetab = \sumin \beta_i\,\yI{i} k\left(\xI{i}, \cdot \right)$, and thus:
\vspace{-2mm}
$$\thetab^\top \thetab = \beta^\top \bm{K} \beta = \bm{f}^\top \bm{K}^{-1} \bm{f}$$. 

\end{itemize}



\framebreak
%%%%%%%%%%%%%%%%%%%%%%%%%%%%%%%%%%%%%%%%%%%%%%%%%%%%%%%%%%%%%%%%%%%%%%%%%%%%%%%%%%%%

\begin{itemize}
\item For log-concave likelihoods $\log\, p(\ydat\mid \bm{f})$, there is a close correspondence between the MAP solution of the GP classifier and the SVM solution:

\lz

$$\argmin_f\,\frac{1}{2}\,\bm{f}^\top \bm{K}^{-1} \bm{f}- \sumin \log\, p(\yI{i}\mid f^{(i)}) + C \quad \text{(GP classifier)}$$

\lz

$$\argmin_f\,\frac{1}{2}\, \bm{f}^\top \bm{K}^{-1} \bm{f} + C \sumin \loss(\yI{i},f(\xI{i})) \quad \text{(SVM classifier)}$$
\end{itemize}


\framebreak
%%%%%%%%%%%%%%%%%%%%%%%%%%%%%%%%%%%%%%%%%%%%%%%%%%%%%%%%%%%%%%%%%%%%%%%%%%%%%%%%%%%%

\begin{itemize}
\item Both the Hinge loss and the Bernoulli loss are monotonically decreasing with increasing margin $\y f(\x)$. 
\item The key difference is that the Hinge loss takes on the value $0$ for $\y f(\x) \ge 1$, while the Bernoulli loss decays slowly.
\item It is this flat part of the Hinge function that gives rise to the sparsity of the SVM solution. 
\item The SVM classifier can be construed as a ``sparse'' GP classifier.
\end{itemize}

\begin{figure}
\includegraphics[width=0.6\textwidth]{figure/los-exp-plot-1.pdf}
\end{figure}
\framebreak
%%%%%%%%%%%%%%%%%%%%%%%%%%%%%%%%%%%%%%%%%%%%%%%%%%%%%%%%%%%%%%%%%%%%%%%%%%%%%%%%%%%%



\framebreak
%%%%%%%%%%%%%%%%%%%%%%%%%%%%%%%%%%%%%%%%%%%%%%%%%%%%%%%%%%%%%%%%%%%%%%%%%%%%%%%%%%%%



\framebreak
%%%%%%%%%%%%%%%%%%%%%%%%%%%%%%%%%%%%%%%%%%%%%%%%%%%%%%%%%%%%%%%%%%%%%%%%%%%%%%%%%%%%



\framebreak
%%%%%%%%%%%%%%%%%%%%%%%%%%%%%%%%%%%%%%%%%%%%%%%%%%%%%%%%%%%%%%%%%%%%%%%%%%%%%%%%%%%%



\framebreak
%%%%%%%%%%%%%%%%%%%%%%%%%%%%%%%%%%%%%%%%%%%%%%%%%%%%%%%%%%%%%%%%%%%%%%%%%%%%%%%%%%%%



\framebreak
%%%%%%%%%%%%%%%%%%%%%%%%%%%%%%%%%%%%%%%%%%%%%%%%%%%%%%%%%%%%%%%%%%%%%%%%%%%%%%%%%%%%



\framebreak
%%%%%%%%%%%%%%%%%%%%%%%%%%%%%%%%%%%%%%%%%%%%%%%%%%%%%%%%%%%%%%%%%%%%%%%%%%%%%%%%%%%%

\end{frame}
%%%%%%%%%%%%%%%%%%%%%%%%%%%%%%%%%%%%%%%%%%%%%%%%%%%%%%%%%%%%%%%%%%%%%%%%%%%%%%%%%%%%
\end{document}
