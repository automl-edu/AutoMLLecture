
\pdfminorversion=4 % for acroread
\documentclass[aspectratio=169,t,xcolor={usenames,dvipsnames}]{beamer}
%\documentclass[t,handout,xcolor={usenames,dvipsnames}]{beamer}
\usepackage{../beamerstyle}
\usepackage{dsfont}
\usepackage{bm}
\usepackage[english]{babel}
\usepackage[utf8]{inputenc}
\usepackage{graphicx}
\usepackage{algorithm}
\usepackage[ruled,vlined,algo2e,linesnumbered]{algorithm2e}
%\usepackage[boxed,vlined]{algorithm2e}
\usepackage{hyperref}
\usepackage{booktabs}
\usepackage{mathtools}

\usepackage{amsmath,amssymb}
\usepackage{listings}
\lstset{frame=lines,framesep=3pt,numbers=left,numberblanklines=false,basicstyle=\ttfamily\small}

\usepackage{subfig}
\usepackage{multicol}
%\usepackage{appendixnumberbeamer}
%
\usepackage{tcolorbox}

\usepackage{pgfplots}
\usepackage{tikz}
\usetikzlibrary{trees} 
\usetikzlibrary{shapes.geometric}
\usetikzlibrary{positioning,shapes,shadows,arrows,calc,mindmap}
\usetikzlibrary{positioning,fadings,through}
\usetikzlibrary{decorations.pathreplacing}
\usetikzlibrary{intersections}
\usetikzlibrary{positioning,fit,calc,shadows,backgrounds}
\pgfdeclarelayer{background}
\pgfdeclarelayer{foreground}
\pgfsetlayers{background,main,foreground}
\tikzstyle{activity}=[rectangle, draw=black, rounded corners, text centered, text width=8em]
\tikzstyle{data}=[rectangle, draw=black, text centered, text width=8em]
\tikzstyle{myarrow}=[->, thick, draw=black]

% Define the layers to draw the diagram
\pgfdeclarelayer{background}
\pgfdeclarelayer{foreground}
\pgfsetlayers{background,main,foreground}

%\usepackage{listings}
%\lstset{numbers=left,
%  showstringspaces=false,
%  frame={tb},
%  captionpos=b,
%  lineskip=0pt,
%  basicstyle=\ttfamily,
%%  extendedchars=true,
%  stepnumber=1,
%  numberstyle=\small,
%  xleftmargin=1em,
%  breaklines
%}

 
\definecolor{blue}{RGB}{0, 74, 153}

\usetheme{Boadilla}
%\useinnertheme{rectangles}
\usecolortheme{whale}
\setbeamercolor{alerted text}{fg=blue}
\useoutertheme{infolines}
\setbeamertemplate{navigation symbols}{\vspace{-5pt}} % to lower the logo
\setbeamercolor{date in head/foot}{bg=blue} % blue
\setbeamercolor{date in head/foot}{fg=white}
\setbeamercolor{author in head/foot}{bg=blue} %blue
\setbeamercolor{title in head/foot}{bg=blue} % blue
\setbeamercolor{title}{fg=white, bg=blue}
\setbeamercolor{block title}{fg=white,bg=blue}
\setbeamercolor{block body}{bg=blue!10}
\setbeamercolor{frametitle}{fg=white, bg=blue}
\setbeamercovered{invisible}

\makeatletter
\setbeamertemplate{footline}
{
  \leavevmode%
  \hbox{%
  \begin{beamercolorbox}[wd=.333333\paperwidth,ht=2.25ex,dp=1ex,center]{author in head/foot}%
    \usebeamerfont{author in head/foot}\insertshortauthor
  \end{beamercolorbox}%
  \begin{beamercolorbox}[wd=.333333\paperwidth,ht=2.25ex,dp=1ex,center]{title in head/foot}%
    \usebeamerfont{title in head/foot}\insertshorttitle
  \end{beamercolorbox}%
  \begin{beamercolorbox}[wd=.333333\paperwidth,ht=2.25ex,dp=1ex,right]{date in head/foot}%
    \usebeamerfont{date in head/foot}Week \@week, Topic \@topicnumber, Slide \insertframenumber{}\hspace*{2em}
%    \insertframenumber\hspace*{2ex} 
  \end{beamercolorbox}}%
  \vskip0pt%
}

\newcommand{\@week}{0}
\newcommand{\@topicnumber}{0}
\newcommand{\week}[1]{\renewcommand{\@week}{#1}}
\newcommand{\topicnumber}[1]{\renewcommand{\@topicnumber}{#1}}

\makeatother

%\pgfdeclareimage[height=1.2cm]{automl}{images/logos/automl.png}
%\pgfdeclareimage[height=1.2cm]{freiburg}{images/logos/freiburg}

%\logo{\pgfuseimage{freiburg}}

\input{../latex_main/macros}




\newcommand{\lz}{\vspace{0.5cm}}
\newcommand{\thetab}{\bm{\weights}}
\newcommand{\zero}{\mathbf{0}}
\newcommand{\Xmat}{\mathbf{X}}
\newcommand{\ydat}{\mathbf{y}}
\newcommand{\id}{\boldsymbol{I}}
\newcommand{\Amat}{\mathbf{A}}
\newcommand{\Xspace}{\mathcal{X}}                                           
\newcommand{\Yspace}{\mathcal{Y}}
\newcommand{\ls}{\ell}
\newcommand{\natnum}{\mathbb{N}}
\newcommand{\intnum}{\mathbb{Z}}

\usepackage{fontawesome}
\usepackage{dirtytalk}
\usepackage{csquotes}

%\setbeamertemplate{section page}[mine]
%\setbeamertemplate{subsection page}[mine]

\title[AutoML: GPs]{AutoML: Gaussian Processes} % week title
\subtitle{Gaussian Processes} % video title
\author[Marius Lindauer]{\underline{Bernd Bischl} \and Frank Hutter \and Lars Kotthoff\newline \and Marius Lindauer \and Joaquin Vanschoren}
\institute{}
\date{}
\week{5}
\topicnumber{2}



% \AtBeginSection[] % Do nothing for \section*
% {
%   \begin{frame}{Outline}
%     \bigskip
%     \vfill
%     \tableofcontents[currentsection]
%   \end{frame}
% }

\begin{document}
	
	\maketitle
	

%----------------------------------------------------------------------
%----------------------------------------------------------------------
\begin{frame}[c]{Weight-Space View}

\begin{itemize}
  \item So far, we have considered a hypothesis space $\Hspace$ of parameterized functions $f(\x\mid\thetab)$ \\(in particular, the space of linear functions). 
  \lz
  \item Using Bayesian inference, we derived distributions for $\thetab$ after having observed data $\datasettrain$.
  \lz
  \item Prior believes about the parameter are expressed via a prior distribution $q(\thetab)$, which is updated according to Bayes' rule 

$$
\underbrace{p(\thetab \mid \Xmat, \ydat)}_{\text{posterior}} = \frac{\overbrace{p(\ydat \mid \Xmat, \thetab)}^{\text{likelihood}}\overbrace{q(\thetab)}^{\text{prior}}}{\underbrace{p(\ydat\mid\Xmat)}_{\text{marginal}}}.
$$
\end{itemize}

\end{frame}



%%%%%%%%%%%%%%%%%%%%%%%%%%%%%%%%%%%%%%%%%%%%%%%%%%%%%%%%%%%%%%%%%%%%%%%%%%%%%%%%%%%%%
\begin{frame}[c,allowframebreaks]{Function-Space View}


Let us change our point of view:

\lz

\begin{itemize}
  \item Instead of ``searching'' for a parameter  $\thetab$ in the parameter space, we directly search in a space of ``allowed'' functions $\Hspace$.
  \lz
  \lz
  \item We will still use Bayesian inference, but instead of specifying a prior distribution over a parameter, we will specify a prior distribution \textbf{over functions} and will update it according to the data points that we observe. 
\end{itemize}



%%%%%%%%%%%%%%%%%%%%%%%%%%%%%%%%%%%%%%%%%%%%%%%%%%%%%%%%%%%%%%%%%%%%%%%%%%%%%%%%%%%%
\framebreak

Intuitively, imagine we could draw a huge number of functions from some prior distribution over functions $^{(*)}$. 

\vspace*{-0.5cm}

\begin{figure}
  \includegraphics[width=0.5\textwidth]{figure_man/gp-sample/gp-sample-1-1.pdf}
\end{figure}

\vspace*{-0.5cm}

\begin{footnotesize}
  $^{(*)}$ We will see in a minute how distributions over functions can be specified. 
\end{footnotesize}


%%%%%%%%%%%%%%%%%%%%%%%%%%%%%%%%%%%%%%%%%%%%%%%%%%%%%%%%%%%%%%%%%%%%%%%%%%%%%%%%%%%%
\framebreak

\foreach \x in{1,2,3} {
    After observing some data points, we are allowed to sample only those functions that are consistent with the data. \\
  \begin{figure}
    \includegraphics[width=0.5\textwidth]{figure_man/gp-sample/gp-sample-2-\x.pdf}
  \end{figure}
}

%%%%%%%%%%%%%%%%%%%%%%%%%%%%%%%%%%%%%%%%%%%%%%%%%%%%%%%%%%%%%%%%%%%%%%%%%%%%%%%%%%%%
\framebreak


As we observe more and more data points, the number of functions that consistent with the data shrinks.

  \begin{figure}
    \includegraphics[width=0.5\textwidth]{figure_man/gp-sample/gp-sample-2-4.pdf}
  \end{figure}
  

%%%%%%%%%%%%%%%%%%%%%%%%%%%%%%%%%%%%%%%%%%%%%%%%%%%%%%%%%%%%%%%%%%%%%%%%%%%%%%%%%%%%
\framebreak

Intuitively, there is something like the ``mean'' and ``variance'' of a distribution over functions. 

  \begin{figure}
    \includegraphics[width=0.5\textwidth]{figure_man/gp-sample/gp-sample-2-4.pdf}
  \end{figure}

\end{frame}
%%%%%%%%%%%%%%%%%%%%%%%%%%%%%%%%%%%%%%%%%%%%%%%%%%%%%%%%%%%%%%%%%%%%%%%%%%%%%%%%%%%%
\begin{frame}[c]{Weight-Space View vs. Function-Space View}

\begin{table}
  \begin{tabular}{cc}
  \textbf{Weight-Space View} & \textbf{Function-Space View} \vspace{4mm}\\ 
  Parameterize functions & \vspace{1mm}\\
  \footnotesize Example: $f(\x\mid\thetab) = \thetab^\top\x$ & \vspace{4mm}\\
  Define distributions on $\thetab$ & Define distributions on $f$ \vspace{4mm}\\
  Inference in parameter space $\Theta$ & Inference in function space $\Hspace$
  \end{tabular}
\end{table}  

\lz
\lz

Next, we will see how we can define distributions over functions mathematically. 


\end{frame}
%%%%%%%%%%%%%%%%%%%%%%%%%%%%%%%%%%%%%%%%%%%%%%%%%%%%%%%%%%%%%%%%%%%%%%%%%%%%%%%%%%%%

\begin{frame}[c]{}
\centering
\huge
\textbf{Distributions on Functions}
\end{frame}



\begin{frame}[c,allowframebreaks]{Discrete Functions}

For simplicity, we will firstly consider functions with finite domains. 

\lz
\lz


\begin{itemize}
\item Let $\mathcal{X} = \left\{\xI{1},\dots, \xI{n}\right\}$ be a finite set of elements and $\Hspace$ the set of all functions $h: \mathcal{X} \to\realnum$.

\lz
\lz

\item Since the domain of any $h(\cdot) \in \Hspace$ has only $n$ elements, we can represent the function $h(\cdot)$ compactly as a $n$-dimensional vector 

$$\bm{h} = \left[h\left(\xI{1}\right),\dots, h\left(\xI{n}\right)\right].$$

\end{itemize}

%%%%%%%%%%%%%%%%%%%%%%%%%%%%%%%%%%%%%%%%%%%%%%%%%%%%%%%%%%%%%%%%%%%%%%%%%%%%%%%%%%%%
\framebreak

\foreach \x in{1,2,3} {
\textbf{Example 1:} Consider function $h: \Xspace \to \Yspace$ where the input space consists of \textbf{two} points $\Xspace = \{0, 1\}$.

\vspace{.3cm}
Examples for functions that live in this space:
\vspace{.3cm}

\begin{figure}
  \includegraphics[width=0.5\linewidth]{figure_man/discrete/example_2_\x.pdf} \par
\end{figure}
}


%%%%%%%%%%%%%%%%%%%%%%%%%%%%%%%%%%%%%%%%%%%%%%%%%%%%%%%%%%%%%%%%%%%%%%%%%%%%%%%%%%%%
\framebreak

\foreach \x in{1,2,3} {
\textbf{Example 2:} Consider $h: \Xspace \to \Yspace$ where the input space consists of \textbf{five} points $\Xspace = \{0, 0.25, 0.5, 0.75, 1\}$.

\vspace{.3cm}
Examples for functions that live in this space:
\vspace{.3cm}


\begin{figure}
  \includegraphics[width=0.5\linewidth]{figure_man/discrete/example_5_\x.pdf}\par
\end{figure}
}

%%%%%%%%%%%%%%%%%%%%%%%%%%%%%%%%%%%%%%%%%%%%%%%%%%%%%%%%%%%%%%%%%%%%%%%%%%%%%%%%%%%%
%\framebreak

\foreach \x in{1,2,3} {
\textbf{Example 3:} Consider $h: \Xspace \to \Yspace$ where the input space consists of \textbf{ten} points.

\vspace{.3cm}

Examples for functions that live in this space:

\vspace{.3cm}
\begin{figure}
  \includegraphics[width=0.5\textwidth]{figure_man/discrete/example_10_\x.pdf}
\end{figure}
}

\end{frame}
%%%%%%%%%%%%%%%%%%%%%%%%%%%%%%%%%%%%%%%%%%%%%%%%%%%%%%%%%%%%%%%%%%%%%%%%%%%%%%%%%%%%
\begin{frame}[c,allowframebreaks]{Distributions on Discrete Functions}

\begin{itemize}
\item One natural way to specify a probability distribution on a discrete function $h \in \Hspace$ is to use the vector representation of the function:
$$\bm{h} = \left[h\left(\xI{1}\right), h\left(\xI{2}\right),\dots, h\left(\xI{n}\right)\right].$$ 
\lz
\item Let us consider $\bm{h}$ as a $n$-dimensional random variable. We will further assume the following normal distribution: 
$$\bm{h} \sim \normaldist\left(\bm{m}, \bm{K}\right).$$ 
\end{itemize}

\textbf{Note: }For now, we set $\bm{m} = \bm{0}$ and take the covariance matrix $\bm{K}$ as given. We will see later how they are chosen / estimated. 


%%%%%%%%%%%%%%%%%%%%%%%%%%%%%%%%%%%%%%%%%%%%%%%%%%%%%%%%%%%%%%%%%%%%%%%%%%%%%%%%%%%%
\framebreak

\foreach \x in{1,2,3} {
\textbf{Example 1 (continued):} Let $h: \Xspace \to \Yspace$ be a function that is defined on \textbf{two} points $\Xspace$. We sample functions by sampling from a two-dimensional normal variable
\vspace{-.1cm}
$$\bm{h} = [h(1), h(2)] \sim \normaldist(\bm{m}, \bm{K}).$$
\vspace{-.5cm}
\begin{figure}
  \includegraphics[width=0.3\textwidth]{figure_man/discrete/example_norm_2_\x-a.pdf} ~  \includegraphics[width=0.3\textwidth]{figure_man/discrete/example_norm_2_\x-b.pdf}
  
\begin{footnotesize}
In this example, $m = (0, 0)$ and $K = \begin{pmatrix} 1 & 0.5 \\ 0.5 & 1 \end{pmatrix}$.
\end{footnotesize}
\end{figure}
}

%%%%%%%%%%%%%%%%%%%%%%%%%%%%%%%%%%%%%%%%%%%%%%%%%%%%%%%%%%%%%%%%%%%%%%%%%%%%%%%%%%%%
\framebreak

\foreach \x in{1,2,3} {
\textbf{Example 2 (continued):} Let us consider $h: \Xspace \to \Yspace$ where the input space consists of \textbf{five} points. We sample functions by sampling from a five-dimensional normal variable
\vspace{-.1cm}
$$\bm{h} = [h(1), h(2), h(3), h(4), h(5)] \sim \normaldist(\bm{m}, \bm{K}).$$
\vspace{-.5cm}
\begin{figure}
  \includegraphics[width=0.3\textwidth]{figure_man/discrete/example_norm_5_\x-a.pdf} ~  \includegraphics[width=0.3\textwidth]{figure_man/discrete/example_norm_5_\x-b.pdf}
  \end{figure}
}


%%%%%%%%%%%%%%%%%%%%%%%%%%%%%%%%%%%%%%%%%%%%%%%%%%%%%%%%%%%%%%%%%%%%%%%%%%%%%%%%%%%%
\framebreak

\foreach \x in{1,2,3} {
\textbf{Example 3 (continued):} Let us consider $h: \Xspace \to \Yspace$ where the input space consists of \textbf{ten} points. We sample functions by sampling from a ten-dimensional normal variable
\vspace{-.1cm}
$$\bm{h} = [h(1), h(2),\dots, h(10)] \sim \normaldist(\bm{m}, \bm{K}).$$
\vspace{-.5cm}
\begin{figure}
  \includegraphics[width=0.3\textwidth]{figure_man/discrete/example_norm_10_\x-a.pdf} ~  \includegraphics[width=0.3\textwidth]{figure_man/discrete/example_norm_10_\x-b.pdf}
  \end{figure}
}

\end{frame}


%%%%%%%%%%%%%%%%%%%%%%%%%%%%%%%%%%%%%%%%%%%%%%%%%%%%%%%%%%%%%%%%%%%%%%%%%%%%%%%%%%%%
\begin{frame}[c,allowframebreaks]{The Role of Covariance Function}
\vspace{-.03cm}

The covariance controls the ``shape'' of drawn functions. Consider two extreme cases where function values are:
\vspace{-.1cm}
\begin{enumerate}
  \item[a)] strongly correlated: $\bm{K} = \begin{footnotesize}\begin{pmatrix} 1 & 0.99 & ... & 0.99 \\
  0.99 & 1 & ... & 0.99 \\
  0.99 & 0.99 & \ddots & 0.99 \\
  0.99 & ... & 0.99 & 1 \end{pmatrix}\end{footnotesize}$
  \vspace{-.4cm}
  \item[b)] uncorrelated: $\bm{K} = \id$.
\end{enumerate}

\begin{figure}
  \includegraphics[width=0.31\textwidth]{figure_man/discrete/example_extreme_50-1.pdf} ~~  \includegraphics[width=0.31\textwidth]{figure_man/discrete/example_extreme_50-2.pdf}
\end{figure}

%%%%%%%%%%%%%%%%%%%%%%%%%%%%%%%%%%%%%%%%%%%%%%%%%%%%%%%%%%%%%%%%%%%%%%%%%%%%%%%%%%%%
\framebreak

\begin{itemize}
  \item On a numeric space $\Xspace$, ``meaningful'' functions may be characterized by the following spatial property:
\end{itemize}

\begin{displayquote}
If $\xI{i}$ and $\xI{j}$ are close in the $\Xspace$-space, their function values $f(\xI{i})$ and $f(\xI{j})$ should be close in $\Yspace$-space.
\end{displayquote}

\lz
\begin{itemize}
  \item[\faLightbulbO] In other words, if two data points are close in $\Xspace$-space, their corresponding values should be \textbf{correlated}!
  
  \lz
  
  \item[\faLightbulbO] We can enforce this condition by choosing a covariance function for which,  
\end{itemize}
 $$\bm{K}_{ij} \text{ is high, if } \xI{i} \text{ and } \xI{j} \text{ are close.}$$



%%%%%%%%%%%%%%%%%%%%%%%%%%%%%%%%%%%%%%%%%%%%%%%%%%%%%%%%%%%%%%%%%%%%%%%%%%%%%%%%%%%%
\framebreak


We can compute the entries of the covariance matrix by a function that is based on the distance between $\xI{i}$ and $\xI{j}$. For example: 

\begin{enumerate}
    \item[c)] spatial correlation: \begin{footnotesize}$K_{ij} = k(\xI{i}, \xI{j}) = \exp\left(-\frac{1}{2}\left|\xI{i} - \xI{j}\right|^2\right)$\end{footnotesize}
\end{enumerate}
  
\begin{figure}
  \includegraphics[width=0.3\textwidth]{figure_man/discrete/example_extreme_50-4.pdf} ~~\includegraphics[width=0.3\textwidth]{figure_man/discrete/example_extreme_50-3.pdf}
\end{figure}


\begin{footnotesize}
\textbf{Note}: $k(\cdot,\cdot)$ is known as the \textbf{covariance function} or \textbf{kernel}. It will be studied in more detail later on.
\end{footnotesize}

\end{frame}

%%%%%%%%%%%%%%%%%%%%%%%%%%%%%%%%%%%%%%%%%%%%%%%%%%%%%%%%%%%%%%%%%%%%%%%%%%%%%%%%%%%%
\begin{frame}[c]{}
\centering
\huge
\textbf{Gaussian Processes}
\end{frame}

%%%%%%%%%%%%%%%%%%%%%%%%%%%%%%%%%%%%%%%%%%%%%%%%%%%%%%%%%%%%%%%%%%%%%%%%%%%%%%%%%%%%
\begin{frame}[c]{From Discrete to Continuous Functions}

\begin{itemize}
  \item We have already considered distributions on functions with discrete domain. We did so, by defining Gaussian distributions on the vector of the respective function values $$\mathbf{h} = [h(\xI{1}), h(\xI{2}),\dots, h(\xI{n})] \sim \normaldist(\bm{m}, \bm{K}).$$
  \item We can generalize this idea for $n \to \infty$.
\end{itemize}

\begin{figure}
    \includegraphics[width = 0.65\textwidth]{figure_man/discrete/example_limit.pdf}
  \end{figure}
\end{frame}
%%%%%%%%%%%%%%%%%%%%%%%%%%%%%%%%%%%%%%%%%%%%%%%%%%%%%%%%%%%%%%%%%%%%%%%%%%%%%%%%%%%%

\begin{frame}[c,allowframebreaks]{Gaussian Processes: Intuition}


\begin{itemize}
  \item  No matter how large $n$ is, we consider functions with discrete domains.
\vspace{.3cm}
  \item But, how can we extend our definition to functions with \textbf{continuous} domains $\Xspace \subset \realnum$?
\vspace{.3cm}
  \item Intuitively, a function $f$ drawn from a \textbf{Gaussian process} can be understood as an ``infinite'' long Gaussian random vector.
\vspace{.3cm}
  \item It is unclear how to handle an ``infinite'' long Gaussian random vector!
\end{itemize}

\vspace{.3cm}

\begin{figure}
\includegraphics[width=0.18\textwidth]{figure_man/question.png}
\end{figure}
%%%%%%%%%%%%%%%%%%%%%%%%%%%%%%%%%%%%%%%%%%%%%%%%%%%%%%%%%%%%%%%%%%%%%%%%%%%%%%%%%%%%
\framebreak

\begin{itemize}
\item Thus, it is required that for \textbf{any finite set} of inputs $\{\xI{1},\dots,\xI{n}\} \subset \Xspace$, the vector $\mathbf{f}$ has a Gaussian distribution with $\bm{m}$ and $\bm{K}$ being calculated by a mean function $m(\cdot)$ and a covariance function $k(\cdot,\cdot)$:\vspace{-.2cm}$$\bm{f} = \left[f\left(\xI{1}\right),\dots, f\left(\xI{n}\right)\right] \sim\normaldist\left(\bm{m},\bm{K}\right).$$
    
\item This property is called the \textbf{Marginalization Property}. 
\vspace{.15cm}
\begin{figure}
\includegraphics[width=0.2\textwidth]{figure_man/discrete/example_marginalization_5.pdf}
\includegraphics[width=0.3\textwidth]{figure_man/discrete/marginalization-5.png}
\end{figure}
\end{itemize}


%%%%%%%%%%%%%%%%%%%%%%%%%%%%%%%%%%%%%%%%%%%%%%%%%%%%%%%%%%%%%%%%%%%%%%%%%%%%%%%%%%%%
\framebreak

\begin{itemize}
\item Thus, it is required that for \textbf{any finite set} of inputs $\{\xI{1},\dots,\xI{n}\} \subset \Xspace$, the vector $\mathbf{f}$ has a Gaussian distribution with $\bm{m}$ and $\bm{K}$ being calculated by a mean function $m(\cdot)$ and a covariance function $k(\cdot,\cdot)$:\vspace{-.2cm}$$\bm{f} = \left[f\left(\xI{1}\right),\dots, f\left(\xI{n}\right)\right] \sim\normaldist\left(\bm{m},\bm{K}\right).$$
    
\item This property is called the \textbf{Marginalization Property}. 
\vspace{.15cm}
\begin{figure}
\includegraphics[width=0.2\textwidth]{figure_man/discrete/example_marginalization_10.pdf}
\includegraphics[width=0.3\textwidth]{figure_man/discrete/marginalization-more.png}
\end{figure}
\end{itemize}


%%%%%%%%%%%%%%%%%%%%%%%%%%%%%%%%%%%%%%%%%%%%%%%%%%%%%%%%%%%%%%%%%%%%%%%%%%%%%%%%%%%%
\framebreak


\begin{itemize}
\item Thus, it is required that for \textbf{any finite set} of inputs $\{\xI{1},\dots,\xI{n}\} \subset \Xspace$, the vector $\mathbf{f}$ has a Gaussian distribution with $\bm{m}$ and $\bm{K}$ being calculated by a mean function $m(\cdot)$ and a covariance function $k(\cdot,\cdot)$:\vspace{-.2cm}$$\bm{f} = \left[f\left(\xI{1}\right),\dots, f\left(\xI{n}\right)\right] \sim\normaldist\left(\bm{m},\bm{K}\right).$$
    
\item This property is called the \textbf{Marginalization Property}. 
\vspace{.15cm}
\begin{figure}
\includegraphics[width=0.2\textwidth]{figure_man/discrete/example_marginalization_50.pdf}
\includegraphics[width=0.3\textwidth]{figure_man/discrete/marginalization-more.png}
\end{figure}
\end{itemize}
\end{frame}

%%%%%%%%%%%%%%%%%%%%%%%%%%%%%%%%%%%%%%%%%%%%%%%%%%%%%%%%%%%%%%%%%%%%%%%%%%%%%%%%%%%%
\begin{frame}[c,allowframebreaks]{Gaussian Processes: Formal Definitions}

\begin{itemize}
\item The above intuitive explanation is formally defined as follows. 
\end{itemize}

\vspace{.7cm}

\begin{displayquote}
A function $f(\x)$ is generated by a Gaussian process $\gp$ if for \textbf{any finite} set of inputs $\left\{\xI{1},\dots,\xI{n}\right\}$, the associated vector of function values has a Gaussian distribution:
$$\bm{f} = \left(f(\xI{1}),\dots, f(\xI{n})\right) \sim\normaldist\biggl(\textbf{m},\textbf{K}\biggr),$$ with 

$$\textbf{m}:= \left(m\left(\xI{i}\right)\right)_{i}, \quad
\textbf{K}:= \left(k\left(\xI{i}, \xI{j}\right)\right)_{i,j},$$ 

where $m(\x)$ is called mean function and $k(\x, \x^\prime)$ is called covariance function.
\end{displayquote}

%%%%%%%%%%%%%%%%%%%%%%%%%%%%%%%%%%%%%%%%%%%%%%%%%%%%%%%%%%%%%%%%%%%%%%%%%%%%%%%%%%%%
\framebreak


\begin{itemize}

\item A GP is \textbf{completely specified} by its mean and covariance functions.
\vspace{.2cm}

\item The mean function $m(\x)$ and the covariance function $k(\x, \x^\prime)$ of a real process $f(\x)$ are defined as:
\vspace{-.4cm}
\begin{eqnarray*}
m(\x) &=& \expectation[f(\x)] \\
k(\x, \x^\prime) &=& \expectation\biggl[\left( f(\x) - \expectation[f(\x)] \right) \left( f(\x^\prime) - \expectation[f(\x^\prime)] \right)\biggr]
\end{eqnarray*}

\vspace{.2cm}

\item We denote a GP by
\vspace{-.1cm}
$$f(\x) \sim \gp\left(m(\x), k\left(\x, \x^\prime\right)\right) $$
\end{itemize}

\vspace{.3cm}

\textbf{Note:} For now, we assume $m(\x)\equiv 0$. This is not a drastic limitation. In fact, it is common to consider GPs with a zero mean function.

\end{frame}
%%%%%%%%%%%%%%%%%%%%%%%%%%%%%%%%%%%%%%%%%%%%%%%%%%%%%%%%%%%%%%%%%%%%%%%%%%%%%%%%%%%%
\begin{frame}[c,allowframebreaks]{Sampling from a Gaussian Process Prior}

\begin{itemize}

\item We can draw functions from a Gaussian process prior. To do so, consider $f(\x) \sim \gp\left(0, k(\x, \x^\prime)\right)$ with the squared exponential covariance function $^{(*)}$

$$
k(\x, \x^\prime) = \exp\left(-\frac{1}{2\ls^2}\|\x - \x^\prime\|^2\right), ~~ \ls = 1.
$$

\lz

\item This covariance function specifies the Gaussian process completely. 

\end{itemize}

\vspace{2cm}
\begin{footnotesize}
$^{(*)}$ We will talk later about different choices of covariance functions. 
\end{footnotesize}


%%%%%%%%%%%%%%%%%%%%%%%%%%%%%%%%%%%%%%%%%%%%%%%%%%%%%%%%%%%%%%%%%%%%%%%%%%%%%%%%%%%%
\framebreak

To visualize a sample function, we 
\begin{itemize}
  \item choose a large number of equidistant points: $\left\{\xI{1},\dots,\xI{n}\right\}$,
  \item compute their corresponding covariance matrix by plugging in all pairs of $\xI{i}$ and $\xI{j}$ in $\textbf{K} = \left(k\left(\xI{i}, \xI{j}\right)\right)_{i,j}$,
  \item sample from a Gaussian $\bm{f} \sim \normaldist (\bm{0}, \bm{K})$.
\end{itemize}

\begin{figure}
\includegraphics[width=0.6\textwidth]{figure_man/gp-sqexp-1-1.pdf}
\begin{footnotesize}
\caption*{We draw $10$ times from the Gaussian, to get $10$ different samples. Since we specified the mean function to be zero, the drawn functions have a zero mean.}
\end{footnotesize}
\end{figure}

\end{frame}

%%%%%%%%%%%%%%%%%%%%%%%%%%%%%%%%%%%%%%%%%%%%%%%%%%%%%%%%%%%%%%%%%%%%%%%%%%%%%%%%%%%%
\framebreak

\begin{frame}[c]{}
\centering
\huge
\textbf{Gaussian Processes as an Indexed Family}
\end{frame}


%%%%%%%%%%%%%%%%%%%%%%%%%%%%%%%%%%%%%%%%%%%%%%%%%%%%%%%%%%%%%%%%%%%%%%%%%%%%%%%%%%%%
\begin{frame}[c]{Gaussian Processes as an Indexed Family}

\begin{itemize}
\item A Gaussian process is a special case of a \textbf{stochastic process} which is defined as a collection of random variables indexed by some index set (also called an \textbf{indexed family}).
\lz
\item What does it mean?
\lz
\item An \textbf{indexed family} is a mathematical function (or ``rule'') that maps indices $t \in T$ to objects in $\mathcal{S}$.
\end{itemize}
\lz


\begin{displayquote} 
\textbf{Definition:} an \textbf{index family} (or a family of elements in $\mathcal{S}$ indexed by $T$) is a surjective function that is defined as follows:
\vspace{-.2cm}
\begin{eqnarray*}
s: T &\to& \mathcal{S} \\
   t &\mapsto& s_t = s(t) 
\end{eqnarray*}

\end{displayquote}

\end{frame}


%%%%%%%%%%%%%%%%%%%%%%%%%%%%%%%%%%%%%%%%%%%%%%%%%%%%%%%%%%%%%%%%%%%%%%%%%%%%%%%%%%%%
\begin{frame}[c,allowframebreaks]{Index Family}

Some simple examples for indexed families are:
\begin{columns}[T]
\begin{column}{.5\textwidth}
\vspace*{1cm}
  \begin{itemize}
  \item Finite sequences (lists): $T = \{1, 2,\dots, n\}$ and $\left(s_t\right)_{t \in T}\in\realnum $
  \vspace{2.5cm}
  \item Infinite sequences: $T = \natnum$ and $\left(s_t\right)_{t \in T} \in \realnum$
  \end{itemize}
  
\end{column}
\begin{column}{.35\textwidth}
\includegraphics[width=\textwidth]{figure_man/indexed_family/indexed_family_1.png}\\
\includegraphics[width=\textwidth]{figure_man/indexed_family/indexed_family_2.png}
\end{column}
\end{columns}

%%%%%%%%%%%%%%%%%%%%%%%%%%%%%%%%%%%%%%%%%%%%%%%%%%%%%%%%%%%%%%%%%%%%%%%%%%%%%%%%%%%%

%%%%%%%%%%%%%%%%%%%%%%%%%%%%%%%%%%%%%%%%%%%%%%%%%%%%%%%%%%%%%%%%%%%%%%%%%%%%%%%%%%%%
\framebreak


But the indexed set $\mathcal{S}$ can be something more complicated, for example functions or \textbf{random variables} (RV):

\begin{columns}[T]
\begin{column}{.5\textwidth}
\vspace*{1cm}
  \begin{itemize}
    \item $T = \{1,\dots, m\}$, $Y_t$'s are RVs: Indexed family is a random vector.
    \vspace*{0.4cm}
    \item $T = \{1,\dots, m\}$, $Y_t$'s are RVs: Indexed family is a stochastic process in discrete time. 
    \vspace*{0.4cm}
    \item $T = \intnum^2$, $Y_t$'s are RVs: Indexed family is a 2D-random walk.
  \end{itemize}
  
\end{column}
\begin{column}{.35\textwidth}
\includegraphics[width=\textwidth]{figure_man/indexed_family/indexed_family_4.png}\\
\includegraphics[width=\textwidth]{figure_man/indexed_family/indexed_family_3.png}
\end{column}
\end{columns}



\begin{itemize}
  \item A Gaussian process is also an indexed family, where the random variables $f(\x)$ are indexed by the input values $\x\in\Xspace$. 
  \item Importantly, any indexed (finite) random vector has a multivariate Gaussian distribution (which comes with all the nice properties of Gaussianity!). 
\end{itemize}

\lz

\begin{figure}
\includegraphics[width=0.5\textwidth]{figure_man/indexed_family/indexed_family_5.png}\par
\begin{footnotesize}
Visualization for a one-dimensional $\Xspace$.
\end{footnotesize}
\end{figure}


%%%%%%%%%%%%%%%%%%%%%%%%%%%%%%%%%%%%%%%%%%%%%%%%%%%%%%%%%%%%%%%%%%%%%%%%%%%%%%%%%%%%
\framebreak

\begin{itemize}
  \item A Gaussian process is also an indexed family, where the random variables $f(\x)$ are indexed by the input values $\x\in\Xspace$. 
  \item Importantly, any indexed (finite) random vector has a multivariate Gaussian distribution (which comes with all the nice properties of Gaussianity!). 
\end{itemize}

\lz

\begin{figure}
\includegraphics[width=0.4\textwidth]{figure_man/indexed_family/indexed_family_6.png}\par
\begin{footnotesize}
Visualization for a two-dimensional $\Xspace$.
\end{footnotesize}
\end{figure}

%%%%%%%%%%%%%%%%%%%%%%%%%%%%%%%%%%%%%%%%%%%%%%%%%%%%%%%%%%%%%%%%%%%%%%%%%%%%%%%%%%%%
\framebreak

%%%%%%%%%%%%%%%%%%%%%%%%%%%%%%%%%%%%%%%%%%%%%%%%%%%%%%%%%%%%%%%%%%%%%%%%%%%%%%%%%%%%
\framebreak

%%%%%%%%%%%%%%%%%%%%%%%%%%%%%%%%%%%%%%%%%%%%%%%%%%%%%%%%%%%%%%%%%%%%%%%%%%%%%%%%%%%%
\framebreak


%%%%%%%%%%%%%%%%%%%%%%%%%%%%%%%%%%%%%%%%%%%%%%%%%%%%%%%%%%%%%%%%%%%%%%%%%%%%%%%%%%%%
\framebreak



\end{frame}


\end{document}
