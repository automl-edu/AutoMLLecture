
\pdfminorversion=4 % for acroread
\documentclass[aspectratio=169,t,xcolor={usenames,dvipsnames}]{beamer}
%\documentclass[t,handout,xcolor={usenames,dvipsnames}]{beamer}
\usepackage{../beamerstyle}
\usepackage{dsfont}
\usepackage{bm}
\usepackage[english]{babel}
\usepackage[utf8]{inputenc}
\usepackage{graphicx}
\usepackage{algorithm}
\usepackage[ruled,vlined,algo2e,linesnumbered]{algorithm2e}
%\usepackage[boxed,vlined]{algorithm2e}
\usepackage{hyperref}
\usepackage{booktabs}
\usepackage{mathtools}

\usepackage{amsmath,amssymb}
\usepackage{listings}
\lstset{frame=lines,framesep=3pt,numbers=left,numberblanklines=false,basicstyle=\ttfamily\small}

\usepackage{subfig}
\usepackage{multicol}
%\usepackage{appendixnumberbeamer}
%
\usepackage{tcolorbox}

\usepackage{pgfplots}
\usepackage{tikz}
\usetikzlibrary{trees} 
\usetikzlibrary{shapes.geometric}
\usetikzlibrary{positioning,shapes,shadows,arrows,calc,mindmap}
\usetikzlibrary{positioning,fadings,through}
\usetikzlibrary{decorations.pathreplacing}
\usetikzlibrary{intersections}
\usetikzlibrary{positioning,fit,calc,shadows,backgrounds}
\pgfdeclarelayer{background}
\pgfdeclarelayer{foreground}
\pgfsetlayers{background,main,foreground}
\tikzstyle{activity}=[rectangle, draw=black, rounded corners, text centered, text width=8em]
\tikzstyle{data}=[rectangle, draw=black, text centered, text width=8em]
\tikzstyle{myarrow}=[->, thick, draw=black]

% Define the layers to draw the diagram
\pgfdeclarelayer{background}
\pgfdeclarelayer{foreground}
\pgfsetlayers{background,main,foreground}

%\usepackage{listings}
%\lstset{numbers=left,
%  showstringspaces=false,
%  frame={tb},
%  captionpos=b,
%  lineskip=0pt,
%  basicstyle=\ttfamily,
%%  extendedchars=true,
%  stepnumber=1,
%  numberstyle=\small,
%  xleftmargin=1em,
%  breaklines
%}

 
\definecolor{blue}{RGB}{0, 74, 153}

\usetheme{Boadilla}
%\useinnertheme{rectangles}
\usecolortheme{whale}
\setbeamercolor{alerted text}{fg=blue}
\useoutertheme{infolines}
\setbeamertemplate{navigation symbols}{\vspace{-5pt}} % to lower the logo
\setbeamercolor{date in head/foot}{bg=blue} % blue
\setbeamercolor{date in head/foot}{fg=white}
\setbeamercolor{author in head/foot}{bg=blue} %blue
\setbeamercolor{title in head/foot}{bg=blue} % blue
\setbeamercolor{title}{fg=white, bg=blue}
\setbeamercolor{block title}{fg=white,bg=blue}
\setbeamercolor{block body}{bg=blue!10}
\setbeamercolor{frametitle}{fg=white, bg=blue}
\setbeamercovered{invisible}

\makeatletter
\setbeamertemplate{footline}
{
  \leavevmode%
  \hbox{%
  \begin{beamercolorbox}[wd=.333333\paperwidth,ht=2.25ex,dp=1ex,center]{author in head/foot}%
    \usebeamerfont{author in head/foot}\insertshortauthor
  \end{beamercolorbox}%
  \begin{beamercolorbox}[wd=.333333\paperwidth,ht=2.25ex,dp=1ex,center]{title in head/foot}%
    \usebeamerfont{title in head/foot}\insertshorttitle
  \end{beamercolorbox}%
  \begin{beamercolorbox}[wd=.333333\paperwidth,ht=2.25ex,dp=1ex,right]{date in head/foot}%
    \usebeamerfont{date in head/foot}Week \@week, Topic \@topicnumber, Slide \insertframenumber{}\hspace*{2em}
%    \insertframenumber\hspace*{2ex} 
  \end{beamercolorbox}}%
  \vskip0pt%
}

\newcommand{\@week}{0}
\newcommand{\@topicnumber}{0}
\newcommand{\week}[1]{\renewcommand{\@week}{#1}}
\newcommand{\topicnumber}[1]{\renewcommand{\@topicnumber}{#1}}

\makeatother

%\pgfdeclareimage[height=1.2cm]{automl}{images/logos/automl.png}
%\pgfdeclareimage[height=1.2cm]{freiburg}{images/logos/freiburg}

%\logo{\pgfuseimage{freiburg}}

\input{../latex_main/macros}





\newcommand{\lz}{\vspace{0.5cm}}
\newcommand{\thetab}{\bm{\weights}}
\newcommand{\zero}{\mathbf{0}}
\newcommand{\Xmat}{\mathbf{X}}
\newcommand{\Kmat}{\mathbf{K}}
\newcommand{\ydat}{\mathbf{y}}
\newcommand{\id}{\boldsymbol{I}}
\newcommand{\Amat}{\mathbf{A}}
\newcommand{\Xspace}{\mathcal{X}}                                           
\newcommand{\Yspace}{\mathcal{Y}}
\newcommand{\ls}{\ell}
\newcommand{\natnum}{\mathbb{N}}
\newcommand{\intnum}{\mathbb{Z}}

\usepackage{fontawesome}
\usepackage{dirtytalk}
\usepackage{csquotes}

%\begin{frame}[c]{}
%\centering
%\huge
%\textbf{}
%\end{frame}


%\item[\faLightbulbO]
\title[AutoML: GPs]{AutoML: Gaussian Processes} % week title
\subtitle{Mean Functions for Gaussian Processes} % video title
\author[Marius Lindauer]{\underline{Bernd Bischl} \and Frank Hutter \and Lars Kotthoff\newline \and Marius Lindauer \and Joaquin Vanschoren}
\institute{}
\date{}
\week{5}
\topicnumber{6}



% \AtBeginSection[] % Do nothing for \section*
% {
%   \begin{frame}{Outline}
%     \bigskip
%     \vfill
%     \tableofcontents[currentsection]
%   \end{frame}
% }

\begin{document}
\maketitle
	


\begin{frame}[c,allowframebreaks]{The Role of Mean Functions}

\begin{itemize}
\item It is common but by no means necessary to consider GPs with zero-mean functions: $$m(\x) \equiv 0$$

\vspace{3mm}
\item This is not necessarily a drastic limitation, since the mean of the posterior process is not confined to be zero:
\vspace{-3mm}
  $$\bm{f}_* | \Xmat_*, \Xmat, \bm{f} \sim \mathcal{N}(\Kmat_{*}^\top\Kmat^{-1}\bm{f}, \Kmat_{**} - \Kmat_*^\top \Kmat ^{-1}\Kmat_*).$$
  
\vspace{3mm}
\item There are several reasons why one might wish to explicitly model a mean function, including interpretability, convenience of expressing prior informations, etc.

%%%%%%%%%%%%%%%%%%%%%%%%%%%%%%%%%%%%%%%%%%%%%%%%%%%%%%%%%%%%%%%%%%%%%%%%%%%%%%%%%%%%

\framebreak

\item When assuming a non-zero mean GP prior $\gp (m(\x), k(\x, \x^\prime)$ with mean $m(\x)$, the predictive mean becomes:\vspace{-5mm}$$m(\Xmat_*) + \Kmat_*\Kmat_y^{-1}\left(\bm{y} - m(\Xmat)\right)$$\vspace{-7mm}but the predictive variance remains unchanged. 

\vspace{20mm}
\item Gaussian processes with non-zero mean Gaussian process priors are also called \textbf{Gaussian processes with trend}.  

\end{itemize}

\framebreak
%%%%%%%%%%%%%%%%%%%%%%%%%%%%%%%%%%%%%%%%%%%%%%%%%%%%%%%%%%%%%%%%%%%%%%%%%%%%%%%%%%%%


\begin{figure}
\includegraphics[width=0.6\textwidth]{figure_man/gp-sample-nonzero/gp-sample-nonzero-1-1.pdf}
\end{figure}

\framebreak
%%%%%%%%%%%%%%%%%%%%%%%%%%%%%%%%%%%%%%%%%%%%%%%%%%%%%%%%%%%%%%%%%%%%%%%%%%%%%%%%%%%%

\foreach \x in{1, 2, 3, 4} {
\begin{figure}
  \includegraphics[width=0.6\textwidth]{figure_man/gp-sample-nonzero/gp-sample-nonzero-2-\x.pdf}
\framebreak
\end{figure}
}

\framebreak
%%%%%%%%%%%%%%%%%%%%%%%%%%%%%%%%%%%%%%%%%%%%%%%%%%%%%%%%%%%%%%%%%%%%%%%%%%%%%%%%%%%%
\begin{itemize}
\item In practice it is often difficult to specify a fixed mean function. Instead, it may be more convenient to specify a few fixed basis functions, whose coefficients, $\bm{\beta}$, are to be inferred from the data.
\vspace{2mm}
\item Consider 
$$g(\x) = f(\x)+\bm{h}(\x)^\top\bm{\beta}, \text{   with } f(\x) \sim\gp \left(0, k(\x, \x^\prime)\right)$$ 

where $f(\x)$ is a zero mean GP, $\bm{h}(\x)$ are a set of fixed basis functions, and $\bm{\beta}$ are additional parameters. 
\vspace{2mm}
\item This formulation expresses that the data is close to a global linear model with the residuals being modelled by a GP.
\vspace{2mm}
\item By taking the prior on $\bm{\beta}$ to be Gaussian, $\bm{\beta} \sim \mathcal{N}(\mathbf{b}, B)$, another GP with an added contribution in the covariance function can be obtained: 
\vspace{-2mm}
$$g(\x) \sim \gp\left(\mathbf{h}(\x^\top \mathbf{b}, k\left(\x, \x^\prime\right)+\mathbf{h}(\x)^\top B\,\, \mathbf{h}\left(\x^\prime\right)\right)$$

\end{itemize}


\framebreak
%%%%%%%%%%%%%%%%%%%%%%%%%%%%%%%%%%%%%%%%%%%%%%%%%%%%%%%%%%%%%%%%%%%%%%%%%%%%%%%%%%%%
\begin{itemize}

\item The mean and covariance functions of $g(\x)$ will be then:
\vspace{-2mm}

$$\begin{aligned}
\overline{\mathbf{g}}\left(X_{*}\right) &=\overline{\mathbf{f}}\left(X_{*}\right)+R^{\top} \overline{\boldsymbol{\beta}}, \\
\operatorname{cov}\left(\mathbf{g}_{*}\right) &=\operatorname{cov}\left(\mathbf{f}_{*}\right)+R^{\top}(B^{-1}+H K_{y}^{-1} H^{\top})^{-1} R,
\end{aligned}\label{41}$$ where $\overline{\boldsymbol{\beta}}=(B^{-1}+H K_{y}^{-1} H^{\top})^{-1}(H K_{y}^{-1} \mathbf{y}+B^{-1} \mathbf{b})$ and  $R=H_{*}-H K_{y}^{-1} K_{*}.$

\vspace{4mm}
\item Notice that $\overline{\boldsymbol{\beta}}$ is the mean of the global linear model parameters and the $H$ matrix collects the $\mathbf{h}(\x)$ vectors for all training and test cases.

\vspace{4mm}
\item Hence, the predictive mean is the sum of the mean linear output and what the GP model predicts from the residuals. The covariance is the usual covariance term plus a non negative contribution.
\end{itemize}

\framebreak
%%%%%%%%%%%%%%%%%%%%%%%%%%%%%%%%%%%%%%%%%%%%%%%%%%%%%%%%%%%%%%%%%%%%%%%%%%%%%%%%%%%%

\begin{itemize}

\item In the limiting case where the prior on the $\boldsymbol{\beta}$ becomes vague ($B^{-1} \rightarrow O$), the predictive distribution becomes independent of $\boldsymbol{b}$:
\vspace{-2mm}

$$\begin{aligned} 
\overline{\mathbf{g}}\left(X_{*}\right) &=\overline{\mathbf{f}}\left(X_{*}\right)+R^{\top} \overline{\boldsymbol{\beta}}, \\
\operatorname{cov}\left(\mathbf{g}_{*}\right) &=\operatorname{cov}(\mathbf{f}_{*})+R^{\top}(H K_{y}^{-1} H^{\top})^{-1} R, 
\end{aligned}
$$ where the limiting $\overline{\boldsymbol{\beta}}=(H K_{y}^{-1} H^{\top})^{-1} H K_{y}^{-1} \mathbf{y}$.

\vspace{1cm}

\item To make predictions under the limit $B^{-1} \rightarrow O$, it is important to use the above equation. Otherwise, by na{\"\i}vely plugging the modified covariance function into the standard prediction equations, the entries of the covariance function tend to infinity making it unsuitable for numerical implementation.

\end{itemize}

\framebreak
%%%%%%%%%%%%%%%%%%%%%%%%%%%%%%%%%%%%%%%%%%%%%%%%%%%%%%%%%%%%%%%%%%%%%%%%%%%%%%%%%%%%
\begin{itemize}

\item The marginal likelihood for the model with a Gaussian prior $\bm{\beta} \sim \mathcal{N}(\mathbf{b}, B)$ can be expressed as what follows (the explicit mean has been included).

%\begin{footnotesize}
$$\begin{aligned} 
\log p(\mathbf{y} | X, \mathbf{b}, B)=&-\frac{1}{2}\left(H^{\top} \mathbf{b}-\mathbf{y}\right)^{\top}\left(K_{y}+H^{\top} B H\right)^{-1}\left(H^{\top} \mathbf{b}-\mathbf{y}\right) \\ &-\frac{1}{2} \log \left|K_{y}+H^{\top} B H\right|-\frac{n}{2} \log 2 \pi. 
\end{aligned}$$
%\end{footnotesize}

\lz
\lz

\item We are interested in exploring the limit where $B^{-1} \rightarrow O$, i.e. when the prior is vague. In this limit the mean of the prior is irrelevant, so without loss of generality we assume for now that the mean is zero, $\mathbf{b} = \mathbf{0}$.

\end{itemize}

\framebreak
%%%%%%%%%%%%%%%%%%%%%%%%%%%%%%%%%%%%%%%%%%%%%%%%%%%%%%%%%%%%%%%%%%%%%%%%%%%%%%%%%%%%
\begin{itemize}

\item This assumption gives: 
\vspace{-5mm}
%\begin{footnotesize} 
$$\begin{aligned} 
\log p(\mathbf{y} | X, \mathbf{b}=\mathbf{0}, B)=&-\frac{1}{2} \mathbf{y}^{\top} K_{y}^{-1} \mathbf{y}+\frac{1}{2} \mathbf{y}^{\top} C \mathbf{y} \\ &-\frac{1}{2} \log \left|K_{y}\right|-\frac{1}{2} \log |B|-\frac{1}{2} \log |A|-\frac{n}{2} \log 2 \pi, 
\end{aligned}$$
where $A=B^{-1}+H K_{y}^{-1} H^{\top}$ and $C=K_{y}^{-1} H^{\top} A^{-1} H K_{y}^{-1}$.
%\end{footnotesize}

\vspace{4mm}
\item The behaviour of the log marginal likelihood in the limiting case can be explored now. The variances of the Gaussian in the directions spanned by columns of $H^\top$ will become infinite, and it will require special treatment. 

\vspace{4mm}
\item[\faLightbulbO] The log marginal likelihood consists of three terms: a quadratic form in $\mathbf{y}$, a log determinant term, and a term involving $\log 2 \pi$.

\end{itemize}

\framebreak
%%%%%%%%%%%%%%%%%%%%%%%%%%%%%%%%%%%%%%%%%%%%%%%%%%%%%%%%%%%%%%%%%%%%%%%%%%%%%%%%%%%%
\begin{itemize}

\item Performing an eigendecomposition of the covariance matrix, the contributions to quadratic form term from the infinite-variance directions will be zero. However, the log determinant term will tend to minus infinity. 

\vspace{3mm}
\item The standard solution in this case is to project $\mathbf{y}$ onto the directions orthogonal to the span of $H^\top$ and compute the marginal likelihood in this subspace. 

\vspace{3mm}
\item By taking $m$ to denote the rank of $H^\top$, we can discard the terms $-\frac{1}{2} \log |B|-\frac{m}{2} \log 2 \pi$ from the previous equation to give:
\vspace{-3mm}
%\begin{footnotesize}
$$\log p(\mathbf{y} | X)=-\frac{1}{2} \mathbf{y}^{\top} K_{y}^{-1} \mathbf{y}+\frac{1}{2} \mathbf{y}^{\top} C \mathbf{y}-\frac{1}{2} \log \left|K_{y}\right|-\frac{1}{2} \log |A|-\frac{n-m}{2} \log 2 \pi,$$
where $A=H K_{y}^{-1} H^{\top}$ and $C=K_{y}^{-1} H^{\top} A^{-1} H K_{y}^{-1}$.
%\end{footnotesize}

\end{itemize}

\framebreak
%%%%%%%%%%%%%%%%%%%%%%%%%%%%%%%%%%%%%%%%%%%%%%%%%%%%%%%%%%%%%%%%%%%%%%%%%%%%%%%%%%%%



\framebreak
%%%%%%%%%%%%%%%%%%%%%%%%%%%%%%%%%%%%%%%%%%%%%%%%%%%%%%%%%%%%%%%%%%%%%%%%%%%%%%%%%%%%



\framebreak
%%%%%%%%%%%%%%%%%%%%%%%%%%%%%%%%%%%%%%%%%%%%%%%%%%%%%%%%%%%%%%%%%%%%%%%%%%%%%%%%%%%%



\framebreak
%%%%%%%%%%%%%%%%%%%%%%%%%%%%%%%%%%%%%%%%%%%%%%%%%%%%%%%%%%%%%%%%%%%%%%%%%%%%%%%%%%%%



\framebreak
%%%%%%%%%%%%%%%%%%%%%%%%%%%%%%%%%%%%%%%%%%%%%%%%%%%%%%%%%%%%%%%%%%%%%%%%%%%%%%%%%%%%



\framebreak
%%%%%%%%%%%%%%%%%%%%%%%%%%%%%%%%%%%%%%%%%%%%%%%%%%%%%%%%%%%%%%%%%%%%%%%%%%%%%%%%%%%%



\framebreak
%%%%%%%%%%%%%%%%%%%%%%%%%%%%%%%%%%%%%%%%%%%%%%%%%%%%%%%%%%%%%%%%%%%%%%%%%%%%%%%%%%%%



\framebreak
%%%%%%%%%%%%%%%%%%%%%%%%%%%%%%%%%%%%%%%%%%%%%%%%%%%%%%%%%%%%%%%%%%%%%%%%%%%%%%%%%%%%



\framebreak
%%%%%%%%%%%%%%%%%%%%%%%%%%%%%%%%%%%%%%%%%%%%%%%%%%%%%%%%%%%%%%%%%%%%%%%%%%%%%%%%%%%%



\framebreak
%%%%%%%%%%%%%%%%%%%%%%%%%%%%%%%%%%%%%%%%%%%%%%%%%%%%%%%%%%%%%%%%%%%%%%%%%%%%%%%%%%%%




\end{frame}
%%%%%%%%%%%%%%%%%%%%%%%%%%%%%%%%%%%%%%%%%%%%%%%%%%%%%%%%%%%%%%%%%%%%%%%%%%%%%%%%%%%%


\end{document}
