
\pdfminorversion=4 % for acroread
\documentclass[aspectratio=169,t,xcolor={usenames,dvipsnames}]{beamer}
%\documentclass[t,handout,xcolor={usenames,dvipsnames}]{beamer}
\usepackage{../beamerstyle}
\usepackage{dsfont}
\usepackage{bm}
\usepackage[english]{babel}
\usepackage[utf8]{inputenc}
\usepackage{graphicx}
\usepackage{algorithm}
\usepackage[ruled,vlined,algo2e,linesnumbered]{algorithm2e}
%\usepackage[boxed,vlined]{algorithm2e}
\usepackage{hyperref}
\usepackage{booktabs}
\usepackage{mathtools}

\usepackage{amsmath,amssymb}
\usepackage{listings}
\lstset{frame=lines,framesep=3pt,numbers=left,numberblanklines=false,basicstyle=\ttfamily\small}

\usepackage{subfig}
\usepackage{multicol}
%\usepackage{appendixnumberbeamer}
%
\usepackage{tcolorbox}

\usepackage{pgfplots}
\usepackage{tikz}
\usetikzlibrary{trees} 
\usetikzlibrary{shapes.geometric}
\usetikzlibrary{positioning,shapes,shadows,arrows,calc,mindmap}
\usetikzlibrary{positioning,fadings,through}
\usetikzlibrary{decorations.pathreplacing}
\usetikzlibrary{intersections}
\usetikzlibrary{positioning,fit,calc,shadows,backgrounds}
\pgfdeclarelayer{background}
\pgfdeclarelayer{foreground}
\pgfsetlayers{background,main,foreground}
\tikzstyle{activity}=[rectangle, draw=black, rounded corners, text centered, text width=8em]
\tikzstyle{data}=[rectangle, draw=black, text centered, text width=8em]
\tikzstyle{myarrow}=[->, thick, draw=black]

% Define the layers to draw the diagram
\pgfdeclarelayer{background}
\pgfdeclarelayer{foreground}
\pgfsetlayers{background,main,foreground}

%\usepackage{listings}
%\lstset{numbers=left,
%  showstringspaces=false,
%  frame={tb},
%  captionpos=b,
%  lineskip=0pt,
%  basicstyle=\ttfamily,
%%  extendedchars=true,
%  stepnumber=1,
%  numberstyle=\small,
%  xleftmargin=1em,
%  breaklines
%}

 
\definecolor{blue}{RGB}{0, 74, 153}

\usetheme{Boadilla}
%\useinnertheme{rectangles}
\usecolortheme{whale}
\setbeamercolor{alerted text}{fg=blue}
\useoutertheme{infolines}
\setbeamertemplate{navigation symbols}{\vspace{-5pt}} % to lower the logo
\setbeamercolor{date in head/foot}{bg=white} % blue
\setbeamercolor{date in head/foot}{fg=white}
\setbeamercolor{author  in head/foot}{bg=white} %blue
\setbeamercolor{title in head/foot}{bg=white} % blue
\setbeamercolor{title}{fg=white, bg=blue}
\setbeamercolor{block title}{fg=white,bg=blue}
\setbeamercolor{block body}{bg=blue!10}
\setbeamercolor{frametitle}{fg=white, bg=blue}
\setbeamercovered{invisible}

\makeatletter
\setbeamertemplate{footline}
{
  \leavevmode%
  \hbox{%
  \begin{beamercolorbox}[wd=.333333\paperwidth,ht=2.25ex,dp=1ex,center]{author in head/foot}%
%    \usebeamerfont{author in head/foot}\insertshortauthor
  \end{beamercolorbox}%
  \begin{beamercolorbox}[wd=.333333\paperwidth,ht=2.25ex,dp=1ex,center]{title in head/foot}%
    \usebeamerfont{title in head/foot}\insertshorttitle
  \end{beamercolorbox}%
  \begin{beamercolorbox}[wd=.333333\paperwidth,ht=2.25ex,dp=1ex,right]{date in head/foot}%
    \usebeamerfont{date in head/foot}\insertshortdate{}\hspace*{2em}
%    \insertframenumber\hspace*{2ex} 
  \end{beamercolorbox}}%
  \vskip0pt%
}
\makeatother

%\pgfdeclareimage[height=1.2cm]{automl}{images/logos/automl.png}
%\pgfdeclareimage[height=1.2cm]{freiburg}{images/logos/freiburg}

%\logo{\pgfuseimage{freiburg}}

\newcommand{\comment}[1]{
	\noindent
	%\vspace{0.25cm}
	{\color{red}{\textbf{TODO:} #1}}
	%\vspace{0.25cm}
}
\renewcommand{\comment}[1]{}
\newcommand{\hide}[1]{}
\newcommand{\cemph}[2]{\emph{\textcolor{#1}{#2}}}

\newcommand{\lit}[1]{{\footnotesize\color{black!70}[#1]}}

\newcommand{\litw}[1]{{\footnotesize\color{black!20}[#1]}}


\newcommand{\myframe}[2]{\begin{frame}[c]{#1}#2\end{frame}}
\newcommand{\myframetop}[2]{\begin{frame}{#1}#2\end{frame}}
\newcommand{\myit}[1]{\begin{itemize}#1\end{itemize}}
\newcommand{\myblock}[2]{\begin{block}{#1}#2\end{block}}


\newcommand{\votepurple}[1]{\textcolor{Purple}{$\bigstar$}}
\newcommand{\voteyellow}[1]{\textcolor{Goldenrod}{$\bigstar$}}
\newcommand{\voteblue}[1]{\textcolor{RoyalBlue}{$\bigstar$}}
\newcommand{\votepink}[1]{\textcolor{Pink}{$\bigstar$}}

\newcommand{\diff}{\mathop{}\!\mathrm{d}}
\newcommand{\refstyle}[1]{{\small{\textcolor{gray}{#1}}}}
\newcommand{\hands}[0]{\includegraphics[height=1.5em]{images/hands}}
\newcommand{\transpose}[0]{{\textrm{\tiny{\sf{T}}}}}
\newcommand{\norm}{{\mathcal{N}}}
\newcommand{\cutoff}[0]{\kappa}
\newcommand{\instD}[0]{\dataset}
\newcommand{\insts}[0]{\mathcal{I}}
\newcommand{\inst}[0]{i}
\newcommand{\pcs}[0]{\mathbf{\Lambda}}
\newcommand{\bx}[0]{\conf}
\newcommand{\conf}[0]{\mathbf{\lambda}}
\newcommand{\defconf}[0]{\mathbf{\lambda}_{\text{def}}}
\newcommand{\finconf}[0]{\mathbf{\lambda}^*}
\newcommand{\incumbent}[0]{\finconf}
\newcommand{\confs}[0]{\pcs}
%\newcommand{\vlambda}[0]{\bm{\lambda}}
%\newcommand{\vLambda}[0]{\bm{\Lambda}}
\newcommand{\dataset}[0]{\mathcal{D}}
\newcommand{\datasets}[0]{\mathbf{D}}
\newcommand{\loss}[0]{\mathcal{L}}

% \renewcommand{\vec}[1]{\mathbf{#1}}
\newcommand{\hist}[0]{\mathcal{H}}
\newcommand{\param}[0]{p}
\newcommand{\algo}[0]{\mathcal{A}}
\newcommand{\algos}[0]{\mathbf{A}}
%\newcommand{\nn}[0]{N}
\newcommand{\feats}[0]{\mathcal{F}}
\newcommand{\feat}[0]{\vec{f}}
\newcommand{\cluster}[0]{\vec{h}}
\newcommand{\clusters}[0]{\vec{H}}
\newcommand{\perf}[0]{\mathbb{R}}
%\newcommand{\surro}[0]{\mathcal{S}}
\newcommand{\surro}[0]{\hat{f}}
\newcommand{\func}[0]{f}
\newcommand{\epm}[0]{\surro}
\newcommand{\portfolio}[0]{\mathcal{P}}
\newcommand{\schedule}[0]{\mathcal{S}}
\newcommand{\mdata}[0]{\dataset_{\text{meta}}}

% Deep Learning
\newcommand{\weights}[0]{\theta}
\newcommand{\metaweights}[0]{\phi}


% reinforcement learning
\newcommand{\policies}[0]{\Pi}
\newcommand{\policy}[0]{\pi}
\newcommand{\actionRL}[0]{a}
\newcommand{\stateRL}[0]{s}
\newcommand{\statesRL}[0]{\mathcal{S}}
\newcommand{\rewardRL}[0]{r}
\newcommand{\rewardfuncRL}[0]{\mathcal{R}}

\RestyleAlgo{algoruled}
\DontPrintSemicolon
\LinesNumbered
\SetAlgoVlined
\SetFuncSty{textsc}

\SetKwInOut{Input}{Input}
\SetKwInOut{Output}{Output}
\SetKw{Return}{return}

%\newcommand{\changed}[1]{{\color{red}#1}}

%\newcommand{\citeN}[1]{\citeauthor{#1}~(\citeyear{#1})}

\renewcommand{\vec}[1]{\mathbf{#1}}
\DeclareMathOperator*{\argmin}{arg\,min}
\DeclareMathOperator*{\argmax}{arg\,max}

\newcommand{\aqme}{\textit{AQME}}
\newcommand{\aslib}{\textit{ASlib}}
\newcommand{\llama}{\textit{LLAMA}}
\newcommand{\satzilla}{\textit{SATzilla}}
\newcommand{\satzillaY}[1]{\textit{SATzilla'{#1}}}
\newcommand{\snnap}{\textit{SNNAP}}
\newcommand{\claspfolioTwo}{\textit{claspfolio~2}}
\newcommand{\flexfolio}{\textit{FlexFolio}}
\newcommand{\claspfolioOne}{\textit{claspfolio~1}}
\newcommand{\isac}{\textit{ISAC}}
\newcommand{\eisac}{\textit{EISAC}}
\newcommand{\sss}{\textit{3S}}
\newcommand{\sunny}{\textit{Sunny}}
\newcommand{\ssspar}{\textit{3Spar}}
\newcommand{\cshc}{\textit{CSHC}}  
\newcommand{\cshcpar}{\textit{CSHCpar}}  
\newcommand{\measp}{\textit{ME-ASP}} 
\newcommand{\aspeed}{\textit{aspeed}}
\newcommand{\autofolio}{\textit{AutoFolio}}
\newcommand{\cedalion}{\textit{Cedalion}}
\newcommand{\fanova}{\textit{fANOVA}}
\newcommand{\sbs}{\textit{SB}}
\newcommand{\oracle}{\textit{VBS}}

% like approaches
\newcommand{\claspfoliolike}[1]{\texttt{claspfolio-#1-like}}
\newcommand{\satzillalike}[1]{\texttt{SATzilla'#1-like}}
\newcommand{\isaclike}{\texttt{ISAC-like}}
\newcommand{\ssslike}{\texttt{3S-like}}
\newcommand{\measplike}{\texttt{ME-ASP-like}}

\newcommand{\aspCoseal}{\textit{ASP-POTASSCO}}
\newcommand{\cspCoseal}{\textit{CSP-2010}}
\newcommand{\maxsatCoseal}{\textit{MAXSAT12-PMS}}
\newcommand{\premarCoseal}{\textit{PRE\-MARSHALLING}}
\newcommand{\qbfCoseal}{\textit{QBF-2011}}
\newcommand{\satallTwelveCoseal}{\textit{SAT12-ALL}}
\newcommand{\sathandTwelveCoseal}{\textit{SAT12-HAND}}
\newcommand{\satinduTwelveCoseal}{\textit{SAT12-INDU}}
\newcommand{\satrandTwelveCoseal}{\textit{SAT12-RAND}}
\newcommand{\sathandElevenCoseal}{\textit{SAT11-HAND}}
\newcommand{\satinduElevenCoseal}{\textit{SAT11-INDU}}
\newcommand{\satrandElevenCoseal}{\textit{SAT11-RAND}}
\newcommand{\proteusCoseal}{\textit{PROTEUS-2014}}

\newcommand{\irace}{\textit{I/F-race}}
\newcommand{\gga}{\textit{GGA}}
\newcommand{\smac}{\textit{SMAC}}
\newcommand{\paramils}{\textit{ParamILS}}
\newcommand{\spearmint}{\textit{Spearmint}}
\newcommand{\tpe}{\textit{TPE}}

\newcommand{\gringo}{\textit{gringo}}
\newcommand{\clasp}{\textit{clasp}}
\newcommand{\lingeling}{\textit{lingeling}}

\newcommand{\hydra}{\textit{Hydra}}

\newcommand{\plingeling}{\textit{Plingeling}}
\newcommand{\ccasat}{\textit{CCASat}}

\usepackage{pifont}
\newcommand{\itarrow}{\mbox{\Pisymbol{pzd}{229}}}
\newcommand{\ithook}{\mbox{\Pisymbol{pzd}{52}}}
\newcommand{\itcross}{\mbox{\Pisymbol{pzd}{56}}}
\newcommand{\ithand}{\mbox{\raisebox{-1pt}{\Pisymbol{pzd}{43}}}}

%\DeclareMathOperator*{\argmax}{arg\,max}

\newcommand{\ie}{{\it{}i.e.\/}}
\newcommand{\eg}{{\it{}e.g.\/}}
\newcommand{\cf}{{\it{}cf.\/}}
\newcommand{\wrt}{\mbox{w.r.t.}}
\newcommand{\vs}{{\it{}vs\/}}
\newcommand{\vsp}{{\it{}vs\/}}
\newcommand{\etc}{{\copyedit{etc.}}}
\newcommand{\etal}{{\it{}et al.\/}}

\newcommand{\pscProc}{{\bf procedure}}
\newcommand{\pscBegin}{{\bf begin}}
\newcommand{\pscEnd}{{\bf end}}
\newcommand{\pscEndIf}{{\bf endif}}
\newcommand{\pscFor}{{\bf for}}
\newcommand{\pscEach}{{\bf each}}
\newcommand{\pscThen}{{\bf then}}
\newcommand{\pscElse}{{\bf else}}
\newcommand{\pscWhile}{{\bf while}}
\newcommand{\pscIf}{{\bf if}}
\newcommand{\pscRepeat}{{\bf repeat}}
\newcommand{\pscUntil}{{\bf until}}
\newcommand{\pscWithProb}{{\bf with probability}}
\newcommand{\pscOtherwise}{{\bf otherwise}}
\newcommand{\pscDo}{{\bf do}}
\newcommand{\pscTo}{{\bf to}}
\newcommand{\pscOr}{{\bf or}}
\newcommand{\pscAnd}{{\bf and}}
\newcommand{\pscNot}{{\bf not}}
\newcommand{\pscFalse}{{\bf false}}
\newcommand{\pscEachElOf}{{\bf each element of}}
\newcommand{\pscReturn}{{\bf return}}

%\newcommand{\param}[1]{{\sl{}#1}}
\newcommand{\var}[1]{{\it{}#1}}
\newcommand{\cond}[1]{{\sf{}#1}}
%\newcommand{\state}[1]{{\sf{}#1}}
%\newcommand{\func}[1]{{\sl{}#1}}
\newcommand{\set}[1]{{\Bbb #1}}
%\newcommand{\inst}[1]{{\tt{}#1}}
\newcommand{\myurl}[1]{{\small\sf #1}}

\newcommand{\Nats}{{\Bbb N}}
\newcommand{\Reals}{{\Bbb R}}
\newcommand{\extset}[2]{\{#1 \; | \; #2\}}

\newcommand{\vbar}{$\,\;|$\hspace*{-1em}\raisebox{-0.3mm}{$\,\;\;|$}}
\newcommand{\vendbar}{\raisebox{+0.4mm}{$\,\;|$}}
\newcommand{\vend}{$\,\:\lfloor$}


\newcommand{\goleft}[2][.7]{\parbox[t]{#1\linewidth}{\strut\raggedright #2\strut}}
\newcommand{\rightimage}[2][.3]{\mbox{}\hfill\raisebox{1em-\height}[0pt][0pt]{\includegraphics[width=#1\linewidth]{#2}}\vspace*{-\baselineskip}}







\newcommand{\lz}{\vspace{0.5cm}}
\newcommand{\thetab}{\bm{\weights}}
\newcommand{\zero}{\mathbf{0}}
\newcommand{\Xmat}{\mathbf{X}}
\newcommand{\ydat}{\mathbf{y}}
\newcommand{\id}{\boldsymbol{I}}
\newcommand{\Amat}{\mathbf{A}}
\newcommand{\Xspace}{\mathcal{X}}                                           
\newcommand{\Yspace}{\mathcal{Y}}
\newcommand{\ls}{\ell}
\newcommand{\natnum}{\mathbb{N}}
\newcommand{\intnum}{\mathbb{Z}}

\usepackage{fontawesome}
\usepackage{dirtytalk}
\usepackage{csquotes}


\title[AutoML: GPs]{AutoML: Gaussian Processes} % week title
\subtitle{The Bayesian Linear Model} % video title
\author[Marius Lindauer]{\underline{Bernd Bischl} \and Frank Hutter \and Lars Kotthoff\newline \and Marius Lindauer \and Joaquin Vanschoren}
\institute{}
\date{}



% \AtBeginSection[] % Do nothing for \section*
% {
%   \begin{frame}{Outline}
%     \bigskip
%     \vfill
%     \tableofcontents[currentsection]
%   \end{frame}
% }

\begin{document}
	
	\maketitle
	

%----------------------------------------------------------------------
%----------------------------------------------------------------------
\begin{frame}[c, allowframebreaks]{Review: The Bayesian Linear Model}

Let $\datasettrain = \left\{(\xI{1},\yI{1}), ..., (\xI{n},\yI{n})\right\}$ be a training set of i.i.d. observations from some unknown distribution.
\begin{figure}
  \includegraphics[width=0.4\textwidth]{figure_man/bayes-lm/example.pdf}
\end{figure}

Let $\textbf{y} = (\yI{1}, ..., \yI{n})^\top$ and $\Xmat \in \realnum^{n \times p}$ be the design matrix where the i-th row contains vector $\xI{i}$. 





%%%%%%%%%%%%%%%%%%%%%%%%%%%%%%%%%%%%%%%%%%%%%%%%%%%%%%%%%%%%%%%%%%%%%%%%%%%%%%%%%%%%
\framebreak

The linear regression model is defined as


$$
\yI{i} = f(\xI{i}) + \epsilon^{(i)} = \bm{\weights}^\top \xI{i} + \epsilon^{(i)} \text{, for all } i \in \{1,\dots,n\}.
$$

\lz 

\lz 

The observed values $\yI{i}$ differ from the function values $f(\xI{i})$ by some additive noise, which is assumed to be i.i.d. Gaussian 
$$\epsilon^{(i)} \sim \normaldist (0, \variance).$$


%%%%%%%%%%%%%%%%%%%%%%%%%%%%%%%%%%%%%%%%%%%%%%%%%%%%%%%%%%%%%%%%%%%%%%%%%%%%%%%%%%%%%
\framebreak
\begin{itemize}
  \item Let us assume we have \textbf{prior beliefs} about the parameter $\thetab$ that are represented in a prior distribution $\thetab \sim \normaldist (\zero, \tau^2 \id_p).$

\lz
\lz

\item Whenever data points are observed, we update the parameters' prior distribution according to Bayes' rule

\end{itemize}




$$
\underbrace{p(\thetab \mid \Xmat, \ydat)}_{\text{posterior}} = \frac{\overbrace{p(\ydat \mid \Xmat, \thetab)}^{\text{likelihood}}\overbrace{q(\thetab)}^{\text{prior}}}{\underbrace{p(\ydat\mid\Xmat)}_{\text{marginal}}}.
$$


%%%%%%%%%%%%%%%%%%%%%%%%%%%%%%%%%%%%%%%%%%%%%%%%%%%%%%%%%%%%%%%%%%%%%%%%%%%%%%%%%%%%%
\framebreak

The posterior distribution of the parameter $\thetab$ is again normal distributed (the Gaussian family is self-conjugate): 

$$
\thetab \mid \Xmat, \ydat \sim \normaldist(\sigma^{-2}\bm{A}^{-1}\Xmat^\top\ydat, \bm{A}^{-1})\text{, where }\bm{A}:= \sigma^{-2}\Xmat^\top\Xmat + \frac{1}{\tau^2} \id_p.
$$


\lz 

\begin{footnotesize}
\textbf{Note:} If the posterior distributions $p(\thetab\mid\Xmat, \ydat)$ are in the same probability distribution family as the prior $q(\thetab)$, the prior and posterior are then called \textbf{conjugate distributions}, and the prior is  called a \textbf{conjugate prior} for the likelihood function $p(\ydat\mid\Xmat, \thetab)$. 
\end{footnotesize}

\lz

\begin{footnotesize}
\textbf{Note:} The Gaussian family is \textbf{self-conjugate} with respect to a Gaussian likelihood function: choosing a Gaussian prior for a Gaussian likelihood ensures that the posterior is also Gaussian.
\end{footnotesize}


%%%%%%%%%%%%%%%%%%%%%%%%%%%%%%%%%%%%%%%%%%%%%%%%%%%%%%%%%%%%%%%%%%%%%%%%%%%%%%%%%%%%%
\framebreak

%\begin{figure}
%  \includegraphics[width=0.5\textwidth]{figure_man/bayes-lm/prior-1.pdf}~\includegraphics[width=0.5\textwidth]{figure_man/bayes-lm/prior-2.pdf}
%\end{figure}


%%%%%%%%%%%%%%%%%%%%%%%%%%%%%%%%%%%%%%%%%%%%%%%%%%%%%%%%%%%%%%%%%%%%%%%%%%%%%%%%%%%%%
\framebreak


%\foreach \x in{5, 10, 20} {
%\begin{figure}
%  \includegraphics[width=0.5\textwidth]{figure_man/bayes-lm/posterior-\x-1.pdf}~  \includegraphics[width=0.5\textwidth]{figure_man/bayes-lm/posterior-\x-2.pdf}
%\end{figure}
%}
%\end{comment}

%%%%%%%%%%%%%%%%%%%%%%%%%%%%%%%%%%%%%%%%%%%%%%%%%%%%%%%%%%%%%%%%%%%%%%%%%%%%%%%%%%%%%
\framebreak


\begin{footnotesize}
\textbf{Theorem:}\\
\begin{itemize}
  \item For a Gaussian prior on $\thetab \sim \normaldist(\zero, \tau^2 \id_p)$ and a Gaussian likelihood $y \mid \Xmat, \thetab \sim \normaldist(\Xmat^\top \thetab, \sigma^2 \id_n)$, the resulting posterior is Gaussian: $\normaldist(\sigma^{-2}\bm{A}^{-1}\Xmat^\top\ydat, \bm{A}^{-1})$, with $\bm{A}:= \sigma^{-2}\Xmat^\top\Xmat + \frac{1}{\tau^2} \id_p$.
  \end{itemize}

\vspace{+.2cm}
\textbf{Proof:}\\
Plugging in Bayes' rule and multiplying out yields

\vspace{-.5cm}

\begin{eqnarray*}
p(\thetab \mid \Xmat, \ydat) &\propto& p(\ydat \mid \Xmat, \thetab) q(\thetab) \propto \exp\biggl[-\frac{1}{2\sigma^2}(\ydat - \Xmat\thetab)^\top(\ydat - \Xmat\thetab)-\frac{1}{2\tau^2}\thetab^\top\thetab\biggr] \\
&=& \exp\biggl[-\frac{1}{2}\biggl(\underbrace{\sigma^{-2}\ydat^\top\ydat}_{\text{doesn't depend on } \thetab} - 2 \sigma^{-2} \ydat^\top \Xmat \thetab + \sigma^{-2}\thetab^\top \Xmat^\top \Xmat \thetab  + \tau^{-2} \thetab^\top\thetab \biggr)\biggr] \\
&\propto& \exp\biggl[-\frac{1}{2}\biggl(\sigma^{-2}\thetab^\top \Xmat^\top \Xmat \thetab  + \tau^{-2} \thetab^\top\thetab  - 2 \sigma^{-2} \ydat^\top \Xmat \thetab \biggr)\biggr] \\
&=& \exp\biggl[-\frac{1}{2}\thetab^\top\underbrace{\biggl(\sigma^{-2} \Xmat^\top \Xmat + \tau^{-2} \id_p \biggr)}_{:= \Amat} \thetab + \textcolor{red}{\sigma^{-2} \ydat^\top \Xmat \thetab}\biggr]
\end{eqnarray*}

This expression resembles a normal density - except for the term in red!


%\end{footnotesize}


%%%%%%%%%%%%%%%%%%%%%%%%%%%%%%%%%%%%%%%%%%%%%%%%%%%%%%%%%%%%%%%%%%%%%%%%%%%%%%%%%%%%%
\framebreak

\textbf{Note:} We need not worry about the normalizing constant since its mere role is to convert probability functions to density functions with a total probability of one.

\vspace{+.2cm}

We subtract a (not yet defined) constant $c$ while compensating for this change by adding the respective terms (``adding $0$''), emphasized in green:

\vspace{-.5cm}

\begin{eqnarray*}
	p(\thetab | \Xmat, \ydat) &\propto&  \exp\biggl[-\frac{1}{2}(\thetab \textcolor{green}{- c})^\top\Amat  (\thetab \textcolor{green}{- c}) \textcolor{green}{- c^\top \Amat \thetab} + \underbrace{\textcolor{green}{\frac{1}{2}c^\top\Amat c}}_{\text{doesn't depend on } \thetab} +\sigma^{-2} \ydat^\top \Xmat \thetab\biggr] \\
	&\propto& \exp\biggl[-\frac{1}{2}(\thetab \textcolor{green}{- c})^\top\Amat  (\thetab \textcolor{green}{- c}) \textcolor{green}{- c^\top \Amat \thetab} +\sigma^{-2} \ydat^\top \Xmat \thetab\biggr]
\end{eqnarray*}

If we choose $c$ such that $- c^\top \Amat \thetab +\sigma^{-2} \ydat^\top \Xmat \thetab = 0$, the posterior is normal with mean $c$ and covariance matrix $\Amat^{-1}$. Taking into account that $\Amat$ is symmetric, this is if we choose

\vspace{-.5cm}

\begin{eqnarray*}
&& \sigma^{-2} \ydat^\top \Xmat = c^\top\Amat \\
&\Leftrightarrow & \sigma^{-2} \ydat^\top \Xmat \Amat^{-1} = c^\top \\
&\Leftrightarrow& c = \sigma^{-2} \Amat^{-1} \Xmat^\top \ydat
\end{eqnarray*}

\vspace{-.3cm}

as claimed.

\end{footnotesize}

%%%%%%%%%%%%%%%%%%%%%%%%%%%%%%%%%%%%%%%%%%%%%%%%%%%%%%%%%%%%%%%%%%%%%%%%%%%%%%%%%%%%%
\framebreak

\begin{itemize}

\item Based on the posterior destribution, 
$\thetab \mid \Xmat, \ydat \sim \normaldist(\sigma^{-2}\bm{A}^{-1}\Xmat^\top\ydat, \bm{A}^{-1})$,
we can derive the predictive distribution for a new observations $\x_*$.

\lz

\item The predictive distribution for the Bayesian linear model, i.e. the distribution of $\thetab^\top \x_*$, is
$$y_* \mid \Xmat, \ydat, \x_* \sim \normaldist(\sigma^{-2}\ydat^\top \Xmat \Amat^{-1}\x_*, \x_*^\top\Amat^{-1}\x_*).$$
\end{itemize}

\lz

\textbf{Note:} This can be obtained by applying the rules for linear transformations of Gaussians.

%%%%%%%%%%%%%%%%%%%%%%%%%%%%%%%%%%%%%%%%%%%%%%%%%%%%%%%%%%%%%%%%%%%%%%%%%%%%%%%%%%%%%
\framebreak

%\foreach \x in{5, 10, 20} {
%\begin{figure}
%  \includegraphics[width=0.5\textwidth]{figure_man/bayes-lm/posterior-\x-3.pdf} \\
%  \begin{footnotesize}
%    For every test input $\xv_*$, we get a distribution over the prediction $y_*$. In particular, we get a posterior mean (orange) and a posterior variance (the grey region, which equals $+/-$ two times the standard deviation).
%  \end{footnotesize}
%\end{figure}
%}

\end{frame}

%%%%%%%%%%%%%%%%%%%%%%%%%%%%%%%%%%%%%%%%%%%%%%%%%%%%%%%%%%%%%%%%%%%%%%%%%%%%%%%%%%%%%

\begin{frame}[c]{Summary: The Bayesian Linear Model}

\begin{itemize}
  \item By switching to a Bayesian perspective, we have not only point estimation for the parameter $\thetab$ but also whole \textbf{distributions}.
  \lz
  \item From the posterior distribution of $\thetab$, we can derive a predictive distribution for $y_* = \thetab^\top \x_*$.  
  \lz
  \item We can perform online updates: the \textbf{posterior distribution} of $\thetab$ can be updated whenever new datapoints are observed. 
\lz
\item In the next step, we would like go beyond the linear funtions and develop a theory for functions with general shapes.
\end{itemize}

\end{frame}
%-----------------------------------------------------------------------

\end{document}
