
\pdfminorversion=4 % for acroread
\documentclass[aspectratio=169,t,xcolor={usenames,dvipsnames}]{beamer}
%\documentclass[t,handout,xcolor={usenames,dvipsnames}]{beamer}
\usepackage{../beamerstyle}
\usepackage{dsfont}
\usepackage{bm}
\usepackage[english]{babel}
\usepackage[utf8]{inputenc}
\usepackage{graphicx}
\usepackage{algorithm}
\usepackage[ruled,vlined,algo2e,linesnumbered]{algorithm2e}
%\usepackage[boxed,vlined]{algorithm2e}
\usepackage{hyperref}
\usepackage{booktabs}
\usepackage{mathtools}

\usepackage{amsmath,amssymb}
\usepackage{listings}
\lstset{frame=lines,framesep=3pt,numbers=left,numberblanklines=false,basicstyle=\ttfamily\small}

\usepackage{subfig}
\usepackage{multicol}
%\usepackage{appendixnumberbeamer}
%
\usepackage{tcolorbox}

\usepackage{pgfplots}
\usepackage{tikz}
\usetikzlibrary{trees} 
\usetikzlibrary{shapes.geometric}
\usetikzlibrary{positioning,shapes,shadows,arrows,calc,mindmap}
\usetikzlibrary{positioning,fadings,through}
\usetikzlibrary{decorations.pathreplacing}
\usetikzlibrary{intersections}
\usetikzlibrary{positioning,fit,calc,shadows,backgrounds}
\pgfdeclarelayer{background}
\pgfdeclarelayer{foreground}
\pgfsetlayers{background,main,foreground}
\tikzstyle{activity}=[rectangle, draw=black, rounded corners, text centered, text width=8em]
\tikzstyle{data}=[rectangle, draw=black, text centered, text width=8em]
\tikzstyle{myarrow}=[->, thick, draw=black]

% Define the layers to draw the diagram
\pgfdeclarelayer{background}
\pgfdeclarelayer{foreground}
\pgfsetlayers{background,main,foreground}

%\usepackage{listings}
%\lstset{numbers=left,
%  showstringspaces=false,
%  frame={tb},
%  captionpos=b,
%  lineskip=0pt,
%  basicstyle=\ttfamily,
%%  extendedchars=true,
%  stepnumber=1,
%  numberstyle=\small,
%  xleftmargin=1em,
%  breaklines
%}

 
\definecolor{blue}{RGB}{0, 74, 153}

\usetheme{Boadilla}
%\useinnertheme{rectangles}
\usecolortheme{whale}
\setbeamercolor{alerted text}{fg=blue}
\useoutertheme{infolines}
\setbeamertemplate{navigation symbols}{\vspace{-5pt}} % to lower the logo
\setbeamercolor{date in head/foot}{bg=blue} % blue
\setbeamercolor{date in head/foot}{fg=white}
\setbeamercolor{author in head/foot}{bg=blue} %blue
\setbeamercolor{title in head/foot}{bg=blue} % blue
\setbeamercolor{title}{fg=white, bg=blue}
\setbeamercolor{block title}{fg=white,bg=blue}
\setbeamercolor{block body}{bg=blue!10}
\setbeamercolor{frametitle}{fg=white, bg=blue}
\setbeamercovered{invisible}

\makeatletter
\setbeamertemplate{footline}
{
  \leavevmode%
  \hbox{%
  \begin{beamercolorbox}[wd=.333333\paperwidth,ht=2.25ex,dp=1ex,center]{author in head/foot}%
    \usebeamerfont{author in head/foot}\insertshortauthor
  \end{beamercolorbox}%
  \begin{beamercolorbox}[wd=.333333\paperwidth,ht=2.25ex,dp=1ex,center]{title in head/foot}%
    \usebeamerfont{title in head/foot}\insertshorttitle
  \end{beamercolorbox}%
  \begin{beamercolorbox}[wd=.333333\paperwidth,ht=2.25ex,dp=1ex,right]{date in head/foot}%
    \usebeamerfont{date in head/foot}Week \@week, Topic \@topicnumber, Slide \insertframenumber{}\hspace*{2em}
%    \insertframenumber\hspace*{2ex} 
  \end{beamercolorbox}}%
  \vskip0pt%
}

\newcommand{\@week}{0}
\newcommand{\@topicnumber}{0}
\newcommand{\week}[1]{\renewcommand{\@week}{#1}}
\newcommand{\topicnumber}[1]{\renewcommand{\@topicnumber}{#1}}

\makeatother

%\pgfdeclareimage[height=1.2cm]{automl}{images/logos/automl.png}
%\pgfdeclareimage[height=1.2cm]{freiburg}{images/logos/freiburg}

%\logo{\pgfuseimage{freiburg}}

\input{../latex_main/macros}






\newcommand{\lz}{\vspace{0.5cm}}
\newcommand{\thetab}{\bm{\weights}}
\newcommand{\zero}{\mathbf{0}}
\newcommand{\Xmat}{\mathbf{X}}
\newcommand{\ydat}{\mathbf{y}}
\newcommand{\id}{\boldsymbol{I}}
\newcommand{\Amat}{\mathbf{A}}
\newcommand{\Xspace}{\mathcal{X}}                                           
\newcommand{\Yspace}{\mathcal{Y}}
\newcommand{\ls}{\ell}
\newcommand{\natnum}{\mathbb{N}}
\newcommand{\intnum}{\mathbb{Z}}

\usepackage{fontawesome}
\usepackage{dirtytalk}
\usepackage{csquotes}


\title[AutoML: GPs]{AutoML: Gaussian Processes} % week title
\subtitle{The Bayesian Linear Model} % video title
\author[Marius Lindauer]{\underline{Bernd Bischl} \and Frank Hutter \and Lars Kotthoff\newline \and Marius Lindauer \and Joaquin Vanschoren}
\institute{}
\date{}
\week{5}
\topicnumber{1}



% \AtBeginSection[] % Do nothing for \section*
% {
%   \begin{frame}{Outline}
%     \bigskip
%     \vfill
%     \tableofcontents[currentsection]
%   \end{frame}
% }

\begin{document}
	
	\maketitle
	

%----------------------------------------------------------------------
%----------------------------------------------------------------------
\begin{frame}[c, allowframebreaks]{Review: The Bayesian Linear Model}

Let $\datasettrain = \left\{(\xI{1},\yI{1}), ..., (\xI{n},\yI{n})\right\}$ be a training set of i.i.d. observations from some unknown distribution.
\begin{figure}
  \includegraphics[width=0.4\textwidth]{figure_man/bayes-lm/example.pdf}
\end{figure}

Let $\textbf{y} = (\yI{1}, ..., \yI{n})^\top$ and $\Xmat \in \realnum^{n \times p}$ be the design matrix where the i-th row contains vector $\xI{i}$. 





%%%%%%%%%%%%%%%%%%%%%%%%%%%%%%%%%%%%%%%%%%%%%%%%%%%%%%%%%%%%%%%%%%%%%%%%%%%%%%%%%%%%
\framebreak

The linear regression model is defined as

$$
y = f(\x) + \epsilon = \bm{\weights}^\top \x + \epsilon 
$$

or on the data:

$$
\yI{i} = f(\xI{i}) + \epsilon^{(i)} = \bm{\weights}^\top \xI{i} + \epsilon^{(i)} \text{, for all } i \in \{1,\dots,n\}.
$$

\lz 
We now assume (from a Bayesian perspective) that also our parameter vector $\bm{\weights}$ is stochastic and follows a distribution.

The observed values $\yI{i}$ differ from the function values $f(\xI{i})$ by some additive noise, which is assumed to be i.i.d. Gaussian 
$$\epsilon^{(i)} \sim \normaldist (0, \variance)$$
and independent of $\x$ and $\theta$.


%%%%%%%%%%%%%%%%%%%%%%%%%%%%%%%%%%%%%%%%%%%%%%%%%%%%%%%%%%%%%%%%%%%%%%%%%%%%%%%%%%%%%
\framebreak

\begin{itemize}
  \item Let us assume we have \textbf{prior beliefs} about the parameter $\thetab$ that are represented in a prior distribution $\thetab \sim \normaldist (\zero, \tau^2 \id_p).$

\lz
\lz

\item Whenever data points are observed, we update the parameters' prior distribution according to Bayes' rule

\end{itemize}




$$
\underbrace{p(\thetab \mid \Xmat, \ydat)}_{\text{posterior}} = \frac{\overbrace{p(\ydat \mid \Xmat, \thetab)}^{\text{likelihood}}\overbrace{q(\thetab)}^{\text{prior}}}{\underbrace{p(\ydat\mid\Xmat)}_{\text{marginal}}}.
$$


%%%%%%%%%%%%%%%%%%%%%%%%%%%%%%%%%%%%%%%%%%%%%%%%%%%%%%%%%%%%%%%%%%%%%%%%%%%%%%%%%%%%%
\framebreak

The posterior distribution of the parameter $\thetab$ is again normal distributed (the Gaussian family is self-conjugate): 

$$
\thetab \mid \Xmat, \ydat \sim \normaldist(\sigma^{-2}\bm{A}^{-1}\Xmat^\top\ydat, \bm{A}^{-1})\text{, where }\bm{A}:= \sigma^{-2}\Xmat^\top\Xmat + \frac{1}{\tau^2} \id_p.
$$


\lz 

\begin{footnotesize}
\textbf{Note:} If the posterior distributions $p(\thetab\mid\Xmat, \ydat)$ are in the same probability distribution family as the prior $q(\thetab)$, the prior and posterior are then called \textbf{conjugate distributions}, and the prior is  called a \textbf{conjugate prior} for the likelihood function $p(\ydat\mid\Xmat, \thetab)$. 
\end{footnotesize}

\lz

\begin{footnotesize}
\textbf{Note:} The Gaussian family is \textbf{self-conjugate} with respect to a Gaussian likelihood function: choosing a Gaussian prior for a Gaussian likelihood ensures that the posterior is also Gaussian.
\end{footnotesize}


%%%%%%%%%%%%%%%%%%%%%%%%%%%%%%%%%%%%%%%%%%%%%%%%%%%%%%%%%%%%%%%%%%%%%%%%%%%%%%%%%%%%%
\framebreak

\begin{figure}
\includegraphics[width=0.4\textwidth]{figure_man/bayes-lm/prior-1.pdf}~\includegraphics[width=0.4\textwidth]{figure_man/bayes-lm/prior-2.pdf}
\end{figure}


%%%%%%%%%%%%%%%%%%%%%%%%%%%%%%%%%%%%%%%%%%%%%%%%%%%%%%%%%%%%%%%%%%%%%%%%%%%%%%%%%%%%%
\framebreak


\foreach \x in{5, 10, 20} {
\begin{figure}
\includegraphics[width=0.4\textwidth]{figure_man/bayes-lm/posterior-\x-1.pdf}~  \includegraphics[width=0.4\textwidth]{figure_man/bayes-lm/posterior-\x-2.pdf}
\end{figure}
\framebreak
}


%%%%%%%%%%%%%%%%%%%%%%%%%%%%%%%%%%%%%%%%%%%%%%%%%%%%%%%%%%%%%%%%%%%%%%%%%%%%%%%%%%%%%
\framebreak


\begin{footnotesize}
\textbf{Theorem:}\\
\begin{itemize}
  \item For a Gaussian prior on $\thetab \sim \normaldist(\zero, \tau^2 \id_p)$ and a Gaussian likelihood $y \mid \Xmat, \thetab \sim \normaldist(\Xmat^\top \thetab, \sigma^2 \id_n)$, the resulting posterior is Gaussian: $\normaldist(\sigma^{-2}\bm{A}^{-1}\Xmat^\top\ydat, \bm{A}^{-1})$, with $\bm{A}:= \sigma^{-2}\Xmat^\top\Xmat + \frac{1}{\tau^2} \id_p$.
  \end{itemize}

\vspace{+.2cm}
\textbf{Proof:}\\
Plugging in Bayes' rule and multiplying out yields

\vspace{-.5cm}

\begin{eqnarray*}
p(\thetab \mid \Xmat, \ydat) &\propto& p(\ydat \mid \Xmat, \thetab) q(\thetab) \propto \exp\biggl[-\frac{1}{2\sigma^2}(\ydat - \Xmat\thetab)^\top(\ydat - \Xmat\thetab)-\frac{1}{2\tau^2}\thetab^\top\thetab\biggr] \\
&=& \exp\biggl[-\frac{1}{2}\biggl(\underbrace{\sigma^{-2}\ydat^\top\ydat}_{\text{doesn't depend on } \thetab} - 2 \sigma^{-2} \ydat^\top \Xmat \thetab + \sigma^{-2}\thetab^\top \Xmat^\top \Xmat \thetab  + \tau^{-2} \thetab^\top\thetab \biggr)\biggr] \\
&\propto& \exp\biggl[-\frac{1}{2}\biggl(\sigma^{-2}\thetab^\top \Xmat^\top \Xmat \thetab  + \tau^{-2} \thetab^\top\thetab  - 2 \sigma^{-2} \ydat^\top \Xmat \thetab \biggr)\biggr] \\
&=& \exp\biggl[-\frac{1}{2}\thetab^\top\underbrace{\biggl(\sigma^{-2} \Xmat^\top \Xmat + \tau^{-2} \id_p \biggr)}_{:= \Amat} \thetab + \textcolor{red}{\sigma^{-2} \ydat^\top \Xmat \thetab}\biggr]
\end{eqnarray*}

This expression resembles a normal density - except for the term in red!


%\end{footnotesize}


%%%%%%%%%%%%%%%%%%%%%%%%%%%%%%%%%%%%%%%%%%%%%%%%%%%%%%%%%%%%%%%%%%%%%%%%%%%%%%%%%%%%%
\framebreak

\textbf{Note:} We need not worry about the normalizing constant since its mere role is to convert probability functions to density functions with a total probability of one.

\vspace{+.2cm}

We subtract a (not yet defined) constant $c$ while compensating for this change by adding the respective terms (``adding $0$''), emphasized in green:

\vspace{-.5cm}

\begin{eqnarray*}
	p(\thetab | \Xmat, \ydat) &\propto&  \exp\biggl[-\frac{1}{2}(\thetab \textcolor{green}{- c})^\top\Amat  (\thetab \textcolor{green}{- c}) \textcolor{green}{- c^\top \Amat \thetab} + \underbrace{\textcolor{green}{\frac{1}{2}c^\top\Amat c}}_{\text{doesn't depend on } \thetab} +\sigma^{-2} \ydat^\top \Xmat \thetab\biggr] \\
	&\propto& \exp\biggl[-\frac{1}{2}(\thetab \textcolor{green}{- c})^\top\Amat  (\thetab \textcolor{green}{- c}) \textcolor{green}{- c^\top \Amat \thetab} +\sigma^{-2} \ydat^\top \Xmat \thetab\biggr]
\end{eqnarray*}

If we choose $c$ such that $- c^\top \Amat \thetab +\sigma^{-2} \ydat^\top \Xmat \thetab = 0$, the posterior is normal with mean $c$ and covariance matrix $\Amat^{-1}$. Taking into account that $\Amat$ is symmetric, this is if we choose

\vspace{-.5cm}

\begin{eqnarray*}
&& \sigma^{-2} \ydat^\top \Xmat = c^\top\Amat \\
&\Leftrightarrow & \sigma^{-2} \ydat^\top \Xmat \Amat^{-1} = c^\top \\
&\Leftrightarrow& c = \sigma^{-2} \Amat^{-1} \Xmat^\top \ydat
\end{eqnarray*}

\vspace{-.3cm}

as claimed.

\end{footnotesize}

%%%%%%%%%%%%%%%%%%%%%%%%%%%%%%%%%%%%%%%%%%%%%%%%%%%%%%%%%%%%%%%%%%%%%%%%%%%%%%%%%%%%%
\framebreak

\begin{itemize}

\item Based on the posterior destribution, 
$\thetab \mid \Xmat, \ydat \sim \normaldist(\sigma^{-2}\bm{A}^{-1}\Xmat^\top\ydat, \bm{A}^{-1})$,
we can derive the predictive distribution for a new observation $\x_*$.

\lz

\item The predictive distribution for the Bayesian linear model, i.e. the distribution of $\thetab^\top \x_*$, is
$$y_* \mid \Xmat, \ydat, \x_* \sim \normaldist(\sigma^{-2}\ydat^\top \Xmat \Amat^{-1}\x_*, \x_*^\top\Amat^{-1}\x_*).$$
\end{itemize}

\lz

\textbf{Note:} This can be obtained by applying the rules for linear transformations of Gaussians.

%%%%%%%%%%%%%%%%%%%%%%%%%%%%%%%%%%%%%%%%%%%%%%%%%%%%%%%%%%%%%%%%%%%%%%%%%%%%%%%%%%%%%
\framebreak

\foreach \j in{5, 10, 20} {
\begin{figure}
\includegraphics[width=0.4\textwidth]{figure_man/bayes-lm/posterior-\j-3.pdf} \par
\begin{footnotesize}
For every test input $\x_*$, we get a distribution over the prediction $y_*$. In particular, we get a posterior mean (orange) and a posterior variance (the grey region, which equals $+/-$ two times the standard deviation).
\end{footnotesize}
\end{figure}
\framebreak
}

\end{frame}

%%%%%%%%%%%%%%%%%%%%%%%%%%%%%%%%%%%%%%%%%%%%%%%%%%%%%%%%%%%%%%%%%%%%%%%%%%%%%%%%%%%%%

\begin{frame}[c]{Summary: The Bayesian Linear Model}

\begin{itemize}
  \item By switching to a Bayesian perspective, we have not only point estimation for the parameter $\thetab$ but also whole \textbf{distributions}.
  \lz
  \item From the posterior distribution of $\thetab$, we can derive a predictive distribution for $y_* = \thetab^\top \x_*$.  
  \lz
  \item We can perform online updates: the \textbf{posterior distribution} of $\thetab$ can be updated whenever new datapoints are observed. 
\lz
\item In the next step, we would like go beyond the linear funtions and develop a theory for functions with general shapes.
\end{itemize}

\end{frame}
%-----------------------------------------------------------------------

\end{document}
