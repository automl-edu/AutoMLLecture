
\pdfminorversion=4 % for acroread
\documentclass[aspectratio=169,t,xcolor={usenames,dvipsnames}]{beamer}
%\documentclass[t,handout,xcolor={usenames,dvipsnames}]{beamer}
\usepackage{../beamerstyle}
\usepackage{dsfont}
\usepackage{bm}
\usepackage[english]{babel}
\usepackage[utf8]{inputenc}
\usepackage{graphicx}
\usepackage{algorithm}
\usepackage[ruled,vlined,algo2e,linesnumbered]{algorithm2e}
%\usepackage[boxed,vlined]{algorithm2e}
\usepackage{hyperref}
\usepackage{booktabs}
\usepackage{mathtools}

\usepackage{amsmath,amssymb}
\usepackage{listings}
\lstset{frame=lines,framesep=3pt,numbers=left,numberblanklines=false,basicstyle=\ttfamily\small}

\usepackage{subfig}
\usepackage{multicol}
%\usepackage{appendixnumberbeamer}
%
\usepackage{tcolorbox}

\usepackage{pgfplots}
\usepackage{tikz}
\usetikzlibrary{trees} 
\usetikzlibrary{shapes.geometric}
\usetikzlibrary{positioning,shapes,shadows,arrows,calc,mindmap}
\usetikzlibrary{positioning,fadings,through}
\usetikzlibrary{decorations.pathreplacing}
\usetikzlibrary{intersections}
\usetikzlibrary{positioning,fit,calc,shadows,backgrounds}
\pgfdeclarelayer{background}
\pgfdeclarelayer{foreground}
\pgfsetlayers{background,main,foreground}
\tikzstyle{activity}=[rectangle, draw=black, rounded corners, text centered, text width=8em]
\tikzstyle{data}=[rectangle, draw=black, text centered, text width=8em]
\tikzstyle{myarrow}=[->, thick, draw=black]

% Define the layers to draw the diagram
\pgfdeclarelayer{background}
\pgfdeclarelayer{foreground}
\pgfsetlayers{background,main,foreground}

%\usepackage{listings}
%\lstset{numbers=left,
%  showstringspaces=false,
%  frame={tb},
%  captionpos=b,
%  lineskip=0pt,
%  basicstyle=\ttfamily,
%%  extendedchars=true,
%  stepnumber=1,
%  numberstyle=\small,
%  xleftmargin=1em,
%  breaklines
%}

 
\definecolor{blue}{RGB}{0, 74, 153}

\usetheme{Boadilla}
%\useinnertheme{rectangles}
\usecolortheme{whale}
\setbeamercolor{alerted text}{fg=blue}
\useoutertheme{infolines}
\setbeamertemplate{navigation symbols}{\vspace{-5pt}} % to lower the logo
\setbeamercolor{date in head/foot}{bg=white} % blue
\setbeamercolor{date in head/foot}{fg=white}
\setbeamercolor{author  in head/foot}{bg=white} %blue
\setbeamercolor{title in head/foot}{bg=white} % blue
\setbeamercolor{title}{fg=white, bg=blue}
\setbeamercolor{block title}{fg=white,bg=blue}
\setbeamercolor{block body}{bg=blue!10}
\setbeamercolor{frametitle}{fg=white, bg=blue}
\setbeamercovered{invisible}

\makeatletter
\setbeamertemplate{footline}
{
  \leavevmode%
  \hbox{%
  \begin{beamercolorbox}[wd=.333333\paperwidth,ht=2.25ex,dp=1ex,center]{author in head/foot}%
%    \usebeamerfont{author in head/foot}\insertshortauthor
  \end{beamercolorbox}%
  \begin{beamercolorbox}[wd=.333333\paperwidth,ht=2.25ex,dp=1ex,center]{title in head/foot}%
    \usebeamerfont{title in head/foot}\insertshorttitle
  \end{beamercolorbox}%
  \begin{beamercolorbox}[wd=.333333\paperwidth,ht=2.25ex,dp=1ex,right]{date in head/foot}%
    \usebeamerfont{date in head/foot}\insertshortdate{}\hspace*{2em}
%    \insertframenumber\hspace*{2ex} 
  \end{beamercolorbox}}%
  \vskip0pt%
}
\makeatother

%\pgfdeclareimage[height=1.2cm]{automl}{images/logos/automl.png}
%\pgfdeclareimage[height=1.2cm]{freiburg}{images/logos/freiburg}

%\logo{\pgfuseimage{freiburg}}

\newcommand{\comment}[1]{
	\noindent
	%\vspace{0.25cm}
	{\color{red}{\textbf{TODO:} #1}}
	%\vspace{0.25cm}
}
\renewcommand{\comment}[1]{}
\newcommand{\hide}[1]{}
\newcommand{\cemph}[2]{\emph{\textcolor{#1}{#2}}}

\newcommand{\lit}[1]{{\footnotesize\color{black!70}[#1]}}

\newcommand{\litw}[1]{{\footnotesize\color{black!20}[#1]}}


\newcommand{\myframe}[2]{\begin{frame}[c]{#1}#2\end{frame}}
\newcommand{\myframetop}[2]{\begin{frame}{#1}#2\end{frame}}
\newcommand{\myit}[1]{\begin{itemize}#1\end{itemize}}
\newcommand{\myblock}[2]{\begin{block}{#1}#2\end{block}}


\newcommand{\votepurple}[1]{\textcolor{Purple}{$\bigstar$}}
\newcommand{\voteyellow}[1]{\textcolor{Goldenrod}{$\bigstar$}}
\newcommand{\voteblue}[1]{\textcolor{RoyalBlue}{$\bigstar$}}
\newcommand{\votepink}[1]{\textcolor{Pink}{$\bigstar$}}

\newcommand{\diff}{\mathop{}\!\mathrm{d}}
\newcommand{\refstyle}[1]{{\small{\textcolor{gray}{#1}}}}
\newcommand{\hands}[0]{\includegraphics[height=1.5em]{images/hands}}
\newcommand{\transpose}[0]{{\textrm{\tiny{\sf{T}}}}}
\newcommand{\norm}{{\mathcal{N}}}
\newcommand{\cutoff}[0]{\kappa}
\newcommand{\instD}[0]{\dataset}
\newcommand{\insts}[0]{\mathcal{I}}
\newcommand{\inst}[0]{i}
\newcommand{\pcs}[0]{\mathbf{\Lambda}}
\newcommand{\bx}[0]{\conf}
\newcommand{\conf}[0]{\mathbf{\lambda}}
\newcommand{\defconf}[0]{\mathbf{\lambda}_{\text{def}}}
\newcommand{\finconf}[0]{\mathbf{\lambda}^*}
\newcommand{\incumbent}[0]{\finconf}
\newcommand{\confs}[0]{\pcs}
%\newcommand{\vlambda}[0]{\bm{\lambda}}
%\newcommand{\vLambda}[0]{\bm{\Lambda}}
\newcommand{\dataset}[0]{\mathcal{D}}
\newcommand{\datasets}[0]{\mathbf{D}}
\newcommand{\loss}[0]{\mathcal{L}}

% \renewcommand{\vec}[1]{\mathbf{#1}}
\newcommand{\hist}[0]{\mathcal{H}}
\newcommand{\param}[0]{p}
\newcommand{\algo}[0]{\mathcal{A}}
\newcommand{\algos}[0]{\mathbf{A}}
%\newcommand{\nn}[0]{N}
\newcommand{\feats}[0]{\mathcal{F}}
\newcommand{\feat}[0]{\vec{f}}
\newcommand{\cluster}[0]{\vec{h}}
\newcommand{\clusters}[0]{\vec{H}}
\newcommand{\perf}[0]{\mathbb{R}}
%\newcommand{\surro}[0]{\mathcal{S}}
\newcommand{\surro}[0]{\hat{f}}
\newcommand{\func}[0]{f}
\newcommand{\epm}[0]{\surro}
\newcommand{\portfolio}[0]{\mathcal{P}}
\newcommand{\schedule}[0]{\mathcal{S}}
\newcommand{\mdata}[0]{\dataset_{\text{meta}}}

% Deep Learning
\newcommand{\weights}[0]{\theta}
\newcommand{\metaweights}[0]{\phi}


% reinforcement learning
\newcommand{\policies}[0]{\Pi}
\newcommand{\policy}[0]{\pi}
\newcommand{\actionRL}[0]{a}
\newcommand{\stateRL}[0]{s}
\newcommand{\statesRL}[0]{\mathcal{S}}
\newcommand{\rewardRL}[0]{r}
\newcommand{\rewardfuncRL}[0]{\mathcal{R}}

\RestyleAlgo{algoruled}
\DontPrintSemicolon
\LinesNumbered
\SetAlgoVlined
\SetFuncSty{textsc}

\SetKwInOut{Input}{Input}
\SetKwInOut{Output}{Output}
\SetKw{Return}{return}

%\newcommand{\changed}[1]{{\color{red}#1}}

%\newcommand{\citeN}[1]{\citeauthor{#1}~(\citeyear{#1})}

\renewcommand{\vec}[1]{\mathbf{#1}}
\DeclareMathOperator*{\argmin}{arg\,min}
\DeclareMathOperator*{\argmax}{arg\,max}

\newcommand{\aqme}{\textit{AQME}}
\newcommand{\aslib}{\textit{ASlib}}
\newcommand{\llama}{\textit{LLAMA}}
\newcommand{\satzilla}{\textit{SATzilla}}
\newcommand{\satzillaY}[1]{\textit{SATzilla'{#1}}}
\newcommand{\snnap}{\textit{SNNAP}}
\newcommand{\claspfolioTwo}{\textit{claspfolio~2}}
\newcommand{\flexfolio}{\textit{FlexFolio}}
\newcommand{\claspfolioOne}{\textit{claspfolio~1}}
\newcommand{\isac}{\textit{ISAC}}
\newcommand{\eisac}{\textit{EISAC}}
\newcommand{\sss}{\textit{3S}}
\newcommand{\sunny}{\textit{Sunny}}
\newcommand{\ssspar}{\textit{3Spar}}
\newcommand{\cshc}{\textit{CSHC}}  
\newcommand{\cshcpar}{\textit{CSHCpar}}  
\newcommand{\measp}{\textit{ME-ASP}} 
\newcommand{\aspeed}{\textit{aspeed}}
\newcommand{\autofolio}{\textit{AutoFolio}}
\newcommand{\cedalion}{\textit{Cedalion}}
\newcommand{\fanova}{\textit{fANOVA}}
\newcommand{\sbs}{\textit{SB}}
\newcommand{\oracle}{\textit{VBS}}

% like approaches
\newcommand{\claspfoliolike}[1]{\texttt{claspfolio-#1-like}}
\newcommand{\satzillalike}[1]{\texttt{SATzilla'#1-like}}
\newcommand{\isaclike}{\texttt{ISAC-like}}
\newcommand{\ssslike}{\texttt{3S-like}}
\newcommand{\measplike}{\texttt{ME-ASP-like}}

\newcommand{\aspCoseal}{\textit{ASP-POTASSCO}}
\newcommand{\cspCoseal}{\textit{CSP-2010}}
\newcommand{\maxsatCoseal}{\textit{MAXSAT12-PMS}}
\newcommand{\premarCoseal}{\textit{PRE\-MARSHALLING}}
\newcommand{\qbfCoseal}{\textit{QBF-2011}}
\newcommand{\satallTwelveCoseal}{\textit{SAT12-ALL}}
\newcommand{\sathandTwelveCoseal}{\textit{SAT12-HAND}}
\newcommand{\satinduTwelveCoseal}{\textit{SAT12-INDU}}
\newcommand{\satrandTwelveCoseal}{\textit{SAT12-RAND}}
\newcommand{\sathandElevenCoseal}{\textit{SAT11-HAND}}
\newcommand{\satinduElevenCoseal}{\textit{SAT11-INDU}}
\newcommand{\satrandElevenCoseal}{\textit{SAT11-RAND}}
\newcommand{\proteusCoseal}{\textit{PROTEUS-2014}}

\newcommand{\irace}{\textit{I/F-race}}
\newcommand{\gga}{\textit{GGA}}
\newcommand{\smac}{\textit{SMAC}}
\newcommand{\paramils}{\textit{ParamILS}}
\newcommand{\spearmint}{\textit{Spearmint}}
\newcommand{\tpe}{\textit{TPE}}

\newcommand{\gringo}{\textit{gringo}}
\newcommand{\clasp}{\textit{clasp}}
\newcommand{\lingeling}{\textit{lingeling}}

\newcommand{\hydra}{\textit{Hydra}}

\newcommand{\plingeling}{\textit{Plingeling}}
\newcommand{\ccasat}{\textit{CCASat}}

\usepackage{pifont}
\newcommand{\itarrow}{\mbox{\Pisymbol{pzd}{229}}}
\newcommand{\ithook}{\mbox{\Pisymbol{pzd}{52}}}
\newcommand{\itcross}{\mbox{\Pisymbol{pzd}{56}}}
\newcommand{\ithand}{\mbox{\raisebox{-1pt}{\Pisymbol{pzd}{43}}}}

%\DeclareMathOperator*{\argmax}{arg\,max}

\newcommand{\ie}{{\it{}i.e.\/}}
\newcommand{\eg}{{\it{}e.g.\/}}
\newcommand{\cf}{{\it{}cf.\/}}
\newcommand{\wrt}{\mbox{w.r.t.}}
\newcommand{\vs}{{\it{}vs\/}}
\newcommand{\vsp}{{\it{}vs\/}}
\newcommand{\etc}{{\copyedit{etc.}}}
\newcommand{\etal}{{\it{}et al.\/}}

\newcommand{\pscProc}{{\bf procedure}}
\newcommand{\pscBegin}{{\bf begin}}
\newcommand{\pscEnd}{{\bf end}}
\newcommand{\pscEndIf}{{\bf endif}}
\newcommand{\pscFor}{{\bf for}}
\newcommand{\pscEach}{{\bf each}}
\newcommand{\pscThen}{{\bf then}}
\newcommand{\pscElse}{{\bf else}}
\newcommand{\pscWhile}{{\bf while}}
\newcommand{\pscIf}{{\bf if}}
\newcommand{\pscRepeat}{{\bf repeat}}
\newcommand{\pscUntil}{{\bf until}}
\newcommand{\pscWithProb}{{\bf with probability}}
\newcommand{\pscOtherwise}{{\bf otherwise}}
\newcommand{\pscDo}{{\bf do}}
\newcommand{\pscTo}{{\bf to}}
\newcommand{\pscOr}{{\bf or}}
\newcommand{\pscAnd}{{\bf and}}
\newcommand{\pscNot}{{\bf not}}
\newcommand{\pscFalse}{{\bf false}}
\newcommand{\pscEachElOf}{{\bf each element of}}
\newcommand{\pscReturn}{{\bf return}}

%\newcommand{\param}[1]{{\sl{}#1}}
\newcommand{\var}[1]{{\it{}#1}}
\newcommand{\cond}[1]{{\sf{}#1}}
%\newcommand{\state}[1]{{\sf{}#1}}
%\newcommand{\func}[1]{{\sl{}#1}}
\newcommand{\set}[1]{{\Bbb #1}}
%\newcommand{\inst}[1]{{\tt{}#1}}
\newcommand{\myurl}[1]{{\small\sf #1}}

\newcommand{\Nats}{{\Bbb N}}
\newcommand{\Reals}{{\Bbb R}}
\newcommand{\extset}[2]{\{#1 \; | \; #2\}}

\newcommand{\vbar}{$\,\;|$\hspace*{-1em}\raisebox{-0.3mm}{$\,\;\;|$}}
\newcommand{\vendbar}{\raisebox{+0.4mm}{$\,\;|$}}
\newcommand{\vend}{$\,\:\lfloor$}


\newcommand{\goleft}[2][.7]{\parbox[t]{#1\linewidth}{\strut\raggedright #2\strut}}
\newcommand{\rightimage}[2][.3]{\mbox{}\hfill\raisebox{1em-\height}[0pt][0pt]{\includegraphics[width=#1\linewidth]{#2}}\vspace*{-\baselineskip}}






\newcommand{\lz}{\vspace{0.5cm}}
\newcommand{\thetab}{\bm{\weights}}
\newcommand{\zero}{\mathbf{0}}
\newcommand{\Xmat}{\mathbf{X}}
\newcommand{\ydat}{\mathbf{y}}
\newcommand{\id}{\boldsymbol{I}}
\newcommand{\Amat}{\mathbf{A}}
\newcommand{\Xspace}{\mathcal{X}}
\newcommand{\Yspace}{\mathcal{Y}}
\newcommand{\ls}{\ell}
\newcommand{\natnum}{\mathbb{N}}
\newcommand{\intnum}{\mathbb{Z}}
\newcommand{\order}{\mathcal{O}}

\usepackage{fontawesome}
\usepackage{dirtytalk}
\usepackage{csquotes}


\def\argmin{\mathop{\sf arg\,min}}

%\begin{frame}[c]{}
%\centering
%\huge
%\textbf{}
%\end{frame}


%\item[\faLightbulbO]

\title[AutoML: GPs]{AutoML: Gaussian Processes} % week title
\subtitle{Gaussian Process Training} % video title
\author[Marius Lindauer]{\underline{Bernd Bischl} \and Frank Hutter \and Lars Kotthoff\newline \and Marius Lindauer \and Joaquin Vanschoren}
\institute{}
\date{}
\week{5}
\topicnumber{5}



\begin{document}
\maketitle



%%%%%%%%%%%%%%%%%%%%%%%%%%%%%%%%%%%%%%%%%%%%%%%%%%%%%%%%%%%%%%%%%%%%%%%%%%%%%%%%%%%%
\begin{frame}[c]{Training of a Gaussian Process}

\begin{itemize}
\vspace{.5cm}
\item To make predictions for a regression task by a Gaussian process, one simply needs to perform matrix computations.
\vspace{.5cm}
\item But for this to work out, we assume that the covariance functions is fully given, including all of its hyperparameters.
\vspace{.5cm}
\item A very nice property of GPs is that we can learn the numerical hyperparameters of a selected covariance function directly during GP training.
\end{itemize}


\end{frame}
%%%%%%%%%%%%%%%%%%%%%%%%%%%%%%%%%%%%%%%%%%%%%%%%%%%%%%%%%%%%%%%%%%%%%%%%%%%%%%%%%%%%

\begin{frame}[c,allowframebreaks]{Training a GP via the Maximum Likelihood}

\begin{itemize}
\item Let us assume $y = f(\x) + \epsilon, ~ \epsilon \sim \mathcal{N}\left(0, \variance\right),$ where $f(\x) \sim \gp\left(\bm{0}, k\left(\x, \x^\prime \mid \thetab \right)\right)$.

\lz
\lz

\item Noticing that $\bm{y} \sim \mathcal{N}\left(\bm{0}, \bm{K} + \variance \id\right)$, we can find the marginal log-likelihood (or evidence):

\vspace{-5mm}

\begin{eqnarray*}
log\, p(\bm{y} ~\mid~ \bm{X}, \thetab) &=& log \left[\left(2 \pi\right)^{-n / 2} |\bm{K}_y|^{-1 / 2} \exp\left(- \frac{1}{2} \bm{y}^\top \bm{K}_y^{-1} \bm{y}\right) \right]\\
&=& -\frac{1}{2}\bm{y}^\top\bm{K}_y^{-1} \bm{y} - \frac{1}{2}\, log \left| \bm{K}_y \right| - \frac{n}{2} log\,2\pi.
\end{eqnarray*}

with $\bm{K}_y:=\bm{K} + \variance \id$ and $\thetab$ denoting the parameters of the covariance function (i.e., the hyperparameters).
\end{itemize}

\framebreak
%%%%%%%%%%%%%%%%%%%%%%%%%%%%%%%%%%%%%%%%%%%%%%%%%%%%%%%%%%%%%%%%%%%%%%%%%%%%%%%%%%%%

Recalling that the increase of the length-scale reduces the model flexibility, the three terms of the marginal likelihood can be interpreted as follows.
\vspace{1cm}

\begin{itemize}
\item The data fit $-\frac{1}{2}\bm{y}^T\bm{K}_y^{-1} \bm{y}$. The data fit tends to decrease by increasing the length-scale.
\vspace{.5cm}

\item The complexity penalty $- \frac{1}{2}\,log\,\left| \bm{K}_y \right|$, which depends on the covariance function. This term decreases with the increase of the length-scale (the model gets less complex as the length-scale grows).
\vspace{.5cm}

\item The normalization constant $- \frac{n}{2}\,log\,2\pi$.
\end{itemize}


\end{frame}
%%%%%%%%%%%%%%%%%%%%%%%%%%%%%%%%%%%%%%%%%%%%%%%%%%%%%%%%%%%%%%%%%%%%%%%%%%%%%%%%%%%%

\begin{frame}[c,allowframebreaks]{Training a GP: Example}

To visualize this, let us consider a zero-mean GP with a squared exponential kernel:

$$
k(\x, \x^\prime) = \exp\left(-\frac{1}{2\ls^2}\|\x - \x^\prime\|^2\right).
$$


\begin{itemize}
	\item Recall that the model becomes smoother and less complex as the length-scale $\ls$ increases.
	\lz
	\item We will show how each of the following terms behaves if the value of $\ls$ increases:
	\vspace{2mm}
	\begin{itemize}
		\item the data fit $-\frac{1}{2}\bm{y}^\top\bm{K}_y^{-1} \bm{y}$,
		\vspace{2mm}
		\item the complexity penalty $-\frac{1}{2}\,log\,\left| \bm{K}_y \right|$,
		\vspace{2mm}
		\item the overall value of the marginal likelihood $log\,p(\bm{y}\mid\bm{X}, \thetab)$.
	\end{itemize}
\end{itemize}



\framebreak
%%%%%%%%%%%%%%%%%%%%%%%%%%%%%%%%%%%%%%%%%%%%%%%%%%%%%%%%%%%%%%%%%%%%%%%%%%%%%%%%%%%%

\begin{figure}
	\includegraphics[width = 0.4\textwidth]{figure_man/training/fit-vs-penalty.pdf}~	\includegraphics[width = 0.4\textwidth]{figure_man/training/datapoints.pdf}
\end{figure}

\begin{footnotesize}
\textcolor{blue}{\faInfoCircle} The left plot depicts how the data fit, the complexity penalty (a higher value means less penalization), and the overall marginal likelihood behave for increasing values of the length-scale.
\end{footnotesize}


\framebreak
%%%%%%%%%%%%%%%%%%%%%%%%%%%%%%%%%%%%%%%%%%%%%%%%%%%%%%%%%%%%%%%%%%%%%%%%%%%%%%%%%%%%


\begin{figure}
	\includegraphics[width = 0.4\textwidth]{figure_man/training/fit-vs-penalty-0_2.pdf}~	\includegraphics[width = 0.4\textwidth]{figure_man/training/datapoints-0_2.pdf}
\end{figure}


\begin{footnotesize}
\textcolor{blue}{\faInfoCircle} The left plot depicts how the data fit, the complexity penalty (a higher value means less penalization), and the overall marginal likelihood behave for increasing values of the length-scale.\\
\vspace{3mm}
\textcolor{blue}{\faLightbulbO} A small $\ls$ leads to a good fit, but, to a high complexity penalty.
\end{footnotesize}




\framebreak
%%%%%%%%%%%%%%%%%%%%%%%%%%%%%%%%%%%%%%%%%%%%%%%%%%%%%%%%%%%%%%%%%%%%%%%%%%%%%%%%%%%%

\begin{figure}
	\includegraphics[width = 0.4\textwidth]{figure_man/training/fit-vs-penalty-2.pdf}~	\includegraphics[width = 0.4\textwidth]{figure_man/training/datapoints-2.pdf}
\end{figure}

\begin{footnotesize}
\textcolor{blue}{\faInfoCircle} The left plot depicts how the data fit, the complexity penalty (a higher value means less penalization), and the overall marginal likelihood behave for increasing values of the length-scale.\\
\vspace{3mm}
\textcolor{blue}{\faLightbulbO} A large $\ls$ results in a poor fit.
\end{footnotesize}


\framebreak
%%%%%%%%%%%%%%%%%%%%%%%%%%%%%%%%%%%%%%%%%%%%%%%%%%%%%%%%%%%%%%%%%%%%%%%%%%%%%%%%%%%%

\begin{figure}
	\includegraphics[width = 0.4\textwidth]{figure_man/training/fit-vs-penalty-0_5.pdf}~	\includegraphics[width = 0.4\textwidth]{figure_man/training/datapoints-0_5.pdf}
\end{figure}


\begin{footnotesize}
\textcolor{blue}{\faInfoCircle} The left plot depicts how the data fit, the complexity penalty (a higher value means less penalization), and the overall marginal likelihood behave for increasing values of the length-scale.\\
\vspace{3mm}
\textcolor{blue}{\faLightbulbO} The maximizer of the log-likelihood ($\ls = 0.5$) balances the complexity and data the fit.
\end{footnotesize}


\end{frame}
%%%%%%%%%%%%%%%%%%%%%%%%%%%%%%%%%%%%%%%%%%%%%%%%%%%%%%%%%%%%%%%%%%%%%%%%%%%%%%%%%%%%

\begin{frame}[c,allowframebreaks]{Training a GP via the Maximum Likelihood}

To choose the hyperparameters by maximizing the marginal likelihood, we need to find the partial derivatives of the likelihood w.r.t. the hyperparameters:

\begin{footnotesize}
\begin{eqnarray*}
\frac{\partial}{\partial\theta_j} \, log\,p(\bm{y} \mid \bm{X}, \thetab) &=& \frac{\partial}{\partial\theta_j}  \left(-\frac{1}{2}\bm{y}^\top\bm{K}_y^{-1} \bm{y} - \frac{1}{2}\,log\, \left| \bm{K}_y \right| - \frac{n}{2} \,log\, 2\pi\right) \\
&=&\frac{1}{2} \bm{y}^\top \bm{K}^{-1} \frac{\partial \bm{K}}{\partial \theta_j}\bm{K}^{-1} \bm{y} - \frac{1}{2} \text{tr}\left(\bm{K}^{-1} \frac{\partial \bm{K}}{\partial \thetab} \right) \\
&=& \frac{1}{2} \text{tr}\left((\bm{K}^{-1}\bm{y}\bm{y}^\top\bm{K}^{-1} - \bm{K}^{-1})\frac{\partial\bm{K}}{\partial\theta_j}\right)
\end{eqnarray*}
\end{footnotesize}

\textcolor{blue}{\faLightbulbO} Above, we used the following identities:
\begin{footnotesize}
$$\frac{\partial}{\partial \theta_j} \bm{K}^{-1} = - \bm{K}^{-1} \frac{\partial \bm{K}}{\partial \theta_j}\bm{K}^{-1}\text{ and } \frac{\partial}{\partial \thetab} log  |\bm{K}| = \text{tr}\left(\bm{K}^{-1} \frac{\partial \bm{K}}{\partial \thetab} \right)$$
\end{footnotesize}
\framebreak
%%%%%%%%%%%%%%%%%%%%%%%%%%%%%%%%%%%%%%%%%%%%%%%%%%%%%%%%%%%%%%%%%%%%%%%%%%%%%%%%%%%%


\begin{itemize}
  \item The complexity and the runtime of training a Gaussian process is dominated by the computational task of inverting $\bm{K}$ - or let's rather say for decomposing it.
  \lz
  \item Standard methods require $\order(n^3)$ time (!) for this.
  \lz
  \item Once $\bm{K}^{-1}$ - or rather the decomposition -is known, the computation of the partial derivatives requires only $\order(n^2)$ time per hyperparameter.
  \lz
  \item[\faLightbulbO] Thus, the computational overhead of computing derivatives is small, and using a gradient based optimizer is advantageous.
\end{itemize}


\framebreak
%%%%%%%%%%%%%%%%%%%%%%%%%%%%%%%%%%%%%%%%%%%%%%%%%%%%%%%%%%%%%%%%%%%%%%%%%%%%%%%%%%%%
Workarounds to make GP estimation feasible for big data include:

\begin{itemize}
\item Using kernels that yield sparse $\bm K$: cheaper to invert.
\vspace{3mm}
\item Subsampling the data to estimate $\theta$; $\order(m^3)$ for subset of size $m$.
\vspace{3mm}
\item Combining estimates on different subsets of size $m$: \textbf{Bayesian committee}; $\order(n m^2)$.
\vspace{3mm}
\item Exploiting low-rank approximations of $\bm{K}$ by using only a representative subset (inducing points) of $m$ training data $\bm X_m$:\textbf{Nyström approximation} $\bm K\approx\bm K_{nm} \bm K_{mm}^{-} \bm K_{mn}$, with $\order(nmk + m^3)$ for a rank-k-approximate inverse of $\bm K_{mm}$.
\vspace{3mm}
\item Utilizing structure in $\bm{K}$ induced by the kernel: exact solutions but complicated maths, not applicable for all kernels.
\end{itemize}

\vspace{3mm}
... this is still an active area of research.




\end{frame}
%%%%%%%%%%%%%%%%%%%%%%%%%%%%%%%%%%%%%%%%%%%%%%%%%%%%%%%%%%%%%%%%%%%%%%%%%%%%%%%%%%%%

\end{document}
