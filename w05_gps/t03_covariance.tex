
\pdfminorversion=4 % for acroread
\documentclass[aspectratio=169,t,xcolor={usenames,dvipsnames}]{beamer}
%\documentclass[t,handout,xcolor={usenames,dvipsnames}]{beamer}
\usepackage{../beamerstyle}
\usepackage{dsfont}
\usepackage{bm}
\usepackage[english]{babel}
\usepackage[utf8]{inputenc}
\usepackage{graphicx}
\usepackage{algorithm}
\usepackage[ruled,vlined,algo2e,linesnumbered]{algorithm2e}
%\usepackage[boxed,vlined]{algorithm2e}
\usepackage{hyperref}
\usepackage{booktabs}
\usepackage{mathtools}

\usepackage{amsmath,amssymb}
\usepackage{listings}
\lstset{frame=lines,framesep=3pt,numbers=left,numberblanklines=false,basicstyle=\ttfamily\small}

\usepackage{subfig}
\usepackage{multicol}
%\usepackage{appendixnumberbeamer}
%
\usepackage{tcolorbox}

\usepackage{pgfplots}
\usepackage{tikz}
\usetikzlibrary{trees} 
\usetikzlibrary{shapes.geometric}
\usetikzlibrary{positioning,shapes,shadows,arrows,calc,mindmap}
\usetikzlibrary{positioning,fadings,through}
\usetikzlibrary{decorations.pathreplacing}
\usetikzlibrary{intersections}
\usetikzlibrary{positioning,fit,calc,shadows,backgrounds}
\pgfdeclarelayer{background}
\pgfdeclarelayer{foreground}
\pgfsetlayers{background,main,foreground}
\tikzstyle{activity}=[rectangle, draw=black, rounded corners, text centered, text width=8em]
\tikzstyle{data}=[rectangle, draw=black, text centered, text width=8em]
\tikzstyle{myarrow}=[->, thick, draw=black]

% Define the layers to draw the diagram
\pgfdeclarelayer{background}
\pgfdeclarelayer{foreground}
\pgfsetlayers{background,main,foreground}

%\usepackage{listings}
%\lstset{numbers=left,
%  showstringspaces=false,
%  frame={tb},
%  captionpos=b,
%  lineskip=0pt,
%  basicstyle=\ttfamily,
%%  extendedchars=true,
%  stepnumber=1,
%  numberstyle=\small,
%  xleftmargin=1em,
%  breaklines
%}

 
\definecolor{blue}{RGB}{0, 74, 153}

\usetheme{Boadilla}
%\useinnertheme{rectangles}
\usecolortheme{whale}
\setbeamercolor{alerted text}{fg=blue}
\useoutertheme{infolines}
\setbeamertemplate{navigation symbols}{\vspace{-5pt}} % to lower the logo
\setbeamercolor{date in head/foot}{bg=blue} % blue
\setbeamercolor{date in head/foot}{fg=white}
\setbeamercolor{author in head/foot}{bg=blue} %blue
\setbeamercolor{title in head/foot}{bg=blue} % blue
\setbeamercolor{title}{fg=white, bg=blue}
\setbeamercolor{block title}{fg=white,bg=blue}
\setbeamercolor{block body}{bg=blue!10}
\setbeamercolor{frametitle}{fg=white, bg=blue}
\setbeamercovered{invisible}

\makeatletter
\setbeamertemplate{footline}
{
  \leavevmode%
  \hbox{%
  \begin{beamercolorbox}[wd=.333333\paperwidth,ht=2.25ex,dp=1ex,center]{author in head/foot}%
    \usebeamerfont{author in head/foot}\insertshortauthor
  \end{beamercolorbox}%
  \begin{beamercolorbox}[wd=.333333\paperwidth,ht=2.25ex,dp=1ex,center]{title in head/foot}%
    \usebeamerfont{title in head/foot}\insertshorttitle
  \end{beamercolorbox}%
  \begin{beamercolorbox}[wd=.333333\paperwidth,ht=2.25ex,dp=1ex,right]{date in head/foot}%
    \usebeamerfont{date in head/foot}Week \@week, Topic \@topicnumber, Slide \insertframenumber{}\hspace*{2em}
%    \insertframenumber\hspace*{2ex} 
  \end{beamercolorbox}}%
  \vskip0pt%
}

\newcommand{\@week}{0}
\newcommand{\@topicnumber}{0}
\newcommand{\week}[1]{\renewcommand{\@week}{#1}}
\newcommand{\topicnumber}[1]{\renewcommand{\@topicnumber}{#1}}

\makeatother

%\pgfdeclareimage[height=1.2cm]{automl}{images/logos/automl.png}
%\pgfdeclareimage[height=1.2cm]{freiburg}{images/logos/freiburg}

%\logo{\pgfuseimage{freiburg}}

\input{../latex_main/macros}





\newcommand{\lz}{\vspace{0.5cm}}
\newcommand{\thetab}{\bm{\weights}}
\newcommand{\zero}{\mathbf{0}}
\newcommand{\Xmat}{\mathbf{X}}
\newcommand{\ydat}{\mathbf{y}}
\newcommand{\id}{\boldsymbol{I}}
\newcommand{\Amat}{\mathbf{A}}
\newcommand{\Xspace}{\mathcal{X}}                                           
\newcommand{\Yspace}{\mathcal{Y}}
\newcommand{\ls}{\ell}
\newcommand{\natnum}{\mathbb{N}}
\newcommand{\intnum}{\mathbb{Z}}

\usepackage{fontawesome}
\usepackage{dirtytalk}
\usepackage{csquotes}


\title[AutoML: GPs]{AutoML: Gaussian Processes} % week title
\subtitle{Covariance Functions for GPs} % video title
\author[Marius Lindauer]{\underline{Bernd Bischl} \and Frank Hutter \and Lars Kotthoff\newline \and Marius Lindauer \and Joaquin Vanschoren}
\institute{}
\date{}
\week{5}
\topicnumber{3}



% \AtBeginSection[] % Do nothing for \section*
% {
%   \begin{frame}{Outline}
%     \bigskip
%     \vfill
%     \tableofcontents[currentsection]
%   \end{frame}
% }

\begin{document}
	
	\maketitle
	

%----------------------------------------------------------------------
%----------------------------------------------------------------------
\begin{frame}[c,allowframebreaks]{Covariance function of a GP}

    The marginalization property of the Gaussian process implies that for any finite set of input values, the corresponding vector of function values is Gaussian:

  $$
    \bm{f} = \left[f\left(\xI{1}\right),\dots, f\left(\xI{n}\right)\right] \sim \mathcal{N}\left(\bm{m}, \bm{K}\right).
  $$ 


\begin{itemize}

  \item The covariance matrix $\bm{K}$ is constructed according to the chosen inputs $\left\{\xI{1},\dots,\xI{n}\right\}$.
  \item Each entry $\bm{K}_{ij}$ is computed by $k\left(\xI{i}, \xI{j}\right)$.
  \item Technically, to be a valid covariance matrix, $\bm{K}$ needs to be positive semi-definite for \textbf{every} choice of inputs $\left\{\xI{1},\dots,\xI{n}\right\}$.
  \item A function $k(\cdot,\cdot)$ that satisfies this condition is called \textbf{positive definite}.
  
\end{itemize}
%%%%%%%%%%%%%%%%%%%%%%%%%%%%%%%%%%%%%%%%%%%%%%%%%%%%%%%%%%%%%%%%%%%%%%%%%%%%%%%%%%%%
\framebreak

\begin{itemize}
\item Recall that the purpose of the covariance function is to control to which degree the following condition is fulfilled:
\end{itemize}
\lz
\begin{displayquote}
If $\xI{i}$ and $\xI{j}$ are close in the $\Xspace$-space, their function values $f(\xI{i})$ and $f(\xI{j})$ should be close in $\Yspace$-space.
\end{displayquote}

\lz
\lz

\begin{itemize}
\item[\faLightbulbO] Closeness of $\xI{i}$ and $\xI{j}$ in the input space $\Xspace$ is measured by $\bm{d} =\xI{i}-\xI{j}$. 

% \lz

% \item[\faLightbulbO] $\bm{K}_{ij}$ is the covariance of $f(\xI{i})$ and $f(\xI{j})$, and \textbf{stationary} covariance functions are those in which the following holds:$$k(\xI{i}, \xI{j}) = k(\bm{d})$$ 
\end{itemize}

\end{frame}
%%%%%%%%%%%%%%%%%%%%%%%%%%%%%%%%%%%%%%%%%%%%%%%%%%%%%%%%%%%%%%%%%%%%%%%%%%%%%%%%%%%%

\begin{frame}[c,allowframebreaks]{Covariance function of a GP: Example}


\begin{itemize}
\item Let $f(\x)$ be a GP with $k(\x, \x^\prime) = \exp(-\frac{1}{2}\|\bm{d}\|^2)$ where $\bm{d} = \x - \x^\prime$.

\vspace{.3cm}

\item Consider two points $\xI{1} = 3$ and $\xI{2} = 2.5$. To investigate how correlated their function values are, compute their correlation!
\end{itemize}

\vspace{.3cm}

\begin{figure}
\includegraphics[width=0.3\textwidth]{figure_man/covariance2point/example_covariance_1.png} ~
\includegraphics[width=0.3\textwidth]{figure_man/covariance2point/example_function_1-1.png}
\end{figure}

%%%%%%%%%%%%%%%%%%%%%%%%%%%%%%%%%%%%%%%%%%%%%%%%%%%%%%%%%%%%%%%%%%%%%%%%%%%%%%%%%%%%
\framebreak


\begin{itemize}
  \item Assume that we observe a value of $\yI{1} = - 0.8$. Under the said assumption for the Gaussian process, the value of $\yI{2}$ should be close to $\yI{1}$.
\end{itemize}
\vspace{1cm}

\begin{figure}
\includegraphics[width=0.3\textwidth]{figure_man/covariance2point/example_covariance_1.png} ~
\includegraphics[width=0.3\textwidth]{figure_man/covariance2point/example_function_1-2.png}
\end{figure}


%%%%%%%%%%%%%%%%%%%%%%%%%%%%%%%%%%%%%%%%%%%%%%%%%%%%%%%%%%%%%%%%%%%%%%%%%%%%%%%%%%%%
\framebreak

\begin{itemize}
\item Now, let us take a new point $\xI{3}$ which is not too close to $\xI{1}$.
  
\vspace{.3cm}
\item Their function values should not be so correlated. That is, $\yI{1}$ and $\yI{3}$ are probably far away from each other.
\end{itemize}

\vspace{.3cm}

\begin{figure}
\includegraphics[width=0.3\textwidth]{figure_man/covariance2point/example_covariance_2.png} ~      
\includegraphics[width=0.3\textwidth]{figure_man/covariance2point/example_function_2-1.png}
\end{figure}

\end{frame}
%%%%%%%%%%%%%%%%%%%%%%%%%%%%%%%%%%%%%%%%%%%%%%%%%%%%%%%%%%%%%%%%%%%%%%%%%%%%%%%%%%%%
%%%%%%%%%%%%%%%%%%%%%%%%%%%%%%%%%%%%%%%%%%%%%%%%%%%%%%%%%%%%%%%%%%%%%%%%%%%%%%%%%%%%

\begin{frame}[c]{Covariance Functions}

Three types of properties are commonly used in covariance functions:

\lz

\begin{itemize}
\item $k$ is \textbf{stationary} if it depends only on $\bm{d} =\x -\x^\prime$ and is denoted by $k(\bm{d})$.

\item $k$ is \textbf{isotropic} if it depends only on $r = \|\x - \x^\prime\|$ and is denoted by $k(r)$.

\item $k$ is a \textbf{dot product} if it depends only on $\x^T \x^\prime$.
\end{itemize}

\lz
\lz

\begin{itemize}
\item[\faLightbulbO] Isotropy implies stationarity.
\item[\faLightbulbO] Isotropic functions are rotationally invariant.
\item[\faLightbulbO] Stationary functions are translationally invariant:
\vspace{-.3cm}

$$k(\x,\x + \bm{d}) = k(\bm{0},\bm{d})=k(\bm{d})$$
\end{itemize}

\end{frame}

%%%%%%%%%%%%%%%%%%%%%%%%%%%%%%%%%%%%%%%%%%%%%%%%%%%%%%%%%%%%%%%%%%%%%%%%%%%%%%%%%%%%
\begin{frame}[c,allowframebreaks]{Commonly Used Covariance Functions}


\begin{table}[]
\centering
\begin{tabular}{|c|c|}
\hline
Name & $k(\x, \x^\prime)$\\
\hline
constant & $\variance_0$ \\ [1em]
linear & $\variance_0 + \x^T\x^\prime$ \\ [1em]
polynomial & $(\variance_0 + \x^T\x^\prime)^p$ \\ [1em]
squared exponential & $\exp(- \frac{\|\x - \x^\prime\|^2}{2\ls^2})$ \\ [1em]
Matérn & \begin{footnotesize} $\frac{1}{2^\nu \Gamma(\nu)}\biggl(\frac{\sqrt{2 \nu}}{\ls}\|\x - \x^\prime\|\biggr)^{\nu} K_\nu\biggl(\frac{\sqrt{2 \nu}}{\ls}\|\x - \x^\prime\|\biggr)$\end{footnotesize}  \\ [1em]
exponential & $\exp\left(- \frac{\|\x - \x^\prime\|}{\ls}\right)$ \\ [1em]
\hline
\end{tabular}
\end{table}
\begin{footnotesize}
\centering
$K_\nu(\cdot)$ is the modified Bessel function of the second kind.
\end{footnotesize}

%%%%%%%%%%%%%%%%%%%%%%%%%%%%%%%%%%%%%%%%%%%%%%%%%%%%%%%%%%%%%%%%%%%%%%%%%%%%%%%%%%%%
\framebreak

\begin{itemize}
\item[\faLightbulbO] Some random functions drawn from Gaussian processes with a Squared Exponential Kernel (left), Polynomial Kernel (middle), and a Matérn Kernel (right, $\ls = 1$). 
\item[\faLightbulbO] The length-scale hyperparameter determines the ``wiggliness'' of the function.
\item[\faLightbulbO] For Matérn, the $\nu$ parameter determines how differentiable the process is.
\end{itemize}

\begin{figure}
\includegraphics[width=0.6\textwidth]{figure_man/covariance.png}
\end{figure}


\end{frame}
%-----------------------------------------------------------------------

\begin{frame}[c]{Squared Exponential Covariance Function}

The squared exponential function is one of the most commonly used covariance functions.
$$
k(\x, \x^\prime) = \exp\biggl(- \frac{\|\x - \x^\prime\|^2}{2\ls^2}\biggr).
$$

\textbf{Properties}:
\begin{itemize}
\item[\faLightbulbO] It depends merely on the distance $r = \|\x - \x^\prime\|$ $\to$ isotropic and stationary.\lz
\item[\faLightbulbO] Infinitely differentiable $\to$ the corresponding GP is too smooth.\lz
\item[\faLightbulbO] It utilizes strong smoothness assumptions $\to$ unrealistic for modeling most of the physical processes.

\end{itemize}

\end{frame}

%%%%%%%%%%%%%%%%%%%%%%%%%%%%%%%%%%%%%%%%%%%%%%%%%%%%%%%%%%%%%%%%%%%%%%%%%%%%%%%%%%%%
\begin{frame}[c,allowframebreaks]{Characteristic Length-Scale}
    $$k(\x, \x^\prime) = \exp\left(-\frac{1}{2\ls^2}\|\x - \x^\prime\|^2\right)$$
% \begin{figure}
	% \includegraphics[width = .8\textwidth]{figure/lengthscale-1.pdf}
% \end{figure}
$\ls$ is called \textbf{characteristic length-scale}. Loosely speaking, the characteristic length-scale describes how far you need to move in input space for the function values to become uncorrelated. Higher $\ls$ induces smoother functions, lower $\ls$ induces more wiggly functions.

\begin{figure}
\includegraphics[width=0.6\textwidth]{figure_man/gp-sqexp-l-1.pdf}
\end{figure}

% \begin{itemize}
    % \item Left: for higher $\ls$ the correlation between function values (for unchanged distance of input points) is also higher
    % \item Right plot: a higher $\ls$ induces a smoother function 
% \end{itemize}

\framebreak
%%%%%%%%%%%%%%%%%%%%%%%%%%%%%%%%%%%%%%%%%%%%%%%%%%%%%%%%%%%%%%%%%%%%%%%%%%%%%%%%%%%%

For more than $p = 2$ dimensions, the squared exponential can be parameterized as follows:

$$
k(\xI{i}, \xI{j}) = \variance_f \exp\,\biggl(- \frac{1}{2}\left(\xI{i} - \xI{j}\right)^\top\bm{M}\left(\xI{i} - \xI{j}\right)\biggr)
$$

\vspace{7mm}

Possible choices for the matrix $\bm{M}$ include

$$
\bm{M}_1 = \ls^{-2}\id \qquad \bm{M}_2 = \text{diag}(\bm{\ls})^{-2} \qquad \bm{M}_3 = \Gamma \Gamma^\top + \text{diag}(\bm{\ls})^{-2}
$$

where $\bm{\ls}$ is a $p$-vector of positive values and $\Gamma$ is a $p \times k$ matrix. 

\lz 
\lz

Here again, $\bm{\ls} = (\ls_1,\dots, \ls_p)$ are characteristic length-scales for each dimension. 


\framebreak
%%%%%%%%%%%%%%%%%%%%%%%%%%%%%%%%%%%%%%%%%%%%%%%%%%%%%%%%%%%%%%%%%%%%%%%%%%%%%%%%%%%%

What is the benefit of having an individual hyperparameter $\ls_i$ for each dimension?

\vspace{4mm}

\begin{itemize} 
\item The $\ls_1,\dots, \ls_p$ hyperparameters play the role of \textbf{characteristic length-scales}.
\vspace{2mm}
\item Loosely speaking, $\ls_i$ describes how far you need to move along axis $i$ in input space for the function values to be uncorrelated.
\vspace{2mm}
\item Such a covariance function implements \textbf{automatic relevance determination} (ARD), since the inverse of the length-scale $\ls_i$ determines the relevancy of input feature $i$ to the regression.
\vspace{2mm}
\item If $\ls_i$ is very large, the covariance will become almost independent of that input, effectively removing it from inference.
\vspace{2mm}
\item If the features are on different scales, the data can be automatically \textbf{rescaled} by estimating $\ls_1,\dots, \ls_p$. 

\end{itemize}



\framebreak
%%%%%%%%%%%%%%%%%%%%%%%%%%%%%%%%%%%%%%%%%%%%%%%%%%%%%%%%%%%%%%%%%%%%%%%%%%%%%%%%%%%%


\begin{figure}
	\includegraphics[width = .8\textwidth]{figure_man/covariance2D.png}
\end{figure}

\vspace{3mm}
%\begin{footnotesize}
For the first plot, we have chosen $\bm{M} = \id$: the function varies the same in all directions. The second plot is for $\bm{M} = \text{diag}(\bm{\ls})^{-2}$ and $\bm{\ls} = \left(1, 3 \right)$: The function varies less rapidly as a function of $x_2$ than $x_1$ as the length-scale for $x_1$ is less. In the third plot $\bm{M} = \Gamma \Gamma^T + \text{diag}(\bm{\ls})^{-2}$ for $\Gamma = (1, -1)^\top$ and $\bm{\ls} = (6, 6)^\top$. Here $\Gamma$ gives the direction of the most rapid variation. \lit{\href{http://www.gaussianprocess.org/gpml/chapters/RW.pdf\#page=125&zoom=auto,-17,731}{Rasmussen and Williams. 2006 }}
%\end{footnotesize}


\end{frame}

\end{document}
