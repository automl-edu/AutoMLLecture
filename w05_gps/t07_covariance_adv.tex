
\pdfminorversion=4 % for acroread
\documentclass[aspectratio=169,t,xcolor={usenames,dvipsnames}]{beamer}
%\documentclass[t,handout,xcolor={usenames,dvipsnames}]{beamer}
\usepackage{../beamerstyle}
\usepackage{dsfont}
\usepackage{bm}
\usepackage[english]{babel}
\usepackage[utf8]{inputenc}
\usepackage{graphicx}
\usepackage{algorithm}
\usepackage[ruled,vlined,algo2e,linesnumbered]{algorithm2e}
%\usepackage[boxed,vlined]{algorithm2e}
\usepackage{hyperref}
\usepackage{booktabs}
\usepackage{mathtools}

\usepackage{amsmath,amssymb}
\usepackage{listings}
\lstset{frame=lines,framesep=3pt,numbers=left,numberblanklines=false,basicstyle=\ttfamily\small}

\usepackage{subfig}
\usepackage{multicol}
%\usepackage{appendixnumberbeamer}
%
\usepackage{tcolorbox}

\usepackage{pgfplots}
\usepackage{tikz}
\usetikzlibrary{trees} 
\usetikzlibrary{shapes.geometric}
\usetikzlibrary{positioning,shapes,shadows,arrows,calc,mindmap}
\usetikzlibrary{positioning,fadings,through}
\usetikzlibrary{decorations.pathreplacing}
\usetikzlibrary{intersections}
\usetikzlibrary{positioning,fit,calc,shadows,backgrounds}
\pgfdeclarelayer{background}
\pgfdeclarelayer{foreground}
\pgfsetlayers{background,main,foreground}
\tikzstyle{activity}=[rectangle, draw=black, rounded corners, text centered, text width=8em]
\tikzstyle{data}=[rectangle, draw=black, text centered, text width=8em]
\tikzstyle{myarrow}=[->, thick, draw=black]

% Define the layers to draw the diagram
\pgfdeclarelayer{background}
\pgfdeclarelayer{foreground}
\pgfsetlayers{background,main,foreground}

%\usepackage{listings}
%\lstset{numbers=left,
%  showstringspaces=false,
%  frame={tb},
%  captionpos=b,
%  lineskip=0pt,
%  basicstyle=\ttfamily,
%%  extendedchars=true,
%  stepnumber=1,
%  numberstyle=\small,
%  xleftmargin=1em,
%  breaklines
%}

 
\definecolor{blue}{RGB}{0, 74, 153}

\usetheme{Boadilla}
%\useinnertheme{rectangles}
\usecolortheme{whale}
\setbeamercolor{alerted text}{fg=blue}
\useoutertheme{infolines}
\setbeamertemplate{navigation symbols}{\vspace{-5pt}} % to lower the logo
\setbeamercolor{date in head/foot}{bg=blue} % blue
\setbeamercolor{date in head/foot}{fg=white}
\setbeamercolor{author in head/foot}{bg=blue} %blue
\setbeamercolor{title in head/foot}{bg=blue} % blue
\setbeamercolor{title}{fg=white, bg=blue}
\setbeamercolor{block title}{fg=white,bg=blue}
\setbeamercolor{block body}{bg=blue!10}
\setbeamercolor{frametitle}{fg=white, bg=blue}
\setbeamercovered{invisible}

\makeatletter
\setbeamertemplate{footline}
{
  \leavevmode%
  \hbox{%
  \begin{beamercolorbox}[wd=.333333\paperwidth,ht=2.25ex,dp=1ex,center]{author in head/foot}%
    \usebeamerfont{author in head/foot}\insertshortauthor
  \end{beamercolorbox}%
  \begin{beamercolorbox}[wd=.333333\paperwidth,ht=2.25ex,dp=1ex,center]{title in head/foot}%
    \usebeamerfont{title in head/foot}\insertshorttitle
  \end{beamercolorbox}%
  \begin{beamercolorbox}[wd=.333333\paperwidth,ht=2.25ex,dp=1ex,right]{date in head/foot}%
    \usebeamerfont{date in head/foot}Week \@week, Topic \@topicnumber, Slide \insertframenumber{}\hspace*{2em}
%    \insertframenumber\hspace*{2ex} 
  \end{beamercolorbox}}%
  \vskip0pt%
}

\newcommand{\@week}{0}
\newcommand{\@topicnumber}{0}
\newcommand{\week}[1]{\renewcommand{\@week}{#1}}
\newcommand{\topicnumber}[1]{\renewcommand{\@topicnumber}{#1}}

\makeatother

%\pgfdeclareimage[height=1.2cm]{automl}{images/logos/automl.png}
%\pgfdeclareimage[height=1.2cm]{freiburg}{images/logos/freiburg}

%\logo{\pgfuseimage{freiburg}}

\input{../latex_main/macros}





\newcommand{\lz}{\vspace{0.5cm}}
\newcommand{\thetab}{\bm{\weights}}
\newcommand{\zero}{\mathbf{0}}
\newcommand{\Xmat}{\mathbf{X}}
\newcommand{\ydat}{\mathbf{y}}
\newcommand{\id}{\boldsymbol{I}}
\newcommand{\Amat}{\mathbf{A}}
\newcommand{\Xspace}{\mathcal{X}}                                           
\newcommand{\Yspace}{\mathcal{Y}}
\newcommand{\ls}{\ell}
\newcommand{\natnum}{\mathbb{N}}
\newcommand{\intnum}{\mathbb{Z}}

\usepackage{fontawesome}
\usepackage{dirtytalk}
\usepackage{csquotes}

%\begin{frame}[c]{}
%\centering
%\huge
%\textbf{}
%\end{frame}


%\item[\faLightbulbO]
\title[AutoML: GPs]{AutoML: Gaussian Processes} % week title
\subtitle{Covariance Functions for GPs - Advanced} % video title
\author[Marius Lindauer]{\underline{Bernd Bischl} \and Frank Hutter \and Lars Kotthoff\newline \and Marius Lindauer \and Joaquin Vanschoren}
\institute{}
\date{}
\week{5}
\topicnumber{7}



\begin{document}
\maketitle
	
%%%%%%%%%%%%%%%%%%%%%%%%%%%%%%%%%%%%%%%%%%%%%%%%%%%%%%%%%%%%%%%%%%%%%%%%%%%%%%%%%%%%
\begin{frame}[c,allowframebreaks]{MS-Continuity and Differentiability}

We wish to describe a Gaussian process in terms of its smoothness. There are several notions of continuity for random variables. One is the continuity/differentiability in mean square (MS).

\begin{block}{Definition}
A Gaussian process $f(\x)$ is said to be \textbf{MS continuous} at $\x_*$, if
\vspace{-3mm}
$$\expectation[|f(\xI{k}) - f(\x_*)|^2] \overset{k \to \infty}{\longrightarrow} 0 \text{ for all converging sequences } \xI{k} \overset{k \to \infty}{\longrightarrow} \x_*.$$

\vspace{5mm}
A Gaussian process $f(\x)$ is said to be \textbf{MS differentiable} along the $i$ direction, if the following limit exists, with $\bm{e}_i = (0,\dots,0,1,0,\dots,0)^\top$ being the unit vector along the $i$-th axis. $$\lim_{h\to 0}\expectation[|\frac{f(\x + h\,\bm{e}_i) - f(\x)}{h}|]$$ 
\end{block}


 

\framebreak
%%%%%%%%%%%%%%%%%%%%%%%%%%%%%%%%%%%%%%%%%%%%%%%%%%%%%%%%%%%%%%%%%%%%%%%%%%%%%%%%%%%%


\begin{itemize}
\item The MS continuity/differentiability do not necessarily lead to the continuity/differentiability of sampled functions!
\vspace{.5cm}

\item The MS continuity/differentiability of a Gaussian process can be derived from the smoothness properties of the kernel.
\vspace{.5cm}

\item The GP is continuous in MS iff the covariance function $k(\x, \x^\prime)$ is continuous.
\vspace{.5cm}

\item The MS derivative of a Gaussian process exists iff the second derivative $\frac{\partial^{2} k(\x, \x^\prime)}{\partial \x\partial \x^\prime}$ exists.

\end{itemize}

\end{frame}
%%%%%%%%%%%%%%%%%%%%%%%%%%%%%%%%%%%%%%%%%%%%%%%%%%%%%%%%%%%%%%%%%%%%%%%%%%%%%%%%%%%%
\begin{frame}[c]{Squared Exponential Covariance Function}

The squared exponential function is one of the most commonly used covariance functions.
$$
k(\x, \x^\prime) = \exp\biggl(- \frac{\|\x - \x^\prime\|^2}{2\ls^2}\biggr).
$$

\textbf{Properties}:
\begin{itemize}
\item[\faLightbulbO] It depends merely on the distance $r = \|\x - \x^\prime\|$ $\to$ isotropic and stationary.\lz
\item[\faLightbulbO] Infinitely differentiable $\to$ the corresponding GP is too smooth.\lz
\item[\faLightbulbO] It utilizes strong smoothness assumptions $\to$ unrealistic for modeling most of the physical processes.
\end{itemize}

\end{frame}
%%%%%%%%%%%%%%%%%%%%%%%%%%%%%%%%%%%%%%%%%%%%%%%%%%%%%%%%%%%%%%%%%%%%%%%%%%%%%%%%%%%%
\begin{frame}[c,allowframebreaks]{Upcrossing Rate and Characteristic Length-Scale}

\begin{itemize}
\item Another way to describe a Gaussian process is the expected number of up-crossings at level-0 on the unit interval, which we denote by $N_0$.
\vspace{3mm}
\begin{figure}
	\includegraphics[width = 0.45\textwidth]{figure/upcrossings-1.pdf}
\end{figure}

\item For an isotropic covariance function $k(r)$, the expected number of up-crossings can be calculated explicitly:
$$
\expectation[N_0] = \frac{1}{2\pi} \sqrt{\frac{- k^{\prime\prime}(0)}{k(0)}}.
$$

\end{itemize}


\framebreak
%%%%%%%%%%%%%%%%%%%%%%%%%%%%%%%%%%%%%%%%%%%%%%%%%%%%%%%%%%%%%%%%%%%%%%%%%%%%%%%%%%%%

Example (squared exponential):
\vspace{-3mm}
\begin{eqnarray*}
k(r) &=& \exp\biggl(-\frac{r^2}{2\ls^2}\biggr)\\
k^\prime(r) &=& - k(r) \cdot \frac{r}{\ls^2} \\
k^{\prime\prime}(r) &=&  k(r) \cdot \frac{r^2}{\ls^4} - k(r) \cdot \frac{1}{\ls^2}
\end{eqnarray*}

\vspace{5mm}
The expected number of upcrossings at level-0 is

$$
\E[N_0] = \frac{1}{2\pi} \sqrt{\frac{- k^{\prime\prime}(0)}{k(0)}} = \frac{1}{2\pi} \sqrt{\frac{1}{\ls^2}} = (2\pi \ls)^{-1}.
$$

%%%%%%%%%%%%%%%%%%%%%%%%%%%%%%%%%%%%%%%%%%%%%%%%%%%%%%%%%%%%%%%%%%%%%%%%%%%%%%%%%%%%

$\ls$ is called \textbf{characteristic length-scale}. Loosely speaking, the characteristic length-scale describes how far you need to move in input space for the function values to become uncorrelated. 

\begin{itemize}
\item[\faLightbulbO] Left plot: for higher $\ls$ the correlation between function values (for unchanged distance of input points) is also higher
\item[\faLightbulbO] Right plot: a higher $\ls$ induces a smoother function and thus fewer level-0 upcrossings
\end{itemize}

\vspace{3mm}
\begin{figure}
	\includegraphics[width = .8\textwidth]{figure/lengthscale-1.pdf}
\end{figure}

\framebreak
%%%%%%%%%%%%%%%%%%%%%%%%%%%%%%%%%%%%%%%%%%%%%%%%%%%%%%%%%%%%%%%%%%%%%%%%%%%%%%%%%%%%

For more than $p = 2$ dimensions, the squared exponential can be parameterized as follows:

$$
k(\xI{i}, \xI{j}) = \variance_f \exp\,\biggl(- \frac{1}{2}\left(\xI{i} - \xI{j}\right)^\top\bm{M}\left(\xI{i} - \xI{j}\right)\biggr)
$$

\vspace{7mm}

Possible choices for the matrix $\bm{M}$ include

$$
\bm{M}_1 = \ls^{-2}\id \qquad \bm{M}_2 = \text{diag}(\bm{\ls})^{-2} \qquad \bm{M}_3 = \Gamma \Gamma^\top + \text{diag}(\bm{\ls})^{-2}
$$

where $\bm{\ls}$ is a $p$-vector of positive values and $\Gamma$ is a $p \times k$ matrix. 

\lz 
\lz

Here again, $\bm{\ls} = (\ls_1,\dots, \ls_p)$ are characteristic length-scales for each dimension. 


\framebreak
%%%%%%%%%%%%%%%%%%%%%%%%%%%%%%%%%%%%%%%%%%%%%%%%%%%%%%%%%%%%%%%%%%%%%%%%%%%%%%%%%%%%

What is the benefit of learning an individual hyperparameter $\ls_i$ for each dimension?

\vspace{4mm}

\begin{itemize} 
\item The $\ls_1,\dots, \ls_p$ hyperparameters play the role of \textbf{characteristic length-scales}.
\vspace{2mm}
\item Losely speaking, $\ls_i$ describes how far you need to move along axis $i$ in input space for the function values to be uncorrelated.
\vspace{2mm}
\item Such a covariance function implements \textbf{automatic relevance determination} (ARD), since the inverse of the length-scale $\ls_i$ determines the relevancy of input feature $i$ to the regression.
\vspace{2mm}
\item If $\ls_i$ is very large, the covariance will become almost independent of that input, effectively removing it from inference.
\vspace{2mm}
\item If the features are on different scales, the data can be automatically \textbf{rescaled} by estimating $\ls_1,\dots, \ls_p$ 

\end{itemize}



\framebreak
%%%%%%%%%%%%%%%%%%%%%%%%%%%%%%%%%%%%%%%%%%%%%%%%%%%%%%%%%%%%%%%%%%%%%%%%%%%%%%%%%%%%


\begin{figure}
	\includegraphics[width = .8\textwidth]{figure_man/covariance2D.png}
\end{figure}

\vspace{3mm}
%\begin{footnotesize}
For the first plot, we have chosen $\bm{M} = \id$: the function varies the same in all directions. The second plot is for $\bm{M} = \text{diag}(\bm{\ls})^{-2}$ and $\bm{\ls} = \left(1, 3 \right)$: The function varies less rapidly as a function of $x_2$ than $x_1$ as the length-scale for $x_1$ is less. In the third plot $\bm{M} = \Gamma \Gamma^T + \text{diag}(\bm{\ls})^{-2}$ for $\Gamma = (1, -1)^\top$ and $\bm{\ls} = (6, 6)^\top$. Here $\Gamma$ gives the direction of the most rapid variation. \lit{\href{http://www.gaussianprocess.org/gpml/chapters/RW.pdf\#page=125&zoom=auto,-17,731}{Rasmussen and Williams. 2006 }}%(Image from Rasmussen \& Williams, 2006)
%\end{footnotesize}


\end{frame}

\end{document}
