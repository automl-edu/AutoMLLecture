
\pdfminorversion=4 % for acroread
\documentclass[aspectratio=169,t,xcolor={usenames,dvipsnames}]{beamer}
%\documentclass[t,handout,xcolor={usenames,dvipsnames}]{beamer}
\usepackage{../beamerstyle}
\usepackage{dsfont}
\usepackage{bm}
\usepackage[english]{babel}
\usepackage[utf8]{inputenc}
\usepackage{graphicx}
\usepackage{algorithm}
\usepackage[ruled,vlined,algo2e,linesnumbered]{algorithm2e}
%\usepackage[boxed,vlined]{algorithm2e}
\usepackage{hyperref}
\usepackage{booktabs}
\usepackage{mathtools}

\usepackage{amsmath,amssymb}
\usepackage{listings}
\lstset{frame=lines,framesep=3pt,numbers=left,numberblanklines=false,basicstyle=\ttfamily\small}

\usepackage{subfig}
\usepackage{multicol}
%\usepackage{appendixnumberbeamer}
%
\usepackage{tcolorbox}

\usepackage{pgfplots}
\usepackage{tikz}
\usetikzlibrary{trees} 
\usetikzlibrary{shapes.geometric}
\usetikzlibrary{positioning,shapes,shadows,arrows,calc,mindmap}
\usetikzlibrary{positioning,fadings,through}
\usetikzlibrary{decorations.pathreplacing}
\usetikzlibrary{intersections}
\usetikzlibrary{positioning,fit,calc,shadows,backgrounds}
\pgfdeclarelayer{background}
\pgfdeclarelayer{foreground}
\pgfsetlayers{background,main,foreground}
\tikzstyle{activity}=[rectangle, draw=black, rounded corners, text centered, text width=8em]
\tikzstyle{data}=[rectangle, draw=black, text centered, text width=8em]
\tikzstyle{myarrow}=[->, thick, draw=black]

% Define the layers to draw the diagram
\pgfdeclarelayer{background}
\pgfdeclarelayer{foreground}
\pgfsetlayers{background,main,foreground}

%\usepackage{listings}
%\lstset{numbers=left,
%  showstringspaces=false,
%  frame={tb},
%  captionpos=b,
%  lineskip=0pt,
%  basicstyle=\ttfamily,
%%  extendedchars=true,
%  stepnumber=1,
%  numberstyle=\small,
%  xleftmargin=1em,
%  breaklines
%}

 
\definecolor{blue}{RGB}{0, 74, 153}

\usetheme{Boadilla}
%\useinnertheme{rectangles}
\usecolortheme{whale}
\setbeamercolor{alerted text}{fg=blue}
\useoutertheme{infolines}
\setbeamertemplate{navigation symbols}{\vspace{-5pt}} % to lower the logo
\setbeamercolor{date in head/foot}{bg=blue} % blue
\setbeamercolor{date in head/foot}{fg=white}
\setbeamercolor{author in head/foot}{bg=blue} %blue
\setbeamercolor{title in head/foot}{bg=blue} % blue
\setbeamercolor{title}{fg=white, bg=blue}
\setbeamercolor{block title}{fg=white,bg=blue}
\setbeamercolor{block body}{bg=blue!10}
\setbeamercolor{frametitle}{fg=white, bg=blue}
\setbeamercovered{invisible}

\makeatletter
\setbeamertemplate{footline}
{
  \leavevmode%
  \hbox{%
  \begin{beamercolorbox}[wd=.333333\paperwidth,ht=2.25ex,dp=1ex,center]{author in head/foot}%
    \usebeamerfont{author in head/foot}\insertshortauthor
  \end{beamercolorbox}%
  \begin{beamercolorbox}[wd=.333333\paperwidth,ht=2.25ex,dp=1ex,center]{title in head/foot}%
    \usebeamerfont{title in head/foot}\insertshorttitle
  \end{beamercolorbox}%
  \begin{beamercolorbox}[wd=.333333\paperwidth,ht=2.25ex,dp=1ex,right]{date in head/foot}%
    \usebeamerfont{date in head/foot}Week \@week, Topic \@topicnumber, Slide \insertframenumber{}\hspace*{2em}
%    \insertframenumber\hspace*{2ex} 
  \end{beamercolorbox}}%
  \vskip0pt%
}

\newcommand{\@week}{0}
\newcommand{\@topicnumber}{0}
\newcommand{\week}[1]{\renewcommand{\@week}{#1}}
\newcommand{\topicnumber}[1]{\renewcommand{\@topicnumber}{#1}}

\makeatother

%\pgfdeclareimage[height=1.2cm]{automl}{images/logos/automl.png}
%\pgfdeclareimage[height=1.2cm]{freiburg}{images/logos/freiburg}

%\logo{\pgfuseimage{freiburg}}

\input{../latex_main/macros}





\newcommand{\lz}{\vspace{0.5cm}}
\newcommand{\thetab}{\bm{\weights}}
\newcommand{\zero}{\mathbf{0}}
\newcommand{\Xmat}{\mathbf{X}}
\newcommand{\Kmat}{\mathbf{K}}
\newcommand{\ydat}{\mathbf{y}}
\newcommand{\id}{\boldsymbol{I}}
\newcommand{\Amat}{\mathbf{A}}
\newcommand{\Xspace}{\mathcal{X}}                                           
\newcommand{\Yspace}{\mathcal{Y}}
\newcommand{\ls}{\ell}
\newcommand{\natnum}{\mathbb{N}}
\newcommand{\intnum}{\mathbb{Z}}
\newcommand{\order}{\mathcal{O}} 

\usepackage{fontawesome}
\usepackage{dirtytalk}
\usepackage{csquotes}

%\begin{frame}[c]{}
%\centering
%\huge
%\textbf{}
%\end{frame}


%\item[\faLightbulbO]
\title[AutoML: GPs]{AutoML: Gaussian Processes} % week title
\subtitle{Gaussian Proccesses: Additional Material} % video title
\author[Marius Lindauer]{\underline{Bernd Bischl} \and Frank Hutter \and Lars Kotthoff\newline \and Marius Lindauer \and Joaquin Vanschoren}
\institute{}
\date{}
\week{5}
\topicnumber{8}



\begin{document}
\maketitle
%%%%%%%%%%%%%%%%%%%%%%%%%%%%%%%%%%%%%%%%%%%%%%%%%%%%%%%%%%%%%%%%%%%%%%%%%%%%%%%%%%%%

\begin{frame}[c]{Notation}

In this part,
\vspace{.3cm}
\begin{itemize}
\item $(\x_*, y_*)$ denotes a single test observation, excluded from the training data.
\vspace{.7cm}
\item $\Xmat_* \in \realnum^{n_* \times p}$ denotes a set of $n_*$ test observations. 
\vspace{.7cm}
\item $\ydat_* \in \realnum^{n_* \times p}$ denotes the corresponding outcomes, excluded from the training data.
\end{itemize}

\end{frame}

%%%%%%%%%%%%%%%%%%%%%%%%%%%%%%%%%%%%%%%%%%%%%%%%%%%%%%%%%%%%%%%%%%%%%%%%%%%%%%%%%%%%

\begin{frame}[c]{}
\centering
\huge
\textbf{Noisy Gaussian Processes}
\end{frame}

%%%%%%%%%%%%%%%%%%%%%%%%%%%%%%%%%%%%%%%%%%%%%%%%%%%%%%%%%%%%%%%%%%%%%%%%%%%%%%%%%%%%

\begin{frame}[c]{Noisy Gaussian Processes}

\begin{itemize}
\item In the previous slides, we implicitly assumed that we access the true function values $f(\x)$. However, in many practical cases, we only have a noisy version of the values:
$$y = f(\x) + \epsilon.$$ 

\item By assuming an additive i.i.d. Gaussian noise, the covariance function becomes:
$$cov\,(\yI{i}, \yI{j})=k\left(\xI{i}, \xI{j}\right) + \variance_n \delta_{ij} \text{, where } \delta_{ij} = 1 \text{ if } i=j.$$

\item In the matrix notation, this becomes:
$$cov\,(\ydat) = \Kmat + \variance_n\id =: \Kmat_y \text{, where } \variance_n \text{ is called \textbf{nugget}.}$$

\end{itemize}
\end{frame}

%%%%%%%%%%%%%%%%%%%%%%%%%%%%%%%%%%%%%%%%%%%%%%%%%%%%%%%%%%%%%%%%%%%%%%%%%%%%%%%%%%%%

\begin{frame}[c]{GP vs. Kernelized Ridge Regression}

\begin{itemize}

\item The predictive function is then 
\vspace{-3mm}
$$\bm{f}_* | \Xmat_*, \Xmat, \ydat \sim \mathcal{N}(\bm{\bar f}_*, \,cov\,(\bm{\bar f}_*)),$$ 
\vspace{-5mm}
with $\bm{\bar f}_* = \Kmat_{*}^{T} \Kmat_y^{-1}\ydat$ and $cov\,(\bm{\bar f}_*) = \Kmat_{**}- \Kmat_{*}^\top\Kmat_y^{-1}\Kmat_*$.
\lz
\lz

\item The predicted mean value at the training points $\bm{\bar f} = \bm{K}\Kmat_y^{-1}\bm{y}$ is a \textbf{linear combination} of the $\bm{y}$ values. 
\end{itemize}

\lz
\textcolor{blue}{\faLightbulbO} Predicting the posterior mean corresponds exactly to the predictions obtained by kernelized Ridge regression. However, a GP as a Bayesian model provides us with much more information (i.e., a posterior distribution), whilst the kernelized Ridge regression does not. 

\end{frame}

%%%%%%%%%%%%%%%%%%%%%%%%%%%%%%%%%%%%%%%%%%%%%%%%%%%%%%%%%%%%%%%%%%%%%%%%%%%%%%%%%%%%

\begin{frame}[c]{}
\centering
\huge
\textbf{Bayesian Linear Regression as a GP}
\end{frame}

%%%%%%%%%%%%%%%%%%%%%%%%%%%%%%%%%%%%%%%%%%%%%%%%%%%%%%%%%%%%%%%%%%%%%%%%%%%%%%%%%%%%

\begin{frame}[c]{Bayesian Linear Regression as a GP}

\begin{itemize}

\item One example for a Gaussian process is the Bayesian linear regression model, and we already discuss it.
\lz
\item For $\thetab \sim \normaldist(\bm{0}, \tau^2 \id)$, the joint distribution of any set of function values is Gaussian:

$$f(\xI{i}) = \thetab^\top \xI{i} + \epsilon.$$
\vspace{3mm}
\item The corresponding mean function is $m(\x) = \bm{0}$, and the covariance function is
\vspace{-2mm}
\begin{eqnarray*}
cov\,(f(\x), f(\x^\prime)) &=& \expectation[f(\x) f(\x^\prime)] - \underbrace{\expectation[f(\x)] \expectation[f(\x^\prime]}_{= 0} \\ &=& \expectation[(\thetab^\top \x + \epsilon)^\top(\thetab^\top \x^\prime + \epsilon)] \\ &=&  \tau^2 \x^\top\x^\prime + \sigma^2 =: k(\x, \x^\prime).
\end{eqnarray*}

\end{itemize}

\end{frame}

%%%%%%%%%%%%%%%%%%%%%%%%%%%%%%%%%%%%%%%%%%%%%%%%%%%%%%%%%%%%%%%%%%%%%%%%%%%%%%%%%%%%

\begin{frame}[c,allowframebreaks]{Feature Spaces and the Kernel Trick}
\begin{itemize}

\item If one relaxes the linearity assumption by projecting the features into a higher dimensional feature space $\mathcal{Z}$ using a basis function $\phi: \Xspace \to \mathcal{Z}$, the corresponding covariance function becomes:
$$k(\x, \x^\prime) = \tau^2 \phi(\x)^\top\phi(\x^\prime) + \variance.$$
\vspace{.4cm}

\item To get arbitrarily complicated functions, we would have to handle high-dimensional feature vectors $\phi(\x)$.
\vspace{.4cm}

\item Fortunately, all we need to know is the inner product $\phi(\x)^T\phi(\bm{x}^\prime)$. That is, the feature vector itself never occurs in calculations.

\end{itemize}
\framebreak
%%%%%%%%%%%%%%%%%%%%%%%%%%%%%%%%%%%%%%%%%%%%%%%%%%%%%%%%%%%%%%%%%%%%%%%%%%%%%%%%%%%%

\textcolor{blue}{\faLightbulbO} If we can get the inner product directly and without calculating the infinite feature vectors, we can infer an infinitely complicated model with a finite amount of computation. This idea is known as \textbf{kernel trick}.

\begin{itemize}
\vspace{.7cm}
\item A Gaussian process can then be defined by either:
\vspace{.3cm}
\begin{itemize}
\item deriving the covariance function from the inner products of the basis functions evaluations, or
\vspace{.3cm}
\item choosing a positive definite kernel function (Mercer Kernel), which- according to Mercer's theorem - corresponds to taking the inner products in some (possibly infinite) feature space.
\end{itemize}
\end{itemize}

\end{frame}
%%%%%%%%%%%%%%%%%%%%%%%%%%%%%%%%%%%%%%%%%%%%%%%%%%%%%%%%%%%%%%%%%%%%%%%%%%%%%%%%%%%%


\begin{frame}[c]{Summary: Gaussian Process Regression}

\begin{itemize}
\item The Gaussian process regression is equivalent to the \textbf{kernelized} Bayesian linear regression.
\vspace{3mm}

\item The covariance function describes the shape of the Gaussian process. Hence, with the right choice of covariance function, remarkably flexible models can be built.
\vspace{3mm}

\item Naive implementations of Gaussian process models scale poorly with large datasets, as
\vspace{3mm}

\begin{itemize}
\item the kernel matrix has to be inverted / factorized, which is $\order(n^3)$,
\vspace{3mm}

\item computing the kernel matrix uses $\order(n^2)$ memory - running out of memory places a hard limit on the size of problems
\vspace{3mm}

\item generating predictions is $\order(n)$ for the mean, but $\order(n^2)$ for the variance.
\end{itemize}
(...special tricks are needed.)
\end{itemize}


\end{frame}

%%%%%%%%%%%%%%%%%%%%%%%%%%%%%%%%%%%%%%%%%%%%%%%%%%%%%%%%%%%%%%%%%%%%%%%%%%%%%%%%%%%%

\end{document}
