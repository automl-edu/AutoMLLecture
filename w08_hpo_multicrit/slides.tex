
\pdfminorversion=4 % for acroread
\documentclass[aspectratio=169,t,xcolor={usenames,dvipsnames}]{beamer}
%\documentclass[t,handout,xcolor={usenames,dvipsnames}]{beamer}
\usepackage{../beamerstyle}
\usepackage{dsfont}
\usepackage{bm}
\usepackage[english]{babel}
\usepackage[utf8]{inputenc}
\usepackage{graphicx}
\usepackage{algorithm}
\usepackage[ruled,vlined,algo2e,linesnumbered]{algorithm2e}
%\usepackage[boxed,vlined]{algorithm2e}
\usepackage{hyperref}
\usepackage{booktabs}
\usepackage{mathtools}

\usepackage{amsmath,amssymb}
\usepackage{listings}
\lstset{frame=lines,framesep=3pt,numbers=left,numberblanklines=false,basicstyle=\ttfamily\small}

\usepackage{subfig}
\usepackage{multicol}
%\usepackage{appendixnumberbeamer}
%
\usepackage{tcolorbox}

\usepackage{pgfplots}
\usepackage{tikz}
\usetikzlibrary{trees} 
\usetikzlibrary{shapes.geometric}
\usetikzlibrary{positioning,shapes,shadows,arrows,calc,mindmap}
\usetikzlibrary{positioning,fadings,through}
\usetikzlibrary{decorations.pathreplacing}
\usetikzlibrary{intersections}
\usetikzlibrary{positioning,fit,calc,shadows,backgrounds}
\pgfdeclarelayer{background}
\pgfdeclarelayer{foreground}
\pgfsetlayers{background,main,foreground}
\tikzstyle{activity}=[rectangle, draw=black, rounded corners, text centered, text width=8em]
\tikzstyle{data}=[rectangle, draw=black, text centered, text width=8em]
\tikzstyle{myarrow}=[->, thick, draw=black]

% Define the layers to draw the diagram
\pgfdeclarelayer{background}
\pgfdeclarelayer{foreground}
\pgfsetlayers{background,main,foreground}

%\usepackage{listings}
%\lstset{numbers=left,
%  showstringspaces=false,
%  frame={tb},
%  captionpos=b,
%  lineskip=0pt,
%  basicstyle=\ttfamily,
%%  extendedchars=true,
%  stepnumber=1,
%  numberstyle=\small,
%  xleftmargin=1em,
%  breaklines
%}

 
\definecolor{blue}{RGB}{0, 74, 153}

\usetheme{Boadilla}
%\useinnertheme{rectangles}
\usecolortheme{whale}
\setbeamercolor{alerted text}{fg=blue}
\useoutertheme{infolines}
\setbeamertemplate{navigation symbols}{\vspace{-5pt}} % to lower the logo
\setbeamercolor{date in head/foot}{bg=white} % blue
\setbeamercolor{date in head/foot}{fg=white}
\setbeamercolor{author  in head/foot}{bg=white} %blue
\setbeamercolor{title in head/foot}{bg=white} % blue
\setbeamercolor{title}{fg=white, bg=blue}
\setbeamercolor{block title}{fg=white,bg=blue}
\setbeamercolor{block body}{bg=blue!10}
\setbeamercolor{frametitle}{fg=white, bg=blue}
\setbeamercovered{invisible}

\makeatletter
\setbeamertemplate{footline}
{
  \leavevmode%
  \hbox{%
  \begin{beamercolorbox}[wd=.333333\paperwidth,ht=2.25ex,dp=1ex,center]{author in head/foot}%
%    \usebeamerfont{author in head/foot}\insertshortauthor
  \end{beamercolorbox}%
  \begin{beamercolorbox}[wd=.333333\paperwidth,ht=2.25ex,dp=1ex,center]{title in head/foot}%
    \usebeamerfont{title in head/foot}\insertshorttitle
  \end{beamercolorbox}%
  \begin{beamercolorbox}[wd=.333333\paperwidth,ht=2.25ex,dp=1ex,right]{date in head/foot}%
    \usebeamerfont{date in head/foot}\insertshortdate{}\hspace*{2em}
%    \insertframenumber\hspace*{2ex} 
  \end{beamercolorbox}}%
  \vskip0pt%
}
\makeatother

%\pgfdeclareimage[height=1.2cm]{automl}{images/logos/automl.png}
%\pgfdeclareimage[height=1.2cm]{freiburg}{images/logos/freiburg}

%\logo{\pgfuseimage{freiburg}}

\newcommand{\comment}[1]{
	\noindent
	%\vspace{0.25cm}
	{\color{red}{\textbf{TODO:} #1}}
	%\vspace{0.25cm}
}
\renewcommand{\comment}[1]{}
\newcommand{\hide}[1]{}
\newcommand{\cemph}[2]{\emph{\textcolor{#1}{#2}}}

\newcommand{\lit}[1]{{\footnotesize\color{black!70}[#1]}}

\newcommand{\litw}[1]{{\footnotesize\color{black!20}[#1]}}


\newcommand{\myframe}[2]{\begin{frame}[c]{#1}#2\end{frame}}
\newcommand{\myframetop}[2]{\begin{frame}{#1}#2\end{frame}}
\newcommand{\myit}[1]{\begin{itemize}#1\end{itemize}}
\newcommand{\myblock}[2]{\begin{block}{#1}#2\end{block}}


\newcommand{\votepurple}[1]{\textcolor{Purple}{$\bigstar$}}
\newcommand{\voteyellow}[1]{\textcolor{Goldenrod}{$\bigstar$}}
\newcommand{\voteblue}[1]{\textcolor{RoyalBlue}{$\bigstar$}}
\newcommand{\votepink}[1]{\textcolor{Pink}{$\bigstar$}}

\newcommand{\diff}{\mathop{}\!\mathrm{d}}
\newcommand{\refstyle}[1]{{\small{\textcolor{gray}{#1}}}}
\newcommand{\hands}[0]{\includegraphics[height=1.5em]{images/hands}}
\newcommand{\transpose}[0]{{\textrm{\tiny{\sf{T}}}}}
\newcommand{\norm}{{\mathcal{N}}}
\newcommand{\cutoff}[0]{\kappa}
\newcommand{\instD}[0]{\dataset}
\newcommand{\insts}[0]{\mathcal{I}}
\newcommand{\inst}[0]{i}
\newcommand{\pcs}[0]{\mathbf{\Lambda}}
\newcommand{\bx}[0]{\conf}
\newcommand{\conf}[0]{\mathbf{\lambda}}
\newcommand{\defconf}[0]{\mathbf{\lambda}_{\text{def}}}
\newcommand{\finconf}[0]{\mathbf{\lambda}^*}
\newcommand{\incumbent}[0]{\finconf}
\newcommand{\confs}[0]{\pcs}
%\newcommand{\vlambda}[0]{\bm{\lambda}}
%\newcommand{\vLambda}[0]{\bm{\Lambda}}
\newcommand{\dataset}[0]{\mathcal{D}}
\newcommand{\datasets}[0]{\mathbf{D}}
\newcommand{\loss}[0]{\mathcal{L}}

% \renewcommand{\vec}[1]{\mathbf{#1}}
\newcommand{\hist}[0]{\mathcal{H}}
\newcommand{\param}[0]{p}
\newcommand{\algo}[0]{\mathcal{A}}
\newcommand{\algos}[0]{\mathbf{A}}
%\newcommand{\nn}[0]{N}
\newcommand{\feats}[0]{\mathcal{F}}
\newcommand{\feat}[0]{\vec{f}}
\newcommand{\cluster}[0]{\vec{h}}
\newcommand{\clusters}[0]{\vec{H}}
\newcommand{\perf}[0]{\mathbb{R}}
%\newcommand{\surro}[0]{\mathcal{S}}
\newcommand{\surro}[0]{\hat{f}}
\newcommand{\func}[0]{f}
\newcommand{\epm}[0]{\surro}
\newcommand{\portfolio}[0]{\mathcal{P}}
\newcommand{\schedule}[0]{\mathcal{S}}
\newcommand{\mdata}[0]{\dataset_{\text{meta}}}

% Deep Learning
\newcommand{\weights}[0]{\theta}
\newcommand{\metaweights}[0]{\phi}


% reinforcement learning
\newcommand{\policies}[0]{\Pi}
\newcommand{\policy}[0]{\pi}
\newcommand{\actionRL}[0]{a}
\newcommand{\stateRL}[0]{s}
\newcommand{\statesRL}[0]{\mathcal{S}}
\newcommand{\rewardRL}[0]{r}
\newcommand{\rewardfuncRL}[0]{\mathcal{R}}

\RestyleAlgo{algoruled}
\DontPrintSemicolon
\LinesNumbered
\SetAlgoVlined
\SetFuncSty{textsc}

\SetKwInOut{Input}{Input}
\SetKwInOut{Output}{Output}
\SetKw{Return}{return}

%\newcommand{\changed}[1]{{\color{red}#1}}

%\newcommand{\citeN}[1]{\citeauthor{#1}~(\citeyear{#1})}

\renewcommand{\vec}[1]{\mathbf{#1}}
\DeclareMathOperator*{\argmin}{arg\,min}
\DeclareMathOperator*{\argmax}{arg\,max}

\newcommand{\aqme}{\textit{AQME}}
\newcommand{\aslib}{\textit{ASlib}}
\newcommand{\llama}{\textit{LLAMA}}
\newcommand{\satzilla}{\textit{SATzilla}}
\newcommand{\satzillaY}[1]{\textit{SATzilla'{#1}}}
\newcommand{\snnap}{\textit{SNNAP}}
\newcommand{\claspfolioTwo}{\textit{claspfolio~2}}
\newcommand{\flexfolio}{\textit{FlexFolio}}
\newcommand{\claspfolioOne}{\textit{claspfolio~1}}
\newcommand{\isac}{\textit{ISAC}}
\newcommand{\eisac}{\textit{EISAC}}
\newcommand{\sss}{\textit{3S}}
\newcommand{\sunny}{\textit{Sunny}}
\newcommand{\ssspar}{\textit{3Spar}}
\newcommand{\cshc}{\textit{CSHC}}  
\newcommand{\cshcpar}{\textit{CSHCpar}}  
\newcommand{\measp}{\textit{ME-ASP}} 
\newcommand{\aspeed}{\textit{aspeed}}
\newcommand{\autofolio}{\textit{AutoFolio}}
\newcommand{\cedalion}{\textit{Cedalion}}
\newcommand{\fanova}{\textit{fANOVA}}
\newcommand{\sbs}{\textit{SB}}
\newcommand{\oracle}{\textit{VBS}}

% like approaches
\newcommand{\claspfoliolike}[1]{\texttt{claspfolio-#1-like}}
\newcommand{\satzillalike}[1]{\texttt{SATzilla'#1-like}}
\newcommand{\isaclike}{\texttt{ISAC-like}}
\newcommand{\ssslike}{\texttt{3S-like}}
\newcommand{\measplike}{\texttt{ME-ASP-like}}

\newcommand{\aspCoseal}{\textit{ASP-POTASSCO}}
\newcommand{\cspCoseal}{\textit{CSP-2010}}
\newcommand{\maxsatCoseal}{\textit{MAXSAT12-PMS}}
\newcommand{\premarCoseal}{\textit{PRE\-MARSHALLING}}
\newcommand{\qbfCoseal}{\textit{QBF-2011}}
\newcommand{\satallTwelveCoseal}{\textit{SAT12-ALL}}
\newcommand{\sathandTwelveCoseal}{\textit{SAT12-HAND}}
\newcommand{\satinduTwelveCoseal}{\textit{SAT12-INDU}}
\newcommand{\satrandTwelveCoseal}{\textit{SAT12-RAND}}
\newcommand{\sathandElevenCoseal}{\textit{SAT11-HAND}}
\newcommand{\satinduElevenCoseal}{\textit{SAT11-INDU}}
\newcommand{\satrandElevenCoseal}{\textit{SAT11-RAND}}
\newcommand{\proteusCoseal}{\textit{PROTEUS-2014}}

\newcommand{\irace}{\textit{I/F-race}}
\newcommand{\gga}{\textit{GGA}}
\newcommand{\smac}{\textit{SMAC}}
\newcommand{\paramils}{\textit{ParamILS}}
\newcommand{\spearmint}{\textit{Spearmint}}
\newcommand{\tpe}{\textit{TPE}}

\newcommand{\gringo}{\textit{gringo}}
\newcommand{\clasp}{\textit{clasp}}
\newcommand{\lingeling}{\textit{lingeling}}

\newcommand{\hydra}{\textit{Hydra}}

\newcommand{\plingeling}{\textit{Plingeling}}
\newcommand{\ccasat}{\textit{CCASat}}

\usepackage{pifont}
\newcommand{\itarrow}{\mbox{\Pisymbol{pzd}{229}}}
\newcommand{\ithook}{\mbox{\Pisymbol{pzd}{52}}}
\newcommand{\itcross}{\mbox{\Pisymbol{pzd}{56}}}
\newcommand{\ithand}{\mbox{\raisebox{-1pt}{\Pisymbol{pzd}{43}}}}

%\DeclareMathOperator*{\argmax}{arg\,max}

\newcommand{\ie}{{\it{}i.e.\/}}
\newcommand{\eg}{{\it{}e.g.\/}}
\newcommand{\cf}{{\it{}cf.\/}}
\newcommand{\wrt}{\mbox{w.r.t.}}
\newcommand{\vs}{{\it{}vs\/}}
\newcommand{\vsp}{{\it{}vs\/}}
\newcommand{\etc}{{\copyedit{etc.}}}
\newcommand{\etal}{{\it{}et al.\/}}

\newcommand{\pscProc}{{\bf procedure}}
\newcommand{\pscBegin}{{\bf begin}}
\newcommand{\pscEnd}{{\bf end}}
\newcommand{\pscEndIf}{{\bf endif}}
\newcommand{\pscFor}{{\bf for}}
\newcommand{\pscEach}{{\bf each}}
\newcommand{\pscThen}{{\bf then}}
\newcommand{\pscElse}{{\bf else}}
\newcommand{\pscWhile}{{\bf while}}
\newcommand{\pscIf}{{\bf if}}
\newcommand{\pscRepeat}{{\bf repeat}}
\newcommand{\pscUntil}{{\bf until}}
\newcommand{\pscWithProb}{{\bf with probability}}
\newcommand{\pscOtherwise}{{\bf otherwise}}
\newcommand{\pscDo}{{\bf do}}
\newcommand{\pscTo}{{\bf to}}
\newcommand{\pscOr}{{\bf or}}
\newcommand{\pscAnd}{{\bf and}}
\newcommand{\pscNot}{{\bf not}}
\newcommand{\pscFalse}{{\bf false}}
\newcommand{\pscEachElOf}{{\bf each element of}}
\newcommand{\pscReturn}{{\bf return}}

%\newcommand{\param}[1]{{\sl{}#1}}
\newcommand{\var}[1]{{\it{}#1}}
\newcommand{\cond}[1]{{\sf{}#1}}
%\newcommand{\state}[1]{{\sf{}#1}}
%\newcommand{\func}[1]{{\sl{}#1}}
\newcommand{\set}[1]{{\Bbb #1}}
%\newcommand{\inst}[1]{{\tt{}#1}}
\newcommand{\myurl}[1]{{\small\sf #1}}

\newcommand{\Nats}{{\Bbb N}}
\newcommand{\Reals}{{\Bbb R}}
\newcommand{\extset}[2]{\{#1 \; | \; #2\}}

\newcommand{\vbar}{$\,\;|$\hspace*{-1em}\raisebox{-0.3mm}{$\,\;\;|$}}
\newcommand{\vendbar}{\raisebox{+0.4mm}{$\,\;|$}}
\newcommand{\vend}{$\,\:\lfloor$}


\newcommand{\goleft}[2][.7]{\parbox[t]{#1\linewidth}{\strut\raggedright #2\strut}}
\newcommand{\rightimage}[2][.3]{\mbox{}\hfill\raisebox{1em-\height}[0pt][0pt]{\includegraphics[width=#1\linewidth]{#2}}\vspace*{-\baselineskip}}





\newcommand{\a}[0]{\mathbf{a}}
\newcommand{\y}[0]{\mathbf{y}}
\newcommand{\q}[0]{\mathbf{q}}
\newcommand{\Xspace}[0]{\mathcal{X}}

\title[AutoML: Overview]{Multi-criteria Optimization}
\subtitle{Overview for this Week}
%TODO: change authors!
\author[Bernd Bischl]{Bernd Bischl \and Frank Hutter \and Lars Kotthoff \and \underline{Marius Lindauer}}
\institute{}
\date{}



% \AtBeginSection[] % Do nothing for \section*
% {
%   \begin{frame}{Outline}
%     \bigskip
%     \vfill
%     \tableofcontents[currentsection]
%   \end{frame}
% }

\begin{document}

	\maketitle


\begin{frame}{Notation}

\begin{itemize}
\item Admissible set $\Xspace \subset \realnum^n$
\item Target region $\realnum^m$
\item Multi-criteria objective function $f: \Xspace \to \realnum^m$
\item Objective function vector $f(\x) = \left(f_1(\x), ..., f_m(\x\right))^\top \in \realnum^m$, which maps $\x$ into the space $\realnum^m$.
% \item $F:=f(\Xspace) = \left\{f(\x) ~|~ \x \in \Xspace\right\}$: Bild einer Funktion (Menge aller möglichen Funktionswerte)
\end{itemize}

w.l.o.g. we look at minimization problems.

\end{frame}

\begin{frame}[allowframebreaks]{Introduction example}

Often we want to solve optimization problems concerning several goals.

\begin{itemize}
\item Medicine: maximum effect, but minimum side effect of a drug.
\item Finances: maximum return, but minimum risk of an equity portfolio.
\item Production planning: maximum revenue, but minimum costs.
\item Booking a hotel: maximum rating, but minimum costs.
\end{itemize}

A \textit{simple} approach would be to formulate all but one objective function simplified as a secondary condition.

\vspace*{0.2cm}

\framebreak

\textbf{Example}:

Maximize proceeds subject to costs $\le C, C \in \realnum$.

\vspace*{0.2cm}

 \textbf{Disadvantages}:
\begin{itemize}
 \item The result depends of course on how we select $C$ and usually returns different solutions for different values of $C$.
 \item The more target functions we optimize, the more difficult such a formulation becomes.
\end{itemize}

\vspace*{0.2cm}

\textbf{Target}: find a general approach to solving multi-criteria problems.


\begin{center}
\includegraphics[width = 0.35\linewidth]{images/booking1.png} ~~~ \includegraphics[width = 0.35\linewidth]{images/booking2.png}
\end{center}

When booking a hotel: find the hotel with

\begin{itemize}
\item Minimum price per night (\textbf{costs}) and
\item Maximum user rating (\textbf{performance}).
\end{itemize}

\vfill

\begin{footnotesize}
Since we limit ourselves to minimizing problems, we minimize negative valuations.
\end{footnotesize}

\framebreak

The goals often conflict with each other:

\begin{itemize}
\item Lower price $\to$ often lower hotel rating.
\item Better rating $\to$ frequently higher price.
\end{itemize}

Example: (negative) average rating by hotel guests (1 - 5) vs. average price per night in USD from hotels on Expedia (excerpt).

\vspace*{0.2cm}

\centering \includegraphics[width=\maxwidth]{images/expedia-1-1}


In addition, targets are often not comparable because they have different units, for example.

\begin{itemize}
\item Left: a hotel with rating $4$ for $89$ Euro ($\textcolor{green}{y_1} = \left(89, - 4.0\right)$ would be preferred to a hotel $\textcolor{red}{y_2} = \left(108, - 4.0\right)$ (left)
\item Right: how to decide if $\textcolor{orange}{y_1} = \left(89, - 4.0\right)$ and $\textcolor{orange}{y_1} = \left(95, - 4.5\right)$?
\item How much is a \textit{scoring point} worth?
\end{itemize}

\centering \includegraphics[width=\maxwidth]{images/expedia-2-1}

\end{frame}


\begin{frame}{Definition: multi-criteria optimization problem}

Be $\Xspace \subset \realnum^n$ and $f: \Xspace \to \realnum^m$, $m \ge 2$. A \textbf{multi-criteria optimization problem} is defined by

$$
\min_{\x \in \Xspace}  f(\x) \Leftrightarrow \min_{\x \in \Xspace} \left(f_1(\x), f_2(\x), ..., f_m(\x)\right).
$$

\begin{itemize}
\item \textbf{Aim: } minimize multiple target functions simultaneously.
\item Objective functions are often contradictory here.
\item Often no clearly best solution, but a set of solutions that are equally good.
\item Synonym terms: multi-criteria optimization, multi-objective optimization, Pareto optimization
\end{itemize}

\end{frame}


\begin{frame}[allowframebreaks]{How to define optimality?}

Be still $\y = (\text{price}, - \text{evaluation})$. In some cases it is \textit{clear} which point is the better one:

\begin{itemize}
\item The solution $\textcolor{green}{\textbf{y}_1} = \left(89, - 4.0\right)$ dominates $\textcolor{red}{\textbf{y}_2} = \left(108, - 4.0\right)$: $\textcolor{green}{\textbf{y}_1}$ is not worse in any dimension and is better in one dimension. $\textcolor{red}{\textbf{y}_2}$ gets \textbf{dominated} of $\textcolor{green}{\textbf{y}_1}$
$$
\textcolor{red}{\textbf{y}_2} \prec \textcolor{green}{\textbf{y}_1}.
$$
\end{itemize}

\centering \includegraphics[width=\maxwidth]{images/expedia-3-1}

\framebreak

For the points $\textcolor{orange}{\textbf{y}_1} = \left(89, - 4.0\right)$ and $\textcolor{orange}{\textbf{y}_2} = \left(95, - 4.5\right)$ we cannot say which point is the better one.

\begin{itemize}
\item We designate the points as \textbf{equivalent} and write

$$
\textcolor{orange}{\textbf{y}_1} \not\prec \textcolor{orange}{\textbf{y}_2} \text{ und } \textcolor{orange}{\textbf{y}_2} \not\prec \textcolor{orange}{\textbf{y}_1}.
$$

\centering \includegraphics[width=\maxwidth]{images/expedia-4-1}


\item The set of all equivalent points that are not dominated by another point is called the \textbf{Pareto front}.

\vspace*{0.3cm}

\centering \includegraphics[width=\maxwidth]{images/expedia-5-1}



\end{itemize}

\end{frame}

\begin{frame}[allowframebreaks]{Pareto sets und Pareto optimality}

\textbf{Definition:}

Given a multicriteria optimization problem $$\min_{\x \in \Xspace} f(\x) = \left(f_1(\x), ..., f_m(\x)\right), \quad f_i: \Xspace \to \R.$$

\begin{itemize}
\item A solution $\x_1$ \textbf{(Pareto-) dominates} $\x_2$, if $f(\x_1) \prec f(\x_2)$, i.e.
\begin{enumerate}
\item $f_i(\x_1) \le f_i(\x_2)$ for all $i \in \{1, 2, ..., m\}$ und
\item $f_j(\x_1) < f_j(\x_2)$ for at least one $j \in \{1, 2, ..., m\}$
\end{enumerate}
\vspace*{0.1cm}
\item A solution $\x^*$ that is not dominated by any other solution is called \textbf{Pareto optimal}.
\vspace*{0.1cm}
\item The set of all Pareto optimal solutions is called \textbf{Pareto set} $\mathcal{P} := \{\x \in \Xspace |\not \exists ~\tilde{\x} \text{ with } f(\tilde{\x}) \prec f(\x)\}$
\item $\mathcal{F} = f(\mathcal{P}) = \{f(\x) | \x \in \mathcal{P}\}$ is called \textbf{Pareto front}.
\end{itemize}

\end{frame}

\begin{frame}[allowframebreaks]{Example: an objective function}

We consider the minimization problem

$$
\min_{x} f(x) = (x - 1)^2, \qquad 0 \le x \le 3.
$$

The optimum is at $x^* = 1$.

\vspace*{0.1cm}


\centering \includegraphics[width=\maxwidth,height=0.5\textheight]{images/expedia-6-1}


\end{frame}

\begin{frame}[allowframebreaks]{Example: two target functions}

We extend the above problem to two objective functions $f_1(x) = (x - 1)^2$ and $f_2(x) = 3(x - 2)^2$, thus

$$
\min_x f(x) := \left(f_1(x), f_2(x)\right), \qquad 0 \le x \le 3.
$$

\vspace*{0.1cm}


\centering \includegraphics[width=\maxwidth,height=0.5\textheight]{images/expedia-7-1}


\framebreak

We consider the functions in the objective function space $f(\Xspace)$ by drawing the objective function values $\left(f_1(x), f_2(x)\right)$ for all $0 \le x \le 3$.

\vspace*{0.1cm}


\centering \includegraphics[width=\maxwidth,height=0.5\textheight]{images/expedia-8-1}

    \vspace*{-0.3cm}

The Pareto front is shown in green.
The Pareto front cannot be \enquote{left} without getting worse in at least one objective function.

\end{frame}


% \begin{frame}[allowframebreaks]{Zusammenfassung}
%
% Bisher:
% \begin{itemize}
% \item Multikriterielle Optimierungsprobleme sind Probleme, in denen mehrere Funktionen gleichzeitig optimiert werden sollen.
% \item Zielfunktionen sind häufig konkurrierend und lassen sich nicht auf derselben Skala vergleichen
% \item $\to$ Einführung eines geeigneten Optimalitätsbegriffs: Pareto-Optimalität
% \end{itemize}
%
% \begin{itemize}
% \item Wie findet man eine Pareto-Front?
% \item $\to$ Lösungsverfahren
% \end{itemize}
%
% \end{frame}

\section{Solvers}

\begin{frame}[allowframebreaks]{Two solutions}

\begin{itemize}
\item The Pareto set is a set of equally optimal solutions.
\item One is often interested in a \textbf{single} optimal solution.
\item Without further information no unambiguous optimal solution can be determined \\
$\to$ decision must be based on other criteria.
\end{itemize}

Basically, there are two possible solutions:
\begin{itemize}
\item \textbf{A-priori approach}: user preferences are considered \textbf{before} the optimization process
\item \textbf{A-posteriori approach}: user preferences are considered \textbf{after} the optimization process
\end{itemize}

\end{frame}

\begin{frame}[allowframebreaks]{A-priori procedure}

\textbf{Example: weighted total}


\textbf{Previous knowledge: } one rating point is worth $50$ Euro to a customer. \\ $\to$ We optimize the weighted sum:

$$
\min_\text{Hotel} \text{(Price / Night)} - 50 \cdot \text{Rating}
$$

\centering \includegraphics[width=0.7\linewidth,height=0.4\textheight]{images/expedia-9-1}

\framebreak

A-priori approach: weighted sum

\begin{eqnarray*}
\min_{x \in \Xspace} && \sum_{i = 1}^m w_i f_i(\x) \\
&\text{with}& w_i \ge 0
 \end{eqnarray*}

% Bei systematischem Variieren der Gewichte können alle Lösungen auf dem konvexen Zielbereich gefunden werden.

\framebreak

\textbf{Example: lexicographic method}

\lz

\textbf{Previous knowledge: } customer prioritizes rating over price. \\
$\to$ optimize target functions one after the other.

\lz

\centering \includegraphics[width=0.8\linewidth,height=0.4\textheight]{images/expedia-10-1}

\framebreak

A-priori approach: lexicographic method

\begin{eqnarray*}
y_1^* &=& \min_{\x \in \Xspace} f_1(\x)\\
y_2^* &=& \min_{\x \in \{\x ~|~ f_1(\x) = y_1^*\}} f_2(\x) \\
y_3^* &=& \min_{\x \in \{\x ~|~ f_1(\x) = y_1^* \land f_2(\x) = y_2^*\}} f_3(\x) \\
&\vdots&
\end{eqnarray*}

Also here: different sequences provide different solutions.

\framebreak

\textbf{Summary a-priori approach:}
\begin{itemize}
\item In a single application, only one solution is obtained, which depends on the a-priori selection of weights, order, etc.
\item In case of repeated use, several solutions are obtained if weights, order, etc. are systematically varied.
\item Usually there are solutions that remain hidden from these methods.
\item Implicit assumption: monocritical optimization \enquote{simple}
\end{itemize}

\end{frame}

\begin{frame}[allowframebreaks]{A-posteriori procedure}

A-posteriori methods, on the other hand, have the goal to

\begin{itemize}
\item find the set of \textbf{all} optimal solutions (the Pareto set),
\item select (if necessary) an optimal solution based on prior knowledge or individual preferences.
\end{itemize}

A-posteriori methods are therefore the more generic approach to solving a multi-criteria optimization problem.


\framebreak

\textbf{Example:}

A user gets more detailed information about all Pareto optimal hotels (left) and chooses an optimal solution (right) based on previous knowledge or additional criteria (e.g. location of the hotel).

\vspace*{0.3cm}


\centering \includegraphics[width=0.8\linewidth,height=0.5\textheight]{images/expedia-11-1}


\end{frame}


\begin{frame}[allowframebreaks]{Evaluation of solutions}

A common metric for evaluating a set of solutions $\mathcal{P} \subset \Xspace$ is the \textbf{dominated hypervolume} (S-metric), which we call $S(\mathcal{P}, R)$.

\vspace*{-0.4cm}

\begin{center}
\includegraphics[width = 0.35\linewidth]{images/dominated_hypervolume.png}
\end{center}

\vspace*{-0.4cm}

The dominated hypervolume of the set of points $\mathcal{P} \subset \Xspace$ (here: 5 black points) is the area in the target function space (regarding a reference point $R$) which is dominated by points $\mathcal{P}$.

\end{frame}



% \begin{frame}[allowframebreaks]{Skalarisierung}
%
% \textbf{Idee:} Führe mehrkriterielle Optimierung auf monokriterielle Optimierungsprobleme zurück.
%
% \lz
%
% Mögliches Verfahren: Optimiere eine gewichtete Summe
%
% $$
% w_1 \cdot f_1(\x) + w_2 \cdot f_2(\x) + ... + w_m \cdot f_m(\x),
% $$
%
% wobei $w_i \ge 0$ und $\sum_{i = 1}^m w_i = 1$ (Konvexkombination).
%
% \framebreak
%
% \textbf{Beispiel:} Es sei weiterhin $f_1(x) = (x - 1)^2$ und $f_2(x) = 3(x - 2)^2$. Wir lösen das skalare Ersatzproblem
%
% $$
% z = w_1 f_1(x) + w_2 f_2(x) = w_1 y_1 + w_2y_2 \to \min!
% $$
%
% Umformen ergibt hier eine Geradengleichung: $y_2 = -\frac{w_1}{w_2}y_1 + z$.
%
% \lz
%
% Grafisch finden wir das Optimum dieses skalaren Optimierungsproblems, indem wir
%
% \begin{itemize}
% \item alle möglichen Geraden $y_2 = -\frac{w_1}{w_2}y_1 + z$ für unterschiedliche Werte von $z$ einzeichnen und
% \item den Punkt finden, in dem sich die Gerade mit dem Bild der Funktion $F = f(\Xspace)$ berührt.
% \end{itemize}
%
% Beispielhaft für $w_2 = \frac{2}{3}$ und $w_1 = \frac{1}{3}$.
%


% \end{frame}
%
% \begin{frame}[allowframebreaks]{Zusammenfassung: Skalarisierungsmethoden}
%
% Es gibt weitere Möglichkeiten, ein mehrkriterielles Problem durch eines oder mehrere skalare Ersatzprobleme approximativ zu löse (z.B. auch lexikografische Methode).
%
% \lz
%
% \textbf{Aber: }
%
% \begin{itemize}
% \item Aber: Es gibt Lösungen, die diesen Methoden verborgen bleiben (insbesondere: bei einmaliger Anwendung nur eine Lösunge)
% \item Auch monokriterielle Optimierung ist nicht immer \enquote{einfach}
% \end{itemize}
%
% \end{frame}
%
% \begin{frame}[allowframebreaks]{Mehrkriterielles Abstiegsverfahren}
%
% Die Definition einer Abstiegsrichtung lässt sich auf den multikriteriellen Fall erweitern.
%
% \lz
%
% \textbf{Definition: } Abstiegsrichtung \\
% Es sei $f: \R^n \to \R^m$ stetig differenzierbar.  Eine Richtung $\bm{v} \in \R^n$ heißt \textbf{Abstiegsrichtung} in $\x \in \R^n$, wenn
%
% $$
% \nabla f_j(\x)^T\bm{v} < 0 \quad \forall ~ j = 1, ..., m.
% $$
%
% \framebreak
%
% Problem: Das Finden einer Abstiegsrichtung ist per se schon ein nicht ganz einfach zu lösendes Problem.





\section{Evolutionary multi-objective optimization algorithms (EMOA)}


\begin{frame}[allowframebreaks]{A-posteriori methods and evolutionary algorithms}

Evolutionary algorithms return as a solution a \textbf{population} of solution candidates. Evolutionary multi-objective (EMO) algorithms aim to provide a set of solution candidates that corresponds to the Pareto set \enquote{as well as possible}.

\vspace*{-0.4cm}

\begin{center}
\includegraphics[width = 0.8\linewidth]{images/EA-steps.png}
\end{center}

\vspace*{-0.4cm}

\begin{footnotesize}
Image of the function (grey) and target function values $(f_1(\x), f_2(\x))$ for $\x \in \mathcal{P}_i, i = 1, 3, 10$.
\end{footnotesize}

\framebreak

\begin{algorithm}[H]
  \begin{center}
  \caption{Evolutionary algorithm}
    \begin{algorithmic}[1]
    \State Initialize and rate population $P_0 \subset \Xspace$ with $|\mathcal{P}| = \mu$
    \State $t \leftarrow 0$
      \Repeat
        \State Variation: generate offspring $Q_t$ with $|Q_t| = \lambda$
        \State Rate fitness of offspring
        \State Selection: select survivors $P_{t + 1}$
 		\State $t \leftarrow t + 1$
      \Until{Stop criterion fulfilled}
            \vspace*{-0.3cm}
    \end{algorithmic}
    \end{center}
\end{algorithm}

The population of solution candidates consists of $\x \in \Xspace$.

\end{frame}


\begin{frame}[allowframebreaks]{Objectives of an evolutionary strategy}

The aim is to select the evolution strategy in such a way that the algorithm provides an approximation of the Pareto front, where

\begin{enumerate}
\item The individuals of the population (or the corresponding functional values in the target function space) \textbf{converge} to the Pareto front.
\item The individuals of the population provide a \textbf{diverse} as possible approximation of the Pareto front.
\end{enumerate}

\vspace*{-0.3cm}

\begin{center}
\includegraphics[width = 0.25\linewidth]{images/EMO_goals.png}
\end{center}

\vspace*{-0.5cm}

\begin{footnotesize}
\textbf{Caution}: in this graphic the objective function values are exceptionally \textbf{maximized}.
\end{footnotesize}

\end{frame}

\begin{frame}[allowframebreaks]{NSGA-II}

The \textbf{non-dominated sorting genetic algorithm (NSGA-II)} was published by K. Deb in 2002.

\begin{itemize}
\item The NSGA-II follows a $(\mu + \lambda)$ strategy
\item All previously discussed strategies can be used as a variation strategy; the original paper uses polynomial mutation and simulated binary crossover.
\item The selection strategy is based on
\begin{itemize}
\item \textbf{Non-dominated sorting}
\item \textbf{Crowding distance assignment}
\end{itemize}
\end{itemize}

\end{frame}

\begin{frame}[allowframebreaks]{NSGA-II: non-dominated sorting}

% \begin{center}
% \includegraphics[width = 0.5\linewidth]{images/NSGA2_1.png}
% \end{center}

% \framebreak

\begin{footnotesize}
We subdivide $R_t = P_t \cup Q_t$ into fronts $F_1, F_2, F_3, ...$ such that

\begin{itemize}
\item the points in the fronts are equivalent to each other, and
\item that any point $\x \in F_1$ dominates any point from $F_2, F_3, F_4...$; any point $\x \in F_2$ dominates all points from $F_3, F_4, ...$, etc. \\
We write $F_1 \prec F_2 \prec F_4 \prec ... $
\end{itemize}
\end{footnotesize}

\begin{center}
\includegraphics[width = 0.5\linewidth]{images/NSGA2_NDS.png}
\end{center}

\framebreak

Which individuals survive? We fill $\mu$ \enquote{places} one by one with $F_1, F_2, ...$ until a front can no longer \textbf{fully} survive (here: $F_3$).

\begin{center}
\includegraphics[width = 0.45\linewidth]{images/NSGA2_2.png}
\end{center}

Which individuals survive from $F_3$? $\to$ \textbf{crowding sort}

\end{frame}

\begin{frame}[allowframebreaks]{NSGA-II: crowding sort}

\textbf{Idea:} add a \enquote{good} representative of the front $F_3$ if possible.


\includegraphics[height = 0.5\textheight]{images/NSGA2_CS1.png}

\begin{footnotesize}
The points on the left (marked by a triangle) do not represent the front very well because they are very close together. The front is better represented by the points on the right plot.
\end{footnotesize}

\framebreak

\textbf{Crowding sort} sorts the individuals based on their crowding distance:
\end{itemize}

\begin{center}
\includegraphics[width = 0.5\linewidth]{images/NSGA2_CS2.png}
\end{center}

\begin{footnotesize}
One point with high crowding distance (red) and one point with very small crowding distance (blue).
\end{footnotesize}

\framebreak

\begin{algorithm}[H]

  \begin{center}
  \caption{NSGA-II}
    \begin{algorithmic}[1]
   	\begin{footnotesize}
    \State Initialize population $P_0$, $t \leftarrow 0$
    \State $F_1, F_2, F_3, ... \leftarrow \texttt{nondominated-sort}(P_0)$
    \State Generate $Q_0$ by binary tournament selection, recombination and mutation
      \Repeat
        \State $F_1, F_2, F_3, ... \leftarrow \texttt{nondominated-sort}(P_t \cup Q_t)$
        \State $i \leftarrow 1$
        \While{$|P_{t + 1} \cup F_i| < \mu$}
        	\State $P_{t + 1} = P_{t + 1} \cup F_i$
        	\State $i \leftarrow i + 1$
    	\EndWhile
        \State $\tilde F_i = (\x_1, \x_2, ..., \x_k)= \texttt{SortByCrowdingDistance}(F_i)$
        \While {$P_{t + 1} < \mu$}
        	\State $P_{t + 1} = P_{t + 1} \cup \x_j$
        	\State $j \leftarrow j + 1$
        \EndWhile
        \State Generate $Q_{t + 1}$ by binary tournament selection, recombination and mutation
      \Until{Stop criterion fulfilled}
      \vspace*{-0.3cm}
      \end{footnotesize}
    \end{algorithmic}
    \end{center}
\end{algorithm}

\end{frame}

% \begin{frame}[allowframebreaks]{SPEA-2}

% Ebenso im Jahr 2002 wurde der \textbf{Strength Pareto EA} (SPEA-2) von Zitzler et al. veröffentlicht.

% \lz

% \begin{itemize}
% \item Neben der aktuellen Population $P_t$ gibt es auch ein sogenanntes Archiv $A_t$, das lediglich zur Bewertung der aktuellen Population dient.

% \begin{images}
% 	\centering
% 	\includegraphics[width=0.6\linewidth]{images/SPEA-archive}
% \end{images}

% \framebreak

% \item Die Bewertung (und damit auch die Selektion) eines Individuums erfolgt anhand von

% $$
% \text{fitness}(x) = \text{raw}(x) + \text{density}(x).
% $$

% Hierbei ist

% \begin{itemize}
% \item $\text{raw}(x)$ die \enquote{Grundfitness} (bzgl. Population und Archiv)
% \vspace*{-0.2cm}
% $$
% \text{raw}(x) = |\{y \in P_t: f(x) \prec f(y)\}| + |\{y \in A_t: f(x) \prec f(y)\}|,
% $$

% \item $\text{density}(x)$ die Dichte des Punktes
% $$
% \text{density}(x) = \frac{1}{\sigma^{(k)}(x) + 2},
% $$
% ($\sigma^{(k)}$ bezeichne den Abstand zum $k$-nächsten Nachbarn).
% \end{itemize}
% \end{itemize}

% \framebreak

% \begin{algorithm}[H]
% \begin{footnotesize}
%   \begin{center}
%   \caption{SPEA-2}
%     \begin{algorithmic}[1]
%     \State Initialisiere Population $P_0$, $|P_0| = \lambda$ und ein leeres Archiv $A_0, |A_0| = \gamma$
%       \Repeat
%         \State Berechne Fitness der Individuen in $P_t$ und $A_t$ anhand der oben definierten Fitnessfunktion
%         \State Fülle $A_{t+1}$ auf mit nichtdominierten Individuen aus $P_t \cup A_t$ $^{(*)}$
%         \State Fülle \enquote{mating pool} durch binäre Turnierselektion mit Zurücklegen auf $A_{t + 1}$
%         \State Generiere $P_{t + 1}$ durch Rekombination und Mutation
%       \Until{Stoppkriterium erfüllt}
%       \State Gib $A_t$ zurück
%     \end{algorithmic}
%     \end{center}
% \end{footnotesize}
% \end{algorithm}

% \vfill
% \begin{footnotesize}
% $^{(*)}$ Wenn $|A_{t + 1}| > \gamma$, dann entferne solange Individuen mit kleinster Distanz zum Nachbarn, bis $|A_{t + 1}| = \gamma$. Sollte $|A_{t + 1}| < \gamma$, füge die besten dominierten Individuen aus $P_t \cup A_t$ hinzu.
% \end{footnotesize}

% \end{frame}

\section{SMS-EMOA}

\begin{frame}[allowframebreaks]{Selection criteria: contribution to the hypervolume}

\begin{itemize}
\item The SMS-EMOA (S-Metric-Selection-EMOA) evaluates the fitness of an individual $\x \in \mathcal{P} \subset \Xspace$ based on its contribution to the dominated hypervolume (S-Metric):
$$
\Delta s(\x, \mathcal{P}) = S(\mathcal{P}, R) - S(\mathcal{P} \setminus \{ \x\}, R).
$$
\end{itemize}
% \framebreak

% \textbf{Berechnung des Hypervolumens im 2 dimensionalen Fall:}
% \begin{enumerate}
% \item Sortiere die Zielfunktionsvektoren bzgl. eines Kriteriums (z.B. aufsteigend bzgl. $f_1$)\\
% $\Rightarrow$ Da Pareto-Front (kein Punkt dominiert anderen): Punkte sind bzgl. $f_2$ absteigend sortiert .
% \item Für das $j$-te Individuum $a^{(j)}, j\in \{2,..., |F_{\nu}|\}$ in der sortierten Sequenz der Front $F_{\nu}$ berechnet sich der Hypervolumensbeitrag als:
% \medskip

% $$
% \Delta s(\y^{(j)}, F_{\nu}) = (y_{1}^{(j+1)} - y_{1}^{(j)}) (y_{2}^{(j-1)} - y_{2}^{(j)})
% $$
% \end{enumerate}

\framebreak

\begin{center}
Hypervolume contribution in a 2-dimensional objective space:\\
\includegraphics[width = 0.45\textwidth]{images/hypervolumenbeitrag.png}
\end{center}

\footnotesize
\vspace*{-0.5cm}
\begin{itemize}
% \item Links: Punkte entsprechen Werten der Individuen in 2-dimensionalem Zielraum.
% \item Links: Punkte ohne Füllung zeigen dominierte Lösungen. Gelbe Fläche zeigt Bereich in dem dominierende Lösungen liegen.
\item Dark rectangles correspond to the hypervolume contribution of the black dots.
\item Grey point is the so-called reference point and limits the space.
\item The hypervolume contribution thus corresponds to the size of the space that is dominated only by the individual $\bm{a}$, and not to any other of the space.
\item $a^\star$ has lowest S-metric contribution .
\end{itemize}
\end{frame}

% \begin{frame}[allowframebreaks]{SMS-EMOA}
% \textbf{Motivation:}
% \begin{itemize}
% \item Pareto-Front bildet Menge von optimalen Parameterkonbinationen ab.
% \item Oft ist die Menge dieser Kombinationen noch sehr groß.
% \item In Praxis ist es meist nicht möglich alle Pareto-Effizienten Lösungen zu prüfen
% \end{itemize}
% $\Rightarrow$ SMS-EMOA soll möglichst guten Kompromiss zwischen Aufwand der Überprüfung der Paretoeffizienten Lösungen bei gleichzeitig umfassender Abdeckung möglicher Kompromisslösungen darstellen.
% \medskip

% $\Rightarrow$ SMS-EMOA ist einfach handhabbar und verzichtet auf Erstellung eines Archivs um den Aufwand zu reduzieren.
% \medskip

% $\Rightarrow$ Optimierung wird allein auf Grundlage der Population durchgeführt.

% \framebreak
% \end{frame}

\begin{frame}[allowframebreaks]{SMS-EMOA algorithm}

\vspace*{-0.5cm}

\begin{algorithm}[H]
  \begin{center}
  \caption{SMS-EMOA}
    \begin{algorithmic}[1]
    \begin{footnotesize}
    \State Generate start population $P_0$ of size $\mu$
    \State $t \leftarrow 0$
      \Repeat
        \State Generate \textbf{one} individual $\q \in \R^n$ by recombination and mutation of $\mathcal{P}_t$
        \State $\{F_{1},..., F_k\} \leftarrow \text{fast-dominated-sort}(P_{t}\cup \q)$
        \State $\bm{a}^\star \leftarrow \text{argmin}_{\bm{a} \in F_{k}}\Delta s(\bm{a}, F_{k})$
        \State $P_{t+1} \leftarrow (P_t \cup \{\q\}) \setminus\{\bm{a}^\star\}$
        \State $ t \leftarrow t+1$
      \Until{Termination criterion fulfilled}
    \end{footnotesize}
    \vspace*{-0.3cm}
    \end{algorithmic}
    \end{center}
\end{algorithm}

\vspace*{-0.6cm}
\footnotesize
\begin{itemize}
\item L5: the set of temporary $(\mu + 1)$ individuals is divided by \textbf{fast-dominated-sort} into $k$ fronts $F_{1},...,F_{k}$.
\item L6: determine individual $\bm{a}^\star \in F_{k}$ with smallest hypervolume contribution.
\item L7: the individual $\bm{a}^\star$ from the worst front with the smallest contribution to the dominated hypervolume does not survive.
\item The fitness of an individual is therefore primarily the rank of its associated front and secondarily its contribution to hypervolume.
\end{itemize}
\end{frame}


\end{frame}
\end{document}
