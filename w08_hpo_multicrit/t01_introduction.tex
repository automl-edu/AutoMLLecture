\pdfminorversion=4 % for acroread
\documentclass[aspectratio=169,t,xcolor={usenames,dvipsnames}]{beamer}
%\documentclass[t,handout,xcolor={usenames,dvipsnames}]{beamer}
\usepackage{../beamerstyle}
\usepackage{dsfont}
\usepackage{bm}
\usepackage[english]{babel}
\usepackage[utf8]{inputenc}
\usepackage{graphicx}
\usepackage{algorithm}
\usepackage[ruled,vlined,algo2e,linesnumbered]{algorithm2e}
%\usepackage[boxed,vlined]{algorithm2e}
\usepackage{hyperref}
\usepackage{booktabs}
\usepackage{mathtools}

\usepackage{amsmath,amssymb}
\usepackage{listings}
\lstset{frame=lines,framesep=3pt,numbers=left,numberblanklines=false,basicstyle=\ttfamily\small}

\usepackage{subfig}
\usepackage{multicol}
%\usepackage{appendixnumberbeamer}
%
\usepackage{tcolorbox}

\usepackage{pgfplots}
\usepackage{tikz}
\usetikzlibrary{trees} 
\usetikzlibrary{shapes.geometric}
\usetikzlibrary{positioning,shapes,shadows,arrows,calc,mindmap}
\usetikzlibrary{positioning,fadings,through}
\usetikzlibrary{decorations.pathreplacing}
\usetikzlibrary{intersections}
\usetikzlibrary{positioning,fit,calc,shadows,backgrounds}
\pgfdeclarelayer{background}
\pgfdeclarelayer{foreground}
\pgfsetlayers{background,main,foreground}
\tikzstyle{activity}=[rectangle, draw=black, rounded corners, text centered, text width=8em]
\tikzstyle{data}=[rectangle, draw=black, text centered, text width=8em]
\tikzstyle{myarrow}=[->, thick, draw=black]

% Define the layers to draw the diagram
\pgfdeclarelayer{background}
\pgfdeclarelayer{foreground}
\pgfsetlayers{background,main,foreground}

%\usepackage{listings}
%\lstset{numbers=left,
%  showstringspaces=false,
%  frame={tb},
%  captionpos=b,
%  lineskip=0pt,
%  basicstyle=\ttfamily,
%%  extendedchars=true,
%  stepnumber=1,
%  numberstyle=\small,
%  xleftmargin=1em,
%  breaklines
%}

 
\definecolor{blue}{RGB}{0, 74, 153}

\usetheme{Boadilla}
%\useinnertheme{rectangles}
\usecolortheme{whale}
\setbeamercolor{alerted text}{fg=blue}
\useoutertheme{infolines}
\setbeamertemplate{navigation symbols}{\vspace{-5pt}} % to lower the logo
\setbeamercolor{date in head/foot}{bg=white} % blue
\setbeamercolor{date in head/foot}{fg=white}
\setbeamercolor{author  in head/foot}{bg=white} %blue
\setbeamercolor{title in head/foot}{bg=white} % blue
\setbeamercolor{title}{fg=white, bg=blue}
\setbeamercolor{block title}{fg=white,bg=blue}
\setbeamercolor{block body}{bg=blue!10}
\setbeamercolor{frametitle}{fg=white, bg=blue}
\setbeamercovered{invisible}

\makeatletter
\setbeamertemplate{footline}
{
  \leavevmode%
  \hbox{%
  \begin{beamercolorbox}[wd=.333333\paperwidth,ht=2.25ex,dp=1ex,center]{author in head/foot}%
%    \usebeamerfont{author in head/foot}\insertshortauthor
  \end{beamercolorbox}%
  \begin{beamercolorbox}[wd=.333333\paperwidth,ht=2.25ex,dp=1ex,center]{title in head/foot}%
    \usebeamerfont{title in head/foot}\insertshorttitle
  \end{beamercolorbox}%
  \begin{beamercolorbox}[wd=.333333\paperwidth,ht=2.25ex,dp=1ex,right]{date in head/foot}%
    \usebeamerfont{date in head/foot}\insertshortdate{}\hspace*{2em}
%    \insertframenumber\hspace*{2ex} 
  \end{beamercolorbox}}%
  \vskip0pt%
}
\makeatother

%\pgfdeclareimage[height=1.2cm]{automl}{images/logos/automl.png}
%\pgfdeclareimage[height=1.2cm]{freiburg}{images/logos/freiburg}

%\logo{\pgfuseimage{freiburg}}

\newcommand{\comment}[1]{
	\noindent
	%\vspace{0.25cm}
	{\color{red}{\textbf{TODO:} #1}}
	%\vspace{0.25cm}
}
\renewcommand{\comment}[1]{}
\newcommand{\hide}[1]{}
\newcommand{\cemph}[2]{\emph{\textcolor{#1}{#2}}}

\newcommand{\lit}[1]{{\footnotesize\color{black!70}[#1]}}

\newcommand{\litw}[1]{{\footnotesize\color{black!20}[#1]}}


\newcommand{\myframe}[2]{\begin{frame}[c]{#1}#2\end{frame}}
\newcommand{\myframetop}[2]{\begin{frame}{#1}#2\end{frame}}
\newcommand{\myit}[1]{\begin{itemize}#1\end{itemize}}
\newcommand{\myblock}[2]{\begin{block}{#1}#2\end{block}}


\newcommand{\votepurple}[1]{\textcolor{Purple}{$\bigstar$}}
\newcommand{\voteyellow}[1]{\textcolor{Goldenrod}{$\bigstar$}}
\newcommand{\voteblue}[1]{\textcolor{RoyalBlue}{$\bigstar$}}
\newcommand{\votepink}[1]{\textcolor{Pink}{$\bigstar$}}

\newcommand{\diff}{\mathop{}\!\mathrm{d}}
\newcommand{\refstyle}[1]{{\small{\textcolor{gray}{#1}}}}
\newcommand{\hands}[0]{\includegraphics[height=1.5em]{images/hands}}
\newcommand{\transpose}[0]{{\textrm{\tiny{\sf{T}}}}}
\newcommand{\norm}{{\mathcal{N}}}
\newcommand{\cutoff}[0]{\kappa}
\newcommand{\instD}[0]{\dataset}
\newcommand{\insts}[0]{\mathcal{I}}
\newcommand{\inst}[0]{i}
\newcommand{\pcs}[0]{\mathbf{\Lambda}}
\newcommand{\bx}[0]{\conf}
\newcommand{\conf}[0]{\mathbf{\lambda}}
\newcommand{\defconf}[0]{\mathbf{\lambda}_{\text{def}}}
\newcommand{\finconf}[0]{\mathbf{\lambda}^*}
\newcommand{\incumbent}[0]{\finconf}
\newcommand{\confs}[0]{\pcs}
%\newcommand{\vlambda}[0]{\bm{\lambda}}
%\newcommand{\vLambda}[0]{\bm{\Lambda}}
\newcommand{\dataset}[0]{\mathcal{D}}
\newcommand{\datasets}[0]{\mathbf{D}}
\newcommand{\loss}[0]{\mathcal{L}}

% \renewcommand{\vec}[1]{\mathbf{#1}}
\newcommand{\hist}[0]{\mathcal{H}}
\newcommand{\param}[0]{p}
\newcommand{\algo}[0]{\mathcal{A}}
\newcommand{\algos}[0]{\mathbf{A}}
%\newcommand{\nn}[0]{N}
\newcommand{\feats}[0]{\mathcal{F}}
\newcommand{\feat}[0]{\vec{f}}
\newcommand{\cluster}[0]{\vec{h}}
\newcommand{\clusters}[0]{\vec{H}}
\newcommand{\perf}[0]{\mathbb{R}}
%\newcommand{\surro}[0]{\mathcal{S}}
\newcommand{\surro}[0]{\hat{f}}
\newcommand{\func}[0]{f}
\newcommand{\epm}[0]{\surro}
\newcommand{\portfolio}[0]{\mathcal{P}}
\newcommand{\schedule}[0]{\mathcal{S}}
\newcommand{\mdata}[0]{\dataset_{\text{meta}}}

% Deep Learning
\newcommand{\weights}[0]{\theta}
\newcommand{\metaweights}[0]{\phi}


% reinforcement learning
\newcommand{\policies}[0]{\Pi}
\newcommand{\policy}[0]{\pi}
\newcommand{\actionRL}[0]{a}
\newcommand{\stateRL}[0]{s}
\newcommand{\statesRL}[0]{\mathcal{S}}
\newcommand{\rewardRL}[0]{r}
\newcommand{\rewardfuncRL}[0]{\mathcal{R}}

\RestyleAlgo{algoruled}
\DontPrintSemicolon
\LinesNumbered
\SetAlgoVlined
\SetFuncSty{textsc}

\SetKwInOut{Input}{Input}
\SetKwInOut{Output}{Output}
\SetKw{Return}{return}

%\newcommand{\changed}[1]{{\color{red}#1}}

%\newcommand{\citeN}[1]{\citeauthor{#1}~(\citeyear{#1})}

\renewcommand{\vec}[1]{\mathbf{#1}}
\DeclareMathOperator*{\argmin}{arg\,min}
\DeclareMathOperator*{\argmax}{arg\,max}

\newcommand{\aqme}{\textit{AQME}}
\newcommand{\aslib}{\textit{ASlib}}
\newcommand{\llama}{\textit{LLAMA}}
\newcommand{\satzilla}{\textit{SATzilla}}
\newcommand{\satzillaY}[1]{\textit{SATzilla'{#1}}}
\newcommand{\snnap}{\textit{SNNAP}}
\newcommand{\claspfolioTwo}{\textit{claspfolio~2}}
\newcommand{\flexfolio}{\textit{FlexFolio}}
\newcommand{\claspfolioOne}{\textit{claspfolio~1}}
\newcommand{\isac}{\textit{ISAC}}
\newcommand{\eisac}{\textit{EISAC}}
\newcommand{\sss}{\textit{3S}}
\newcommand{\sunny}{\textit{Sunny}}
\newcommand{\ssspar}{\textit{3Spar}}
\newcommand{\cshc}{\textit{CSHC}}  
\newcommand{\cshcpar}{\textit{CSHCpar}}  
\newcommand{\measp}{\textit{ME-ASP}} 
\newcommand{\aspeed}{\textit{aspeed}}
\newcommand{\autofolio}{\textit{AutoFolio}}
\newcommand{\cedalion}{\textit{Cedalion}}
\newcommand{\fanova}{\textit{fANOVA}}
\newcommand{\sbs}{\textit{SB}}
\newcommand{\oracle}{\textit{VBS}}

% like approaches
\newcommand{\claspfoliolike}[1]{\texttt{claspfolio-#1-like}}
\newcommand{\satzillalike}[1]{\texttt{SATzilla'#1-like}}
\newcommand{\isaclike}{\texttt{ISAC-like}}
\newcommand{\ssslike}{\texttt{3S-like}}
\newcommand{\measplike}{\texttt{ME-ASP-like}}

\newcommand{\aspCoseal}{\textit{ASP-POTASSCO}}
\newcommand{\cspCoseal}{\textit{CSP-2010}}
\newcommand{\maxsatCoseal}{\textit{MAXSAT12-PMS}}
\newcommand{\premarCoseal}{\textit{PRE\-MARSHALLING}}
\newcommand{\qbfCoseal}{\textit{QBF-2011}}
\newcommand{\satallTwelveCoseal}{\textit{SAT12-ALL}}
\newcommand{\sathandTwelveCoseal}{\textit{SAT12-HAND}}
\newcommand{\satinduTwelveCoseal}{\textit{SAT12-INDU}}
\newcommand{\satrandTwelveCoseal}{\textit{SAT12-RAND}}
\newcommand{\sathandElevenCoseal}{\textit{SAT11-HAND}}
\newcommand{\satinduElevenCoseal}{\textit{SAT11-INDU}}
\newcommand{\satrandElevenCoseal}{\textit{SAT11-RAND}}
\newcommand{\proteusCoseal}{\textit{PROTEUS-2014}}

\newcommand{\irace}{\textit{I/F-race}}
\newcommand{\gga}{\textit{GGA}}
\newcommand{\smac}{\textit{SMAC}}
\newcommand{\paramils}{\textit{ParamILS}}
\newcommand{\spearmint}{\textit{Spearmint}}
\newcommand{\tpe}{\textit{TPE}}

\newcommand{\gringo}{\textit{gringo}}
\newcommand{\clasp}{\textit{clasp}}
\newcommand{\lingeling}{\textit{lingeling}}

\newcommand{\hydra}{\textit{Hydra}}

\newcommand{\plingeling}{\textit{Plingeling}}
\newcommand{\ccasat}{\textit{CCASat}}

\usepackage{pifont}
\newcommand{\itarrow}{\mbox{\Pisymbol{pzd}{229}}}
\newcommand{\ithook}{\mbox{\Pisymbol{pzd}{52}}}
\newcommand{\itcross}{\mbox{\Pisymbol{pzd}{56}}}
\newcommand{\ithand}{\mbox{\raisebox{-1pt}{\Pisymbol{pzd}{43}}}}

%\DeclareMathOperator*{\argmax}{arg\,max}

\newcommand{\ie}{{\it{}i.e.\/}}
\newcommand{\eg}{{\it{}e.g.\/}}
\newcommand{\cf}{{\it{}cf.\/}}
\newcommand{\wrt}{\mbox{w.r.t.}}
\newcommand{\vs}{{\it{}vs\/}}
\newcommand{\vsp}{{\it{}vs\/}}
\newcommand{\etc}{{\copyedit{etc.}}}
\newcommand{\etal}{{\it{}et al.\/}}

\newcommand{\pscProc}{{\bf procedure}}
\newcommand{\pscBegin}{{\bf begin}}
\newcommand{\pscEnd}{{\bf end}}
\newcommand{\pscEndIf}{{\bf endif}}
\newcommand{\pscFor}{{\bf for}}
\newcommand{\pscEach}{{\bf each}}
\newcommand{\pscThen}{{\bf then}}
\newcommand{\pscElse}{{\bf else}}
\newcommand{\pscWhile}{{\bf while}}
\newcommand{\pscIf}{{\bf if}}
\newcommand{\pscRepeat}{{\bf repeat}}
\newcommand{\pscUntil}{{\bf until}}
\newcommand{\pscWithProb}{{\bf with probability}}
\newcommand{\pscOtherwise}{{\bf otherwise}}
\newcommand{\pscDo}{{\bf do}}
\newcommand{\pscTo}{{\bf to}}
\newcommand{\pscOr}{{\bf or}}
\newcommand{\pscAnd}{{\bf and}}
\newcommand{\pscNot}{{\bf not}}
\newcommand{\pscFalse}{{\bf false}}
\newcommand{\pscEachElOf}{{\bf each element of}}
\newcommand{\pscReturn}{{\bf return}}

%\newcommand{\param}[1]{{\sl{}#1}}
\newcommand{\var}[1]{{\it{}#1}}
\newcommand{\cond}[1]{{\sf{}#1}}
%\newcommand{\state}[1]{{\sf{}#1}}
%\newcommand{\func}[1]{{\sl{}#1}}
\newcommand{\set}[1]{{\Bbb #1}}
%\newcommand{\inst}[1]{{\tt{}#1}}
\newcommand{\myurl}[1]{{\small\sf #1}}

\newcommand{\Nats}{{\Bbb N}}
\newcommand{\Reals}{{\Bbb R}}
\newcommand{\extset}[2]{\{#1 \; | \; #2\}}

\newcommand{\vbar}{$\,\;|$\hspace*{-1em}\raisebox{-0.3mm}{$\,\;\;|$}}
\newcommand{\vendbar}{\raisebox{+0.4mm}{$\,\;|$}}
\newcommand{\vend}{$\,\:\lfloor$}


\newcommand{\goleft}[2][.7]{\parbox[t]{#1\linewidth}{\strut\raggedright #2\strut}}
\newcommand{\rightimage}[2][.3]{\mbox{}\hfill\raisebox{1em-\height}[0pt][0pt]{\includegraphics[width=#1\linewidth]{#2}}\vspace*{-\baselineskip}}





%The following might look confusing but allows us to switch the notation of the optimization problem independently from the notation of the hyper parameter optimization
\newcommand{\xx}{\conf} %x of the optimizer
\newcommand{\xxi}[1][i]{\lambda_{#1}} %i-th component of xx (not confuse with i-th individual)
\newcommand{\XX}{\pcs} %search space / domain of f
\newcommand{\f}{\cost} %objective function
\newcommand{\yy}{\cost} %outcome of objective function

\title[AutoML: Overview]{Multi-criteria Optimization}
\subtitle{Introduction}
\author[Bernd Bischl]{\underline{Bernd Bischl} \and Frank Hutter \and Lars Kotthoff\newline \and Marius Lindauer \and Joaquin Vanschoren}
\institute{}
\date{}



% \AtBeginSection[] % Do nothing for \section*
% {
%   \begin{frame}{Outline}
%     \bigskip
%     \vfill
%     \tableofcontents[currentsection]
%   \end{frame}
% }

\begin{document}

	\maketitle



\begin{frame}[allowframebreaks]{Introductory example}

Often we want to solve optimization problems concerning several goals.

    \vspace{0.5cm}
    \textbf{General applications:}
\begin{itemize}
\item Medicine: maximum effect, but minimum side effect of a drug.
\item Finances: maximum return, but minimum risk of an equity portfolio.
\item Production planning: maximum revenue, but minimum costs.
\item Booking a hotel: maximum rating, but minimum costs.
\end{itemize}

\vspace{0.5cm}
    \textbf{In machine learning:}
\begin{itemize}
\item Sparse models: maximum predictive performance, but minimal number of features.
\item Fast models: maximum predictive performance, but short prediction time.
\item ...
\end{itemize}

%A \textit{simple} approach would be to formulate all but one objective function simplified as a secondary condition.

\vspace*{0.2cm}

\framebreak

\textbf{Example}:

Choose the best hotel to stay at by maximizing ratings subject to a maximum price per night.

\vspace*{0.5cm}

 \textbf{Problems}:

\begin{itemize}
 \item The result depends on how we select the maximum price and usually returns different solutions for different maximum price values.
 \item We could also choose a minimum rating and optimize the price per night.
 \item The more objectives we optimize, the more difficult such a definition becomes.
\end{itemize}

\vspace*{0.5cm}

\textbf{Goal}:

Find a more general approach to solve multi-criteria problems.


\begin{center}
\includegraphics[width = 0.35\linewidth]{images/booking1.png} ~~~ \includegraphics[width = 0.35\linewidth]{images/booking2.png}
\end{center}

When booking a hotel: find the hotel with

\begin{itemize}
\item minimum price per night (\textbf{costs}) and
\item maximum user rating (\textbf{performance}).
\end{itemize}

\vfill

\begin{footnotesize}
Since our standard is to minimize objectives, we minimize negative ratings.
\end{footnotesize}

\framebreak

The objectives often conflict with each other:

\begin{itemize}
\item Lower price $\to$ usually lower hotel rating.
\item Better rating $\to$ usually higher price.
\end{itemize}

Example: (negative) average rating by hotel guests (1 - 5) vs. average price per night (excerpt).

\vspace*{0.2cm}

\begin{center}
\includegraphics[scale=1]{images/expedia-1-1}
\end{center}

\framebreak

Often, objectives are not directly comparable as they are measured on different scales:

\begin{itemize}
    \item Left: A hotel with rating $4$ for $89$ Euro ($\textcolor{green}{\cost^{(1)}} = \left(89, - 4.0\right)$) would be preferred to a hotel for $108$ Euro with the same rating ($\textcolor{red}{\cost^{(2)}} = \left(108, - 4.0\right)$).
\item Right: How to decide if $\textcolor{orange}{\cost^{(1)}} = \left(89, - 4.0\right)$ or $\textcolor{orange}{\cost^{(1)}} = \left(95, - 4.5\right)$ is preferred?
\item How much is one \textit{rating point} worth?
\end{itemize}

\centering \includegraphics[scale=1]{images/expedia-2-1}

\end{frame}


\begin{frame}{Definition: multi-criteria optimization problem}

A \textbf{multi-criteria optimization problem} is defined by

$$
\min_{\conf \in \pcs}  \cost(\conf) \Leftrightarrow \min_{\conf \in \pcs} \left(\cost_1(\conf), \cost_2(\conf), ..., \cost_m(\conf)\right),
$$

with $\pcs \subset \realnum^n$ and multi-criteria objective function $\cost: \pcs \to \realnum^m$, $m \ge 2$.

\begin{itemize}
\item \textbf{Goal:} minimize multiple target functions simultaneously.
\item $\left(\cost_1(\conf), ..., \cost_m(\conf)\right)^\top$ maps each candidate $\conf$ into the objecive space $\realnum^m$.
\item Often no clear best solution, as objective are usually conflicting and we cannot totally order in $\realnum^m$. 
% \item Objective functions are often conflicting.
\item W.l.o.g. we always minimize.
\item Alternative names: multi-criteria optimization, multi-objective optimization, Pareto optimization.
\end{itemize}

\end{frame}

\begin{frame}{Pareto sets and Pareto optimality}

\textbf{Definition:}

Given a multi-criteria optimization problem
    $$\min_{\conf \in \pcs} \left(\cost_1(\conf), ..., \cost_m(\conf)\right), \quad \cost_i: \pcs \to \realnum.$$

\begin{itemize}
    \item A candidate $\conf^{(1)}$ \textbf{(Pareto-) dominates} $\conf^{(2)}$, if $\cost(\conf^{(1)}) \prec \cost(\conf^{(2)})$, i.e.
\begin{enumerate}
    \item $\cost_i(\conf^{(1)}) \le \cost_i(\conf^{(2)})$ for all $i \in \{1, 2, ..., m\}$ and
    \item $\cost_j(\conf^{(1)}) < \cost_j(\conf^{(2)})$ for at least one $j \in \{1, 2, ..., m\}$
\end{enumerate}
\vspace*{0.1cm}
\item A candidate $\optconf$ that is not dominated by any other candidate is called \textbf{Pareto optimal}.
\vspace*{0.1cm}
\item The set of all Pareto optimal candidates is called \textbf{Pareto set} $\mathcal{P} := \{\conf \in \pcs |\not \exists ~\tilde{\conf} \text{ with } \cost(\tilde{\conf}) \prec \cost(\conf)\}$
\item $\mathcal{F} = \cost(\mathcal{P}) = \{\cost(\conf) | \conf \in \mathcal{P}\}$ is called \textbf{Pareto front}.
\end{itemize}

\end{frame}


\begin{frame}[allowframebreaks]{How to define optimality?}

Let $\cost = (\text{price}, - \text{rating})$. For some cases it is \textit{clear} which point is the better one:

\begin{itemize}
    \item The candidate $\textcolor{green}{\cost^{(1)}} = \left(89, - 4.0\right)$ dominates $\textcolor{red}{\cost^{(2)}} = \left(108, - 4.0\right)$: $\textcolor{green}{\cost^{(1)}}$ is not worse in any dimension and is better in one dimension. Therefore, $\textcolor{red}{\cost^{(2)}}$ gets \textbf{dominated} by $\textcolor{green}{\cost^{(1)}}$
$$
\textcolor{red}{\cost^{(2)}} \prec \textcolor{green}{\cost^{(1)}}.
$$
\end{itemize}

\centering \includegraphics[width=0.5\linewidth]{images/expedia-3-1}

\framebreak

For the points $\textcolor{orange}{\cost^{(1)}} = \left(89, - 4.0\right)$ and $\textcolor{orange}{\cost^{(2)}} = \left(95, - 4.5\right)$ we cannot say which one is better.

\begin{itemize}
\item We define the points as \textbf{equivalent} and write

$$
\textcolor{orange}{\cost^{(1)}} \not\prec \textcolor{orange}{\cost^{(2)}} \text{ and } \textcolor{orange}{\cost^{(2)}} \not\prec \textcolor{orange}{\cost^{(1)}}.
$$

\centering \includegraphics[width=0.5\linewidth]{images/expedia-4-1}


\item The set of all equivalent points that are not dominated by another point is called the \textbf{Pareto front}.

\vspace*{0.3cm}

\centering \includegraphics[width=0.8\linewidth]{images/expedia-5-1}
%FIXME: JR: I would prefer the pareto front as a step diagram, as this would be in accordance to the HV. Is the code to generate these plots checked in?

\end{itemize}

\end{frame}

\begin{frame}{Example: One objective function}


We consider the minimization problem

$$
\min_{\conf} \cost(\conf) = (\conf - 1)^2, \qquad 0 \le \conf \le 3.
$$

The optimum is at $\optconf = 1$.

\vspace*{0.1cm}


\centering \includegraphics[scale=0.2]{images/graph1}


\end{frame}

\begin{frame}[allowframebreaks]{Example: Two target functions}

We extend the above problem to two objective functions $\cost_1(\conf) = (\conf - 1)^2$ and $\cost_2(\conf) = 3(\conf - 2)^2$, thus

$$
    \min_{\conf} \cost(\conf) = \left(\cost_1(\conf), \cost_2(\conf)\right), \qquad 0 \le \conf \le 3.
$$

    \begin{center}
    \includegraphics[scale=0.2]{images/graph2}
    \end{center}

\framebreak

We consider the functions in the objective function space $\cost(\pcs)$ by drawing the objective function values $\left(\cost_1(\conf), \cost_2(\conf)\right)$ for all $0 \le \conf \le 3$.

\vspace*{0.1cm}


    \begin{center}
    \includegraphics[scale=0.2]{images/graph3}
    \end{center}
    \vspace*{-0.3cm}

The Pareto front is shown in green.
    The Pareto front cannot be \emph{left} without getting worse in at least one objective function.

\end{frame}


\begin{frame}{A-priori vs. A-posteriori}

\begin{itemize}
\item The Pareto set is a set of equally optimal solutions.
\item In many applications one is often interested in a \textbf{single} optimal solution.
\item Without further information no unambiguous optimal solution can be determined. \\
$\to$ The decision must be based on other criteria.
\end{itemize}

    \vspace{0.5cm}

There are two possible approaches:
\begin{itemize}
\item \textbf{A-priori approach}: User preferences are considered \textbf{before} the optimization process
\item \textbf{A-posteriori approach}: User preferences are considered \textbf{after} the optimization process
\end{itemize}

\end{frame}

\begin{frame}[allowframebreaks]{A-priori procedure}

\textbf{Example: Weighted total}


\textbf{Prior knowledge:} One rating point is worth $50$ Euro to a customer. \\
    $\to$ We optimize the weighted sum:

$$
\min_\text{Hotel} \text{(Price / Night)} - 50 \cdot \text{Rating}
$$

    \begin{center}
\includegraphics[scale=0.555555]{images/expedia-9-1}
    \end{center}

Alternative a weighted sum: $\min_{\conf \in \pcs} \sum_{i = 1}^m w_i \cost_i(\conf) \qquad \text{with} \quad w_i \ge 0 $

\framebreak

\textbf{Example: Lexicographic method}

\textbf{Prior knowledge:} Customer prioritizes rating over price. \\
$\to$ Optimize target functions one after the other.


    \begin{center}
\includegraphics[scale=1]{images/expedia-10-1}
    \end{center}

\framebreak

A-priori approach: Lexicographic method

\begin{eqnarray*}
\yy_1^* &=& \min_{\conf \in \pcs} \cost_1(\conf)\\
\yy_2^* &=& \min_{\conf \in \{\conf ~|~ \cost_1(\conf) = \yy_1^*\}} \cost_2(\conf) \\
\yy_3^* &=& \min_{\conf \in \{\conf ~|~ \cost_1(\conf) = \yy_1^* \land \cost_2(\conf) = \yy_2^*\}} \cost_3(\conf) \\
&\vdots&
\end{eqnarray*}

    \textbf{But:} Different sequences provide different solutions.

\framebreak

\textbf{Summary a-priori approach:}
\begin{itemize}
\item Implicit assumption: Single-objective optimization is \emph{easy}.
\item Only one solution is obtained, which depends on a-priori weights, order, etc.
\item Several solutions can be obtained if weights, order, etc. are systematically varied.
\item Usually not all non-dominated candidates can be found by these methods.
\end{itemize}

\end{frame}

\begin{frame}[allowframebreaks]{A-posteriori procedure}

A-posteriori methods try to

\begin{itemize}
\item find the set of \textbf{all} optimal candidates (the Pareto set),
\item select (if necessary) an optimal candidate based on prior knowledge or individual preferences.
\item Implicit assumption: Specifying your hidden preferences / making a selection from a pool of candidates is easier, if you see the non-dominated solutions.
\end{itemize}

A-posteriori methods are therefore the more generic approach to solving a multi-criteria optimization problem.


\framebreak

\textbf{Example:} A user is displayed all Pareto optimal hotels (left) and chooses an optimal candidate (right) based on his hidden preferences or additional criteria (e.g. location of the hotel).

\vspace*{0.1cm}


\centering \includegraphics[scale=1]{images/expedia-11-1}


\end{frame}


\begin{frame}[allowframebreaks]{Evaluation of solutions}


\begin{columns}
\begin{column}{0.5\textwidth}

A common metric for evaluating the performance of a set of candidates $\mathcal{P} \subset \pcs$ is the \textbf{dominated hypervolume}
$$
    S(\mathcal{P}, R) = \Lambda\left(\bigcup_{\tilde{\conf} \in \mathcal{P}}\left\{\conf | \tilde{\conf} \prec \conf \prec R\right\}\right),
$$
where $\Lambda$ is the Lebuesge measure.
\end{column}
\begin{column}{0.5\textwidth}
\begin{center}
\includegraphics[width=0.8\textwidth]{images/dominated_hypervolume.png}
\end{center}
\end{column}
\end{columns}
    

\framebreak

\begin{itemize}
            \item HV is calculated w.r.t the reference point $R$, which often reflects in each component the natural maximum of the respective objective -- if possible
            \item The dominated hypervolume is also often called \textbf{S-Metric}.
            \item Computation of HV scales exponentially in the number of objective functions $\mathcal{O}(n^{m-1})$.
            \item Fast approximations exist for small values of $m$ and especially for machine learning applications we rarely optimize $m > 3$ objectives.
    \end{itemize}

\end{frame}

\end{document}
