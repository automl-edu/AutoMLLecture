\documentclass{exam}
\usepackage{amsmath, amsfonts}
\usepackage{verbatim}
\usepackage{graphicx}
\usepackage[super]{nth}

\DeclareMathOperator*{\argmin}{argmin}

\usepackage[hyperfootnotes=false]{hyperref}

\usepackage[usenames,dvipsnames]{color}
\newcommand{\note}[1]{
	\noindent~\\
	\vspace{0.25cm}
	\fcolorbox{Red}{Orange}{\parbox{0.99\textwidth}{#1\\}}
	%{\parbox{0.99\textwidth}{#1\\}}
	\vspace{0.25cm}
}


\renewcommand{\vec}[1]{\mathbf{#1}}
\newcommand{\lecture}{ML4AAD}
\newcommand{\lecturelong}{Machine Learning for Automated Algorithm Design}
\newcommand{\semester}{WS 2018/19}
\newcommand{\assignment}[1]{\nth{#1} Assignment}
\newcommand{\lectors}{M. Lindauer \& A. Biedenkapp}
\newcommand{\hide}[1]{}


\newcommand{\gccs}{\paragraph{General constraints for code submissions}
    
    \begin{itemize}
        \item The program can be called as stated on the exercise sheet.
        \item The program exactly returns the required output (neither less nor more) -- please use a \texttt{--verbose} option to increase the verbosity level for debugging\footnote{You might want to use \texttt{argparse} for simplicity's sake.}.
        \item Your scripts should be commented to be readable for the tutors. All functions and classes are documented with a docstring. 
        \item Provide a README ($\to$ how to install requirements and run your program(s)) and (if necessary) an installation script if your program requires any other packages.
        \item Programs are to be submitted in python $3.5$ or newer.
        \item Adding new packages to the requirements.txt is fine. If you do this however, you'll have to give a brief description why you use that package and a link to it's documentation or github page.
        \item All prepared unittests have to pass.
        \item We don't accept ipython notebook submissions.
        \item Points will be deducted if you don't fullfill these constraints.
        \item You are allowed (sometimes required) to reuse code from previous exercises.
    \end{itemize}
    \rule{\textwidth}{.5pt}
    \smallskip\\
    \noindent}
%\renewcommand{\hide}[1]{#1}

\qformat{\thequestion. \textbf{\thequestiontitle}\hfill[\thepoints]}
\bonusqformat{\thequestion. \textbf{\thequestiontitle}\hfill[\thepoints]}

\pagestyle{headandfoot}

%%%%%% MODIFY FOR EACH SHEET!!!! %%%%%%
\newcommand{\duedate}{18.11.18 (14:00)}
\newcommand{\due}{{\bf This assignment is due on \duedate.} }
\firstpageheader
{Due: \duedate \\ Points: 9}
{{\bf\lecture}\\ \assignment{5}}
{\lectors\\ \semester}

\runningheader
{Due: \duedate}
{\assignment{5}}
{\semester}
%%%%%% MODIFY FOR EACH SHEET!!!! %%%%%%

\firstpagefooter
{}
{\thepage}
{}

\runningfooter
{}
{\thepage}
{}

\headrule
\pointsinrightmargin
\bracketedpoints
\marginpointname{pt.}


\begin{document}
	\gccs
	
	Having learned about different ways to empirically evaluate the performances of algorithms you will have to implement some of these techniques to compare the performance of different configurations on provided data (\texttt{read\_data.py}). You will have to add your code to \texttt{main.py}, however if you run \texttt{python read\_data.py} you will be presented some statistics about the data.
	\begin{questions}
		\titledquestion{Empirical Evaluations}[9]
		We have provided you with data of 1000 configurations that have been evaluated on 604 instances and a script with which you can load the data in an easy to use format.
		\begin{parts}
			\part[1] Your first task is to implement a paired permutation test to determine if \texttt{conf\_0} and \texttt{conf\_1} have equal performance.
			\part[2] Complete \texttt{scatter\_and\_box} to compare both configurations using scatter plots and box plots. Timeouts should be highlighted in the scatter plot.
			\part[2] Implement the cdf-plots as presented in the lecture.
			\part[1] Complete \texttt{feature\_scatter} to visually determine if there are multiple clusters in the instance set. You are only required to scatter plot the data.
			\part[2] Add all plots (comparing \texttt{conf\_0} and \texttt{conf\_1}) to a pdf and \textit{interpret} the results.
		\end{parts}
	
	\titledquestion{Feedback}[Bonus: 1]
	For each question in this assignment, state:
	\begin{itemize}
	\item How long you worked on it.
	\item What you learned.
	\item Anything you would improve in this question if you were teaching the course.
	\end{itemize}
	\end{questions}
	
	\noindent
	\due Submit your solution for the tasks by uploading a PDF to your groups BitBucket repository. The PDF has to include the name of the submitter(s). Teams of at most $2$ students are allowed.
\end{document}