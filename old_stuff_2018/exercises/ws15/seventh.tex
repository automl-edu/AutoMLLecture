\documentclass{exam}
\usepackage{amsmath,amssymb,amsthm,mathrsfs,amsfonts,dsfont}
\usepackage{verbatim}
\usepackage{graphicx}


\usepackage[hyperfootnotes=false]{hyperref}

\usepackage[usenames,dvipsnames]{color}
\newcommand{\note}[1]{
	\noindent~\\
	\vspace{0.25cm}
	\fcolorbox{Red}{Orange}{\parbox{0.99\textwidth}{#1\\}}
	%{\parbox{0.99\textwidth}{#1\\}}
	\vspace{0.25cm}
}
\renewcommand{\note}[1]{}
\newcommand{\hide}[1]{#1}
\renewcommand{\hide}[1]{}

\renewcommand{\vec}[1]{\mathbf{#1}}
\DeclareMathOperator*{\argmin}{argmin}

\qformat{\thequestion. \textbf{\thequestiontitle}\hfill[\thepoints]}
\bonusqformat{\thequestion. \textbf{\thequestiontitle}\hfill[\thepoints]}

\pagestyle{headandfoot}
\firstpageheader{Due: 04.02.2016 (23:59 GMT)}{ {\bf MLOAD} \\ Seventh Assignment}{M. Lindauer \& F. Hutter\\ WS 2015/16}
\runningheader{Due: 04.02.2016 (23:59 GMT)}{Seventh Assignment}{WS 2015/16}
\runningfooter{}{}{}
\headrule
\pointsinrightmargin
\bracketedpoints
\marginpointname{pt.}


\begin{document}

After you now know how to use algorithm configuration,
your next task is to configure the SAT solver \textit{Spear} to optimize its performance.   

\bigskip

For this exercise assignment, 
we provide a configuration scenarios at ILIAS (\texttt{spear-scenario.tar.gz}).
The scenario consists of the following directories
\begin{itemize}
  \item \texttt{spear}: a directory with the binary of \textit{Spear} and a basic framework for a algorithm wrapper (called genericWrapper)
  \item \texttt{instances}: a directory with instances from software verification
  \item \texttt{runsolver}: a directory with the binary of the runsolver (compiled for Ubuntu 14.04; please download the source and recompile it if it does not run on your machine\footnote{\url{http://www.cril.univ-artois.fr/~roussel/runsolver/}} -- runsolver is only compatible with Linux.)
\end{itemize}

\textbf{To solve this assignment, we recommend to use the virtual machine (VM) that we used in the hands-on session. Furthermore, please note that we will solve most of this exercise live on the 28.01.} 

\begin{questions}

\titledquestion{Configuration of \textit{Spear}}[50]

Given the above mentioned files, your task is to optimize the performance of \textit{Spear} on the provided instances with \textit{SMAC}\footnote{Either use the installed SMAC in the environment used in the hands-on session or download it at \url{http://www.cs.ubc.ca/labs/beta/Projects/SMAC/}}. 
To do so, please follow the following steps:

\begin{itemize}
  \item Evenly split the instances in a training (\texttt{train.txt}) and test set (\texttt{test.txt}) -- write a bash or Python script to do it [5 points] 
  \item Complete the Python script \texttt{spearWrapper.py}; the call to \textit{Spear} should look like [10 points]:\\
  		\texttt{spear/Spear-32\_1.2.1 --seed <int> --model-stdout --dimacs <instance> --param\_1 value\_1}
  \item Write a PCS file (\texttt{spear/pcs.txt}) for the \emph{search parameters} of \textit{Spear} [15 points] -- look at\\ \texttt{./spear/Spear-32\_1.2.1 --hidden} to see the list of parameters and its domains.   		 
  \item Write a scenario file (\texttt{scenario.txt}) that defines the following characteristics of the configuration scenario [5 points]
  \begin{itemize}
    \item the algorithm should be called like: \texttt{python -u spear/SATCSSCWrapper.py --mem-limit 1024 --script spear/spearWrapper.py}
    \item use the above specified files (\texttt{pcs.txt}, \texttt{train.txt} and \texttt{test.txt})
    \item the algorithm is non-deterministic
    \item optimize runtime
    \item the cutoff time time will be $300$ seconds
    \item the configuration budget will be $10000$ seconds
  \end{itemize}
  You can call \texttt{smac --help} to see all required option names of \textit{SMAC} 
  \item Run \textit{SMAC} with the above defined scenario and report the performance of the configured \textit{Spear} on the test instances [$\to$ PDF] [10 points]
  \item Validate the default configuration of \textit{Spear} on the test instances -- you can use\\
  		\texttt{smac-validate --includeDefaultAsFirstRandom true --random-configurations 1  --numRun 1 --scenarioFile scenario.txt}\\
  		You will find the mean performance in the file \texttt{validationResults-cli-1-walltime.csv} (third column).
  		Report the performance of the default configuration [$\to$ PDF] \mbox{[5 points]}
\end{itemize}

\titledquestion{Feedback}[Bonus: 5]
For each question in this assignment, state:
\begin{itemize}
	\item How long you worked on it.
	\item What you learned.
	\item Anything you would improve in this question if you were teaching the course.
\end{itemize}

\end{questions}


% {\bf This assignment is due on 16.11.2015 (23:59 GMT).} Submit your solution for the tasks by uploading a PDF to our ILIAS\footnote{ \url{https://ilias.uni-freiburg.de/goto.php?target=crs_465155&client_id=unifreiburg}.} course page. The PDF has to include the name of the submitter(s). Teams of at most $2$ students are allowed. Everyone has to submit his/her solution.\\

% Please note that this assignment is optional and you can get bonus points. However, we strongly encourage you to solve the given tasks since we will solve such CSP problems in the next exercise and you will benefit from some experience with modelling and understanding CSP problems. 

\noindent
{\bf This assignment is due on 04.02.2016 (23:59 GMT).} Submit your solution for the tasks by uploading an archive (tar.gz) to our ILIAS\footnote{ \url{https://ilias.uni-freiburg.de/goto.php?target=crs_465155&client_id=unifreiburg}.} course page. The archive has to include the name of the submitter(s). Teams of at most $2$ students are allowed. Everyone has to submit his/her solution. 

\bigskip
\paragraph{General constraints for code submissions}

\begin{itemize}
  \item The program can be called as stated on the exercise sheet.
  \item The program exactly returns the required output (neither less nor more) -- please use a \texttt{--verbose} option to increase the verbosity level for debugging.
  \item Your scripts should be commented to be readable for the tutors. All functions and classes are documented with a docstring. 
  \item Provide a README ($\to$ how to install requirements and run your program(s)) and (if necessary) an installation script if your program requires any other packages.
\end{itemize}

\bigskip

Submissions will get $0$ points if they do not satisfy these constraints.

\end{document}