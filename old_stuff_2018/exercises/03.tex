\documentclass{exam}
\usepackage{amsmath, amsfonts}
\usepackage{verbatim}
\usepackage{graphicx}
\usepackage[super]{nth}

\DeclareMathOperator*{\argmin}{argmin}

\usepackage[hyperfootnotes=false]{hyperref}

\usepackage[usenames,dvipsnames]{color}
\newcommand{\note}[1]{
	\noindent~\\
	\vspace{0.25cm}
	\fcolorbox{Red}{Orange}{\parbox{0.99\textwidth}{#1\\}}
	%{\parbox{0.99\textwidth}{#1\\}}
	\vspace{0.25cm}
}


\renewcommand{\vec}[1]{\mathbf{#1}}
\newcommand{\lecture}{ML4AAD}
\newcommand{\lecturelong}{Machine Learning for Automated Algorithm Design}
\newcommand{\semester}{WS 2018/19}
\newcommand{\assignment}[1]{\nth{#1} Assignment}
\newcommand{\lectors}{M. Lindauer \& A. Biedenkapp}
\newcommand{\hide}[1]{}


\newcommand{\gccs}{\paragraph{General constraints for code submissions}
    
    \begin{itemize}
        \item The program can be called as stated on the exercise sheet.
        \item The program exactly returns the required output (neither less nor more) -- please use a \texttt{--verbose} option to increase the verbosity level for debugging\footnote{You might want to use \texttt{argparse} for simplicity's sake.}.
        \item Your scripts should be commented to be readable for the tutors. All functions and classes are documented with a docstring. 
        \item Provide a README ($\to$ how to install requirements and run your program(s)) and (if necessary) an installation script if your program requires any other packages.
        \item Programs are to be submitted in python $3.5$ or newer.
        \item Adding new packages to the requirements.txt is fine. If you do this however, you'll have to give a brief description why you use that package and a link to it's documentation or github page.
        \item All prepared unittests have to pass.
        \item We don't accept ipython notebook submissions.
        \item Points will be deducted if you don't fullfill these constraints.
        \item You are allowed (sometimes required) to reuse code from previous exercises.
    \end{itemize}
    \rule{\textwidth}{.5pt}
    \smallskip\\
    \noindent}
%\renewcommand{\hide}[1]{#1}

\qformat{\thequestion. \textbf{\thequestiontitle}\hfill[\thepoints]}
\bonusqformat{\thequestion. \textbf{\thequestiontitle}\hfill[\thepoints]}

\pagestyle{headandfoot}

%%%%%% MODIFY FOR EACH SHEET!!!! %%%%%%
\newcommand{\duedate}{4.11.18 (14:00)}
\newcommand{\due}{{\bf This assignment is due on \duedate.} }
\firstpageheader
{Due: \duedate \\ Points: 4}
{{\bf\lecture}\\ \assignment{3}}
{\lectors\\ \semester}

\runningheader
{Due: \duedate}
{\assignment{3}}
{\semester}
%%%%%% MODIFY FOR EACH SHEET!!!! %%%%%%

\firstpagefooter
{}
{\thepage}
{}

\runningfooter
{}
{\thepage}
{}

\headrule
\pointsinrightmargin
\bracketedpoints
\marginpointname{pt.}


\begin{document}
	\gccs
	
	For this exercise assignment, 
	you are required to use the same ASlib scenarios as on the last sheet.
	Where necessary, you can also reuse code from the last exercise.
	\begin{questions}
		
		\titledquestion{Algorithm Selection}[4]
		
		Your task is to implement three of the presented algorithm selection approaches. You can choose between:
		
		\begin{itemize}
			\item regression model for each algorithm,
			\item pairwise regression models to predict the performance difference,
			\item cost-sensitive pairwise classification with voting,
			\item clustering,
			\item and $k$-nearest neighbour.
		\end{itemize} 
		
		For the underlying machine learning techniques, you should use again \texttt{sklearn},
		e.g., \texttt{sklearn.linear\_model.Ridge} for regression,
		\texttt{sklearn.ensemble.RandomForestClassifier} for classification,
		\texttt{sklearn.cluster.KMeans} for clustering
		and \texttt{sklearn.neighbors.KNeighborsClassifier} for kNN.
		
		To assess the performance of your algorithm selector,
		you should use the provided cross-validation splits in cv.arff.
		Accordingly, the call of your algorithm selector should look like this:
		
		\begin{verbatim}
		python aslib_selection.py --algoruns algorithm_runs.arff --features feature_values.arff\
		--cv cv.arff
		\end{verbatim}
		
		In the end, your algorithm selector should output the cross-validated PAR$10$ performance
		and the selected algorithm on each instance in the following format:
		
		\begin{verbatim}
		PAR10 performance: 1.2345
		Selection per instance:
		instance1.cnf, algo2
		instance2.cnf, algo3
		instance3.cnf, algo1
		...
		\end{verbatim}
		
		Please report the cross-validated performance ($\to$ PDF)
		and how you handled the hyper-parameters of your approach.
		We note that you will probably only get satisfactory results if you impute missing feature values (e.g., \texttt{sklearn.preprocessing.Imputer})
		and scale the instance features\\ (e.g., \texttt{sklearn.preprocessing.MinMaxScaler}) -- depending on your used approach.
		
		\titledquestion{Feedback}[Bonus: 1]
		For each question in this assignment, state:
		\begin{itemize}
			\item How long you worked on it.
			\item What you learned.
			\item Anything you would improve in this question if you were teaching the course.
		\end{itemize}
	\end{questions}
	
	\noindent
	\due Submit your solution for the tasks by uploading a PDF to your groups BitBucket repository. The PDF has to include the name of the submitter(s). Teams of at most $2$ students are allowed.
\end{document}