\documentclass{exam}
\usepackage{amsmath, amsfonts}
\usepackage{verbatim}
\usepackage{graphicx}
\usepackage[super]{nth}

\DeclareMathOperator*{\argmin}{argmin}

\usepackage[hyperfootnotes=false]{hyperref}

\usepackage[usenames,dvipsnames]{color}
\newcommand{\note}[1]{
	\noindent~\\
	\vspace{0.25cm}
	\fcolorbox{Red}{Orange}{\parbox{0.99\textwidth}{#1\\}}
	%{\parbox{0.99\textwidth}{#1\\}}
	\vspace{0.25cm}
}


\renewcommand{\vec}[1]{\mathbf{#1}}
\newcommand{\lecture}{ML4AAD}
\newcommand{\lecturelong}{Machine Learning for Automated Algorithm Design}
\newcommand{\semester}{WS 2018/19}
\newcommand{\assignment}[1]{\nth{#1} Assignment}
\newcommand{\lectors}{M. Lindauer \& A. Biedenkapp}
\newcommand{\hide}[1]{}


\newcommand{\gccs}{\paragraph{General constraints for code submissions}
    
    \begin{itemize}
        \item The program can be called as stated on the exercise sheet.
        \item The program exactly returns the required output (neither less nor more) -- please use a \texttt{--verbose} option to increase the verbosity level for debugging\footnote{You might want to use \texttt{argparse} for simplicity's sake.}.
        \item Your scripts should be commented to be readable for the tutors. All functions and classes are documented with a docstring. 
        \item Provide a README ($\to$ how to install requirements and run your program(s)) and (if necessary) an installation script if your program requires any other packages.
        \item Programs are to be submitted in python $3.5$ or newer.
        \item Adding new packages to the requirements.txt is fine. If you do this however, you'll have to give a brief description why you use that package and a link to it's documentation or github page.
        \item All prepared unittests have to pass.
        \item We don't accept ipython notebook submissions.
        \item Points will be deducted if you don't fullfill these constraints.
        \item You are allowed (sometimes required) to reuse code from previous exercises.
    \end{itemize}
    \rule{\textwidth}{.5pt}
    \smallskip\\
    \noindent}
%\renewcommand{\hide}[1]{#1}

\qformat{\thequestion. \textbf{\thequestiontitle}\hfill[\thepoints]}
\bonusqformat{\thequestion. \textbf{\thequestiontitle}\hfill[\thepoints]}

\pagestyle{headandfoot}

%%%%%% MODIFY FOR EACH SHEET!!!! %%%%%%
\newcommand{\duedate}{13.01.19 (14:00)}
\newcommand{\due}{{\bf This assignment is due on \duedate.} }
\firstpageheader
{Due: \duedate \\ Points: 22}
{{\bf\lecture}\\ \assignment{10}}
{\lectors\\ \semester}

\runningheader
{Due: \duedate}
{\assignment{10}}
{\semester}
%%%%%% MODIFY FOR EACH SHEET!!!! %%%%%%

\firstpagefooter
{}
{\thepage}
{}

\runningfooter
{}
{\thepage}
{}

\headrule
\pointsinrightmargin
\bracketedpoints
\marginpointname{pt.}


\begin{document}
	\gccs
	Now that you know all about algorithm selection and algorithm configuration, you are able to tackle the per instance algorithm configuration problem.
	\begin{questions}
		\titledquestion{Hydra}[16]
		Your task is to implement Hydra in order to \textit{construct a portfolio}. To be able to decide for unseen instances, which configuration should be used,
		you could reuse your algorithm selection code (although it is not required for this exercise).
		The scenario contains three test instances on which you are to evaluate the final performance of your portfolio (report oracle performances).
		
		The simplest way to implement Hydra is to modify SMAC such that it can be called iteratively with an adapted performance metric. For that you can use
		the target algorithm evaluator (TAE) we already implemented for use with hydra (see \url{https://github.com/automl/SMAC3/blob/master/smac/tae/execute_ta_run_hydra.py}).
		
		The remaining steps for you to implement are:
		\begin{itemize}
			\item Implement the Hydra loop
			\item Validate the best configuration after each iteration on all test instances
			\item Keep track of the Portfolio performance
			\item Modify the performance metric (using the provided Hydra TAE)
		\end{itemize}
		
		To give you an understanding how an optimizer interacts with the provided target algorithm (\texttt{branin.py}) we've provided an example of how you
		were to call \textit{SMAC} to find a well performing configuration across all instances (see \texttt{runsmac.py}).
		
		The \textit{scenario} folder contains all necessary files you will need to solve the task. \texttt{scenario.txt} lists the maximum number of function evaluations.
		You should not use more function evaluations (per iteration) when running your PIAC system.
		
		The instance set consists of three subsets so you only need to run Hydra for at most 4 iterations.
		
		\titledquestion{Comparison}[6]
		Compare your systems performance against the best configuration you can find using only SMAC (\texttt{runsmac.py}) on the training as well as the test set.\\
		Generate scatter plots to visually compare SMAC against your PIAC system.
		
		\titledquestion{Feedback}[Bonus: 1]
		For each question in this assignment, state:
		\begin{itemize}
			\item How long you worked on it.
			\item What you learned.
			\item Anything you would improve in this question if you were teaching the course.
		\end{itemize}
		
	\end{questions}

\noindent
\due Submit your solution for the tasks by uploading a PDF to your groups BitBucket repository. The PDF has to include the name of the submitter(s). Teams of at most $2$ students are allowed.
\end{document}