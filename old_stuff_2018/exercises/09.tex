\documentclass{exam}
\usepackage{amsmath, amsfonts}
\usepackage{verbatim}
\usepackage{graphicx}
\usepackage[super]{nth}

\DeclareMathOperator*{\argmin}{argmin}

\usepackage[hyperfootnotes=false]{hyperref}

\usepackage[usenames,dvipsnames]{color}
\newcommand{\note}[1]{
	\noindent~\\
	\vspace{0.25cm}
	\fcolorbox{Red}{Orange}{\parbox{0.99\textwidth}{#1\\}}
	%{\parbox{0.99\textwidth}{#1\\}}
	\vspace{0.25cm}
}


\renewcommand{\vec}[1]{\mathbf{#1}}
\newcommand{\lecture}{ML4AAD}
\newcommand{\lecturelong}{Machine Learning for Automated Algorithm Design}
\newcommand{\semester}{WS 2018/19}
\newcommand{\assignment}[1]{\nth{#1} Assignment}
\newcommand{\lectors}{M. Lindauer \& A. Biedenkapp}
\newcommand{\hide}[1]{}


\newcommand{\gccs}{\paragraph{General constraints for code submissions}
    
    \begin{itemize}
        \item The program can be called as stated on the exercise sheet.
        \item The program exactly returns the required output (neither less nor more) -- please use a \texttt{--verbose} option to increase the verbosity level for debugging\footnote{You might want to use \texttt{argparse} for simplicity's sake.}.
        \item Your scripts should be commented to be readable for the tutors. All functions and classes are documented with a docstring. 
        \item Provide a README ($\to$ how to install requirements and run your program(s)) and (if necessary) an installation script if your program requires any other packages.
        \item Programs are to be submitted in python $3.5$ or newer.
        \item Adding new packages to the requirements.txt is fine. If you do this however, you'll have to give a brief description why you use that package and a link to it's documentation or github page.
        \item All prepared unittests have to pass.
        \item We don't accept ipython notebook submissions.
        \item Points will be deducted if you don't fullfill these constraints.
        \item You are allowed (sometimes required) to reuse code from previous exercises.
    \end{itemize}
    \rule{\textwidth}{.5pt}
    \smallskip\\
    \noindent}
%\renewcommand{\hide}[1]{#1}

\qformat{\thequestion. \textbf{\thequestiontitle}\hfill[\thepoints]}
\bonusqformat{\thequestion. \textbf{\thequestiontitle}\hfill[\thepoints]}

\pagestyle{headandfoot}

%%%%%% MODIFY FOR EACH SHEET!!!! %%%%%%
\newcommand{\duedate}{12.01.19 (14:00)}
\newcommand{\due}{{\bf This assignment is due on \duedate.} }
\firstpageheader
{Due: \duedate \\ Points: 10}
{{\bf\lecture}\\ \assignment{9}}
{\lectors\\ \semester}

\runningheader
{Due: \duedate}
{\assignment{9}}
{\semester}
%%%%%% MODIFY FOR EACH SHEET!!!! %%%%%%

\firstpagefooter
{}
{\thepage}
{}

\runningfooter
{}
{\thepage}
{}

\headrule
\pointsinrightmargin
\bracketedpoints
\marginpointname{pt.}


\begin{document}
	\gccs
	After you now know how to use SMAC,
	your next task is to configure an algorithm of your own choice. This will enable you to use SMAC in your own projects
	
	\begin{questions}
		
		\titledquestion{Configuration of \textit{XYZ}}[Bonus: 9]
		The most demanding task in this exercise is to correctly define the configuration space. A quick explanation of the syntax of how to encode the configuration space
		to work with SMAC can be found here: \url{https://automl.github.io/SMAC3/stable/options.html#paramcs}\\
		You will have to define meaningful boundaries to keep the search space small enough such that SMAC does not have to deal with too many irrelevant values but large
		enough to not exclude good regions.
		
		For the remaining setup you can follow the previous exercise sheet.\\
		Briefly describe what algorithm you are optimizing, how your instances (if you have any) look like and how your derived your search space.
		Reuse the plotting code you implemented to compare your optimized configuration to the default configuration you started from.
		
		If you don't have an algorithm you want to optimize in mind, you can optimize any \texttt{sklearn} model on any of the \texttt{sklearn} datasets\footnote{\url{http://scikit-learn.org/stable/datasets/index.html}}. As instances you will have to draw 100 random splits of the dataset you choose to configure the model on.
		
		\titledquestion{Feedback}[Bonus: 1]
		For each question in this assignment, state:
		\begin{itemize}
			\item How long you worked on it.
			\item What you learned.
			\item Anything you would improve in this question if you were teaching the course.
		\end{itemize}
	\end{questions}

\noindent
\due Submit your solution for the tasks by uploading a PDF to your groups BitBucket repository. The PDF has to include the name of the submitter(s). Teams of at most $2$ students are allowed.
\end{document}