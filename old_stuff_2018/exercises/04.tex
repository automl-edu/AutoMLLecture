\documentclass{exam}
\usepackage{amsmath, amsfonts}
\usepackage{verbatim}
\usepackage{graphicx}
\usepackage[super]{nth}

\DeclareMathOperator*{\argmin}{argmin}

\usepackage[hyperfootnotes=false]{hyperref}

\usepackage[usenames,dvipsnames]{color}
\newcommand{\note}[1]{
	\noindent~\\
	\vspace{0.25cm}
	\fcolorbox{Red}{Orange}{\parbox{0.99\textwidth}{#1\\}}
	%{\parbox{0.99\textwidth}{#1\\}}
	\vspace{0.25cm}
}


\renewcommand{\vec}[1]{\mathbf{#1}}
\newcommand{\lecture}{ML4AAD}
\newcommand{\lecturelong}{Machine Learning for Automated Algorithm Design}
\newcommand{\semester}{WS 2018/19}
\newcommand{\assignment}[1]{\nth{#1} Assignment}
\newcommand{\lectors}{M. Lindauer \& A. Biedenkapp}
\newcommand{\hide}[1]{}


\newcommand{\gccs}{\paragraph{General constraints for code submissions}
    
    \begin{itemize}
        \item The program can be called as stated on the exercise sheet.
        \item The program exactly returns the required output (neither less nor more) -- please use a \texttt{--verbose} option to increase the verbosity level for debugging\footnote{You might want to use \texttt{argparse} for simplicity's sake.}.
        \item Your scripts should be commented to be readable for the tutors. All functions and classes are documented with a docstring. 
        \item Provide a README ($\to$ how to install requirements and run your program(s)) and (if necessary) an installation script if your program requires any other packages.
        \item Programs are to be submitted in python $3.5$ or newer.
        \item Adding new packages to the requirements.txt is fine. If you do this however, you'll have to give a brief description why you use that package and a link to it's documentation or github page.
        \item All prepared unittests have to pass.
        \item We don't accept ipython notebook submissions.
        \item Points will be deducted if you don't fullfill these constraints.
        \item You are allowed (sometimes required) to reuse code from previous exercises.
    \end{itemize}
    \rule{\textwidth}{.5pt}
    \smallskip\\
    \noindent}
%\renewcommand{\hide}[1]{#1}

\qformat{\thequestion. \textbf{\thequestiontitle}\hfill[\thepoints]}
\bonusqformat{\thequestion. \textbf{\thequestiontitle}\hfill[\thepoints]}

\pagestyle{headandfoot}

%%%%%% MODIFY FOR EACH SHEET!!!! %%%%%%
\newcommand{\duedate}{11.11.18 (14:00)}
\newcommand{\due}{{\bf This assignment is due on \duedate.} }
\firstpageheader
{Due: \duedate \\ Points: 6}
{{\bf\lecture}\\ \assignment{4}}
{\lectors\\ \semester}

\runningheader
{Due: \duedate}
{\assignment{4}}
{\semester}
%%%%%% MODIFY FOR EACH SHEET!!!! %%%%%%

\firstpagefooter
{}
{\thepage}
{}

\runningfooter
{}
{\thepage}
{}

\headrule
\pointsinrightmargin
\bracketedpoints
\marginpointname{pt.}


\begin{document}
	\gccs
	After you now know some examples of hard combinatorial problems, the goal of this exercise is to let you implement some simple Python programs to solve CSP and SAT problem with SLS. Further you will implement the k-nearest neighbor algorithm to get you thinking about (hyper-)parameters and their influence on the algorithm.
	
	\bigskip
	
	\noindent All tasks include the submission of some results (besides the code).
	To submit these results, please submit a PDF with all the results and your name(s).
	
	\noindent We provide a simple Makefile which you can use to install all packages listed in your requirements file (\texttt{make init}) and run all provided unit tests (\texttt{make test}).
	
	
	\begin{questions}
		
		
		\titledquestion{Solving SAT with uninformed random walk (URW)}[2]
		
		Consider the model-finding variant of SAT, i.e., finding a satisfying assignment of a given Boolean formula in CNF (conjunctive normal form).
		A CNF is given in the DIMACS format\footnote{\url{http://www.satcompetition.org/2009/format-benchmarks2009.html}} as follows:
		
		\begin{verbatim}
		c start with comments
		p cnf 5 3
		1 -5 4 0
		-1 5 3 4 0
		-3 -4 0
		\end{verbatim}
		
		\begin{itemize}
			\item Any line starting with a \texttt{c} is a comment;
			\item The single line starting with \texttt{p} defines the number of variables (here $5$) and the number of clauses (here $3$);
			\item Each other line encodes a clause (a disjunction of literals); e.g., \texttt{1 -5 4 0} corresponds to $x_1 \vee \neg x_5 \vee x_4$. Please note that each line terminates with a $0$ -- that is not a variable.
		\end{itemize}	
		
		Your first task is to implement an uninformed random work in Python to find a satisfying assignment of a given CNF in DIMACS format.
		For satisfiable formulas, the output of your program should include lines of the following form:
		
		\begin{verbatim}
		s SATISFIABLE
		v 1 2 3 -4 5
		\end{verbatim}
		
		The first line indicates that the given CNF is satisfiable.
		The second line encodes a satisfying assignment, in this case $x_1 \to True$, $x_2 \to True$, $x_3 \to True$, $x_4 \to False$, $x_5 \to True$. Given this assignment, each clause (/line) of the given CNF should be satisfied.
		
		It should be possible to call your program as \texttt{python solver.py inst1.dimacs}.
		
		We provide you with some example code which reads in a DIMACS file and parses it for you to easily use it. You don't have to use our script. However if you choose to change things, you'll have to adjust the prepared unittest accordingly.
		
		Report the results of urw for all provided instance sets in the pdf.
		
		\titledquestion{Solving SAT with SLS}[4]
		
		Your second task is to extend your program by adding one alternative SLS methods of your choice (which we discussed in the lecture) to solve a given CNF. Furthermore, you have to measure the runtime of your three methods on the provided example CNFs to identify the best approach. Discuss the results briefly as well as the design decisions you had to face when implementing both solvers.\\
		Write your own unittests (one for each) that shows the functionality of your code.
		
		Report the results of both methods for all provided instance sets in the pdf.
		
		\titledquestion{Feedback}[Bonus: 1]
		For each question in this assignment, state:
		\begin{itemize}
			\item How long you worked on it.
			\item What you learned.
			\item Anything you would improve in this question if you were teaching the course.
		\end{itemize}
	\end{questions}
	
	\noindent
	\due Submit your solution for the tasks by uploading a PDF to your groups BitBucket repository. The PDF has to include the name of the submitter(s). Teams of at most $2$ students are allowed.
\end{document}