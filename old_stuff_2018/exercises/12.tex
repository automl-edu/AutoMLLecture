\documentclass{exam}
\usepackage{amsmath, amsfonts}
\usepackage{verbatim}
\usepackage{graphicx}
\usepackage[super]{nth}
\usepackage[multiple]{footmisc}

\DeclareMathOperator*{\argmin}{argmin}

\usepackage[hyperfootnotes=false]{hyperref}

\usepackage[usenames,dvipsnames]{color}
\newcommand{\note}[1]{
	\noindent~\\
	\vspace{0.25cm}
	\fcolorbox{Red}{Orange}{\parbox{0.99\textwidth}{#1\\}}
	%{\parbox{0.99\textwidth}{#1\\}}
	\vspace{0.25cm}
}


\renewcommand{\vec}[1]{\mathbf{#1}}
\newcommand{\lecture}{ML4AAD}
\newcommand{\lecturelong}{Machine Learning for Automated Algorithm Design}
\newcommand{\semester}{WS 2018/19}
\newcommand{\assignment}[1]{\nth{#1} Assignment}
\newcommand{\lectors}{M. Lindauer \& A. Biedenkapp}
\newcommand{\hide}[1]{}


\newcommand{\gccs}{\paragraph{General constraints for code submissions}
    
    \begin{itemize}
        \item The program can be called as stated on the exercise sheet.
        \item The program exactly returns the required output (neither less nor more) -- please use a \texttt{--verbose} option to increase the verbosity level for debugging\footnote{You might want to use \texttt{argparse} for simplicity's sake.}.
        \item Your scripts should be commented to be readable for the tutors. All functions and classes are documented with a docstring. 
        \item Provide a README ($\to$ how to install requirements and run your program(s)) and (if necessary) an installation script if your program requires any other packages.
        \item Programs are to be submitted in python $3.5$ or newer.
        \item Adding new packages to the requirements.txt is fine. If you do this however, you'll have to give a brief description why you use that package and a link to it's documentation or github page.
        \item All prepared unittests have to pass.
        \item We don't accept ipython notebook submissions.
        \item Points will be deducted if you don't fullfill these constraints.
        \item You are allowed (sometimes required) to reuse code from previous exercises.
    \end{itemize}
    \rule{\textwidth}{.5pt}
    \smallskip\\
    \noindent}
%\renewcommand{\hide}[1]{#1}

\qformat{\thequestion. \textbf{\thequestiontitle}\hfill[\thepoints]}
\bonusqformat{\thequestion. \textbf{\thequestiontitle}\hfill[\thepoints]}

\pagestyle{headandfoot}

%%%%%% MODIFY FOR EACH SHEET!!!! %%%%%%
\newcommand{\duedate}{27.01.19 (14:00)}
\newcommand{\due}{{\bf This assignment is due on \duedate.} }
\firstpageheader
{Due: \duedate \\ Points: 34}
{{\bf\lecture}\\ \assignment{12}}
{\lectors\\ \semester}

\runningheader
{Due: \duedate}
{\assignment{12}}
{\semester}
%%%%%% MODIFY FOR EACH SHEET!!!! %%%%%%

\firstpagefooter
{}
{\thepage}
{}

\runningfooter
{}
{\thepage}
{}

\headrule
\pointsinrightmargin
\bracketedpoints
\marginpointname{pt.}


\begin{document}
	\gccs
	Now that you've learned all about algorithm analysis you are tasked with using CAVE\footnote{\url{https://ml.informatik.uni-freiburg.de/papers/18-LION12-CAVE.pdf}}\footnote{\url{https://github.com/automl/CAVE}}. CAVE is a versatile analysis tool for automatic algorithm configurators. It generates comprehensive reports hat give you insights into the configured algorithm, the used instance set and also the configuration tool itself.
	
	If you don't want to clone CAVE, you can simply install CAVE using pip. After the installation cave can be called in the terminal via \texttt{cave --folder smac3-output*/*} (to also load any validation data you would have to specify \texttt{--validation\_format SMAC3}).
	\begin{questions}
		\titledquestion{Configuration Analysis, Visualization and Evaluation}[34]
		We provide you with 15 validated configuration results when running SATenstein on the industrial instances from exercise sheet 8. Your task for this exercise sheet
		is to evaluate/analyze the provided results with the help of CAVE.
	\begin{parts}
		\part[2] You first have to run CAVE to generate an HTML report\footnote{Similar to \url{http://ml.informatik.uni-freiburg.de/~biedenka/cave.html}.\\\hspace*{.625cm}However the given report is generated with an older version of CAVE.} using all provided SMAC outputs (including the validated results).\\It might take a while for parts of the report to be generated. (When handing in your solution please upload the whole CAVE output folder).
		\part[16] For each part of the generated report (Meta Data, Performance Analysis, Configurator's behavior, Parameter Importance and Feature Analysis) state what you can learn from each subsection. Where possible, support your statement(s) by using plots from the report in your PDF.
		\part[16] Generate at least 4 more reports using subsets (always including \texttt{run\_00} as that is the only output with a validated default configuration) of the provided data. How does the report change when only using results of one SMAC run, 10, ...? Do your conclusions from part (b) still hold? Why are repeated experiments important?
	\end{parts}
		
		
	\titledquestion{Feedback}[Bonus: 1]
	For each question in this assignment, state:
	\begin{itemize}
		\item How long you worked on it.
		\item What you learned.
		\item Anything you would improve in this question if you were teaching the course.
	\end{itemize}
	\end{questions}

\noindent
\due Submit your solution for the tasks by uploading a PDF to your groups BitBucket repository. The PDF has to include the name of the submitter(s). Teams of at most $2$ students are allowed.
\end{document}