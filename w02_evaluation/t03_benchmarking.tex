
\pdfminorversion=4 % for acroread
\documentclass[aspectratio=169,t,xcolor={usenames,dvipsnames}]{beamer}
%\documentclass[t,handout,xcolor={usenames,dvipsnames}]{beamer}
\usepackage{../beamerstyle}
\usepackage{dsfont}
\usepackage{bm}
\usepackage[english]{babel}
\usepackage[utf8]{inputenc}
\usepackage{graphicx}
\usepackage{algorithm}
\usepackage[ruled,vlined,algo2e,linesnumbered]{algorithm2e}
%\usepackage[boxed,vlined]{algorithm2e}
\usepackage{hyperref}
\usepackage{booktabs}
\usepackage{mathtools}

\usepackage{amsmath,amssymb}
\usepackage{listings}
\lstset{frame=lines,framesep=3pt,numbers=left,numberblanklines=false,basicstyle=\ttfamily\small}

\usepackage{subfig}
\usepackage{multicol}
%\usepackage{appendixnumberbeamer}
%
\usepackage{tcolorbox}

\usepackage{pgfplots}
\usepackage{tikz}
\usetikzlibrary{trees} 
\usetikzlibrary{shapes.geometric}
\usetikzlibrary{positioning,shapes,shadows,arrows,calc,mindmap}
\usetikzlibrary{positioning,fadings,through}
\usetikzlibrary{decorations.pathreplacing}
\usetikzlibrary{intersections}
\usetikzlibrary{positioning,fit,calc,shadows,backgrounds}
\pgfdeclarelayer{background}
\pgfdeclarelayer{foreground}
\pgfsetlayers{background,main,foreground}
\tikzstyle{activity}=[rectangle, draw=black, rounded corners, text centered, text width=8em]
\tikzstyle{data}=[rectangle, draw=black, text centered, text width=8em]
\tikzstyle{myarrow}=[->, thick, draw=black]

% Define the layers to draw the diagram
\pgfdeclarelayer{background}
\pgfdeclarelayer{foreground}
\pgfsetlayers{background,main,foreground}

%\usepackage{listings}
%\lstset{numbers=left,
%  showstringspaces=false,
%  frame={tb},
%  captionpos=b,
%  lineskip=0pt,
%  basicstyle=\ttfamily,
%%  extendedchars=true,
%  stepnumber=1,
%  numberstyle=\small,
%  xleftmargin=1em,
%  breaklines
%}

 
\definecolor{blue}{RGB}{0, 74, 153}

\usetheme{Boadilla}
%\useinnertheme{rectangles}
\usecolortheme{whale}
\setbeamercolor{alerted text}{fg=blue}
\useoutertheme{infolines}
\setbeamertemplate{navigation symbols}{\vspace{-5pt}} % to lower the logo
\setbeamercolor{date in head/foot}{bg=white} % blue
\setbeamercolor{date in head/foot}{fg=white}
\setbeamercolor{author  in head/foot}{bg=white} %blue
\setbeamercolor{title in head/foot}{bg=white} % blue
\setbeamercolor{title}{fg=white, bg=blue}
\setbeamercolor{block title}{fg=white,bg=blue}
\setbeamercolor{block body}{bg=blue!10}
\setbeamercolor{frametitle}{fg=white, bg=blue}
\setbeamercovered{invisible}

\makeatletter
\setbeamertemplate{footline}
{
  \leavevmode%
  \hbox{%
  \begin{beamercolorbox}[wd=.333333\paperwidth,ht=2.25ex,dp=1ex,center]{author in head/foot}%
%    \usebeamerfont{author in head/foot}\insertshortauthor
  \end{beamercolorbox}%
  \begin{beamercolorbox}[wd=.333333\paperwidth,ht=2.25ex,dp=1ex,center]{title in head/foot}%
    \usebeamerfont{title in head/foot}\insertshorttitle
  \end{beamercolorbox}%
  \begin{beamercolorbox}[wd=.333333\paperwidth,ht=2.25ex,dp=1ex,right]{date in head/foot}%
    \usebeamerfont{date in head/foot}\insertshortdate{}\hspace*{2em}
%    \insertframenumber\hspace*{2ex} 
  \end{beamercolorbox}}%
  \vskip0pt%
}
\makeatother

%\pgfdeclareimage[height=1.2cm]{automl}{images/logos/automl.png}
%\pgfdeclareimage[height=1.2cm]{freiburg}{images/logos/freiburg}

%\logo{\pgfuseimage{freiburg}}

\newcommand{\comment}[1]{
	\noindent
	%\vspace{0.25cm}
	{\color{red}{\textbf{TODO:} #1}}
	%\vspace{0.25cm}
}
\renewcommand{\comment}[1]{}
\newcommand{\hide}[1]{}
\newcommand{\cemph}[2]{\emph{\textcolor{#1}{#2}}}

\newcommand{\lit}[1]{{\footnotesize\color{black!70}[#1]}}

\newcommand{\litw}[1]{{\footnotesize\color{black!20}[#1]}}


\newcommand{\myframe}[2]{\begin{frame}[c]{#1}#2\end{frame}}
\newcommand{\myframetop}[2]{\begin{frame}{#1}#2\end{frame}}
\newcommand{\myit}[1]{\begin{itemize}#1\end{itemize}}
\newcommand{\myblock}[2]{\begin{block}{#1}#2\end{block}}


\newcommand{\votepurple}[1]{\textcolor{Purple}{$\bigstar$}}
\newcommand{\voteyellow}[1]{\textcolor{Goldenrod}{$\bigstar$}}
\newcommand{\voteblue}[1]{\textcolor{RoyalBlue}{$\bigstar$}}
\newcommand{\votepink}[1]{\textcolor{Pink}{$\bigstar$}}

\newcommand{\diff}{\mathop{}\!\mathrm{d}}
\newcommand{\refstyle}[1]{{\small{\textcolor{gray}{#1}}}}
\newcommand{\hands}[0]{\includegraphics[height=1.5em]{images/hands}}
\newcommand{\transpose}[0]{{\textrm{\tiny{\sf{T}}}}}
\newcommand{\norm}{{\mathcal{N}}}
\newcommand{\cutoff}[0]{\kappa}
\newcommand{\instD}[0]{\dataset}
\newcommand{\insts}[0]{\mathcal{I}}
\newcommand{\inst}[0]{i}
\newcommand{\pcs}[0]{\mathbf{\Lambda}}
\newcommand{\bx}[0]{\conf}
\newcommand{\conf}[0]{\mathbf{\lambda}}
\newcommand{\defconf}[0]{\mathbf{\lambda}_{\text{def}}}
\newcommand{\finconf}[0]{\mathbf{\lambda}^*}
\newcommand{\incumbent}[0]{\finconf}
\newcommand{\confs}[0]{\pcs}
%\newcommand{\vlambda}[0]{\bm{\lambda}}
%\newcommand{\vLambda}[0]{\bm{\Lambda}}
\newcommand{\dataset}[0]{\mathcal{D}}
\newcommand{\datasets}[0]{\mathbf{D}}
\newcommand{\loss}[0]{\mathcal{L}}

% \renewcommand{\vec}[1]{\mathbf{#1}}
\newcommand{\hist}[0]{\mathcal{H}}
\newcommand{\param}[0]{p}
\newcommand{\algo}[0]{\mathcal{A}}
\newcommand{\algos}[0]{\mathbf{A}}
%\newcommand{\nn}[0]{N}
\newcommand{\feats}[0]{\mathcal{F}}
\newcommand{\feat}[0]{\vec{f}}
\newcommand{\cluster}[0]{\vec{h}}
\newcommand{\clusters}[0]{\vec{H}}
\newcommand{\perf}[0]{\mathbb{R}}
%\newcommand{\surro}[0]{\mathcal{S}}
\newcommand{\surro}[0]{\hat{f}}
\newcommand{\func}[0]{f}
\newcommand{\epm}[0]{\surro}
\newcommand{\portfolio}[0]{\mathcal{P}}
\newcommand{\schedule}[0]{\mathcal{S}}
\newcommand{\mdata}[0]{\dataset_{\text{meta}}}

% Deep Learning
\newcommand{\weights}[0]{\theta}
\newcommand{\metaweights}[0]{\phi}


% reinforcement learning
\newcommand{\policies}[0]{\Pi}
\newcommand{\policy}[0]{\pi}
\newcommand{\actionRL}[0]{a}
\newcommand{\stateRL}[0]{s}
\newcommand{\statesRL}[0]{\mathcal{S}}
\newcommand{\rewardRL}[0]{r}
\newcommand{\rewardfuncRL}[0]{\mathcal{R}}

\RestyleAlgo{algoruled}
\DontPrintSemicolon
\LinesNumbered
\SetAlgoVlined
\SetFuncSty{textsc}

\SetKwInOut{Input}{Input}
\SetKwInOut{Output}{Output}
\SetKw{Return}{return}

%\newcommand{\changed}[1]{{\color{red}#1}}

%\newcommand{\citeN}[1]{\citeauthor{#1}~(\citeyear{#1})}

\renewcommand{\vec}[1]{\mathbf{#1}}
\DeclareMathOperator*{\argmin}{arg\,min}
\DeclareMathOperator*{\argmax}{arg\,max}

\newcommand{\aqme}{\textit{AQME}}
\newcommand{\aslib}{\textit{ASlib}}
\newcommand{\llama}{\textit{LLAMA}}
\newcommand{\satzilla}{\textit{SATzilla}}
\newcommand{\satzillaY}[1]{\textit{SATzilla'{#1}}}
\newcommand{\snnap}{\textit{SNNAP}}
\newcommand{\claspfolioTwo}{\textit{claspfolio~2}}
\newcommand{\flexfolio}{\textit{FlexFolio}}
\newcommand{\claspfolioOne}{\textit{claspfolio~1}}
\newcommand{\isac}{\textit{ISAC}}
\newcommand{\eisac}{\textit{EISAC}}
\newcommand{\sss}{\textit{3S}}
\newcommand{\sunny}{\textit{Sunny}}
\newcommand{\ssspar}{\textit{3Spar}}
\newcommand{\cshc}{\textit{CSHC}}  
\newcommand{\cshcpar}{\textit{CSHCpar}}  
\newcommand{\measp}{\textit{ME-ASP}} 
\newcommand{\aspeed}{\textit{aspeed}}
\newcommand{\autofolio}{\textit{AutoFolio}}
\newcommand{\cedalion}{\textit{Cedalion}}
\newcommand{\fanova}{\textit{fANOVA}}
\newcommand{\sbs}{\textit{SB}}
\newcommand{\oracle}{\textit{VBS}}

% like approaches
\newcommand{\claspfoliolike}[1]{\texttt{claspfolio-#1-like}}
\newcommand{\satzillalike}[1]{\texttt{SATzilla'#1-like}}
\newcommand{\isaclike}{\texttt{ISAC-like}}
\newcommand{\ssslike}{\texttt{3S-like}}
\newcommand{\measplike}{\texttt{ME-ASP-like}}

\newcommand{\aspCoseal}{\textit{ASP-POTASSCO}}
\newcommand{\cspCoseal}{\textit{CSP-2010}}
\newcommand{\maxsatCoseal}{\textit{MAXSAT12-PMS}}
\newcommand{\premarCoseal}{\textit{PRE\-MARSHALLING}}
\newcommand{\qbfCoseal}{\textit{QBF-2011}}
\newcommand{\satallTwelveCoseal}{\textit{SAT12-ALL}}
\newcommand{\sathandTwelveCoseal}{\textit{SAT12-HAND}}
\newcommand{\satinduTwelveCoseal}{\textit{SAT12-INDU}}
\newcommand{\satrandTwelveCoseal}{\textit{SAT12-RAND}}
\newcommand{\sathandElevenCoseal}{\textit{SAT11-HAND}}
\newcommand{\satinduElevenCoseal}{\textit{SAT11-INDU}}
\newcommand{\satrandElevenCoseal}{\textit{SAT11-RAND}}
\newcommand{\proteusCoseal}{\textit{PROTEUS-2014}}

\newcommand{\irace}{\textit{I/F-race}}
\newcommand{\gga}{\textit{GGA}}
\newcommand{\smac}{\textit{SMAC}}
\newcommand{\paramils}{\textit{ParamILS}}
\newcommand{\spearmint}{\textit{Spearmint}}
\newcommand{\tpe}{\textit{TPE}}

\newcommand{\gringo}{\textit{gringo}}
\newcommand{\clasp}{\textit{clasp}}
\newcommand{\lingeling}{\textit{lingeling}}

\newcommand{\hydra}{\textit{Hydra}}

\newcommand{\plingeling}{\textit{Plingeling}}
\newcommand{\ccasat}{\textit{CCASat}}

\usepackage{pifont}
\newcommand{\itarrow}{\mbox{\Pisymbol{pzd}{229}}}
\newcommand{\ithook}{\mbox{\Pisymbol{pzd}{52}}}
\newcommand{\itcross}{\mbox{\Pisymbol{pzd}{56}}}
\newcommand{\ithand}{\mbox{\raisebox{-1pt}{\Pisymbol{pzd}{43}}}}

%\DeclareMathOperator*{\argmax}{arg\,max}

\newcommand{\ie}{{\it{}i.e.\/}}
\newcommand{\eg}{{\it{}e.g.\/}}
\newcommand{\cf}{{\it{}cf.\/}}
\newcommand{\wrt}{\mbox{w.r.t.}}
\newcommand{\vs}{{\it{}vs\/}}
\newcommand{\vsp}{{\it{}vs\/}}
\newcommand{\etc}{{\copyedit{etc.}}}
\newcommand{\etal}{{\it{}et al.\/}}

\newcommand{\pscProc}{{\bf procedure}}
\newcommand{\pscBegin}{{\bf begin}}
\newcommand{\pscEnd}{{\bf end}}
\newcommand{\pscEndIf}{{\bf endif}}
\newcommand{\pscFor}{{\bf for}}
\newcommand{\pscEach}{{\bf each}}
\newcommand{\pscThen}{{\bf then}}
\newcommand{\pscElse}{{\bf else}}
\newcommand{\pscWhile}{{\bf while}}
\newcommand{\pscIf}{{\bf if}}
\newcommand{\pscRepeat}{{\bf repeat}}
\newcommand{\pscUntil}{{\bf until}}
\newcommand{\pscWithProb}{{\bf with probability}}
\newcommand{\pscOtherwise}{{\bf otherwise}}
\newcommand{\pscDo}{{\bf do}}
\newcommand{\pscTo}{{\bf to}}
\newcommand{\pscOr}{{\bf or}}
\newcommand{\pscAnd}{{\bf and}}
\newcommand{\pscNot}{{\bf not}}
\newcommand{\pscFalse}{{\bf false}}
\newcommand{\pscEachElOf}{{\bf each element of}}
\newcommand{\pscReturn}{{\bf return}}

%\newcommand{\param}[1]{{\sl{}#1}}
\newcommand{\var}[1]{{\it{}#1}}
\newcommand{\cond}[1]{{\sf{}#1}}
%\newcommand{\state}[1]{{\sf{}#1}}
%\newcommand{\func}[1]{{\sl{}#1}}
\newcommand{\set}[1]{{\Bbb #1}}
%\newcommand{\inst}[1]{{\tt{}#1}}
\newcommand{\myurl}[1]{{\small\sf #1}}

\newcommand{\Nats}{{\Bbb N}}
\newcommand{\Reals}{{\Bbb R}}
\newcommand{\extset}[2]{\{#1 \; | \; #2\}}

\newcommand{\vbar}{$\,\;|$\hspace*{-1em}\raisebox{-0.3mm}{$\,\;\;|$}}
\newcommand{\vendbar}{\raisebox{+0.4mm}{$\,\;|$}}
\newcommand{\vend}{$\,\:\lfloor$}


\newcommand{\goleft}[2][.7]{\parbox[t]{#1\linewidth}{\strut\raggedright #2\strut}}
\newcommand{\rightimage}[2][.3]{\mbox{}\hfill\raisebox{1em-\height}[0pt][0pt]{\includegraphics[width=#1\linewidth]{#2}}\vspace*{-\baselineskip}}





\usepackage{multirow}


\title[AutoML: Risks]{AutoML: Evaluation} % week title
\subtitle{Benchmarking and Comparing Learners} % video title
\author[Lars Kotthoff]{\underline{Bernd Bischl} \and Frank Hutter \and \underline{Lars Kotthoff}\newline \and Marius Lindauer \and Joaquin Vanschoren}
\institute{}
\date{}

\newcommand\reffootnote[1]{%
    \begingroup
    \renewcommand\thefootnote{}\footnote{
        \tiny #1
    \vspace*{1em}}%
    \addtocounter{footnote}{-1}%
    \endgroup
}

% \AtBeginSection[] % Do nothing for \section*
% {
%   \begin{frame}{Outline}
%     \bigskip
%     \vfill
%     \tableofcontents[currentsection]
%   \end{frame}
% }

\begin{document}
	
	\maketitle

    \begin{frame}[c]{Benchmark Experiments}

    \begin{itemize}
    \item In benchmark experiments, different learning algorithms are applied to one or more data sets with the aim to compare and rank their performances.
    \item To ensure comparability, synchronized train and test sets, i.e.\ the same resampling method with the same train-test splits should be used to calculate and compare the performances.
    \item Results of benchmark experiments produce a data set, which can be further analyzed and visualized. 
    \newline
    \textbf{Example}: Benchmark results (per CV-fold) of CART and random forest using 2-fold CV with MSE as performance measure:

    \begin{center}
        \scriptsize
        \begin{tabular}{c|c|c|c}
        \hline
        data set & k-th fold & MSE (rpart) & MSE (randomForest)\\
        \hline
        BostonHousing & 1 & 29.4 & 17.13\\
        \hline
        BostonHousing & 2 & 20.5 & 8.90\\
        \hline
        mtcars & 1 & 35.0 & 7.53\\
        \hline
        mtcars & 2 & 38.9 & 6.73\\
        \hline
        \end{tabular}
    \end{center}
    \end{itemize}

    \end{frame}

    \begin{frame}[c,allowframebreaks]{Hypothesis Testing in Benchmarking}
    We want to know if the difference in performance between models (or algorithms) is significant or if it only occurred by chance.

    \textbf{Null Hypothesis Statistical Testing (NHST)} in Benchmarking:

    \begin{itemize}
    \item Formulate a null hypothesis $H_0$ (e.g.\ the expected generalization error of two algorithms is equivalent).
    \item Use a hypothesis test to reject $H_0$ (not rejecting $H_0$ does not mean that we accept it).
    \item Rejecting $H_0$ gives us some confidence in the belief that the observed results may not be only random.
    \end{itemize}

    Typical example in machine learning:

    \begin{itemize}
    \item $H_0$: on average, model 1 does not perform better than model 2.
    \item $H_1$: on average, model 1 outperforms the model 2.
    \item Aim: Reject $H_0$ with confidence level of $1-\alpha$.
    \end{itemize}

    \framebreak

    Selection of an appropriate hypothesis test is at least based on the type of
    problem, i.e.\ if the aim is to compare
    \begin{itemize}
    \item 2 models / algorithms on a single domain (i.e.\ on a single data set)
    \item 2 algorithms across different domains (i.e.\ on multiple data sets)
    \item multiple algorithms across different domains / data sets
    \end{itemize}

    Overview of different hypothesis tests for matched samples:

    \begin{center}
    \includegraphics[height=.4\textheight]{tests_overview.png}
    \end{center}

    \end{frame}

    \begin{frame}[c,allowframebreaks]{McNemar Test}
    \begin{itemize}
    \item The McNemar test is a non-parametric test used on paired nominal data and does not make any distributional assumptions.
    \item It can be applied to compare the performance of two \textbf{models}
        when the considered performance measure is based on an outer loss with a
        nominal or binary output, e.g.\ accuracy is based on a binary outer loss.
    \item Both models are trained on a training set and evaluated on a test set. Based on the test set, a \textbf{contingency table} that compares the two models (model 1 and model 2) is calculated:
    \end{itemize}

    \medskip
    \begin{minipage}{0.25\textwidth}
    \includegraphics[width=\textwidth]{mcnemar_1.png}
    \end{minipage}
    \begin{minipage}{0.74\textwidth}
    \begin{itemize}
    \item A: $\#$obs.\ misclassified by both.
    \item B: $\#$obs.\ misclassified by model 1 but not by model 2.
    \item C: $\#$obs.\ misclassified by model 2 but not by model 1.
    \item D: $\#$obs.\ correctly classified by both.
    \end{itemize}
    \end{minipage}

    \framebreak

    \begin{minipage}[c]{0.625\linewidth}
    Given such a contingency table, the accuracy of each model can be computed as follows:
    \begin{itemize}
      \item Model 1: (A+B)/(A+B+C+D)
      \item Model 2: (A+C)/(A+B+C+D)
    \end{itemize}

    Even if the models have the \textbf{same} accuracies (indicating equal performance), cells B and C may be different because the models may misclassify different instances.
    \end{minipage}
    \begin{minipage}[c]{0.2\linewidth}
        \includegraphics[width=\textwidth]{mcnemar_1.png}
    \end{minipage}

    McNemar tests the following hypothesis:
    \begin{itemize}
    \item $H_0:$ Both models have the same performance (we expect B = C).
    \item $H_1:$ Performances of the two models are not equal.
    \end{itemize}

    The test statistic is computed as
    $$\chi^2_{Mc} =  \tfrac{(|B-C| - 1)^2}{B + C} \sim \chi^2_{1}.$$

    \textbf{Note}: The McNemar test should only be used if $B + C > 20$.

    \framebreak

    \textbf{Example}:

    \begin{center}
      \begin{tabular}{cc|cc}
          & & \multicolumn{2}{c}{Random Forest} \\
          & & $0$ & $1$ \\
          \hline
          \multirow{2}{*}{Tree} & $0$ & 30 & 5 \\
          & $1$ & 17 & 42 \\
      \end{tabular}
    \end{center}

    Calculating the test statistic:

    $$\chi^2_{Mc} =  \frac{(|5-17| - 1)^2}{5 + 17} = 5.5 > 3.841 = \chi^2_{1,0.95}$$

    We can reject $H_0$ at a significance level of 0.05, i.e.\ we reject the hypothesis that the tree and the random forest have the same performance.
    \end{frame}


    \begin{frame}[c,allowframebreaks]{Two-Matched-Samples t-Test}

    A two-matched-samples t-test (i.e.\ a paired t-test) is the simplest hypothesis test if the aim is to compare two \textbf{models} on a single test set based on arbitrary performance measures.

    However, it is a parametric test and distributional assumptions must be made (which are often problematic).

    The t-test relies on several assumptions:

    \begin{itemize}
    \item \textbf{(pseudo-)normality}, usually met when sample size $>$ 30.
    \item \textbf{i.i.d.\ samples}, usually met if the loss of individual observations from a single test set are considered. % (this assumption is violated in case of resampling as ). %difficult as data is often limited (assumption is violated in the case of resampling).
    \item \textbf{equal variances of populations}, can be investigated through plots.
    \end{itemize}

    \framebreak

    A paired t-test to compare two different models $\fh_1$ and $\fh_2$ w.r.t.\ a performance measure calculated on a test set of size $n_{\text{test}}$:

    \begin{itemize}
    \item $H_0$: $GE(\fh_{1}) = GE(\fh_{2})$ vs.\ $H_1$: $GE(\fh_{1}) \neq GE(\fh_{2})$
    \item Test statistic $T = \sqrt{n_{\text{test}}} \frac{\bar{d}}{\sigma_{d}}$, with
    \begin{itemize}
    \item the mean performance difference of both models
        $\bar{d} = \hat{GE}_{\datasettest}(\fh_{1}) - \hat{GE}_{\datasettest}(\fh_{2})$, and
    \item the standard deviation of this mean difference
    $$\sigma_{d} = \sqrt{\frac{1}{n_{\text{test}} - 1}\sum_{i=1}^{n_{\text{test}}} \left(d_i - \bar{d} \right)^2},$$
    where $d_i = L(\yI{i}, \fh_1 (\xI{i})) - L(\yI{i}, \fh_2 (\xI{i}))$ and $\bar{d} = \sum\limits_{i=1}^{n_{\text{test}}} d_i$.
    \end{itemize}
    \end{itemize}

    \textbf{Note}: Here, $d_i$ is the difference of the outer loss of individual observations from the test set between the two models to be compared.

    \begin{itemize}
    \item We could also use a \textbf{$k$-fold CV paired t-test} to compare two \textbf{algorithms} (instead of two models) on a single data set.
    \item Instead of comparing the outer loss of individual observations, we
        would then compare the individual generalization errors per CV fold
        (i.e.\ the generalization error of the $k$ prediction models induced by the learning algorithm in each CV fold).
    \item Although the test sets do not overlap, the performance differences are
        not independent across CV folds due to overlapping training sets (which
        violates the assumption of i.i.d.\ samples).
    \item To partly overcome the issue of overlapping training sets across folds, Dietterich\footnote{Dietterich (1998). Approximate statistical tests for comparing supervised classification learning algorithms.} suggests using 5 times 2-fold CV so that at least within each repetition neither the training nor the test sets overlap.
    \end{itemize}
    \end{frame}

    \begin{frame}[c,allowframebreaks]{Friedman Test}
    Until now, we have only compared 2 models / algorithms on one data set. The \textbf{Friedman test} can be used to compare multiple classifiers on multiple data sets. The hypothesis to be tested are:
    \begin{itemize}
    \item $H_0:$ All algorithms are equivalent in their performance and hence their average ranks should be equal.
    \item $H_1:$ The average ranks for at least one algorithm is different.
    \end{itemize}

    Suppose we want to evaluate $n$ data sets and $k$ algorithms. The construction of \textbf{Friedman test} works as follows:

    \begin{itemize}
      \item Rank each algorithm for each data set separately from the best-performing algorithm (rank 1) to worst-performing algorithm using any performance measure of interest.
      \item If there is a $d$-way tie after rank $r$, assign a rank of $ \left[(r+1) + (r+2) + ... + (r+d)\right] /d $ to each tied classifiers.
      \item $R_{ij}$ is the rank of algorithm $j$ on data set $i$.
    \end{itemize}

    \framebreak

    After obtaining the rank for each algorithm $j$ on different data sets $i$, calculate the following quantities:

    \begin{itemize}
      \item The overall mean rank:
      $ \bar{R} = \frac{1}{nk} \sum_{i=1}^{n} \sum_{j=1}^{k} R_{ij} $
      \item The sum of squares total:
      $ SS_{Total} = n \sum_{j=1}^{k} (\bar{R}_{.j} - \bar{R})^2 $ where $\bar{R}_{.j} =  \frac{1}{n} \sum_{i=1}^{n} R_{ij}$.
      \item The sum of squares error:
      $ SS_{Error} = \frac{1}{n(k-1)} \sum_{i=1}^{n} \sum_{j=1}^{k} (R_{ij} - \bar{R})^2 $
    \end{itemize}

    The Friedman test statistic is calculated as:

    $${\chi_F}^2 = \frac{SS_{Total}}{SS_{Error}} \sim \chi_{k-1}^2 \text{ for
    large n ($>$15) and k ($>$5).}$$

    For smaller n and k, the $\chi^2$ approximation is imprecise and a look up of $\chi_F^2$ values that were approximated specifically for the Friedman test is suggested.
    \end{frame}


    \begin{frame}[c,allowframebreaks]{Post-Hoc Tests}
    A Friedman test checks if all algorithms are ranked equally or not. However,
    it does not provide information w.r.t.\ the best performing algorithm.
    To address this issue, post-hoc tests can be used.

    \bigskip

    \textbf{Post-hoc Nemenyi test}:
    \begin{itemize}
    \item Compares all algorithms pairwise to find the best-performing algorithm after $H_0$ of the Friedman-test was rejected.
    \item For $n$ data sets and $k$ algorithms $\frac{n(n-1)}{2}$ comparisons are made.
    \item Calculate the average rank of algorithm $j$ on all $n$ data sets: $\bar{R}_{.j} =\frac{1}{n} \sum_{i=1}^n R_{ij}$
    \end{itemize}

    For any two algorithms $j_1$ and $j_2$, we compute the test statistic as:
    $$q = \frac{\bar{R}_{.j_1} - \bar{R}_{.j_2}}{\sqrt{\frac{k(k+1)}{6n}}}.$$

    \framebreak

    Critical Difference Plot:
    \begin{itemize}
        \item quick way to see what differences are significant across all
            compared learners
        \item all learners that do not differ by at least the critical
            difference are connected by line
        \item a learner not connected to another learner and of lower rank can
            be considered better according to the chosen significance level
    \end{itemize}

    \begin{center}
        \includegraphics[height=.5\textheight]{crit-diff-nemenyi}
    \end{center}

    \framebreak

    \textbf{Post-hoc Bonferonni-Dunn test}:

    \begin{itemize}
    \item Compares all algorithms with a baseline (i.e.\ $k-1$ comparisons).
    \item It is used after a Friedman test to find which algorithms differ from the baseline significantly.
    \item It uses the Bonferonni correction to prevent randomly accepting one of the algorithms as significant due to multiple testing.
    \end{itemize}
    The test statistic is the same as before:
    $$q = \frac{\bar{R}_{.j_1} - \bar{R}_{.j_2}}{\sqrt{\frac{k(k+1)}{6n}}}.$$

    The performance of $j_1$ and $j_2$ are significantly different when $|q| > q_{\alpha}$, where the critical value $q_{\alpha}$ is obtained from a table of the studentized range statistic scaled by dividing it by $\sqrt{2}$.

    \framebreak

    \begin{itemize}
        \item learners within the baseline interval (gray line) perform not
            significantly different from the baseline
    \end{itemize}
    \begin{center}
        \includegraphics[height=.6\textheight]{crit-diff-bd}
    \end{center}

    \end{frame}

\end{document}
