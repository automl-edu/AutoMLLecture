
\pdfminorversion=4 % for acroread
\documentclass[aspectratio=169,t,xcolor={usenames,dvipsnames}]{beamer}
%\documentclass[t,handout,xcolor={usenames,dvipsnames}]{beamer}
\usepackage{../beamerstyle}
\usepackage{dsfont}
\usepackage{bm}
\usepackage[english]{babel}
\usepackage[utf8]{inputenc}
\usepackage{graphicx}
\usepackage{algorithm}
\usepackage[ruled,vlined,algo2e,linesnumbered]{algorithm2e}
%\usepackage[boxed,vlined]{algorithm2e}
\usepackage{hyperref}
\usepackage{booktabs}
\usepackage{mathtools}

\usepackage{amsmath,amssymb}
\usepackage{listings}
\lstset{frame=lines,framesep=3pt,numbers=left,numberblanklines=false,basicstyle=\ttfamily\small}

\usepackage{subfig}
\usepackage{multicol}
%\usepackage{appendixnumberbeamer}
%
\usepackage{tcolorbox}

\usepackage{pgfplots}
\usepackage{tikz}
\usetikzlibrary{trees} 
\usetikzlibrary{shapes.geometric}
\usetikzlibrary{positioning,shapes,shadows,arrows,calc,mindmap}
\usetikzlibrary{positioning,fadings,through}
\usetikzlibrary{decorations.pathreplacing}
\usetikzlibrary{intersections}
\usetikzlibrary{positioning,fit,calc,shadows,backgrounds}
\pgfdeclarelayer{background}
\pgfdeclarelayer{foreground}
\pgfsetlayers{background,main,foreground}
\tikzstyle{activity}=[rectangle, draw=black, rounded corners, text centered, text width=8em]
\tikzstyle{data}=[rectangle, draw=black, text centered, text width=8em]
\tikzstyle{myarrow}=[->, thick, draw=black]

% Define the layers to draw the diagram
\pgfdeclarelayer{background}
\pgfdeclarelayer{foreground}
\pgfsetlayers{background,main,foreground}

%\usepackage{listings}
%\lstset{numbers=left,
%  showstringspaces=false,
%  frame={tb},
%  captionpos=b,
%  lineskip=0pt,
%  basicstyle=\ttfamily,
%%  extendedchars=true,
%  stepnumber=1,
%  numberstyle=\small,
%  xleftmargin=1em,
%  breaklines
%}

 
\definecolor{blue}{RGB}{0, 74, 153}

\usetheme{Boadilla}
%\useinnertheme{rectangles}
\usecolortheme{whale}
\setbeamercolor{alerted text}{fg=blue}
\useoutertheme{infolines}
\setbeamertemplate{navigation symbols}{\vspace{-5pt}} % to lower the logo
\setbeamercolor{date in head/foot}{bg=blue} % blue
\setbeamercolor{date in head/foot}{fg=white}
\setbeamercolor{author in head/foot}{bg=blue} %blue
\setbeamercolor{title in head/foot}{bg=blue} % blue
\setbeamercolor{title}{fg=white, bg=blue}
\setbeamercolor{block title}{fg=white,bg=blue}
\setbeamercolor{block body}{bg=blue!10}
\setbeamercolor{frametitle}{fg=white, bg=blue}
\setbeamercovered{invisible}

\makeatletter
\setbeamertemplate{footline}
{
  \leavevmode%
  \hbox{%
  \begin{beamercolorbox}[wd=.333333\paperwidth,ht=2.25ex,dp=1ex,center]{author in head/foot}%
    \usebeamerfont{author in head/foot}\insertshortauthor
  \end{beamercolorbox}%
  \begin{beamercolorbox}[wd=.333333\paperwidth,ht=2.25ex,dp=1ex,center]{title in head/foot}%
    \usebeamerfont{title in head/foot}\insertshorttitle
  \end{beamercolorbox}%
  \begin{beamercolorbox}[wd=.333333\paperwidth,ht=2.25ex,dp=1ex,right]{date in head/foot}%
    \usebeamerfont{date in head/foot}Week \@week, Topic \@topicnumber, Slide \insertframenumber{}\hspace*{2em}
%    \insertframenumber\hspace*{2ex} 
  \end{beamercolorbox}}%
  \vskip0pt%
}

\newcommand{\@week}{0}
\newcommand{\@topicnumber}{0}
\newcommand{\week}[1]{\renewcommand{\@week}{#1}}
\newcommand{\topicnumber}[1]{\renewcommand{\@topicnumber}{#1}}

\makeatother

%\pgfdeclareimage[height=1.2cm]{automl}{images/logos/automl.png}
%\pgfdeclareimage[height=1.2cm]{freiburg}{images/logos/freiburg}

%\logo{\pgfuseimage{freiburg}}

\input{../latex_main/macros}






\title[AutoML: Risks]{AutoML: Evaluation} % week title
\subtitle{Nested Resampling} % video title
\author[Lars Kotthoff]{\underline{Bernd Bischl}\footnote{Some slides taken from Bernd Bischl's ``Introduction to Machine Learning'' lecture at LMU. \lit{\href{https://introduction-to-machine-learning.netlify.app/}{Bischl }} %\url{https://compstat-lmu.github.io/lecture_i2ml/index.html}
} \and Frank Hutter \and \underline{Lars Kotthoff}\newline \and Marius Lindauer \and Joaquin Vanschoren}
\institute{}
\date{}
\week{2}
\topicnumber{4}

\newcommand\reffootnote[1]{%
    \begingroup
    \renewcommand\thefootnote{}\footnote{
        \tiny #1
    \vspace*{1em}}%
    \addtocounter{footnote}{-1}%
    \endgroup
}

% \AtBeginSection[] % Do nothing for \section*
% {
%   \begin{frame}{Outline}
%     \bigskip
%     \vfill
%     \tableofcontents[currentsection]
%   \end{frame}
% }

\begin{document}
	
	\maketitle

    \begin{frame}[c]{Motivation}
    Selecting the best model from a set of potential candidates (e.g.\ different
    classes of learners, different hyperparameter settings, different feature
    sets, different preprocessing\ldots) is an important part of most  machine
    learning problems. However,

    \begin{itemize}
        \item cannot evaluate selected learner on the same
            resampling splits used to select it
        \item repeatedly evaluating learner on same test set or same CV splits
            ``leaks'' information about test set into evaluation
        \item danger of overfitting to the resampling splits or overtuning
        \item final performance estimate would be optimistically biased
        \item similar to multiple hypothesis testing
    \end{itemize}
    \end{frame}

    \begin{frame}[c,allowframebreaks]{Motivating Example}
    \begin{itemize}
        \item binary classification problem with equal class sizes
        \item learner with hyperparameter $\conf$
        \item learner is (nonsense) feature-independent classifier that picks
            random labels with equal probability,
              $\conf$ has no effect
        \item true generalization error is 50\%
        \item cross-validation of learner (with any fixed $\conf$) will easily show this
          (if the partitioned data set for CV is not too small)
        \item let's ``tune'' it by trying out 100 different $\conf$ values
        \item repeat this experiment 50 times and average results
    \end{itemize}

    \framebreak

    \begin{center}
        \includegraphics[height=.5\textheight]{example-nested-resampling}
    \end{center}

    \begin{itemize}
    \item shown is best ``tuning error'' (i.e.\ performance of
        model with fixed $\conf$ in cross-validation) after $k$ tuning iterations
    \item evaluated for different data set sizes
    \end{itemize}

    \begin{center}
        \includegraphics[height=.6\textheight]{dist-tuning1}
    \end{center}

    \begin{itemize}
    \item for one experiment, CV score is close to 0.5, as expected
    \item errors essentially sampled from (rescaled) binomial distribution
    \item scores from multiple experiments also arranged around expected mean of 0.5
    \end{itemize}

    \framebreak

    \begin{center}
        \includegraphics[height=.55\textheight]{dist-tuning2}
    \end{center}

    \begin{itemize}
    \item tuning means we take the minimum of the scores
    \item not estimate of average performance, but best-case performance
    \item the more we sample, the more ``biased'' this value becomes
        $\rightarrow$ unrealistic generalization performance estimate
    \end{itemize}
    \end{frame}

    \begin{frame}[c,allowframebreaks]{Untouched Test Set Principle}
    Instead: simulate what actually happens when applying machine learning
    models

    \begin{itemize}
    \item all parts of model construction (including model selection,
      preprocessing) evaluated \textbf{on training data}
    \item test set only touched once, so no way of ``cheating''
    \item test dataset is only used once \emph{after} model is completely
        trained (including e.g.\ deciding hyper-parameter values)
    \item performance estimates from test set now \textbf{unbiased estimates} of the true performance

    \framebreak

    \item for steps that themselves require resampling (e.g.\ hyperparameter tuning) this results
      in \textbf{nested resampling}, i.e.\ resampling strategies for both
      \begin{itemize} 
      \item inner evaluation to find what works best based on training data 
      \item outer evaluation on data not used in inner evaluation to get unbiased estimates of expected performance on new data
      \end{itemize}
    \end{itemize}
    \end{frame}

    \begin{frame}[c,allowframebreaks]{Nested Resampling}

        \begin{itemize}
            \item holdout can be generalized to resampling for more reliable
                generalization performance estimates
            \item resampling can be generalized to nested resampling
            \item nested resampling loops for inner and outer evaluation for
                hyperparameter tuning
        \end{itemize}

    \framebreak

    Example: four-fold CV for inner resampling, three-fold CV in outer
    resampling

    \begin{center}
        \includegraphics[height=0.6\textheight]{Nested_Resampling.png}
    \end{center}

    \framebreak

    \begin{footnotesize}
    In each iteration of the outer loop:
    \begin{itemize}
    \item split light green testing data
    \item run hyperparameter tuner on dark green part of data, i.e.\
      evaluate each $\conf_i$ through four-fold CV on dark green part
    \end{itemize}
    \end{footnotesize}

    \begin{center}
        \includegraphics[height=0.55\textheight]{Nested_Resampling.png}
    \end{center}

    \framebreak

    \begin{footnotesize}
    In each iteration of the outer loop:
    \begin{itemize}
    \item return winning $\finconf$ that performed best on the grey inner test sets
    \item re-train model on full outer dark green training set
    \item evaluate model on outer light green test set
    \end{itemize}
    \end{footnotesize}

    \begin{center}
        \includegraphics[height=0.55\textheight]{Nested_Resampling.png}
    \end{center}

    \framebreak

    \begin{footnotesize}
    $\rightarrow$ error estimates on outer samples (light green) are unbiased
    because this data was not used process constructing the tested model
    \end{footnotesize}

    \begin{center}
        \includegraphics[height=0.6\textheight]{Nested_Resampling.png}
    \end{center}

    \end{frame}


    \begin{frame}[c]{Nested Resampling Example}

    Revisited motivating example: expected performance estimate with nested
    resampling:

    \begin{center}
        \includegraphics[width=0.8\textwidth]{nested-resampling-example}
    \end{center}


    \end{frame}

\end{document}
