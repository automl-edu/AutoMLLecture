
\pdfminorversion=4 % for acroread
\documentclass[aspectratio=169,t,xcolor={usenames,dvipsnames}]{beamer}
%\documentclass[t,handout,xcolor={usenames,dvipsnames}]{beamer}
\usepackage{../beamerstyle}
\usepackage{dsfont}
\usepackage{bm}
\usepackage[english]{babel}
\usepackage[utf8]{inputenc}
\usepackage{graphicx}
\usepackage{algorithm}
\usepackage[ruled,vlined,algo2e,linesnumbered]{algorithm2e}
%\usepackage[boxed,vlined]{algorithm2e}
\usepackage{hyperref}
\usepackage{booktabs}
\usepackage{mathtools}

\usepackage{amsmath,amssymb}
\usepackage{listings}
\lstset{frame=lines,framesep=3pt,numbers=left,numberblanklines=false,basicstyle=\ttfamily\small}

\usepackage{subfig}
\usepackage{multicol}
%\usepackage{appendixnumberbeamer}
%
\usepackage{tcolorbox}

\usepackage{pgfplots}
\usepackage{tikz}
\usetikzlibrary{trees} 
\usetikzlibrary{shapes.geometric}
\usetikzlibrary{positioning,shapes,shadows,arrows,calc,mindmap}
\usetikzlibrary{positioning,fadings,through}
\usetikzlibrary{decorations.pathreplacing}
\usetikzlibrary{intersections}
\usetikzlibrary{positioning,fit,calc,shadows,backgrounds}
\pgfdeclarelayer{background}
\pgfdeclarelayer{foreground}
\pgfsetlayers{background,main,foreground}
\tikzstyle{activity}=[rectangle, draw=black, rounded corners, text centered, text width=8em]
\tikzstyle{data}=[rectangle, draw=black, text centered, text width=8em]
\tikzstyle{myarrow}=[->, thick, draw=black]

% Define the layers to draw the diagram
\pgfdeclarelayer{background}
\pgfdeclarelayer{foreground}
\pgfsetlayers{background,main,foreground}

%\usepackage{listings}
%\lstset{numbers=left,
%  showstringspaces=false,
%  frame={tb},
%  captionpos=b,
%  lineskip=0pt,
%  basicstyle=\ttfamily,
%%  extendedchars=true,
%  stepnumber=1,
%  numberstyle=\small,
%  xleftmargin=1em,
%  breaklines
%}

 
\definecolor{blue}{RGB}{0, 74, 153}

\usetheme{Boadilla}
%\useinnertheme{rectangles}
\usecolortheme{whale}
\setbeamercolor{alerted text}{fg=blue}
\useoutertheme{infolines}
\setbeamertemplate{navigation symbols}{\vspace{-5pt}} % to lower the logo
\setbeamercolor{date in head/foot}{bg=blue} % blue
\setbeamercolor{date in head/foot}{fg=white}
\setbeamercolor{author in head/foot}{bg=blue} %blue
\setbeamercolor{title in head/foot}{bg=blue} % blue
\setbeamercolor{title}{fg=white, bg=blue}
\setbeamercolor{block title}{fg=white,bg=blue}
\setbeamercolor{block body}{bg=blue!10}
\setbeamercolor{frametitle}{fg=white, bg=blue}
\setbeamercovered{invisible}

\makeatletter
\setbeamertemplate{footline}
{
  \leavevmode%
  \hbox{%
  \begin{beamercolorbox}[wd=.333333\paperwidth,ht=2.25ex,dp=1ex,center]{author in head/foot}%
    \usebeamerfont{author in head/foot}\insertshortauthor
  \end{beamercolorbox}%
  \begin{beamercolorbox}[wd=.333333\paperwidth,ht=2.25ex,dp=1ex,center]{title in head/foot}%
    \usebeamerfont{title in head/foot}\insertshorttitle
  \end{beamercolorbox}%
  \begin{beamercolorbox}[wd=.333333\paperwidth,ht=2.25ex,dp=1ex,right]{date in head/foot}%
    \usebeamerfont{date in head/foot}Week \@week, Topic \@topicnumber, Slide \insertframenumber{}\hspace*{2em}
%    \insertframenumber\hspace*{2ex} 
  \end{beamercolorbox}}%
  \vskip0pt%
}

\newcommand{\@week}{0}
\newcommand{\@topicnumber}{0}
\newcommand{\week}[1]{\renewcommand{\@week}{#1}}
\newcommand{\topicnumber}[1]{\renewcommand{\@topicnumber}{#1}}

\makeatother

%\pgfdeclareimage[height=1.2cm]{automl}{images/logos/automl.png}
%\pgfdeclareimage[height=1.2cm]{freiburg}{images/logos/freiburg}

%\logo{\pgfuseimage{freiburg}}

\input{../latex_main/macros}






\title[AutoML: Risks]{AutoML: Evaluation} % week title
\subtitle{Overview and Motivation} % video title
\author[Lars Kotthoff]{Bernd Bischl \and Frank Hutter \and \underline{Lars Kotthoff}\newline \and Marius Lindauer \and Joaquin Vanschoren}
\institute{}
\date{}
\week{2}
\topicnumber{1}

\newcommand\reffootnote[1]{%
    \begingroup
    \renewcommand\thefootnote{}\footnote{
        \tiny #1
    \vspace*{1em}}%
    \addtocounter{footnote}{-1}%
    \endgroup
}

% \AtBeginSection[] % Do nothing for \section*
% {
%   \begin{frame}{Outline}
%     \bigskip
%     \vfill
%     \tableofcontents[currentsection]
%   \end{frame}
% }

\begin{document}
	
	\maketitle
	
\begin{frame}[c]{Training Machine Learning Models}
    \begin{itemize}
        \item fundamentally an optimization problem
        \item determine model parameters such that loss on data is minimized
        \item quality of fit depends on model class (i.e.\ degrees of freedom)
    \end{itemize}
    \begin{center}
    \includegraphics{fit}\\
    Which model is best?
    \end{center}
\end{frame}


\begin{frame}[c]{Generalization}
    \begin{itemize}
        \item we want models that \emph{generalize} -- make ``reasonable''
            predictions on new data
            \begin{itemize}
                \item ignore outliers
                \item smooth
                \item captures general trend
            \end{itemize}
    \end{itemize}
    \begin{center}
        \includegraphics[width=.5\textwidth]{overfitting}
    \end{center}
    Usually model performance gets better with more data/higher model complexity
    and then worse, but see \lit{\href{https://arxiv.org/pdf/1912.02292.pdf}{Nakkiran et al. 2019}}%\url{https://openai.com/blog/deep-double-descent/}.
    %\reffootnote{\url{https://towardsdatascience.com/three-crazily-simple-recipes-to-fight-overfitting-in-deep-learning-models-314ae7660495}}
\end{frame}

\begin{frame}[c]{Overview}
    \begin{itemize}
        \item evaluating machine learning models and quantifying generalization
            performance
        \item learning curves
        \item comparing multiple models/learners on multiple data sets
        \item statistical tests
        \item higher levels of optimization, higher levels of evaluation
            \begin{itemize}
                \item automated machine learning (meta-optimization) can lead to
                    meta-overfitting
                \item simple training/testing split(s) no longer sufficient
                    $\rightarrow$ nested evaluation
            \end{itemize}
    \end{itemize}
\end{frame}

\begin{frame}[c]{Evaluate Early, Evaluate Often}
    \begin{center}
        \includegraphics[width=.6\textwidth]{datasaurus}
        \reffootnote{\url{https://www.autodeskresearch.com/publications/samestats}}
    \end{center}
\end{frame}

\end{document}
